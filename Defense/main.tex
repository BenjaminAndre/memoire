\documentclass[french,compress]{beamer}

\usepackage{amsmath}
\usepackage{tikz}
\usepackage[utf8]{inputenc}
\usepackage{longtable}
\usepackage[T1]{fontenc}
\usepackage{epigraph}
\usepackage{fancyhdr}
\usepackage{float}
\usepackage{xcolor}
\usepackage{eurosym}
\usepackage{calc}
\usepackage{hyperref}
\usepackage{multirow}
\usepackage{caption}
\usepackage[Algorithme]{algorithm}
\usepackage{algorithmic}
\usepackage{enumerate}
\usepackage[french]{babel}
\usepackage{tcolorbox}
\usepackage{multicol}
\usepackage{etoolbox,refcount}
\usepackage{listings}
\usepackage{amssymb}
\usepackage{subcaption}
\usepackage{lmodern}
\usepackage{thmtools}
\usepackage{tikz-cd}
\usepackage{xcolor, colortbl}
\usepackage[title]{appendix}

%
%%%%%%% Sub librairies

\usetikzlibrary{arrows,automata}
\usetikzlibrary{decorations.pathreplacing,shapes,arrows,positioning}
\usetikzlibrary{calc}
\usetikzlibrary{positioning}
\usetikzlibrary{snakes}
\usetikzlibrary{shapes,fit}
\usetikzlibrary{positioning}
\usetikzlibrary{automata,arrows,trees,positioning,shapes,calc}

%
%%%%% Custom commands
%



\lstset{
	keywordstyle=\color{red},
	basicstyle=\scriptsize\ttfamily,
	commentstyle=\ttfamily\itshape\color{gray},
	stringstyle=\ttfamily,
	showstringspaces=false,
	breaklines=true,
	frameround=ffff,
	rulecolor=\color{black}
}


%  █████  ██       ██████   ██████
% ██   ██ ██      ██       ██    ██
% ███████ ██      ██   ███ ██    ██
% ██   ██ ██      ██    ██ ██    ██
% ██   ██ ███████  ██████   ██████


\renewcommand{\algorithmicrequire}{\textbf{Requis:}}
\renewcommand{\algorithmicensure}{\textbf{Promet:}}
\renewcommand{\algorithmicend}{\textbf{fin}}
\renewcommand{\algorithmicif}{\textbf{si}}
\renewcommand{\algorithmicthen}{\textbf{alors}}
\renewcommand{\algorithmicelse}{\textbf{sinon}}
\renewcommand{\algorithmicelsif}{\algorithmicelse\ \algorithmicif}
\renewcommand{\algorithmicendif}{\algorithmicend\ \algorithmicif}
\renewcommand{\algorithmicfor}{\textbf{pour}}
\renewcommand{\algorithmicforall}{\textbf{pour chaque}}
\renewcommand{\algorithmicdo}{\textbf{faire}}
\renewcommand{\algorithmicendfor}{\algorithmicend\ \algorithmicfor}
\renewcommand{\algorithmicwhile}{\textbf{tant que}}
\renewcommand{\algorithmicendwhile}{\algorithmicend\ \algorithmicwhile}
\renewcommand{\algorithmicloop}{\textbf{boucle}}
\renewcommand{\algorithmicendloop}{\algorithmicend\ \algorithmicloop}
\renewcommand{\algorithmicrepeat}{\textbf{répéter}}
\renewcommand{\algorithmicuntil}{\textbf{jusqu'à}}
\renewcommand{\algorithmicprint}{\textbf{afficher}}
\renewcommand{\algorithmicreturn}{\textbf{retourner}}
\renewcommand{\algorithmictrue}{\textbf{vrai}}
\renewcommand{\algorithmicfalse}{\textbf{faux}}

\renewcommand{\algorithmicand}{\textbf{et}}
\renewcommand{\algorithmicor}{\textbf{ou}}
%%%%%%%%%%%%%%%%%%%%%%%%%%%%%%%%%%%%%%%%%%%%%%%%%%%

\newcommand{\todo}[1]{\textcolor{red}{\emph{\textbf{TODO} : #1}}}
%%%%%%%%%%%%%%%%%%%%%%%%%%%%%%%%%%%%%%%%%%%%%%%%%%%%

% ██   ██ ███████ ██      ██████
% ██   ██ ██      ██      ██   ██
% ███████ █████   ██      ██████
% ██   ██ ██      ██      ██
% ██   ██ ███████ ███████ ██



\newcommand{\hdelta}{\hat{\delta}}
\newcommand{\automaton}{$A=(Q,\Sigma, q_0, \delta, F)$\ }
\newcommand{\automatonbis}{$B=(Q_B,\Sigma_b, q_b, \delta_b, F_b)$\ }
\newcommand{\fifo}{$F=(Q,C, \Sigma, q_0, \Theta, \delta)$\ }
\newcommand{\fifoA}{$A=(Q_A,C, \Sigma, q_{0A}, \Theta_A, \delta_A)$\ }
\newcommand{\fifoB}{$B=(Q_B,C, \Sigma, q_{0B}, \Theta_B, \delta_B)$\ }
\newcommand{\tsys}{$\mathcal{T}=(S,\Theta, \rightarrow)$\ }
\newcommand{\barTheta}{$\bar{\Theta}$\ }

\newcommand{\rf}{$R_F$\ } % F comme états finaux
\newcommand{\ra}{$R_A$\ } % comme ça on sait de quel automate on parle
\newcommand{\rl}{$R_L$\ } % comme ça on sait de quel langage on parle
\newcommand{\ro}{$R_O$\ } % comme ça on sait de quelle table O on parle
\newcommand{\fl}{$\mathcal{F}(L)$\ }
\newcommand{\wl}{$\mathcal{W}(L)$\ }
\newcommand{\xor}{\oplus}
\newcommand{\alfx}{$AL(F)\xor L$\ }
\newcommand{\bt}{\bar{\theta}}

%%%%%%%%%%%%%%%%%%%%%%%%%%%%%%%%%%%%%%%%%%%%%%%%%%%%

\usetheme[navigation]{UMONS}
%\usetheme[navigation, no-subsection, no-totalframenumber]{UMONS}

\newcommand{\IR}{\mathbb{R}}


\title{Vérification de la sécurité d'automates à files par apprentissage actif}
\author[B. André]{Benjamin André}
\institute[(Info)]{%
	Département d'Informatique\\
	Université de Mons
	\\[2ex]
	\includegraphics[height=4ex]{res/UMONS}\hspace{2em}%
	\raisebox{-1ex}{\includegraphics[height=6ex]{res/UMONS_FS}}
}

\begin{document}
	
	\begin{frame}[plain]
		\titlepage
	\end{frame}
	
	\begin{frame}
		\tableofcontents
	\end{frame}
	
	
	\section{Introduction}
	\subsection{Automates à files}
		\begin{frame}
			\frametitle{Automates à files}
			\vspace{-0.5cm}
			\only<1>{\vspace{-1.3cm}
			\begin{figure}[H]
				\centering
				\begin{tikzpicture}[->,>=stealth',shorten >=1pt,auto,node distance=2.5cm, semithick, bend angle=10,  scale=0.7, every node/.style={scale=0.7}]
					
					\tikzstyle{every state}=[circle]
					
					\node[initial,state] (A)                    {$q_0$};
					\node[state]         (B) [above right= 1cm and 3 cm of A] {$q_1$};
					\node[state]         (C) [below right= 1cm and 3 cm of A] {$q_2$};
					\node[state]         (D) [above right= 1cm and 3 cm of C] {$q_3$};
					
					\path
					(A) edge node {$\theta_1(a!0)$} (B)
					(A) edge node[below left] {$\theta_2(a!1)$} (C)
					(B) edge node {$\theta_4(a?0)$} (D)
					(B) edge[loop above] node {$\theta_3(b!1)$} (B)
					(C) edge node[below right] {$\theta_6(a?1)$} (D)
					(C) edge [loop below] node{$\theta_5(b!0)$} (C)
					(D) edge node[above] {$\theta_7(b?0)$} (A)
					;
				\end{tikzpicture}
			\end{figure}
			}
		
			\only<2>{
				\begin{figure}[H]
					\centering
					\begin{tikzpicture}[->,>=stealth',shorten >=1pt,auto,node distance=2.5cm, semithick, bend angle=10,  scale=0.7, every node/.style={scale=0.7}]
						
						\tikzstyle{every state}=[circle]
						
						\node[initial,state,fill=green!60!white] (A)                    {$q_0$};
						\node[state]         (B) [above right= 1cm and 3 cm of A] {$q_1$};
						\node[state]         (C) [below right= 1cm and 3 cm of A] {$q_2$};
						\node[state]         (D) [above right= 1cm and 3 cm of C] {$q_3$};
						
						\path
						(A) edge node {$\theta_1(a!0)$} (B)
						(A) edge node[below left] {$\theta_2(a!1)$} (C)
						(B) edge node {$\theta_4(a?0)$} (D)
						(B) edge[loop above] node {$\theta_3(b!1)$} (B)
						(C) edge node[below right] {$\theta_6(a?1)$} (D)
						(C) edge [loop below] node{$\theta_5(b!0)$} (C)
						(D) edge node[above] {$\theta_7(b?0)$} (A)
						;
					\end{tikzpicture}
				\end{figure}
				$$\epsilon$$
				\vspace{-0.9cm}
				\begin{figure}
					a:
					\begin{tabular}{|c|}
						\hline
						$\epsilon$\\
						\hline
					\end{tabular}
				\end{figure}
				\vspace{-0.5cm}
				\begin{figure}
					b:
					\begin{tabular}{|c|}
						\hline
						$\epsilon$\\
						\hline
					\end{tabular}
				\end{figure}
			}
		
			\only<3>{
				\begin{figure}[H]
					\centering
					\begin{tikzpicture}[->,>=stealth',shorten >=1pt,auto,node distance=2.5cm, semithick, bend angle=10,  scale=0.7, every node/.style={scale=0.7}]
						
						\tikzstyle{every state}=[circle]
						
						\node[initial,state] (A)                    {$q_0$};
						\node[state]         (B) [above right= 1cm and 3 cm of A] {$q_1$};
						\node[state,fill=green!60!white]         (C) [below right= 1cm and 3 cm of A] {$q_2$};
						\node[state]         (D) [above right= 1cm and 3 cm of C] {$q_3$};
						
						\path
						(A) edge node {$\theta_1(a!0)$} (B)
						(A) edge node[below left] {$\theta_2(a!1)$} (C)
						(B) edge node {$\theta_4(a?0)$} (D)
						(B) edge[loop above] node {$\theta_3(b!1)$} (B)
						(C) edge node[below right] {$\theta_6(a?1)$} (D)
						(C) edge [loop below] node{$\theta_5(b!0)$} (C)
						(D) edge node[above] {$\theta_7(b?0)$} (A)
						;
					\end{tikzpicture}
				\end{figure}
				$$\theta_2$$
				\vspace{-0.9cm}
				\begin{figure}
					a:
					\begin{tabular}{|c|}
						\hline
						1\\
						\hline
					\end{tabular}
				\end{figure}
				\vspace{-0.5cm}
				\begin{figure}
					b:
					\begin{tabular}{|c|}
						\hline
						$\epsilon$\\
						\hline
					\end{tabular}
				\end{figure}
			}
		
			\only<4>{
				\begin{figure}[H]
					\centering
					\begin{tikzpicture}[->,>=stealth',shorten >=1pt,auto,node distance=2.5cm, semithick, bend angle=10,  scale=0.7, every node/.style={scale=0.7}]
						
						\tikzstyle{every state}=[circle]
						
						\node[initial,state] (A)                    {$q_0$};
						\node[state]         (B) [above right= 1cm and 3 cm of A] {$q_1$};
						\node[state,fill=green!60!white]         (C) [below right= 1cm and 3 cm of A] {$q_2$};
						\node[state]         (D) [above right= 1cm and 3 cm of C] {$q_3$};
						
						\path
						(A) edge node {$\theta_1(a!0)$} (B)
						(A) edge node[below left] {$\theta_2(a!1)$} (C)
						(B) edge node {$\theta_4(a?0)$} (D)
						(B) edge[loop above] node {$\theta_3(b!1)$} (B)
						(C) edge node[below right] {$\theta_6(a?1)$} (D)
						(C) edge [loop below] node{$\theta_5(b!0)$} (C)
						(D) edge node[above] {$\theta_7(b?0)$} (A)
						;
					\end{tikzpicture}
				\end{figure}
				$$\theta_2\theta_5$$
				\vspace{-0.9cm}
				\begin{figure}
					a:
					\begin{tabular}{|c|}
						\hline
						1\\
						\hline
					\end{tabular}
				\end{figure}
				\vspace{-0.5cm}
				\begin{figure}
					b:
					\begin{tabular}{|c|}
						\hline
						0\\
						\hline
					\end{tabular}
				\end{figure}
			}
		
			\only<5>{
				\begin{figure}[H]
					\centering
					\begin{tikzpicture}[->,>=stealth',shorten >=1pt,auto,node distance=2.5cm, semithick, bend angle=10,  scale=0.7, every node/.style={scale=0.7}]
						
						\tikzstyle{every state}=[circle]
						
						\node[initial,state] (A)                    {$q_0$};
						\node[state]         (B) [above right= 1cm and 3 cm of A] {$q_1$};
						\node[state,fill=green!60!white]         (C) [below right= 1cm and 3 cm of A] {$q_2$};
						\node[state]         (D) [above right= 1cm and 3 cm of C] {$q_3$};
						
						\path
						(A) edge node {$\theta_1(a!0)$} (B)
						(A) edge node[below left] {$\theta_2(a!1)$} (C)
						(B) edge node {$\theta_4(a?0)$} (D)
						(B) edge[loop above] node {$\theta_3(b!1)$} (B)
						(C) edge node[below right] {$\theta_6(a?1)$} (D)
						(C) edge [loop below] node{$\theta_5(b!0)$} (C)
						(D) edge node[above] {$\theta_7(b?0)$} (A)
						;
					\end{tikzpicture}
				\end{figure}
				$$\theta_2\theta_5\theta_5$$
				\vspace{-0.9cm}
				\begin{figure}
					a:
					\begin{tabular}{|c|}
						\hline
						0\\
						\hline
					\end{tabular}
				\end{figure}
				\vspace{-0.5cm}
				\begin{figure}
					b:
					\begin{tabular}{|c|c|}
						\hline
						0&0\\
						\hline
					\end{tabular}
				\end{figure}
			}
		
			\only<6>{
				\begin{figure}[H]
					\centering
					\begin{tikzpicture}[->,>=stealth',shorten >=1pt,auto,node distance=2.5cm, semithick, bend angle=10,  scale=0.7, every node/.style={scale=0.7}]
						
						\tikzstyle{every state}=[circle]
						
						\node[initial,state] (A)                    {$q_0$};
						\node[state]         (B) [above right= 1cm and 3 cm of A] {$q_1$};
						\node[state]         (C) [below right= 1cm and 3 cm of A] {$q_2$};
						\node[state,fill=green!60!white]         (D) [above right= 1cm and 3 cm of C] {$q_3$};
						
						\path
						(A) edge node {$\theta_1(a!0)$} (B)
						(A) edge node[below left] {$\theta_2(a!1)$} (C)
						(B) edge node {$\theta_4(a?0)$} (D)
						(B) edge[loop above] node {$\theta_3(b!1)$} (B)
						(C) edge node[below right] {$\theta_6(a?1)$} (D)
						(C) edge [loop below] node{$\theta_5(b!0)$} (C)
						(D) edge node[above] {$\theta_7(b?0)$} (A)
						;
					\end{tikzpicture}
				\end{figure}
				$$\theta_2\theta_5\theta_5\theta_6$$
				\vspace{-0.9cm}
				\begin{figure}
					a:
					\begin{tabular}{|c|}
						\hline
						$\epsilon$\\
						\hline
					\end{tabular}
				\end{figure}
				\vspace{-0.5cm}
				\begin{figure}
					b:
					\begin{tabular}{|c|c|}
						\hline
						0&0\\
						\hline
					\end{tabular}
				\end{figure}
			}
		
			\only<7>{
				\begin{figure}[H]
					\centering
					\begin{tikzpicture}[->,>=stealth',shorten >=1pt,auto,node distance=2.5cm, semithick, bend angle=10,  scale=0.7, every node/.style={scale=0.7}]
						
						\tikzstyle{every state}=[circle]
						
						\node[initial,state,fill=green!60!white] (A)                    {$q_0$};
						\node[state]         (B) [above right= 1cm and 3 cm of A] {$q_1$};
						\node[state]         (C) [below right= 1cm and 3 cm of A] {$q_2$};
						\node[state]         (D) [above right= 1cm and 3 cm of C] {$q_3$};
						
						\path
						(A) edge node {$\theta_1(a!0)$} (B)
						(A) edge node[below left] {$\theta_2(a!1)$} (C)
						(B) edge node {$\theta_4(a?0)$} (D)
						(B) edge[loop above] node {$\theta_3(b!1)$} (B)
						(C) edge node[below right] {$\theta_6(a?1)$} (D)
						(C) edge [loop below] node{$\theta_5(b!0)$} (C)
						(D) edge node[above] {$\theta_7(b?0)$} (A)
						;
					\end{tikzpicture}
				\end{figure}
				$$\theta_2\theta_5\theta_5\theta_6\theta_7$$
				\vspace{-0.9cm}
				\begin{figure}
					a:
					\begin{tabular}{|c|}
						\hline
						$\epsilon$\\
						\hline
					\end{tabular}
				\end{figure}
				\vspace{-0.5cm}
				\begin{figure}
					b:
					\begin{tabular}{|c|}
						\hline
						0\\
						\hline
					\end{tabular}
				\end{figure}
			}
		
			\only<8>{
				\begin{figure}[H]
					\centering
					\begin{tikzpicture}[->,>=stealth',shorten >=1pt,auto,node distance=2.5cm, semithick, bend angle=10,  scale=0.7, every node/.style={scale=0.7}]
						
						\tikzstyle{every state}=[circle]
						
						\node[initial,state] (A)                    {$q_0$};
						\node[state,fill=green!60!white]         (B) [above right= 1cm and 3 cm of A] {$q_1$};
						\node[state]         (C) [below right= 1cm and 3 cm of A] {$q_2$};
						\node[state]         (D) [above right= 1cm and 3 cm of C] {$q_3$};
						
						\path
						(A) edge node {$\theta_1(a!0)$} (B)
						(A) edge node[below left] {$\theta_2(a!1)$} (C)
						(B) edge node {$\theta_4(a?0)$} (D)
						(B) edge[loop above] node {$\theta_3(b!1)$} (B)
						(C) edge node[below right] {$\theta_6(a?1)$} (D)
						(C) edge [loop below] node{$\theta_5(b!0)$} (C)
						(D) edge node[above] {$\theta_7(b?0)$} (A)
						;
					\end{tikzpicture}
				\end{figure}
				$$\theta_2\theta_5\theta_5\theta_6\theta_7\theta_1$$
				\vspace{-0.9cm}
				\begin{figure}
					a:
					\begin{tabular}{|c|}
						\hline
						0\\
						\hline
					\end{tabular}
				\end{figure}
				\vspace{-0.5cm}
				\begin{figure}
					b:
					\begin{tabular}{|c|}
						\hline
						0\\
						\hline
					\end{tabular}
				\end{figure}
			}
		
			\only<9>{
				\begin{figure}[H]
					\centering
					\begin{tikzpicture}[->,>=stealth',shorten >=1pt,auto,node distance=2.5cm, semithick, bend angle=10,  scale=0.7, every node/.style={scale=0.7}]
						
						\tikzstyle{every state}=[circle]
						
						\node[initial,state] (A)                    {$q_0$};
						\node[state,fill=green!60!white]         (B) [above right= 1cm and 3 cm of A] {$q_1$};
						\node[state]         (C) [below right= 1cm and 3 cm of A] {$q_2$};
						\node[state]         (D) [above right= 1cm and 3 cm of C] {$q_3$};
						
						\path
						(A) edge node {$\theta_1(a!0)$} (B)
						(A) edge node[below left] {$\theta_2(a!1)$} (C)
						(B) edge node {$\theta_4(a?0)$} (D)
						(B) edge[loop above] node {$\theta_3(b!1)$} (B)
						(C) edge node[below right] {$\theta_6(a?1)$} (D)
						(C) edge [loop below] node{$\theta_5(b!0)$} (C)
						(D) edge node[above] {$\theta_7(b?0)$} (A)
						;
					\end{tikzpicture}
				\end{figure}
				$$\theta_2\theta_5\theta_5\theta_6\theta_7\theta_1\theta_3$$
				\vspace{-0.9cm}
				\begin{figure}
					a:
					\begin{tabular}{|c|}
						\hline
						0\\
						\hline
					\end{tabular}
				\end{figure}
				\vspace{-0.5cm}
				\begin{figure}
					b:
					\begin{tabular}{|c|c|}
						\hline
						0&1\\
						\hline
					\end{tabular}
				\end{figure}
			}
		
			\only<10>{
				\begin{figure}[H]
					\centering
					\begin{tikzpicture}[->,>=stealth',shorten >=1pt,auto,node distance=2.5cm, semithick, bend angle=10,  scale=0.7, every node/.style={scale=0.7}]
						
						\tikzstyle{every state}=[circle]
						
						\node[initial,state] (A)                    {$q_0$};
						\node[state]         (B) [above right= 1cm and 3 cm of A] {$q_1$};
						\node[state]         (C) [below right= 1cm and 3 cm of A] {$q_2$};
						\node[state,fill=green!60!white]         (D) [above right= 1cm and 3 cm of C] {$q_3$};
						
						\path
						(A) edge node {$\theta_1(a!0)$} (B)
						(A) edge node[below left] {$\theta_2(a!1)$} (C)
						(B) edge node {$\theta_4(a?0)$} (D)
						(B) edge[loop above] node {$\theta_3(b!1)$} (B)
						(C) edge node[below right] {$\theta_6(a?1)$} (D)
						(C) edge [loop below] node{$\theta_5(b!0)$} (C)
						(D) edge node[above] {$\theta_7(b?0)$} (A)
						;
					\end{tikzpicture}
				\end{figure}
				$$\theta_2\theta_5\theta_5\theta_6\theta_7\theta_1\theta_3\theta_4$$
				\vspace{-0.9cm}
				\begin{figure}
					a:
					\begin{tabular}{|c|}
						\hline
						$\epsilon$\\
						\hline
					\end{tabular}
				\end{figure}
				\vspace{-0.5cm}
				\begin{figure}
					b:
					\begin{tabular}{|c|c|}
						\hline
						0&1\\
						\hline
					\end{tabular}
				\end{figure}
			}
		
			\only<11>{
				\begin{figure}[H]
					\centering
					\begin{tikzpicture}[->,>=stealth',shorten >=1pt,auto,node distance=2.5cm, semithick, bend angle=10,  scale=0.7, every node/.style={scale=0.7}]
						
						\tikzstyle{every state}=[circle]
						
						\node[initial,state,fill=green!60!white] (A)                    {$q_0$};
						\node[state]         (B) [above right= 1cm and 3 cm of A] {$q_1$};
						\node[state]         (C) [below right= 1cm and 3 cm of A] {$q_2$};
						\node[state]         (D) [above right= 1cm and 3 cm of C] {$q_3$};
						
						\path
						(A) edge node {$\theta_1(a!0)$} (B)
						(A) edge node[below left] {$\theta_2(a!1)$} (C)
						(B) edge node {$\theta_4(a?0)$} (D)
						(B) edge[loop above] node {$\theta_3(b!1)$} (B)
						(C) edge node[below right] {$\theta_6(a?1)$} (D)
						(C) edge [loop below] node{$\theta_5(b!0)$} (C)
						(D) edge node[above] {$\theta_7(b?0)$} (A)
						;
					\end{tikzpicture}
				\end{figure}
				$$\theta_2\theta_5\theta_5\theta_6\theta_7\theta_1\theta_3\theta_4\theta_7$$
				\vspace{-0.9cm}
				\begin{figure}
					a:
					\begin{tabular}{|c|}
						\hline
						$\epsilon$\\
						\hline
					\end{tabular}
				\end{figure}
				\vspace{-0.5cm}
				\begin{figure}
					b:
					\begin{tabular}{|c|}
						\hline
						1\\
						\hline
					\end{tabular}
				\end{figure}
			}
		\end{frame}
	
		
	
	
	
	
	
	\subsection{Sécurité}
		\begin{frame}
			\frametitle{Sécurité}
			\vspace{-0.5cm}
			\begin{figure}[H]
				\centering
				\begin{tikzpicture}[->,>=stealth',shorten >=1pt,auto,node distance=2.5cm, semithick, bend angle=10,  scale=0.7, every node/.style={scale=0.7}]
					
					\tikzstyle{every state}=[circle]
					
					\node[initial,state,fill=red!60!white] (A)                    {$q_0$};
					\node[state]         (B) [above right= 1cm and 3 cm of A] {$q_1$};
					\node[state]         (C) [below right= 1cm and 3 cm of A] {$q_2$};
					\node[state]         (D) [above right= 1cm and 3 cm of C] {$q_3$};
					
					\path
					(A) edge node {$\theta_1(a!0)$} (B)
					(A) edge node[below left] {$\theta_2(a!1)$} (C)
					(B) edge node {$\theta_4(a?0)$} (D)
					(B) edge[loop above] node {$\theta_3(b!1)$} (B)
					(C) edge node[below right] {$\theta_6(a?1)$} (D)
					(C) edge [loop below] node{$\theta_5(b!0)$} (C)
					(D) edge node[above] {$\theta_7(b?0)$} (A)
					;
				\end{tikzpicture}
			\end{figure}
			$$?$$
			\vspace{-0.9cm}
			\begin{figure}
				a:
				\begin{tabular}{|c|}
					\hline
					\rowcolor{red!60!white}
					$\epsilon$\\
					\hline
				\end{tabular}
			\end{figure}
			\vspace{-0.5cm}
			\begin{figure}
				b:
				\begin{tabular}{|c|c|}
					\hline
					\rowcolor{red!60!white}
					1 & 1\\
					\hline
				\end{tabular}
			\end{figure}
	\end{frame}


	\subsection{Complexité}
	\begin{frame}
		\frametitle{Complexité}
		\vspace{-1cm}
		\begin{figure}[H]
			\centering
			\begin{tikzpicture}[->,>=stealth',shorten >=1pt,auto,node distance=2.5cm, semithick, bend angle=10,scale=0.6, every node/.style={scale=0.6}]
				
				\tikzstyle{every state}=[rectangle, rounded corners=.2cm]
				
				
				\node[state,initial] (A)                                   {$(q_0,[\epsilon,\epsilon])$};
				\node[state] (B) [above right= 0.5cm and 1 cm of A]  {$(q_1,[0,\epsilon])$};
				\node[state] (C) [below right= 0.5cm and 1 cm of A]  {$(q_2,[1,\epsilon])$};
				\node[state] (D) [above      = 1cm            of B]  {$(q_1,[0,1])$};
				\node[state] (E) [      right=           1 cm of B]  {$(q_3,[\epsilon,\epsilon])$};
				
				\node[state] (F) [below      = 1cm            of C] {\dots};
				\node[state] (G) [right      = 1cm            of C] {\dots};
				\node[state] (H) [above      = 1cm            of D] {\dots};
				\node[state] (I) [right      = 1cm            of D] {\dots};
				\node[state] (J) [right      = 1cm            of E] {\dots};
				\node[state] (K) [below       = 1cm            of A] {\dots};
				
				
				\path
				(A) edge node {$\theta_1$} (B)
				(A) edge node {$\theta_2$} (C)
				(B) edge node {$\theta_3$} (D)
				(B) edge node {$\theta_4$} (E)
				(C) edge node {$\theta_5$} (F)
				(C) edge node {$\theta_6$} (G)
				(D) edge node {$\theta_3$} (H)
				(D) edge node {$\theta_4$} (I)
				(E) edge node {$\theta_7$} (J)
				(K) edge node {$\theta_7$} (A)
				;
			\end{tikzpicture}
		\end{figure}
	\end{frame}

	\subsection{Apprentissage actif}
		\begin{frame}
			\frametitle{Apprentissage actif}
			
			\begin{figure}[H]
				\centering
				\begin{tikzpicture}[scale=0.8, every node/.style={scale=0.8}]
					\tikzset{>=latex}
					
					\draw (0,1) rectangle (5,0) node[pos=.5] {Élève ($L^*$)};
					\draw (8,1) rectangle (13,-2);
					
					\draw (8.2,-0.1) rectangle (12.8,-0.9) node[pos=.5] {Oracle d'appartenance};
					\draw (8.2,-1.1) rectangle (12.8,-1.9) node[pos=.5] {Oracle d'équivalence};
					
					\draw (10.5, -3.5) circle (1);
					\draw (2.5, -3.5) circle (1);
					
					\node[draw=none] at (10.5, 0.5) {Professeur} ;
					\node[draw=none] at (10.5,-3.5) {ADF};
					\node[draw=none] at (2.5,-3.5) {$A_O$};
					
					\draw[<->] (5,0.8) -- (8,0.8) node[pos=0.5,above] {appartenance ?};
					\draw[<->] (5,0.2) -- (8,0.2) node[pos=0.5,below] {équivalence ?};
					
					\draw[<->] (10.5, -2) -- (10.5,-2.5);
					\draw[<->] (2.5, 0) -- (2.5,-2.5);
				\end{tikzpicture}
			\end{figure}
		\end{frame}	
	
	\section{Méthode utilisée}
	
	\subsection{Apprentissage de AL(F)}
	\begin{frame}
		\frametitle{Apprentissage de AL(F)}
		\vspace{-1cm}
		\begin{figure}[H]
			\centering
			
			\resizebox{\textwidth}{!}{
				\begin{tikzpicture}
					\tikzset{>=latex}
					
					\draw (-3,-2.9) rectangle (0,-4.4) node[pos=.5] {Élève ($L^*$)};
					\draw (5,0) rectangle (17.5,-4);
					\draw[->] (0,-3.1) -- (5.3,-3.1) node[pos=0.5,below,text width=4cm] {L est-il le language à apprendre ?};
					
					\draw (-3,-0.3) rectangle (0,-1.5) node[pos=.5,text width=3cm,align=center] {Oracle\\ d'appartenance};
					\draw[->] (-1.4,-2.9) -- (-1.4,-1.5) node[pos=0.5,right,text width=5cm] {$\gamma$ appartient-il au language à apprendre ?};
					\draw[<-] (-1.6,-2.9) -- (-1.6,-1.5) node[pos=0.5,left] {oui/non};
					
					\node[draw=none] at (11.25, -3.7) {Oracle d'équivalence};
					
					\draw (5.3, -0.3) rectangle (8.6, -3.2) node[pos=0.5,text width=3cm,align=center] {L est-il un point fixe ?};
					\draw (9.4, -0.3) rectangle (12.9, -3.2) node[pos=0.5,text width=3cm,align=center] {L intersecte-t-il une région à risque ?};
					\draw (13.7, -0.3) rectangle (17.2, -3.2) node[pos=0.5,text width=3cm,align=center] {Le chemin vers l'état à risque est-il valide ?};
					
					\draw[->] (8.6, -1.75) -- (9.4,-1.75) node[pos=0.5,below] {oui};
					\draw[->] (12.9, -1.75) -- (13.7,-1.75) node[pos=0.5,below] {oui};
					\draw[->] (15.45, -4.3) -- (0, -4.3) node[pos=0.35,below] {non,contre-exemple fourni};
					\draw[-] (7.05, -3.2) -- (7.05, -4.3);
					\draw[-] (15.45, -3.2) -- (15.45, -4.3);
					
					\draw[->] (11.25,-0.3) -- (11.25,0.9)node[pos=0.5,left] {non};
					\draw[->] (15.45,-0.3) -- (15.45,0.9)node[pos=0.5,left] {oui};
					
					\node[draw=none] at (11.25, 1.15) {Automate sûr};
					\node[draw=none,text width=4.5cm] at (15.45, 1.36) {Automate à risque, contre-exemple fourni};
					
					
				\end{tikzpicture}
			}
		\end{figure}		
	\end{frame}
	
	\subsection{Langage de traces annotées}
		\begin{frame}
			\frametitle{Langage de traces annotées}
			
			\vspace{-0.5cm}
			\only<1>{
				\begin{figure}[H]
					\centering
					\begin{tikzpicture}[->,>=stealth',shorten >=1pt,auto,node distance=2.5cm, semithick, bend angle=10,  scale=0.7, every node/.style={scale=0.7}]
						
						\tikzstyle{every state}=[circle]
						
						\node[initial,state,fill=green!60!white] (A)                    {$q_0$};
						\node[state]         (B) [above right= 1cm and 3 cm of A] {$q_1$};
						\node[state]         (C) [below right= 1cm and 3 cm of A] {$q_2$};
						\node[state]         (D) [above right= 1cm and 3 cm of C] {$q_3$};
						
						\path
						(A) edge node {$\theta_1(a!0)$} (B)
						(A) edge node[below left] {$\theta_2(a!1)$} (C)
						(B) edge node {$\theta_4(a?0)$} (D)
						(B) edge[loop above] node {$\theta_3(b!1)$} (B)
						(C) edge node[below right] {$\theta_6(a?1)$} (D)
						(C) edge [loop below] node{$\theta_5(b!0)$} (C)
						(D) edge node[above] {$\theta_7(b?0)$} (A)
						;
					\end{tikzpicture}
				\end{figure}
				$$\epsilon$$
				\vspace{-0.5cm}
				$$\epsilon$$
			}
			
			\only<2>{
				\begin{figure}[H]
					\centering
					\begin{tikzpicture}[->,>=stealth',shorten >=1pt,auto,node distance=2.5cm, semithick, bend angle=10,  scale=0.7, every node/.style={scale=0.7}]
						
						\tikzstyle{every state}=[circle]
						
						\node[initial,state] (A)                    {$q_0$};
						\node[state]         (B) [above right= 1cm and 3 cm of A] {$q_1$};
						\node[state,fill=green!60!white]         (C) [below right= 1cm and 3 cm of A] {$q_2$};
						\node[state]         (D) [above right= 1cm and 3 cm of C] {$q_3$};
						
						\path
						(A) edge node {$\theta_1(a!0)$} (B)
						(A) edge node[below left] {$\theta_2(a!1)$} (C)
						(B) edge node {$\theta_4(a?0)$} (D)
						(B) edge[loop above] node {$\theta_3(b!1)$} (B)
						(C) edge node[below right] {$\theta_6(a?1)$} (D)
						(C) edge [loop below] node{$\theta_5(b!0)$} (C)
						(D) edge node[above] {$\theta_7(b?0)$} (A)
						;
					\end{tikzpicture}
				\end{figure}
				$$\theta_2$$
				\vspace{-0.5cm}
				$$\theta_2$$
			}
			
			\only<3>{
				\begin{figure}[H]
					\centering
					\begin{tikzpicture}[->,>=stealth',shorten >=1pt,auto,node distance=2.5cm, semithick, bend angle=10,  scale=0.7, every node/.style={scale=0.7}]
						
						\tikzstyle{every state}=[circle]
						
						\node[initial,state] (A)                    {$q_0$};
						\node[state]         (B) [above right= 1cm and 3 cm of A] {$q_1$};
						\node[state,fill=green!60!white]         (C) [below right= 1cm and 3 cm of A] {$q_2$};
						\node[state]         (D) [above right= 1cm and 3 cm of C] {$q_3$};
						
						\path
						(A) edge node {$\theta_1(a!0)$} (B)
						(A) edge node[below left] {$\theta_2(a!1)$} (C)
						(B) edge node {$\theta_4(a?0)$} (D)
						(B) edge[loop above] node {$\theta_3(b!1)$} (B)
						(C) edge node[below right] {$\theta_6(a?1)$} (D)
						(C) edge [loop below] node{$\theta_5(b!0)$} (C)
						(D) edge node[above] {$\theta_7(b?0)$} (A)
						;
					\end{tikzpicture}
				\end{figure}
				$$\theta_2\theta_5$$
				\vspace{-0.5cm}
				$$\theta_2\theta_5$$
			}
			
			\only<4>{
				\begin{figure}[H]
					\centering
					\begin{tikzpicture}[->,>=stealth',shorten >=1pt,auto,node distance=2.5cm, semithick, bend angle=10,  scale=0.7, every node/.style={scale=0.7}]
						
						\tikzstyle{every state}=[circle]
						
						\node[initial,state] (A)                    {$q_0$};
						\node[state]         (B) [above right= 1cm and 3 cm of A] {$q_1$};
						\node[state,fill=green!60!white]         (C) [below right= 1cm and 3 cm of A] {$q_2$};
						\node[state]         (D) [above right= 1cm and 3 cm of C] {$q_3$};
						
						\path
						(A) edge node {$\theta_1(a!0)$} (B)
						(A) edge node[below left] {$\theta_2(a!1)$} (C)
						(B) edge node {$\theta_4(a?0)$} (D)
						(B) edge[loop above] node {$\theta_3(b!1)$} (B)
						(C) edge node[below right] {$\theta_6(a?1)$} (D)
						(C) edge [loop below] node{$\theta_5(b!0)$} (C)
						(D) edge node[above] {$\theta_7(b?0)$} (A)
						;
					\end{tikzpicture}
				\end{figure}
				$$\theta_2\theta_5\theta_5$$
				\vspace{-0.5cm}
				$$\theta_2\theta_5\theta_5$$
			}
			
			\only<5>{
				\begin{figure}[H]
					\centering
					\begin{tikzpicture}[->,>=stealth',shorten >=1pt,auto,node distance=2.5cm, semithick, bend angle=10,  scale=0.7, every node/.style={scale=0.7}]
						
						\tikzstyle{every state}=[circle]
						
						\node[initial,state] (A)                    {$q_0$};
						\node[state]         (B) [above right= 1cm and 3 cm of A] {$q_1$};
						\node[state]         (C) [below right= 1cm and 3 cm of A] {$q_2$};
						\node[state,fill=green!60!white]         (D) [above right= 1cm and 3 cm of C] {$q_3$};
						
						\path
						(A) edge node {$\theta_1(a!0)$} (B)
						(A) edge node[below left] {$\theta_2(a!1)$} (C)
						(B) edge node {$\theta_4(a?0)$} (D)
						(B) edge[loop above] node {$\theta_3(b!1)$} (B)
						(C) edge node[below right] {$\theta_6(a?1)$} (D)
						(C) edge [loop below] node{$\theta_5(b!0)$} (C)
						(D) edge node[above] {$\theta_7(b?0)$} (A)
						;
					\end{tikzpicture}
				\end{figure}
				$$\theta_2\theta_5\theta_5\theta_6$$
				\vspace{-0.5cm}
				$$\bt_2\theta_5\theta_5$$
			}
			
			\only<6>{
				\begin{figure}[H]
					\centering
					\begin{tikzpicture}[->,>=stealth',shorten >=1pt,auto,node distance=2.5cm, semithick, bend angle=10,  scale=0.7, every node/.style={scale=0.7}]
						
						\tikzstyle{every state}=[circle]
						
						\node[initial,state,fill=green!60!white] (A)                    {$q_0$};
						\node[state]         (B) [above right= 1cm and 3 cm of A] {$q_1$};
						\node[state]         (C) [below right= 1cm and 3 cm of A] {$q_2$};
						\node[state]         (D) [above right= 1cm and 3 cm of C] {$q_3$};
						
						\path
						(A) edge node {$\theta_1(a!0)$} (B)
						(A) edge node[below left] {$\theta_2(a!1)$} (C)
						(B) edge node {$\theta_4(a?0)$} (D)
						(B) edge[loop above] node {$\theta_3(b!1)$} (B)
						(C) edge node[below right] {$\theta_6(a?1)$} (D)
						(C) edge [loop below] node{$\theta_5(b!0)$} (C)
						(D) edge node[above] {$\theta_7(b?0)$} (A)
						;
					\end{tikzpicture}
				\end{figure}
				$$\theta_2\theta_5\theta_5\theta_6\theta_7$$
				\vspace{-0.5cm}
				$$\bt_2\bt_5\theta_5$$
			}			
		\end{frame}
	
	

	\subsection{Oracle d'appartenance}
		\begin{frame}
			\frametitle{Oracle d'appartenance}
			
			
			$$\bt_2\bt_5\bt_5\bt_1\theta_3$$
			
			
			$$\downarrow Existe?$$
			
			$$\theta_2\theta_5\theta_5\theta_6\theta_7\theta_1\theta_3\theta_4\theta_7$$
			
			
			
		\end{frame}
	
		\begin{frame}
			\frametitle{Oracle d'appartenance}
			% MAIS ATTENDEZ, REVENONS AU SCHEMA D'AVANT. CA VEUT DIRE QU'UN DES "BLOCS" EST BON NON ?
			
			\vspace{-1cm}
			\begin{figure}[H]
				\centering
				
				\resizebox{\textwidth}{!}{
					\begin{tikzpicture}
						\tikzset{>=latex}
						
						\draw (-3,-2.9) rectangle (0,-4.4) node[pos=.5] {Élève ($L^*$)};
						\draw (5,0) rectangle (17.5,-4);
						\draw[->] (0,-3.1) -- (5.3,-3.1) node[pos=0.5,below,text width=4cm] {L est-il le language à apprendre ?};
						
						\draw [fill=black!30!green] (-3,-0.3) rectangle (0,-1.5) node[pos=.5,text width=3cm,align=center] {Oracle\\ d'appartenance};
						\draw[->] (-1.4,-2.9) -- (-1.4,-1.5) node[pos=0.5,right,text width=5cm] {$\gamma$ appartient-il au language à apprendre ?};
						\draw[<-] (-1.6,-2.9) -- (-1.6,-1.5) node[pos=0.5,left] {oui/non};
						
						\node[draw=none] at (11.25, -3.7) {Oracle d'équivalence};
						
						\draw (5.3, -0.3) rectangle (8.6, -3.2) node[pos=0.5,text width=3cm,align=center] {L est-il un point fixe ?};
						\draw (9.4, -0.3) rectangle (12.9, -3.2) node[pos=0.5,text width=3cm,align=center] {L intersecte-t-il une région à risque ?};
						\draw (13.7, -0.3) rectangle (17.2, -3.2) node[pos=0.5,text width=3cm,align=center] {Le chemin vers l'état à risque est-il valide ?};
						
						\draw[->] (8.6, -1.75) -- (9.4,-1.75) node[pos=0.5,below] {oui};
						\draw[->] (12.9, -1.75) -- (13.7,-1.75) node[pos=0.5,below] {oui};
						\draw[->] (15.45, -4.3) -- (0, -4.3) node[pos=0.35,below] {non,contre-exemple fourni};
						\draw[-] (7.05, -3.2) -- (7.05, -4.3);
						\draw[-] (15.45, -3.2) -- (15.45, -4.3);
						
						\draw[->] (11.25,-0.3) -- (11.25,0.9)node[pos=0.5,left] {non};
						\draw[->] (15.45,-0.3) -- (15.45,0.9)node[pos=0.5,left] {oui};
						
						\node[draw=none] at (11.25, 1.15) {Automate sûr};
						\node[draw=none,text width=4.5cm] at (15.45, 1.36) {Automate à risque, contre-exemple fourni};
						
						
					\end{tikzpicture}
				}
			\end{figure}
			
		\end{frame}
	
	
	\subsection{Point fixe}
		\begin{frame}
			\frametitle{Point fixe}
			
			\only<1>{$$\mathcal{F}(L)$$}
			
			\only<2>{
			\begin{figure}[H]
				\centering
				\begin{subfigure}{0.5\linewidth}
					\centering
					\begin{tikzpicture}[->,>=stealth',shorten >=1pt,auto,node distance=3cm, semithick, bend angle=10]
						\tikzstyle{every state}=[circle,scale=0.5]
						\node[state] (0) {};
						\node[state,accepting] (1)   [right of=0] {};
						\node[ellipse] [left of=0] {};
						\path
						(0) edge node {$t_p$} (1)
						;
						\draw (-1,0) ellipse[x radius=1.5,y radius=1];
					\end{tikzpicture}
					\caption{Avant}
				\end{subfigure}\hfill
				\begin{subfigure}{0.5\linewidth}
					\centering
					\begin{tikzpicture}[->,>=stealth',shorten >=1pt,auto,node distance=3cm, semithick, bend angle=10]
						\tikzstyle{every state}=[circle,scale=0.5]
						\node[state] (0) {};
						\node[state] (1) [right of=0] {};
						\node[state,accepting] (2)   [right of=1] {};
						
						\path
						(0) edge node {$\theta$} (1)
						(1) edge node {$t_q$} (2)
						;
						\draw (-1,0) ellipse[x radius=1.5,y radius=1];
					\end{tikzpicture}
					\caption{Après}
				\end{subfigure}
				\caption{$Post(L',\theta)$ avec $\theta$ de forme $c!a$}
			\end{figure}
			}
		
			\only<3>{
			\begin{figure}[H]
				\centering
				\begin{subfigure}{0.5\linewidth}
					\centering
					\begin{tikzpicture}[->,>=stealth',shorten >=1pt,auto,node distance=3cm, semithick, bend angle=10]
						\tikzstyle{every state}=[circle,scale=0.5]
						\node[state] (0) {};
						\node[state,accepting] (1)   [right of=0] {};
						\node[ellipse] [left of=0] {};
						\path
						(0) edge node {$t_p$} (1)
						;
						\draw (-1,0) ellipse[x radius=1.5,y radius=1];
					\end{tikzpicture}
					\caption{Avant}
				\end{subfigure}\hfill
				\begin{subfigure}{0.5\linewidth}
					\centering
					\begin{tikzpicture}[->,>=stealth',shorten >=1pt,auto,node distance=3cm, semithick, bend angle=10]
						\tikzstyle{every state}=[circle,scale=0.5]
						\node[state] (0) {};
						\node[state,accepting] (1)   [right of=0] {};
						\node[ellipse] [left of=0] {};
						\path
						(0) edge node {$t_q$} (1)
						;
						\draw (-1,0) ellipse[x radius=1.5,y radius=1];
					\end{tikzpicture}
					\caption{Après}
				\end{subfigure}
				\caption{$Post(L',\theta)$ avec $\theta$ de forme $\tau$}
			\end{figure}
			}
		
			\only<4>{
				\begin{figure}[H]
					\centering
					\begin{tikzpicture}[->,>=stealth',shorten >=1pt,auto,node distance=3cm, semithick, bend angle=10]
						\tikzstyle{every state}=[circle,scale=0.5]
						\node[state] (0) {$r$};
						\node[state] (1) [right of=0] {$s$};
						
						\node[state] (2) [right=4.5cm of 0] {$r'$};
						\node[state] (3) [right of=2] {$s'$};
						
						\path
						(0) edge[bend right=30] node {$\bt_s$} (3)
						(2) edge node {$\theta_s$} (3)
						;
						\path [dashed]
						(0) edge node {$\theta_s$} (1)
						;
						\draw (-1,-1.5) rectangle ++(3,3);
						\draw (4,-1.5) rectangle ++(3,3);
						
						\node at(0.5,-2) {$A'$};
						\node at(5.5,-2) {$A'_{copie}$};
						
					\end{tikzpicture}
					\caption{Première étape de $Post(L',\theta)$ avec $\theta$ de forme $c?a$}
				\end{figure}		
			}
		
			\only<5>{
			\begin{figure}[H]
				\centering
				\begin{subfigure}{0.5\linewidth}
					\centering
					\begin{tikzpicture}[->,>=stealth',shorten >=1pt,auto,node distance=3cm, semithick, bend angle=10]
						\tikzstyle{every state}=[circle,scale=0.5]
						\node[state] (0) {};
						\node[state,accepting] (1)   [right of=0] {};
						\path
						(0) edge node {$t_p$} (1)
						;
						\draw (-2,-1) rectangle ++(4,2);
					\end{tikzpicture}
					\caption{Avant}
				\end{subfigure}\hfill
				\begin{subfigure}{0.5\linewidth}
					\centering
					\begin{tikzpicture}[->,>=stealth',shorten >=1pt,auto,node distance=3cm, semithick, bend angle=10]
						\tikzstyle{every state}=[circle,scale=0.5]
						\node[state] (0) {};
						\node[state,accepting] (1)   [right of=0] {};
						
						\path
						(0) edge node {$t_q$} (1)
						;
						\draw (-2,-1) rectangle ++(4,2);
					\end{tikzpicture}
					\caption{Après}
				\end{subfigure}
				\caption{Fin du calcul de $Post(L',\theta)$ avec $\theta$ de la forme $c?a$}
			\end{figure}
			}
			
			
		\end{frame}
	
	
			\begin{frame}
			\frametitle{Point fixe}
			
			\vspace{-1cm}
			\begin{figure}[H]
				\centering
				
				\resizebox{\textwidth}{!}{
					\begin{tikzpicture}
						\tikzset{>=latex}
						
						\draw (-3,-2.9) rectangle (0,-4.4) node[pos=.5] {Élève ($L^*$)};
						\draw (5,0) rectangle (17.5,-4);
						\draw[->] (0,-3.1) -- (5.3,-3.1) node[pos=0.5,below,text width=4cm] {L est-il le language à apprendre ?};
						
						\draw [fill=black!30!green] (-3,-0.3) rectangle (0,-1.5) node[pos=.5,text width=3cm,align=center] {Oracle\\ d'appartenance};
						\draw[->] (-1.4,-2.9) -- (-1.4,-1.5) node[pos=0.5,right,text width=5cm] {$\gamma$ appartient-il au language à apprendre ?};
						\draw[<-] (-1.6,-2.9) -- (-1.6,-1.5) node[pos=0.5,left] {oui/non};
						
						\node[draw=none] at (11.25, -3.7) {Oracle d'équivalence};
						
						\draw [fill=black!30!green] (5.3, -0.3) rectangle (8.6, -3.2) node[pos=0.5,text width=3cm,align=center] {L est-il un point fixe ?};
						\draw (9.4, -0.3) rectangle (12.9, -3.2) node[pos=0.5,text width=3cm,align=center] {L intersecte-t-il une région à risque ?};
						\draw (13.7, -0.3) rectangle (17.2, -3.2) node[pos=0.5,text width=3cm,align=center] {Le chemin vers l'état à risque est-il valide ?};
						
						\draw[->] (8.6, -1.75) -- (9.4,-1.75) node[pos=0.5,below] {oui};
						\draw[->] (12.9, -1.75) -- (13.7,-1.75) node[pos=0.5,below] {oui};
						\draw[->] (15.45, -4.3) -- (0, -4.3) node[pos=0.35,below] {non,contre-exemple fourni};
						\draw[-] (7.05, -3.2) -- (7.05, -4.3);
						\draw[-] (15.45, -3.2) -- (15.45, -4.3);
						
						\draw[->] (11.25,-0.3) -- (11.25,0.9)node[pos=0.5,left] {non};
						\draw[->] (15.45,-0.3) -- (15.45,0.9)node[pos=0.5,left] {oui};
						
						\node[draw=none] at (11.25, 1.15) {Automate sûr};
						\node[draw=none,text width=4.5cm] at (15.45, 1.36) {Automate à risque, contre-exemple fourni};
						
						
					\end{tikzpicture}
				}
			\end{figure}
			
		\end{frame}
	
	\subsection{Intersection avec la zone à risque}
		\begin{frame}
			\frametitle{Intersection avec la zone à risque}
		\end{frame}
	
		\begin{frame}
			\frametitle{Intersection avec la zone à risque}
			
			\vspace{-1cm}
			\begin{figure}[H]
				\centering
				
				\resizebox{\textwidth}{!}{
					\begin{tikzpicture}
						\tikzset{>=latex}
						
						\draw (-3,-2.9) rectangle (0,-4.4) node[pos=.5] {Élève ($L^*$)};
						\draw (5,0) rectangle (17.5,-4);
						\draw[->] (0,-3.1) -- (5.3,-3.1) node[pos=0.5,below,text width=4cm] {L est-il le language à apprendre ?};
						
						\draw [fill=black!30!green] (-3,-0.3) rectangle (0,-1.5) node[pos=.5,text width=3cm,align=center] {Oracle\\ d'appartenance};
						\draw[->] (-1.4,-2.9) -- (-1.4,-1.5) node[pos=0.5,right,text width=5cm] {$\gamma$ appartient-il au language à apprendre ?};
						\draw[<-] (-1.6,-2.9) -- (-1.6,-1.5) node[pos=0.5,left] {oui/non};
						
						\node[draw=none] at (11.25, -3.7) {Oracle d'équivalence};
						
						\draw [fill=black!30!green] (5.3, -0.3) rectangle (8.6, -3.2) node[pos=0.5,text width=3cm,align=center] {L est-il un point fixe ?};
						\draw [fill=black!30!green] (9.4, -0.3) rectangle (12.9, -3.2) node[pos=0.5,text width=3cm,align=center] {L intersecte-t-il une région à risque ?};
						\draw (13.7, -0.3) rectangle (17.2, -3.2) node[pos=0.5,text width=3cm,align=center] {Le chemin vers l'état à risque est-il valide ?};
						
						\draw[->] (8.6, -1.75) -- (9.4,-1.75) node[pos=0.5,below] {oui};
						\draw[->] (12.9, -1.75) -- (13.7,-1.75) node[pos=0.5,below] {oui};
						\draw[->] (15.45, -4.3) -- (0, -4.3) node[pos=0.35,below] {non,contre-exemple fourni};
						\draw[-] (7.05, -3.2) -- (7.05, -4.3);
						\draw[-] (15.45, -3.2) -- (15.45, -4.3);
						
						\draw[->] (11.25,-0.3) -- (11.25,0.9)node[pos=0.5,left] {non};
						\draw[->] (15.45,-0.3) -- (15.45,0.9)node[pos=0.5,left] {oui};
						
						\node[draw=none] at (11.25, 1.15) {Automate sûr};
						\node[draw=none,text width=4.5cm] at (15.45, 1.36) {Automate à risque, contre-exemple fourni};
						
						
					\end{tikzpicture}
				}
			\end{figure}
			
		\end{frame}
	
	
	\subsection{Validité du chemin}
		\begin{frame}
			\frametitle{Validité du chemin}
		\end{frame}
	
		\begin{frame}
			% AYANT CODE CETTE REPONSE, BOOM, ON A UN BLOC VERT EN PLUS
			\frametitle{Validité du chemin}
			
			\vspace{-1cm}
			\begin{figure}[H]
				\centering
				
				\resizebox{\textwidth}{!}{
					\begin{tikzpicture}
						\tikzset{>=latex}
						
						\draw (-3,-2.9) rectangle (0,-4.4) node[pos=.5] {Élève ($L^*$)};
						\draw (5,0) rectangle (17.5,-4);
						\draw[->] (0,-3.1) -- (5.3,-3.1) node[pos=0.5,below,text width=4cm] {L est-il le language à apprendre ?};
						
						\draw [fill=black!30!green] (-3,-0.3) rectangle (0,-1.5) node[pos=.5,text width=3cm,align=center] {Oracle\\ d'appartenance};
						\draw[->] (-1.4,-2.9) -- (-1.4,-1.5) node[pos=0.5,right,text width=5cm] {$\gamma$ appartient-il au language à apprendre ?};
						\draw[<-] (-1.6,-2.9) -- (-1.6,-1.5) node[pos=0.5,left] {oui/non};
						
						\node[draw=none] at (11.25, -3.7) {Oracle d'équivalence};
						
						\draw [fill=black!30!green] (5.3, -0.3) rectangle (8.6, -3.2) node[pos=0.5,text width=3cm,align=center] {L est-il un point fixe ?};
						\draw [fill=black!30!green] (9.4, -0.3) rectangle (12.9, -3.2) node[pos=0.5,text width=3cm,align=center] {L intersecte-t-il une région à risque ?};
						\draw  [fill=black!30!green] (13.7, -0.3) rectangle (17.2, -3.2) node[pos=0.5,text width=3cm,align=center] {Le chemin vers l'état à risque est-il valide ?};
						
						\draw[->] (8.6, -1.75) -- (9.4,-1.75) node[pos=0.5,below] {oui};
						\draw[->] (12.9, -1.75) -- (13.7,-1.75) node[pos=0.5,below] {oui};
						\draw[->] (15.45, -4.3) -- (0, -4.3) node[pos=0.35,below] {non,contre-exemple fourni};
						\draw[-] (7.05, -3.2) -- (7.05, -4.3);
						\draw[-] (15.45, -3.2) -- (15.45, -4.3);
						
						\draw[->] (11.25,-0.3) -- (11.25,0.9)node[pos=0.5,left] {non};
						\draw[->] (15.45,-0.3) -- (15.45,0.9)node[pos=0.5,left] {oui};
						
						\node[draw=none] at (11.25, 1.15) {Automate sûr};
						\node[draw=none,text width=4.5cm] at (15.45, 1.36) {Automate à risque, contre-exemple fourni};
						
						
					\end{tikzpicture}
				}
			\end{figure}
		\end{frame}

		\begin{frame}
			% MAIS ATTENDEZ ! COMME TOUT EST BON DANS L'ORACLE D'EQUIVALENCE, L'ORACLE EST PRET NON ?
			\frametitle{Validité du chemin}
			
			\vspace{-1cm}
			\begin{figure}[H]
				\centering
				
				\resizebox{\textwidth}{!}{
					\begin{tikzpicture}
						\tikzset{>=latex}
						
						\draw (-3,-2.9) rectangle (0,-4.4) node[pos=.5] {Élève ($L^*$)};
						\draw [fill=black!30!green] (5,0) rectangle (17.5,-4);
						\draw[->] (0,-3.1) -- (5.3,-3.1) node[pos=0.5,below,text width=4cm] {L est-il le language à apprendre ?};
						
						\draw [fill=black!30!green] (-3,-0.3) rectangle (0,-1.5) node[pos=.5,text width=3cm,align=center] {Oracle\\ d'appartenance};
						\draw[->] (-1.4,-2.9) -- (-1.4,-1.5) node[pos=0.5,right,text width=5cm] {$\gamma$ appartient-il au language à apprendre ?};
						\draw[<-] (-1.6,-2.9) -- (-1.6,-1.5) node[pos=0.5,left] {oui/non};
						
						\node[draw=none] at (11.25, -3.7) {Oracle d'équivalence};
						
						\draw  (5.3, -0.3) rectangle (8.6, -3.2) node[pos=0.5,text width=3cm,align=center] {L est-il un point fixe ?};
						\draw  (9.4, -0.3) rectangle (12.9, -3.2) node[pos=0.5,text width=3cm,align=center] {L intersecte-t-il une région à risque ?};
						\draw   (13.7, -0.3) rectangle (17.2, -3.2) node[pos=0.5,text width=3cm,align=center] {Le chemin vers l'état à risque est-il valide ?};
						
						\draw[->] (8.6, -1.75) -- (9.4,-1.75) node[pos=0.5,below] {oui};
						\draw[->] (12.9, -1.75) -- (13.7,-1.75) node[pos=0.5,below] {oui};
						\draw[->] (15.45, -4.3) -- (0, -4.3) node[pos=0.35,below] {non,contre-exemple fourni};
						\draw[-] (7.05, -3.2) -- (7.05, -4.3);
						\draw[-] (15.45, -3.2) -- (15.45, -4.3);
						
						\draw[->] (11.25,-0.3) -- (11.25,0.9)node[pos=0.5,left] {non};
						\draw[->] (15.45,-0.3) -- (15.45,0.9)node[pos=0.5,left] {oui};
						
						\node[draw=none] at (11.25, 1.15) {Automate sûr};
						\node[draw=none,text width=4.5cm] at (15.45, 1.36) {Automate à risque, contre-exemple fourni};
						
						
					\end{tikzpicture}
				}
			\end{figure}
			
		\end{frame}
	
	
		\begin{frame}
			% MAIS ATTENDEZ ! J'UTILISE UNE LIBRAIRIE QUI A DEJA CODE ANGLUIN. TOUT LE TRAVAILLE FAIT PRECEDEMMENT PAYE ICI !!!!
			\frametitle{Validité du chemin}
			
			\vspace{-1cm}
			\begin{figure}[H]
				\centering
				
				\resizebox{\textwidth}{!}{
					\begin{tikzpicture}
						\tikzset{>=latex}
						
						\draw [fill=black!30!green] (-3,-2.9) rectangle (0,-4.4) node[pos=.5] {Élève ($L^*$)};
						\draw [fill=black!30!green] (5,0) rectangle (17.5,-4);
						\draw[->] (0,-3.1) -- (5.3,-3.1) node[pos=0.5,below,text width=4cm] {L est-il le language à apprendre ?};
						
						\draw [fill=black!30!green] (-3,-0.3) rectangle (0,-1.5) node[pos=.5,text width=3cm,align=center] {Oracle\\ d'appartenance};
						\draw[->] (-1.4,-2.9) -- (-1.4,-1.5) node[pos=0.5,right,text width=5cm] {$\gamma$ appartient-il au language à apprendre ?};
						\draw[<-] (-1.6,-2.9) -- (-1.6,-1.5) node[pos=0.5,left] {oui/non};
						
						\node[draw=none] at (11.25, -3.7) {Oracle d'équivalence};
						
						\draw  (5.3, -0.3) rectangle (8.6, -3.2) node[pos=0.5,text width=3cm,align=center] {L est-il un point fixe ?};
						\draw  (9.4, -0.3) rectangle (12.9, -3.2) node[pos=0.5,text width=3cm,align=center] {L intersecte-t-il une région à risque ?};
						\draw   (13.7, -0.3) rectangle (17.2, -3.2) node[pos=0.5,text width=3cm,align=center] {Le chemin vers l'état à risque est-il valide ?};
						
						\draw[->] (8.6, -1.75) -- (9.4,-1.75) node[pos=0.5,below] {oui};
						\draw[->] (12.9, -1.75) -- (13.7,-1.75) node[pos=0.5,below] {oui};
						\draw[->] (15.45, -4.3) -- (0, -4.3) node[pos=0.35,below] {non,contre-exemple fourni};
						\draw[-] (7.05, -3.2) -- (7.05, -4.3);
						\draw[-] (15.45, -3.2) -- (15.45, -4.3);
						
						\draw[->] (11.25,-0.3) -- (11.25,0.9)node[pos=0.5,left] {non};
						\draw[->] (15.45,-0.3) -- (15.45,0.9)node[pos=0.5,left] {oui};
						
						\node[draw=none] at (11.25, 1.15) {Automate sûr};
						\node[draw=none,text width=4.5cm] at (15.45, 1.36) {Automate à risque, contre-exemple fourni};
						
						
					\end{tikzpicture}
				}
			\end{figure}
		\end{frame}
	
	
	
	\section{Conclusion}
		\begin{frame}
			\frametitle{Conclusion}
		\end{frame}
	
\newpage
\bibliographystyle{siam}
\bibliography{refs.bib}
	
	
\end{document}
%%% Local Variables: 
%%% mode: latex
%%% TeX-master: t
%%% End: 
