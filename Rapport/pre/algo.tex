
Le pseudocode de l'algorithme d'Angluin est fourni par l'algorithme \ref{alg:angluin}\cite{Neider14}. Celui-ci repose sur les oracles du professeur et l'algorithme \ref{alg:corr}, remplissant les lignes de $T$ encore vides. Le code est suivi d'un exemple d'exécution.

\begin{algo}[Algorithme d'Angluin $L^*$]\label{alg:angluin}
  \begin{algorithmic}[1]
    \REQUIRE Un professeur pour le langage régulier $L\subseteq\Sigma^*$
		\ENSURE Un automate canonique décrivant $L$
		\STATE Initialiser la table d'observation $O=(R,S,T)$ avec $R=S={\epsilon}$
		\STATE $corriger(O)$
		\REPEAT
			\WHILE {$O$ n'est pas fermée ou pas cohérente}
				\IF {$O$ n'est pas fermée}
					\STATE Choisir $r\in R$ et $a\in\Sigma$ tels que $[[ua]]_O\bigcap R=\emptyset$
					\STATE $R\leftarrow R\bigcup {ua}$
					\STATE $corriger(O)$
				\ENDIF
				\IF {$O$ n'est pas cohérente}
					\STATE Choisir $uR_Lv \in R, a\in\Sigma$ et $w\in S$ tels que $T(uaw)\neq T(vaw)$
					\STATE $S\leftarrow S \bigcup {aw}$
					\STATE $corriger(O)$
				\ENDIF
			\ENDWHILE
			\STATE Construire $A_O$
			\STATE Soumettre $A_O$ à l'oracle d'équivalence
			\IF {le professeur retourne un contre-exemple $u$}
				\STATE $R\leftarrow R\bigcup Pref(u)$
				\STATE $corriger(O)$
			\ENDIF
		\UNTIL {ce que le professeur réponde "oui" à l'équivalence}
		\RETURN $A_O$
  \end{algorithmic}
\end{algo}

\begin{algo}[$corriger(O)$]\label{alg:corr}
  \begin{algorithmic}[1]
		\REQUIRE une table d'observation $O$, un professeur pour le langage régulier $L\subseteq\Sigma^*$
		\ENSURE les entrées de $O$ sont valide dans $L$
		\FORALL {entrée $u \in (R\bigcup R\Sigma)$ pour laquelle $T(u)$ n'est pas encore définie}
			\IF {$u\in L$ par le test d'appartenance }
				\STATE $T(u)\leftarrow 1$
			\ELSE
				\STATE $T(u)\leftarrow 0$
			\ENDIF
		\ENDFOR
	\end{algorithmic}
\end{algo}

Considérons l'automate $A_3$ de la figure \ref{fig:a3} construit à la section \ref{ss:mini} sur la minimisation.

%\todo{Marquer la différence entre $R_L$ et $R_O$}

\subsubsection{Première itération}

L'algorithme d'Angluin précise, pour son cas de base, une initialisation de la table $T$ avec les ensembles $R$ et $S$ contenant uniquement $\epsilon$. Le champ $R.\{a,b\}$ ($R.\Sigma$) est rempli via des requêtes d'appartenance sur les mots $a$ et $b$.

\begin{minipage}{0.5\linewidth}
	\centering
	\begin{tabular}{|c|c|}
		\hline
		$O_0$ & $\epsilon$\\
		\hline
		$\epsilon$ & 0\\
		\hline
		$a$ & 0\\
		$b$ & 0\\
		\hline
	\end{tabular}
\end{minipage}
\begin{minipage}{0.5\linewidth}
	\centering
	\begin{figure}[H]
		\centering
		\begin{tikzpicture}[->,>=stealth',shorten >=1pt,auto,node distance=3cm, semithick, bend angle=10]
		\tikzstyle{every state}=[circle]
		\node[initial, state] (A) {$[[\epsilon]]$};
		\path (A) edge [loop above] node {a,b} (A);
		\end{tikzpicture}
		\caption*{Automate $O_0$}
	\end{figure}
\end{minipage}


\vspace{1cm}
L'étape suivante consiste à vérifier la fermeture de la table d'observation $O_0$. Pour rappel :

$$ \forall u \in R, \forall a \in \Sigma, \exists v \in R, ua R_O v$$

Intuitivement, pour chaque symbole (ici, $\{a,b\}$, et ce sera vrai jusqu'à la dernière itération), tout mot candidat (dans $R$, la partie supérieure de la table) doit se retrouver, complété de ce symbole, dans une classe d'équivalence d'un autre candidat de $R$. Ici, de toute évidence, les mots $a$ et $b$ sont dans la même classe d'équivalence que $\epsilon$. Dès lors, la propriété de fermeture est respectée.

Si la fermeture est respectée, alors la question de la cohérence se pose. Pour rappel :

$$ \forall u,v \in R, u R_O v \Rightarrow \forall a \in \Sigma, ua R_O va$$

Intuitivement, si deux candidats semblent être dans la même classe d'équivalence (leur lignes dans la table supérieure sont identiques), alors pour n'importe quel symbole, les deux nouveaux mots sont également dans une même classe d'équivalence (leur lignes, potentiellement dans la partie inférieure de la table, sont identiques). N'ayant qu'un seul candidat, cette propriété est forcément respectée ($R_L$ est réflexive).

Les deux propriétés étant respectées, les classes d'équivalences sont calculées (trivialement ici), et un automate $O_0$ est proposé à l'enseignant pour vérification.

Sur cette itération, un automate initial a été proposé, et aucun état final ne pouvant être atteint avec un seul symbole, la version est minime.

\subsubsection{Seconde itération}

L'enseignant répond que non, les automates ne sont pas équivalents. Il fourni le contre-exemple $ba$. Comme il est rejeté par $O_0$, cela signifie qu'il est accepté par $A_3$. Une nouvelle table est alors construite, en ajoutant $ba$ et ses préfixes (ici, juste $b$) à $R$. $R.\Sigma$ est calculé et les mots n'ayant pas encore reçu une valeur dans $T$ sont soumis à l'enseignant pour un test d'appartenance.
Les valeurs ajoutées ou modifiées dans la table d'observation sont mises en évidence \textcolor{red}{en rouge}.
\vspace{1cm}

\begin{minipage}{0.25\linewidth}
	\centering
	\begin{tabular}{|c|c|}
		\hline
		$O_1$ & $\epsilon$\\
		\hline
		$\epsilon$ & 0\\
		\textcolor{red}{$b$} & \textcolor{red}{0}\\
		\textcolor{red}{$ba$} & \textcolor{red}{1}\\
		\hline
		$a$ & 0\\
		\textcolor{red}{$bb$} & \textcolor{red}{0}\\
		\textcolor{red}{$baa$} & \textcolor{red}{1}\\
		\textcolor{red}{$bab$} & \textcolor{red}{1}\\
		\hline
	\end{tabular}
\end{minipage}
\begin{minipage}{0.25\linewidth}
	\centering
	\begin{tabular}{|c|cc|}
		\hline
		$O_2$ & $\epsilon$ & \textcolor{red}{$a$}\\
		\hline
		$\epsilon$ & 0& \textcolor{red}{0}\\
		$b$ & 0&\textcolor{red}{1}\\
		$ba$ & 1&\textcolor{red}{1}\\
		\hline
		$a$ & 0&\textcolor{red}{0}\\
		$bb$ & 0&\textcolor{red}{1}\\
		$baa$ & 1&\textcolor{red}{1}\\
		$bab$ & 1&\textcolor{red}{1}\\
		\hline
	\end{tabular}
\end{minipage}
\begin{minipage}{0.5\linewidth}
	\centering
	\begin{figure}[H]
		\centering
		\begin{tikzpicture}[->,>=stealth',shorten >=1pt,auto,node distance=3cm, semithick, bend angle=10]
		\tikzstyle{every state}=[circle]

		\node[initial, state] (A) {$[[\epsilon]]$};
		\node[state] (B) [right of=A] {$[[b]]$};
		\node[accepting, state] [right of=B] (C) {$[[ba]]$};

		\path
		(A) edge [loop above] node {a} (A)
		(A) edge node {b} (B)
		(B) edge node {a} (C)
		(B) edge [loop above] node {b} (B)
		(C) edge [loop above] node {a,b} (C);


		\end{tikzpicture}
		\caption*{Automate $O_2$}
	\end{figure}
\end{minipage}

\vspace{1cm}
Comme pour la première itération, la fermeture est testée, suivie de la cohérence. Celle-ci n'est pas respectée : si on considère les mots $\epsilon$ et $b$, qui ont la même ligne dans la table $T$ ($\epsilon R_O b$), le symbole $a$, on obtient les mots $a$ et $ba$ qui n'ont pas la même ligne : ($\neg a R_O ba$). Le symbole $a$ est alors ajouté à $S$ et une nouvelle table $O_2$ est calculée.

La fermeture étant déjà vérifiée, la question de la cohérence est reposée, et cette fois-ci elle est vérifiée ; l'automate est construit et proposé à l'enseignant.

Sur cette itération, l'algorithme a reçu le mot $ba$ comme étant accepté. Il a du ajouter $a$ à $S$ pour permettre de différencier certains états. L'automate se voit ajouter les états $[[b]]$ et $[[ba]]$.

\subsubsection{Troisième itération}

Suivant toujours l'algorithme de comparaison d'automates détaillé dans la section \ref{ss:eqauto}, l'enseignant découvre qu'ils sont différents.

Il sort le contre-exemple $aaba$. Si c'est un contre-exemple et qu'il est accepté par $O_2$, c'est qu'il ne l'est pas (0) par $A_4$. Une nouvelle table $O_3$ doit être construite.

\begin{minipage}{0.25\linewidth}
	\centering
	\begin{tabular}{|c|cc|}
		\hline
		$O_3$ & $\epsilon$ & $a$\\
		\hline
		$\epsilon$ & 0 &0\\
		\textcolor{red}{$a$}&\textcolor{red}{0}&\textcolor{red}{0}\\
		$b$&0&1\\
		\textcolor{red}{$aa$}&\textcolor{red}{0}&\textcolor{red}{0}\\
		$ba$&1&1\\
		\textcolor{red}{$aab$}&\textcolor{red}{0}&\textcolor{red}{0}\\
		\textcolor{red}{$aaba$}&\textcolor{red}{0}&\textcolor{red}{0}\\
		\hline
		\textcolor{red}{$ab$}&\textcolor{red}{0}&\textcolor{red}{1}\\
		$bb$&0&1\\
		\textcolor{red}{$aaa$}&\textcolor{red}{0}&\textcolor{red}{0}\\
		$baa$&1&1\\
		$bab$&1&1\\
		\textcolor{red}{$aabb$}&\textcolor{red}{0}&\textcolor{red}{0}\\
		\textcolor{red}{$aabaa$}&\textcolor{red}{0}&\textcolor{red}{0}\\
		\textcolor{red}{$aabab$}&\textcolor{red}{0}&\textcolor{red}{0}\\
		\hline
	\end{tabular}
\end{minipage}
\begin{minipage}{0.33\linewidth}
	\centering
	\begin{tabular}{|c|cc|}
		\hline
		$O_4$ & $\epsilon$ & $a$\\
		\hline
		$\epsilon$ & 0 &0\\
		$a$&0&0\\
		$b$&0&1\\
		$aa$&0&0\\
		\textcolor{red}{$ab$}&\textcolor{red}{0}&\textcolor{red}{1}\\
		$ba$&1&1\\
		$aab$&0&0\\
		$aaba$&0&0\\
		\hline
		$bb$&0&1\\
		$aaa$&0&0\\
		\textcolor{red}{$aba$}&\textcolor{red}{1}&\textcolor{red}{1}\\
		\textcolor{red}{$abb$}&\textcolor{red}{0}&\textcolor{red}{1}\\
		$baa$&1&1\\
		$bab$&1&1\\
		$aabb$&0&0\\
		$aabaa$&0&0\\
		$aabab$&0&0\\
		\hline
	\end{tabular}
\end{minipage}
\begin{minipage}{0.33\linewidth}
	\centering
	\begin{tabular}{|c|ccccc|}
		\hline
		$O_7$ & $\epsilon$ & $a$&\textcolor{red}{$b$}&\textcolor{red}{$ab$}&\textcolor{red}{$ba$}\\
		\hline
		$\epsilon$&0&0&\textcolor{red}{0}&\textcolor{red}{1}&\textcolor{red}{1}\\
		$a$&0&0&\textcolor{red}{0}&\textcolor{red}{0}&\textcolor{red}{1}\\
		$b$&0&1&\textcolor{red}{0}&\textcolor{red}{1}&\textcolor{red}{1}\\
		$aa$&0&0&\textcolor{red}{0}&\textcolor{red}{0}&\textcolor{red}{0}\\
		$ab$&0&1&\textcolor{red}{0}&\textcolor{red}{1}&\textcolor{red}{1}\\
		$ba$&1&1&\textcolor{red}{1}&\textcolor{red}{1}&\textcolor{red}{1}\\
		$aab$&0&0&\textcolor{red}{0}&\textcolor{red}{0}&\textcolor{red}{0}\\
		$aaba$&0&0&\textcolor{red}{0}&\textcolor{red}{0}&\textcolor{red}{0}\\
		\hline
		$bb$&0&1&\textcolor{red}{0}&\textcolor{red}{1}&\textcolor{red}{1}\\
		$aaa$&0&0&\textcolor{red}{0}&\textcolor{red}{0}&\textcolor{red}{0}\\
		$aba$&1&1&\textcolor{red}{1}&\textcolor{red}{1}&\textcolor{red}{1}\\
		$abb$&0&1&\textcolor{red}{0}&\textcolor{red}{1}&\textcolor{red}{1}\\
		$baa$&1&1&\textcolor{red}{1}&\textcolor{red}{1}&\textcolor{red}{1}\\
		$bab$&1&1&\textcolor{red}{1}&\textcolor{red}{1}&\textcolor{red}{1}\\
		$aabb$&0&0&\textcolor{red}{0}&\textcolor{red}{0}&\textcolor{red}{0}\\
		$aabaa$&0&0&\textcolor{red}{0}&\textcolor{red}{0}&\textcolor{red}{0}\\
		$aabab$&0&0&\textcolor{red}{0}&\textcolor{red}{0}&\textcolor{red}{0}\\
		\hline
	\end{tabular}
\end{minipage}

	\begin{figure}[H]
		\centering
		\begin{tikzpicture}[->,>=stealth',shorten >=1pt,auto,node distance=3cm, semithick, bend angle=10]
		\tikzstyle{every state}=[circle]

		\node[initial, state] (A) {$[[\epsilon]]$};
		\node[state] (B) [above right of=A] {$[[a]]$};
		\node [state] (E) [right of=B] {$[[aa]]$};
		\node[state] (C) [below right of =A] {$[[b]]$};
		\node[accepting, state] [right of=C] (D) {$[[ba]]$};

		\path
		(A) edge node {a} (B)
		(A) edge node {b} (C)
		(B) edge node {b} (C)
		(B) edge node {a} (E)
		(C) edge [loop below] node {b} (C)
		(C) edge node {a} (D)
		(D) edge [loop above] node {a,b} (D)
		(E) edge [loop below] node {a,b} (E);

		\end{tikzpicture}
		\caption*{Automate $O_7$}
	\end{figure}


Ayant reçu $aaba$, ce mot et tous ses préfixes sont ajoutés à la table. L'extension $R.\Sigma$ est recalculée et la table $O_3$ est construite.

Un manquement à la fermeture est détecté : le mot $a$. En effet, en lui ajoutant le symbole $b$, on obtient $ab$ qui n'est ni dans $R$ ni en relation $R_O$ avec $a$. $ab$ est alors ajouté à $R$, et $R.\Sigma$ est étendu. La nouvelle table, $O_4$ est de nouveau testée.

$O_4$ ne respecte pas la cohérence. Les mots $\epsilon$ et $aa$ respectent \ro (leur ligne a la même valeur dans la table) mais $\neg b R_O aab$. $b$ est alors ajouté à $S$ et une nouvelle colonne associée est ajoutée à la table, donnant le table $O_5$. Celle-ci a toujours un soucis de cohérence entre $\epsilon$ et $aa$, menant à l'ajout de $ab$ à $S$ et à la création de $O_6$. Finalement, pour régler le soucis de cohérence dans $O_6$ entre $a$ et $aa$, le mot $ba$ est ajouté à $S$ et une table $O_7$ est ainsi créée avec la nouvelle colonne associée.

Cette table $O_7$ respectant la fermeture et la cohérence, l'automate associé $O_7$ est construit et soumis à l'enseignant pour être comparé à $A_3$. Celui-ci valide l'égalité et l'algorithme s'arrête : l'automate a été construit.
