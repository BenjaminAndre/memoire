Ce chapitre, comme mentionné dans l'introduction \ref{intro}, développe les différentes notions théoriques qui servent de blocs de base à l'apprentissage de la sûreté d'un automate à file avec LeVer.

En particulier, la section \ref{lang} défini les langages, suivies des expressions régulières à la section \ref{regex}, donnant leur nom aux langages réguliers.

Ensuite, avant d'énoncer les automates à files, un modèle plus simple et fini est présenté : les automates finis de la section \ref{adf}. Le Table Filling Algorithme est ensuite détaillé à la section \ref{tfa}. Celui-ci permet notemment de tester l'équivalence entre deux automates finis. Cette équivalence est un des éléments utilisés par l'algorithme d'Angluin de la section \ref{angluin}, permettant l'apprentissage actif d'automates finis.

Finalement, les automates à files sont introduits dans la section \ref{fifo}.

\section{Langage}\label{lang}Cette section s'intéresse à la notion de langage. Langages qui peuvent éventuellement être associés à un automate comme énoncé dans les sections \ref{adf:lang} pour un automate déterministe fini et \ref{fifo:lang} pour un automate à file. La section \ref{lr:def} en donne une définition. La section \ref{lr:regex} définit les expression régulières en faisant le lien avec les langages.

Un \emph{alphabet} $\Sigma$ est un ensemble fini et non vide de \emph{symboles}. Un \emph{mot} sur cet alphabet $\Sigma$ est une suite finie de $k$ éléments de $\Sigma$ notée $ w = a_1a_2\dots a_k$ où $k$ est un nombre naturel. $k$ est la \emph{longueur} de ce mot aussi notée $|w|=k$. Le \emph{mot vide} est un mot de taille $k=0$ noté $w=\epsilon$.

La \emph{concaténation} de deux mots $w=a_1a_2\dots a_k$ et $x=b_1b_2\dots b_j$ est l'opération consistant à créer un nouveau mot $wx=a_1a_2\dots a_kb_1b_2\dots b_j$ de longueur $i=k+j$.

\begin{proposition}[$\epsilon$ et la concaténation]\emph{
$\epsilon$ est \emph{l'identité pour la concaténation}, à savoir pour tout mot $w$, $w\epsilon = \epsilon w = w$.}
\end{proposition}

Cette proposition est triviale par la définition de la concaténation.

\emph{L'exponentiation} d'un symbole $a$ à la puissance $k$, notée $a^k$, retourne un mot de longueur $k$ obtenu par la concaténation de $k$ copies du symbole $a$. Noter que $a^0=\epsilon$. $\Sigma^k$ est \emph{l'ensemble des mots sur $\Sigma$} de longueur $k$. L'ensemble de \emph{tous les mots possibles sur $\Sigma$} est noté $\Sigma^* = \bigcup_{k=0}^{\infty}\Sigma^k$.


Un ensemble quelconque de mots sur $\Sigma$ est un \emph{langage}, noté $L \subseteq \Sigma^*$. Étant donné que $\Sigma^*$ est infini, $L$ peut l'être également.

\begin{example}[Langages] Voici des exemples utilisant plusieurs modes de définition. $\Sigma$ y est implicite mais peut être donné explicitement.
	\begin{itemize}
		\item $L=\{12,35,42,7,0\}$, un langage défini explicitement
		\item $L=\{0^k1^j|k+j=7\}$, les mots de 7 symboles sur $\Sigma=\{0,1\}$ commençant par zéro, un ou plusieurs $0$ et finissant par zéro, un ou plusieurs $1$. Ici, $L$ est donné par notation ensembliste
		\item $L$ contient tous les noms de villes belges. Ici $L$ est défini en langage courant.
		\item $\emptyset$ est un langage sur tout alphabet.
		\item $L=\{\epsilon\}$ ne contient que le mot vide, et est un langage sur tout alphabet.
	\end{itemize}
\end{example}

Soient $L$ et $M$ deux langages. Le langage $L \cup M = \{w | w \in L\vee w \in M\}$ est l'\emph{union} de ces deux langages. Il est composé des mots venant d'un des deux langages.

Le langage composé de tous les mots produits par la concaténation d'un mot de $L$ avec un mot de $M$ est une \emph{concaténation} de ces deux langages et s'écrit $LM = \{vw | v \in L \wedge w \in M\}$.

La \emph{fermeture} de $L$ est un langage constitué de tous les mots qui peuvent être construits par un concaténation d'un nombre arbitraire de mots de $L$, noté $L^*=\{w_1w_2\dots w_n|n\in \mathbb{N},\forall i \in \{1,2,\dots,n\}, w_i \in L\}$.

\section{Expressions régulières}\label{regex}Certains langages peuvent être exprimés par une \emph{expression régulière}.

Une expression régulière est un mot utilisant les symboles à représenter ainsi que les symboles (,),*,| qui sont réservés pour différentes opérations. Une expression régulière est construite à partir d'éléments atomiques (les symboles du langage à représenter) assemblés pour obtenir des langages plus complexes. Un langage qui peut-être représenté par une expression régulière est dit \emph{langage régulier}.

Si plusieurs expressions régulières peuvent être composées en une expression plus complexe, une expression régulière peut aussi être décomposée en ses différents composants.

\textbf{Cas de base}
Certains langages peuvent être construits directement sans passer par l'induction:

\begin{itemize}
	\item $\epsilon$ est une expression régulière. Elle décrit le langage $L(\epsilon)=\{\epsilon\}$
	\item $\emptyset$ est une expression régulière décrivant $L(\emptyset)=\emptyset$
	\item Si $a$ est un symbole, alors $a$ est une expression régulière décrivant le langage $L(a) = \{a\}$.
\end{itemize}


\textbf{Pas de récurrence}
Les autres langages réguliers sont construits suivant différentes règles d'induction présentées par ordre décroissant de priorité :

\begin{itemize}
	\item Si $E$ est une expression régulière, alors $(E)$ est une expression régulière et $L((E)) = L(E)$.
	\item Si $E$ est une expression régulière, alors $E^*$ est une expression régulière représentant la fermeture de $L(E)$, à savoir $L(E^*) = L(E)^*$.
	\item Si $E$ et $F$ sont des expressions régulières, alors $EF$ est une expression régulière décrivant la concaténation des deux langages représentés, à savoir $L(EF)=L(E)L(F)$.
	\item Si $E$ et $F$ sont des expressions régulières, alors $E+F$ est une expression régulière donnant l'union des deux langages représentés, à savoir $L(E+F)=L(E)\cup L(F)$. L'opération est associative et la priorité est à gauche.
\end{itemize}

\begin{example}[Expressions régulières]
	Soit l'expression $E = (b+ab)b^*a(a+b)^*$ qui décrit le langage $L=L(E)$.\\
	\begin{itemize}
		\item Le mot $ba$ fait partie de $L$. En effet, $ba=b\epsilon a \epsilon=(b)b^0a(a+b)^0$, ce qui respecte bien la définition de $E$.
		\item Le mot $ababbab$ fait partie de $L$. A nouveau, $ababbab=ab\epsilon a (a+b)(a+b)(a+b)(a+b)=(ab)b^0a(a+b)^4$.
		\item Le mot $aa$ ne fait \textbf{pas} partie de $L$. Supposons par l'absurde que $aa \in L$, alors il existerait une façon de décomposer $E$ en $aa$. Or, les premiers symboles doivent être soit $b$, soit $ab$. Il y a contradiction. Donc, $aa \notin L$.
	\end{itemize}
	\label{ex:regex}
\end{example}

Un autre exemple d'expression régulière est $E=01^*0$. $E$ décrit le langage $L(E)$ constitué de tous les mots commençant et finissant par $0$ avec uniquement des $1$ entre les deux.

\section{Automates finis}\label{adf}	\subsection{Définition}\label{sub:dfa}
	
	Un \emph{automate déterministe fini (ADF)} \automaton est défini comme suit :
	\begin{itemize}
		\item $Q$ est un ensemble fini d'\emph{états}
		\item $\Sigma$ est un alphabet
		\item $q_0 \in Q$ est l'\emph{état initial}
		\item $\delta : Q \times \Sigma \rightarrow Q$ est la \emph{fonction de transition}. A partir d'un état de $Q$, en fonction d'un symbole, elle retourne un état de $Q$.
		\item $F \subseteq Q$ est un ensemble d'\emph{états acceptants}.
	\end{itemize}
	
	\begin{exemple}\label{ex:adf}
		On considère l'automate \automaton défini comme suit :
		\begin{itemize}
			\item $Q=\{q_0,q_1,q_2,q_3,q_4,q_5,q_6\}$
			\item $\Sigma=\{0,1\}$
			\item $q_0$ est l'état du même nom
			\item La fonction de transition $\delta$ est décrite par la table \ref{fig:transdelta}. L'intersection d'une ligne reprenant un élément $q \in Q$ et d'une colonne $a \in \Sigma$ donne l'état $\delta(q,a)$.
			\item $F=\{q_d\}$
		\end{itemize}
	
		\begin{figure}[H]
			\centering
			\begin{tabular}{|r||c|c|}
				\hline
				&a&b\\
				\hline\hline
				$\rightarrow q_0$&$q_2$&$q_1$\\\hline
				$q_1$&$q_3$&$q_5$\\\hline
				$q_2$&$q_4$&$q_5$\\\hline
				$q_3^*$&$q_3$&$q_3$\\\hline
				$q_4$&$q_4$&$q_4$\\\hline
				$q_5$&$q_3$&$q_1$\\\hline
				$q_6$&$q_4$&$q_5$\\\hline
			\end{tabular}
			\caption{La table de transitions $\delta$}
			\label{fig:transdelta}
		\end{figure}
	\end{exemple}
	 
	Via cette notation, $Q$ et $\Sigma$ sont explicites. En dénotant l'état initial par $\rightarrow$ et les états acceptants par $*$ en exposant, on obtient une définition complète d'un automate : $(Q,\Sigma, q_0, \delta, F)$.
 	
 	
 	\subsection{Graphe d'automate déterministe fini}\label{ss:grapheadf}
 	
 	Le \emph{graphe d'un automate déterministe fini} \automaton est un graphe dirigé construit comme suit :
 	
 	\begin{itemize}
 		\item Chaque nœud du graphe correspond à un état de $Q$
 		\item Chaque arc a un symbole de $\Sigma$ comme étiquette. Un arc relie un état $q_0$ à un état $q_1$. Cet arc défini $\delta(q_0,a)=q_1$, un transition de la fonction de transition. Si plusieurs symboles causent une même transition de $q_0$ à $q_1$, il n'y a qu'une seule étiquette sur l'arc, listant ces différents symboles.
 		\item L'état initial est mis en évidence par une flèche entrante.
 		\item Les états acceptants sont représentés par un double cercle, en opposition au simple cercle des autres nœuds.
 	\end{itemize}
 	
 	\begin{exemple}
	 Voici le graphe représentant l'automate défini par la table \ref{fig:transdelta}
	 \begin{figure}[H]
	 	\centering
	 	\begin{tikzpicture}[->,>=stealth',shorten >=1pt,auto,node distance=3cm, semithick, bend angle=10]
	 	
	 	\tikzstyle{every state}=[circle]
	 	
	 	\node[initial,state] (A)                    {$q_0$};
	 	\node[state]         (B) [below right of=A] {$q_1$};
	 	\node[state]         (C) [below left of=A] {$q_2$};
	 	\node[accepting, state]         (D) [below right of=B] {$q_3$};
	 	\node[state]         (E) [below left of=C]       {$q_4$};
	 	\node[state]         (G) [below right of=E]       {$q_6$};
	 	\node[state]         (F) [above right of=G]       {$q_5$};
	 	
	 	\path 	(A) 	edge              node {a} (C)
	 	edge              node {b} (B)
	 	(B) 	edge              node {a} (D)
	 	edge [bend left]  node {b} (F)
	 	(C) 	edge              node {a} (E)
	 	edge              node {b} (F)
	 	(D) 	edge [loop above] node {a,b} (D)
	 	(E) 	edge [loop above] node {a,b} (E)
	 	(F) 	edge              node {a} (D)
	 	edge [bend left]  node {b} (B)
	 	(G) 	edge              node {a} (E)
	 	edge              node {b} (F);
	 	\end{tikzpicture}
	 	\caption{Automate $A_1$}\label{fig:a1}
	 \end{figure}
	 
 	\end{exemple}

	 Cette représentation d'un automate peut sembler plus naturelle pour un humain alors que la table de transitions est plus proche d'un langage informatique. De plus, dans la représentation par graphe, les ensembles $Q$ et $\Sigma$ sont implicites et doivent être définis ou déduits à part.
	 
	 \subsection{Chemin}
	 
	 La \emph{fonction de transition étendue} 
	 $$\hdelta : Q \times \Sigma^* \rightarrow Q$$
	 prend en entrée un état de $Q$ et un mot $w$ sur $\Sigma$ et retourne un état de $Q$.
	 
	 $\hdelta$ est définie de façon récursive par sur $w$:
	 \paragraph{Cas de base} Il y a deux cas de base:
	 \begin{itemize}
	 	\item $w$ est un mot vide : $\hdelta(q, \epsilon) = q$
	 	\item $w$ est un symbole : $\hdelta(q, w)$ avec $w=a\in \Sigma$. Alors, le chemin utilise la fonction de transition : $\hdelta(q,a)=\delta(q,a)$.
	 \end{itemize}
	 \paragraph{Pas de récurrence} Si $|w|>1$, alors $w=xa$ avec $x$ un mot sur $\Sigma$ et $a$ un symbole de $\Sigma$. Les chemins sur des mots de longueur strictement supérieure à 1 sont définis comme $\hdelta(q,w) = \hdelta(q,xa)= \delta(\hdelta(q,x),a)$.
	 
	 Il se peut que $\delta$ ne soit pas définie pour une paire d'arguments. Auquel cas, $\hdelta$ ne l'est pas non plus.\\
	 
	 Un \emph{chemin} est une application de cette fonction sur un état et un mot.
	 
	 \begin{exemple}
	 	Considérons l'automate $A$ de la figure \ref{fig:a1}. Il existe un chemin de $q_0$ à $q_5$ : $\hdelta(q_0, ab) = \delta(\hdelta(q_0,a),b) = \delta(\delta(q_0,a),b) = \delta(q_2, b)=q_5$.
	 \end{exemple}
	 
	 \subsection{Langage défini par un automate}
	 
	 Le langage représenté par un automate \automaton peut alors se définir comme les mots qui, par l'application de $\hdelta$ sur l'état initial, donnent un état acceptant :
	 $$
	 \{w \in \Sigma^* | \hdelta(q_0,w) \in F\}
	 $$
	 Ainsi, un mot $w$ appartient à un langage $L$ défini par l'automate $A$ si $\hdelta(q_0,w) \in F$.	 
	 
	 \subsection{La relation $R_M$}\label{ss:rm}
	 
	 Soit un automate \automaton. Définissons la relation $R_M$ entre deux états : 
	 $$xR_My \iff (\forall w \in \Sigma^*,\hdelta(x,w) \in F \iff \hdelta(y,w) \in F)$$
	 
	 Intuitivement, ces deux états sont en relation si tout mot lu à partir de celui-ci mène à des états étant simultanément acceptants ou non. 
	 
	 \begin{proposition}
	 	\rm est une relation d'équivalence.
	 \end{proposition}
	
	 \begin{proof} Montrer que \rm est une relation d'équivalence revient à montrer qu'elle est réflexive, transitive et symétrique.
	 	\begin{itemize}
	 		\item \textbf{Réflexive :} Soient un état $x \in Q_M$ et $w \in \Sigma^*$. Alors, $\hat{\delta}(x,w) \in F \iff \hat{\delta}(x,w) \in F$ et par définition, $xR_Mx$.
	 		\item \textbf{Transitive :} Soient les états $x,y,z \in Q_M$ tels que $xR_My$ et $yR_Mz$ ainsi que $w \in \Sigma^*$. Par hypothèse, $\hat{\delta}(x,w) \in F \iff \hat{\delta}(y,w)\in F$ et $\hat{\delta}(y,w) \in F\iff \hat{\delta}(z,w) \in F$. Par transitivité de l'implication, on obtient $\hat{\delta}(x,w) \in F \iff \hat{\delta}(z,w)\in F$. On a donc $xR_Mz$.
	 		\item \textbf{Symétrique : } Soient les états $x,y \in Q_M$ tels que $xR_My$ et un mot $w \in \Sigma^*$. Par hypothèse, $\hat{\delta}(x, w)\in F \iff \hat{\delta}(y, w)\in F$. En lisant la double implication depuis la droite, on a bien $\hat{\delta}(y, w) \in F\iff \hat{\delta}(x, w)\in F$ et donc $yR_Mx$.
	 	\end{itemize}
	 \end{proof}
	 
	 \begin{corollary}
	 	\rm sépare les états de $Q$ en classes d'équivalence.
	 \end{corollary}
	 
	 La classe d'équivalence de tous les états en relation \rm avec $q$ (qui sert alors de \emph{représentant}) se note $[[q]]$ ou par une lettre majuscule, typiquement $S$ ou $T$.
	 
	 La \emph{congruence à droite} d'une relation $R$ entre des mots sur un alphabet $\Sigma$ est définie comme :
	 $$
	 \forall x,y \in \Sigma^*, xRy \Rightarrow \forall a \in \Sigma, xaRya 
	 $$  
	 
	 \begin{proposition}
	 	\rm est congruente à droite.
	 \end{proposition}
	 
	 \begin{proof}
	 	Si la relation est vraie pour deux état, elle reste valable pour les états atteints par la lecture d'un symbole quelconque. Soient les états $x,y \in Q_M$ tels que $xR_My$. Soit un symbole $a \in \Sigma$. Par hypothèse, 
	 	$$\forall w \in \Sigma^*, \hat{\delta}(x, w) \in F \iff \hat{\delta}(y, w) \in F$$
	 	C'est donc vrai en particulier pour $w = au, u \in \Sigma*$. Dès lors,
	 	$$\hat{\delta}(x, au) \in F\iff \hat{\delta}(y, au)\in F$$
	 	$$\hat{\delta}(\delta(x,a),u) \in F\iff\hat{\delta}(\delta(y,a),u)\in F$$
	 	$$\hat{\delta}(p,u) \in F\iff \hat{\delta}(q,u)\in F$$
	 \end{proof}
 
 	\begin{corollary}
 		Pour chaque symbole, toutes les transitions sortant d'une classe d'équivalence mènent à une même classe d'équivalence :
 		$\forall a \in \Sigma, \exists T, \forall q \in S, \delta(q,a)\in T$ avec $T$ une classe d'équivalence.
 	\end{corollary}
	 
	 
	 \subsection{Automate et problème de décision}
	 
	 Une notion liée aux langages est celle de \emph{problème}. Une forme de problème est celle dite de \emph{décision} : une question à laquelle la réponse est oui ou non.
	 
	 Ces problèmes de décision peuvent être exprimés en terme d'appartenance d'un mot à un langage.
	 
	 Par exemple, prenons l'alphabet des chiffres $\Sigma=\{0,1,2,3,4,5,6,7,8,9\}$. Considérons ensuite le langage $L = \{w | \text{le nombre représenté par } w \text{ est pair}\}$.
	 
	 Demander si un nombre est pair peut alors être traduit par l'appartenance d'un mot le représentant à $L$. Si le langage peut être représenté par un automate déterministe fini, la réponse peut être trouvée par l'exécution de celui-ci.
	 
	 
	 
	 
\section{Algorithme Table Filling}\label{tfa}
% ██████  ███████
% ██   ██ ██
% ██████  █████
% ██   ██ ██
% ██   ██ ███████



\subsection{Relation \re}\label{ss:re}

Soit un automate \automaton. Définissons la relation \re entre deux états :
$$xR_Ey \iff (\forall w \in \Sigma^*,\hdelta(x,w) \in F \iff \hdelta(y,w) \in F)$$

Intuitivement, ces deux états sont en relation si tout mot lu à partir de celui-ci mène à des états étant simultanément acceptants ou non.

\begin{proposition}[\re]
 \re est une relation d'équivalence.
\end{proposition}

\begin{proof}[\re est une relation d'équivalence] Montrer que \re est une relation d'équivalence revient à montrer qu'elle est réflexive, transitive et symétrique.
 \begin{itemize}
	 \item \textbf{Réflexive :} Soient un état $x \in Q_M$ et $w \in \Sigma^*$. Alors, $\hat{\delta}(x,w) \in F \iff \hat{\delta}(x,w) \in F$ et par définition, $xR_Ex$.
	 \item \textbf{Transitive :} Soient les états $x,y,z \in Q_M$ tels que $xR_Ey$ et $yR_Ez$ ainsi que $w \in \Sigma^*$. Par hypothèse, $\hat{\delta}(x,w) \in F \iff \hat{\delta}(y,w)\in F$ et $\hat{\delta}(y,w) \in F\iff \hat{\delta}(z,w) \in F$. Par transitivité de l'implication, on obtient $\hat{\delta}(x,w) \in F \iff \hat{\delta}(z,w)\in F$. On a donc $xR_Ez$.
	 \item \textbf{Symétrique : } Soient les états $x,y \in Q_M$ tels que $xR_Ey$ et un mot $w \in \Sigma^*$. Par hypothèse, $\hat{\delta}(x, w)\in F \iff \hat{\delta}(y, w)\in F$. En lisant la double implication depuis la droite, on a bien $\hat{\delta}(y, w) \in F\iff \hat{\delta}(x, w)\in F$ et donc $yR_Ex$.
 \end{itemize}
 \hfill$\square$
\end{proof}

\begin{corollary}
 \re sépare les états de $Q$ en classes d'équivalence.
\end{corollary}

La classe d'équivalence de tous les états en relation \re avec $q$ (qui sert alors de \emph{représentant}) se note $[[q]]$ ou par une lettre majuscule, typiquement $S$ ou $T$.

La \emph{congruence à droite} d'une relation $R$ entre des mots sur un alphabet $\Sigma$ est définie comme :
$$
\forall x,y \in \Sigma^*, xRy \Rightarrow \forall a \in \Sigma, xaRya
$$

\begin{proposition}[Congruence de \re]
 \re est congruente à droite.
\end{proposition}

\begin{proof}[Congruence de \re]\label{proof:rmcongruency}
 Si la relation est vraie pour deux état, elle reste valable pour les états atteints par la lecture d'un symbole quelconque. Soient les états $x,y \in Q_M$ tels que $xR_Ey$. Soit un symbole $a \in \Sigma$. Par hypothèse,
 $$\forall w \in \Sigma^*, \hat{\delta}(x, w) \in F \iff \hat{\delta}(y, w) \in F$$
 C'est donc vrai en particulier pour $w = au, u \in \Sigma*$. Dès lors,
 $$\hat{\delta}(x, au) \in F\iff \hat{\delta}(y, au)\in F$$
 $$\hat{\delta}(\delta(x,a),u) \in F\iff\hat{\delta}(\delta(y,a),u)\in F$$
 $$\hat{\delta}(p,u) \in F\iff \hat{\delta}(q,u)\in F$$

\hfill$\square$
\end{proof}

\begin{corollary}\label{col:st}
 Pour chaque symbole, toutes les transitions sortant d'une classe d'équivalence mènent à une même classe d'équivalence :
 $\forall a \in \Sigma, \exists T, \forall q \in S, \delta(q,a)\in T$ avec $T$ une classe d'équivalence.
\end{corollary}


% ████████ ███████  █████
%    ██    ██      ██   ██
%    ██    █████   ███████
%    ██    ██      ██   ██
%    ██    ██      ██   ██

\subsection{Table Filling Algorithm}
Certains états d'un automate peuvent être \emph{équivalents} selon la relation \re. Celui-ci peut alors être simplifié. Une façon de détecter ces équivalences est de construire un tableau via l'\emph{algorithme de remplissage de tableau}.

Celui-ci détecte les paires \emph{différenciables}, récursivement sur un automate \automaton. Un paire $\{p,q\}$ est différenciable s'il existe un mot $w$ tel qu'un chemin $\hdelta(p,w)$ mène à un état acceptant et $\hdelta(q,w)$ mène à un état non-acceptant ou vice-versa. $w$ sert alors de \emph{mot témoin}.

\textbf{Cas de base :} Si $p$ est un état acceptant et que $q$ ne l'est pas, alors la paire $\{p,q\}$ est différenciable. Le mot témoin est $\epsilon$.

\textbf{Pas de récurrence : } Soient $p,q$ des états de $Q$ et un symbole $a \in \Sigma$ tel que $\delta(p,a)=r$ et $\delta(q,a)=s$. Si $r$ et $s$ sont différenciables, alors $p$ et $q$ le sont aussi. En effet, il existe un mot \emph{témoin} $w$ qui permet de différencier $r$ et $s$. Alors le mot $aw$ est le mot témoin qui permet de différencier $p$ et $q$.

\begin{theorem}[Table d'équivalence]
 Si deux états ne sont pas distingués par l'algorithme de remplissage de tableau, les états sont équivalents (ils respectent la relation \re).
\end{theorem}

\begin{proof}

Considérons un automate déterministe fini quelconque \automaton. Supposons par l'absurde qu'il existe une paire d'états $\{p,q\}$ tels que :
\begin{enumerate}
	 \item $p$ et $q$ ne sont pas distingués par l'algorithme de remplissage de table.
	 \item Les états ne sont pas équivalents, $\not pR_E q$. Par extension, il existe un mot témoin $w$ différenciant $p$ et $q$.
\end{enumerate}

Une telle paire est une \emph{mauvaise paire}. Si il y a des mauvaises paires, chacune distinguée par un mot témoin, il doit exister un paire distinguée par le mot témoin le plus court. Posons $\{p,q\}$ comme étant cette paire et $w=a_1a_2\dots a_n$ le mot témoin le plus court qui les distingue. Dès lors, soit $\hdelta(p,w)$ est acceptant, soit $\hdelta(q,w)$ l'est, mais pas les deux.

Ce mot $w$ ne peut pas être $\epsilon$. Auquel cas, la table aurait été remplie dès l'étape d'induction de l'algorithme. La paire $\{p,q\}$ ne serait pas une mauvaise paire, ne respectant pas l'hypothèse 1.

$w$ n'étant pas $\epsilon$, $ |w| \ge 1$. Considérons les états $r = \delta(p,a_1)$ et $s=\delta(q,a_1)$. Ces états sont différenciés par $a_2a_3\dots a_n$ car $\hdelta(p,w) = \hdelta(r, a_2a_3\dots a_n)$ et $\hdelta(q,w) = \hdelta(s, a_2a_3\dots a_n)$ et $p$ et $q$ sont différenciables.

Cela signifie qu'il existe un mot plus petit que $w$ qui différencie deux états: le mot $a_2a_3\dots a_n$. Comme on a supposé que $w$ est le mot le plus petit qui différencie une mauvaise paire, $r$ et $s$ ne peuvent pas être une mauvaise paire. Donc, l'algorithme a du découvrir qu'ils sont différenciables.

Cependant, le pas de récurrence impose que $\delta(p, a_1)$ et $\delta(q, a_1)$ mènent à deux états différentiables implique que $p$ et $q$ le sont aussi. On a une contradiction de notre hypothèse : $\{p,q\}$ n'est pas une mauvaise paire.

Ainsi, s'il n'existe pas de mauvaise paire, c'est que chaque paire différenciable est reconnue par l'algorithme.

\hfill$\square$
\end{proof}

\begin{example}[Table d'équivalence] Voici une application de cet algorithme sur l'automate $A_2$, version réduite de l'automate $A_1$ de la figure \ref{fig:a1}.

\begin{figure}[H]
 \centering
 \begin{tikzpicture}[->,>=stealth',shorten >=1pt,auto,node distance=3cm, semithick, bend angle=10]

 \tikzstyle{every state}=[circle]

 \node[initial,state] (A)                    {$q_0$};
 \node[state]         (B) [below right of=A] {$q_1$};
 \node[state]         (C) [below left of=A] {$q_2$};
 \node[accepting,state]         (D) [below right of=B] {$q_3$};
 \node[state]         (E) [below left of=C]       {$q_4$};
 \node[state]         (F) [below right of=C]       {$q_5$};

 \path 	(A) 	edge              node {a} (C)
 edge              node {b} (B)
 (B) 	edge              node {a} (D)
 edge [bend left]  node {b} (F)
 (C) 	edge              node {a} (E)
 edge              node {b} (F)
 (D) 	edge [loop above] node {a,b} (D)
 (E) 	edge [loop above] node {a,b} (E)
 (F) 	edge              node {a} (D)
 edge [bend left]  node {b} (B);
 \end{tikzpicture}
 \caption{Automate $A_2$}\label{fig:a2}
\end{figure}

La première étape est de remplir la table avec l'algorithme précédant. Tout état est distinguable de $q_3$ : il est le seul état acceptant. 5 cases peuvent déjà êtres cochées. Le reste de la table est remplie par induction.

\begin{figure}[H]
 \centering
 \begin{tabular}{ccccccc}
	 \cline{2-2}
	 \multicolumn{1}{c|}{$q_1$} & \multicolumn{1}{c|}{x} &&&&\\
	 \cline{2-3}
	 \multicolumn{1}{c|}{$q_2$} & \multicolumn{1}{c|}{x} &\multicolumn{1}{c|}{x}&&&\\
	 \cline{2-4}
	 \multicolumn{1}{c|}{$q_3$} & \multicolumn{1}{c|}{x} &\multicolumn{1}{c|}{x}&\multicolumn{1}{c|}{x}&&\\
	 \cline{2-5}
	 \multicolumn{1}{c|}{$q_4$} & \multicolumn{1}{c|}{x} &\multicolumn{1}{c|}{x}&\multicolumn{1}{c|}{x}&\multicolumn{1}{c|}{x}&\\
	 \cline{2-6}
	 \multicolumn{1}{c|}{$q_5$} & \multicolumn{1}{c|}{x} & \multicolumn{1}{c|}{}&\multicolumn{1}{c|}{x}&\multicolumn{1}{c|}{x}&\multicolumn{1}{c|}{x}\\
	 \cline{2-6}
	 \multicolumn{1}{c}{} & $q_0$&$q_1$&$q_2$&$q_3$&$q_4$\\

 \end{tabular}
 \caption{Table filling pour $A_2$, décelant des équivalences d'états}
 \label{fig:ta2}
\end{figure}
\end{example}
\stepcounter{algo}
\begin{complexity}

Considérons $n$ le nombre d'états d'un automate, et $k$ la taille de l'alphabet $\Sigma$ supporté.

Si il y a $n$ états, il y a $\begin{pmatrix}n\\2\end{pmatrix}$ soit $\frac{n(n-1)}{2}$ paires d'états. A chaque itération (sur l'ensemble de la table), il faut considérer chaque paire, et vérifier si un de leur successeurs est différentiable. Cette étape prend au plus $\mathcal{O}(k)$ pour tester chaque successeurs potentiel (en fonction du symbole lu).  Ainsi, une itération sur la table se fait en $\mathcal{O}(kn^2)$. Si une itération ne découvre pas de nouveaux état différentiable s'arrête. Comme la table a une taille en $\mathcal{O}(n^2)$ et qu'à chaque étape un élément au minimum doit y être coché, la complexité totale de l'algorithme est en $\mathcal{O}(kn^4)$.

Cependant, il existe des pistes d'amélioration. La première est d'avoir, pour chaque paire $\{r,s\}$ une liste des paire $\{p,q\}$ qui, pour un même symbole, mènent à $\{r,s\}$. On dit de ces paires qu'elles sont dépendantes. Si la paire $\{r,s\}$ est marquée comme différenciable, leurs paires dépendantes seront de facto différenciables.

Cette liste peut être construite en considérant chaque symbole $a \in \Sigma$ et ajoutant les paires $\{p,q\}$ à chacune de leur dépendance $\{\delta(p,a),\delta(q,a)\}$. Cette étape prend au plus $k.\mathcal{O}(n^2)=\mathcal{O}(kn^2)$. (Le nombre de symboles multiplié par le nombre de paires à considérer).

Ensuite, il suffit de partir des cas initiaux (se reposant sur le cas de base de l'algorithme), et de marquer tous leurs états dépendants comme différentiables, tout en ajoutant leur propre liste à chaque fois. La complexité de cette exploration est bornée par le nombre d'éléments dans une liste et le nombre de listes. Respectivement, $k$ et $\mathcal{O}(n^2)$, ce qui donne $\mathcal{O}(kn^2)$ pour cette exploration.

La complexité totale revient à $\mathcal{O}(kn^2)$.
\end{complexity}


% ███    ███ ██ ███    ██ ██ ███    ███
% ████  ████ ██ ████   ██ ██ ████  ████
% ██ ████ ██ ██ ██ ██  ██ ██ ██ ████ ██
% ██  ██  ██ ██ ██  ██ ██ ██ ██  ██  ██
% ██      ██ ██ ██   ████ ██ ██      ██

\subsection{Minimisation}
La minimisation d'automate se fait en deux étapes :
\begin{enumerate}
 \item Se débarrasser de tous les états injoignables : ils ne participent pas à la construction du langage représenté
 \item Grâce aux équivalences d'états trouvées grâce à l'algorithme de remplissage de tableau défini au point \ref{ss:tfa}, construire un nouvel automate.
\end{enumerate}

Soit un automate déterministe fini \automaton. Les états non-atteignables peuvent être supprimés de $Q$ et de $\delta$.

Pour minimiser cet automate, il faut :
\begin{enumerate}
 \item Générer la table de différenciation.
 \item Séparer $Q$ en classes d'équivalences
 \item Construire l'automate canonique $C=(Q_C,\Sigma, \delta_C, q_C, F_C)$:
 \begin{itemize}
	 \item Soit $S$ une des classes d'équivalence obtenues par la table de différenciation.
	 \item Ajouter $S$ à $Q_C$ et à $F_C$ si $S$ contient un état acceptant : $q\in S, q\in F$.
	 \item Si $S$ contient $q_0$ l'état initial de $A$, alors $S$ est $q_C$ l'état initial de $C$.
	 \item Pour un symbole $a \in \Sigma$, alors il doit exister une classe d'équivalence $T$ tel que pour chaque état $\forall q \in S,\delta(q,a) \in T$. Si ce n'est pas le cas, c'est que deux états $p$ et $q$ dans $S$ mènent à différentes classes d'équivalences. Or, ces deux états sont différenciables, et ne pourraient pas appartenir tous deux à $S$ par construction. Ce fait est déjà mentionné dans le corollaire \ref{col:st}. On peut écrire $\delta_C(S,a)=T$. Pour rappel, la fonction $\delta$ est définie pour tout état et tout symbole. Rien n'empêche $T=S$.
 \end{itemize}
\end{enumerate}


\begin{example}

 Considérons l'automate $A_1$ représenté à la figure \ref{fig:a1}. En supprimant l'état $q_6$ qui n'est pas atteignable, on obtient l'automate $A_2$ de la figure \ref{fig:a2}.

 Le tableau de la figure \ref{fig:ta2} sert d'exemple pour l'algorithme de remplissage de tableau, sur $A_2$.
 $A_3$.

 En appliquant l'algorithme, qui peut se résumer intuitivement à fusionner les états équivalents, on obtient l'automate $A_3$ de la figure \ref{fig:a3}.

 \begin{figure}[H]
	 \centering
	 \begin{tikzpicture}[->,>=stealth',shorten >=1pt,auto,node distance=3cm, semithick, bend angle=10]

	 \tikzstyle{every state}=[circle]

	 \node[initial,state] (A)                    {$q_0$};
	 \node[state]         (B) [below right of=A] {$q_1$};
	 \node[state]         (C) [below left of=A] {$q_2$};
	 \node[accepting, state]         (D) [below right of=B] {$q_3$};
	 \node[state]         (E) [below left of=C]       {$q_4$};

	 \path
	 (A) 	edge              node {a} (C)
	 edge              node {b} (B)
	 (B) 	edge              node {a} (D)
	 edge [loop above] node {b} (B)
	 (C) 	edge              node {a} (E)
	 edge              node {b} (B)
	 (D) 	edge [loop above] node {a,b} (D)
	 (E) 	edge [loop above] node {a,b} (E);
	 \end{tikzpicture}
	 \caption{Automate $A_3$}\label{fig:a3}
 \end{figure}

 Une expression régulière ($(b+ab)b^*a(a+b)^*$) peut être déduite pour $L$ grâce à cet automate. Cette expression régulière est celle de l'exemple \ref{ex:regex}
\end{example}


\begin{theorem}[Minimalité de l'automate réduit]
 Soit un ADF $A$ et soit $C$ l'automate construit par cet algorithme de minimisation. Aucun automate équivalent à $A$ n'a moins d'états que $C$. De plus, chaque automate ayant autant d'états que $C$ peut être transformé en celui-ci par homomorphisme.
\end{theorem}


\begin{proof}
 Prouvons que l'algorithme de minimisation fourni un automate minimum (il n'en existe aucun comportant moins d'états pour un même langage)
 Soient un ADF $A$ et $C$ l'automate obtenu par l'algorithme de minimisation. Posons que $C$ comporte $k$ états.

 Par l'absurde, supposons qu'il existe $M$ un ADF minimisé équivalent à $A$ mais comptant moins d'états que $C$. Posons qu'il en comporte $l<k$.
 Appliquons l'algorithme de remplissage de table sur $C$ et $M$, comme s'ils étaient un seul ADF, comme proposé dans la section \ref{ss:eqauto}. Les états initiaux sont équivalents (pas différentiables) puisque $L(C)=L(M)$. Dès lors, les successeurs pour chaque symboles sont eux aussi équivalent. Le cas contraire impliquerait que états initiaux sont différentiables, ce qui n'est pas le cas.
 De plus, ni $C$ ni $M$ n'ont un état inaccessible, sinon il pourrait être éliminé, résultant en un automate comportant moins d'états pour un même langage.
 Soit $p$ un état de $C$. Soit un mot $a_1a_2\dots a_i$, qui mène de l'état initial de $C$ à $p$. Alors, il existe un état $q$ de $M$ équivalent à $p$. Puisque les états initiaux sont équivalents, et que par induction, les états obtenus par la lecture d'un symbole le sont aussi, l'état $q$ dans $M$ obtenu par la lecture du mot $a_1a_2\dots a_i$ est équivalent à $p$. Ceci signifie que tout état de $C$ est équivalent à au moins un état de $M$.
 Or, $lk>l$. Cela signifie qu'il doit exister au moins deux états de $C$ équivalents à un même état de $M$ et donc équivalent entre eux. Il y a la contradiction : par construction, les états de $C$ sont tous différentiables les uns des autres. La supposition de l'existence de $M$ est fausse. Il n'existe pas d'automate équivalent à $A$ comportant moins d'états que $C$.

 \hfill$\square$
\end{proof}

\begin{proof}
 Prouvons que tout automate minimal pour un langage est $C$, à un isomorphisme sur les noms des états près.

 Soit $A$ un ADF pour un langage $L$. Soient $C$ un ADF obtenu par l'algorithme de minimisation et $M$ un automate minimal comportant autant d'états que $C$.

 Comme mentionné dans la preuve précédente, il doit y avoir une équivalence 1 à 1 entre chaque état de $C$ et de $M$. (Au minimum 1 et au plus 1). De plus, aucun état de $M$ ne peut être équivalent à 2 états de $C$, selon le même argument.

 Dès lors, l'automate minimisé, dit \emph{canonique} est unique à l'exception du renommage des différents états.

 \hfill$\square$
\end{proof}


% ███████  ██████  ██    ██ ██ ██    ██
% ██      ██    ██ ██    ██ ██ ██    ██
% █████   ██    ██ ██    ██ ██ ██    ██
% ██      ██ ▄▄ ██ ██    ██ ██  ██  ██
% ███████  ██████   ██████  ██   ████
%             ▀▀


\subsection{Appartenance et équivalence}
Considérons les automates $A_H$ et $A_I$ donnés dans les figures \ref{fig:ah} et \ref{fig:ai}

\begin{minipage}{0.4\linewidth}
 \begin{figure}[H]
	 \centering
	 \begin{tikzpicture}[->,>=stealth',shorten >=1pt,auto,node distance=2cm and 5cm, semithick, bend angle=10]

	 \tikzstyle{every state}=[circle]

	 \node[initial,state]	(A)					{$q_0$};
	 \node[state]			(B)	[right= of A]	{$q_1$};
	 \node[accepting,state]	(C) [below of=A]	{$q_2$};
	 \node[accepting,state]	(D)	[below of=B]	{$q_3$};
	 \node[accepting,state]	(E)	[below of=C]	{$q_4$};
	 \node[state]			(F)	[below of=D]	{$q_5$};

	 \path
	 (A)	edge	[bend left]		node{a}		(B)
	 (A)	edge					node{b}		(C)
	 (B) edge	[bend left]		node{a}		(A)
	 (B) edge					node{b}		(D)
	 (C)	edge					node{a}		(E)
	 (C)	edge					node[near start]{b}		(F)
	 (D)	edge					node[near start, above]{a}		(E)
	 (D)	edge					node{b}		(F)
	 (E)	edge	[loop below]	node{a}	(E)
	 (E) edge					node{b} (F)
	 (F)	edge	[loop below]	node{a,b}	(F)

	 ;
	 \end{tikzpicture}
	 \caption{Automate $A_H$, du livre d'Hopcraft et al. de 1979\cite{Hopcroft79} (Fig3.2)}\label{fig:ah}
 \end{figure}
\end{minipage}\hspace{0.2\linewidth}
\begin{minipage}{0.4\linewidth}
 \begin{figure}[H]
	 \centering
	 \begin{tikzpicture}[->,>=stealth',shorten >=1pt,auto,node distance=1cm and 1cm, semithick, bend angle=10]

	 \tikzstyle{every state}=[circle]

	 \node[initial,state]	(A)					{$q_6$};
	 \node[accepting,state]	(B)	[right= of A]	{$q_7$};
	 \node[state]			(C) [right= of B]	{$q_8$};

	 \path
	 (A)	edge					node{b}		(B)
	 (A)	edge	[loop above]	node{a}		(A)
	 (B) edge					node{b}		(C)
	 (B) edge	[loop above]	node{a}		(B)
	 (C)	edge	[loop above]	node{a,b}	(C)

	 ;
	 \end{tikzpicture}
	 \caption{Automate $A_I$, provenant également de \cite{Hopcroft79}. Les états ont été renommés. }\label{fig:ai}
 \end{figure}
\end{minipage}

Il est possible de remplir un tableau via l'algorithme éponyme. Pour ce faire, les deux automates sont considérés comme un seul dont les états sont disjoints.

\begin{figure}[H]
 \centering
 \begin{tabular}{ccccccccc}
	 \cline{2-2}
	 \multicolumn{1}{c|}{$q_1$}&\multicolumn{1}{c|}{} &&&&&&&\\
	 \cline{2-3}
	 \multicolumn{1}{c|}{$q_2$}&\multicolumn{1}{c|}{x} &\multicolumn{1}{c|}{x}&&&&&&\\
	 \cline{2-4}
	 \multicolumn{1}{c|}{$q_3$}&\multicolumn{1}{c|}{x}&\multicolumn{1}{c|}{x}&\multicolumn{1}{c|}{}&&&&&\\
	 \cline{2-5}
	 \multicolumn{1}{c|}{$q_4$}&\multicolumn{1}{c|}{x}&\multicolumn{1}{c|}{x}&\multicolumn{1}{c|}{}&\multicolumn{1}{c|}{}&&&&\\
	 \cline{2-6}
	 \multicolumn{1}{c|}{$q_5$}&\multicolumn{1}{c|}{x}&\multicolumn{1}{c|}{x}&\multicolumn{1}{c|}{x}&\multicolumn{1}{c|}{x}&\multicolumn{1}{c|}{x}&&&\\
	 \cline{2-7}
	 \multicolumn{1}{c|}{$q_6$}&\multicolumn{1}{c|}{}&\multicolumn{1}{c|}{}&\multicolumn{1}{c|}{x}&\multicolumn{1}{c|}{x}&\multicolumn{1}{c|}{x}&\multicolumn{1}{c|}{x}&&\\
	 \cline{2-8}
	 \multicolumn{1}{c|}{$q_7$}&\multicolumn{1}{c|}{x}&\multicolumn{1}{c|}{x}&\multicolumn{1}{c|}{}&\multicolumn{1}{c|}{}&\multicolumn{1}{c|}{}&\multicolumn{1}{c|}{x}&\multicolumn{1}{c|}{x}&\\
	 \cline{2-9}
	 \multicolumn{1}{c|}{$q_8$}&\multicolumn{1}{c|}{x}&\multicolumn{1}{c|}{x}&\multicolumn{1}{c|}{x}&\multicolumn{1}{c|}{x}&\multicolumn{1}{c|}{x}&\multicolumn{1}{c|}{}&\multicolumn{1}{c|}{x}&\multicolumn{1}{c|}{x}\\
	 \cline{2-9}
	 \multicolumn{1}{c}{} & $q_0$& $q_1$ & $q_2$ & $q_3$ & $q_4$ & $q_5$ & $q_6$ & $q_7$\\

 \end{tabular}
 \caption{Tableau généré par l'application de l'algorithme sur $A_H$ et $A_I$}\label{fig:tahi}
\end{figure}

De cette table, toujours grâce aux conclusions précédentes, il est possible d'extraire des classes d'équivalences :
\begin{itemize}
 \item $C_0 = \{q_0, q_1, q_6\}$
 \item $C_1 = \{q_2, q_3, q_4, q_7\}$
 \item $C_2 = \{q_5, q_8\}$
\end{itemize}

En particulier, la classe $C_0$ souligne que les états initiaux sont équivalents. Cela signifie, par définition, que tout mot $w$ lu en partant d'un de ces états sera soit accepté dans les deux automates, soit refusé dans les deux. $A_H$ et $A_I$ définissent donc le même langage.
\stepcounter{algo}
\begin{complexity}
 Reposant sur la construction de la table d'équivalence d'états, la complexité est en $\mathcal{O}(kn^2)$, avec $k$ la taille de l'alphabet et $n$ le nombre d'états. L'étape supplémentaire, la lecture de cette table, est en temps constant et n'impacte pas la complexité.
\end{complexity}


Les différentes notions liées à l'égalité : les propriétés de réflexivité, transitivité et symétrie ont été démontrées dans la section \ref{ss:rm}.

\section{Algorithme d'Angluin}\label{angluin}
L'algorithme d'Angluin repose, en plus des éléments précédents sur quatre concepts :

\begin{itemize}
	\item Une table d'observation
	\item La relation $R_O$, se basant sur la table d'observation et semblable à la relation $R_L$
	\item La propriété de fermeture (closure en anglais)
	\item La propriété de cohérence (consistence en anglais)
\end{itemize}

Cette section commence par décrire cette table en \ref{ss:a_tblo}, la relation $R_O$ en \ref{ss:a_ro}, la fermeture en \ref{ss:a_fermeture}, la cohérence en \ref{ss:a_coherence}.

Une fois toutes ces bases posées, une exécution de l'algorithme sur un exemple est proposée en \ref{ss:a_exemple}, suivie du fonctionnement formel de l'algorithme et des preuves sur son exactitude et sa complexité en \ref{ss:a_algo}, \ref{ss:a_proof} et \ref{ss:a_comp}.


\subsection{Table d'observation}\label{ss:a_tblo}

\subsection{Relation $R_O$}\label{ss:a_ro}

\subsection{Fermeture}\label{ss:a_fermeture}

La propriété de fermeture (closure) s'exprime mathématiquement par 

$$ \forall u \in R, \forall a \in \Sigma, \exists v \in R, ua R_O v$$

En pratique, pour vérifier cette propriété, il suffit de de suivre cet algorithme, expliqué de façon visuelle sur la table O :

\begin{algorithm}[H]
	\begin{algorithmic}[1]
		\ENSURE si la fermeture est respectée ou non
		
		\FORALL {élément $w$ de la section $R$}
		\FORALL {symbole $a$ dans $\Sigma$}		
			\IF {$wa$ est dans $R$} 
				\STATE continuer
			\ELSE
				\STATE \COMMENT{$wa$ est dans $R.\Sigma$ par construction}
				\IF {La ligne de $wa$ dans $T$ est différente de celle de $w$}
					\RETURN Faux
				\ENDIF
			\ENDIF
		\ENDFOR
		\ENDFOR
		\RETURN Vrai
	\end{algorithmic}
	\caption{Vérification de la fermeture}\label{alg:closure}
\end{algorithm}

\subsection{Cohérence}\label{ss:a_coherence}

La propriété de cohérence (consistence) se définit mathématiquement comme 

$$ \forall u,v \in R, u R_O v \Rightarrow \forall a \in \Sigma, ua R_O va$$

Concrètement, il s'agit de prendre deux mots ($u,v$) dans $R$ ayant la même ligne dans $T$ et vérifier, pour chaque symbole ($a$), s'ils ($ua,va$) ont la même ligne dans $T$.



\subsection{Exemple}\label{ss:a_exemple}
Soit l'automate $A_3$ construit à la section \ref{ss:miniauto} sur la minimisation. L'automate $A_4$ recopié ici n'est qu'une isomorphie : les symboles $\{0,1\}$ ont été remplacés par $\{a,b\}$ pour plus de lisibilité dans les tables d'observation.
\begin{figure}[H]
	\centering
	\begin{tikzpicture}[->,>=stealth',shorten >=1pt,auto,node distance=3cm, semithick, bend angle=10]
	
	\tikzstyle{every state}=[circle]
	
	\node[initial,state] (A)                    {$q_a$};
	\node[state]         (B) [below right of=A] {$q_b$};
	\node[state]         (C) [below left of=A] {$q_c$};
	\node[accepting, state]         (D) [below right of=B] {$q_d$};
	\node[state]         (E) [below left of=C]       {$q_e$};
	
	\path 	
	(A) 	edge              node {a} (C)
	edge              node {b} (B)
	(B) 	edge              node {a} (D)
	edge [loop above] node {b} (B)
	(C) 	edge              node {a} (E)
	edge              node {b} (B)
	(D) 	edge [loop above] node {a,b} (D)
	(E) 	edge [loop above] node {a,b} (E);
	\end{tikzpicture}
	\caption{Automate $A_4$}
\end{figure}

\todo{Marquer la différence entre $R_L$ et $R_O$}

\subsubsection{Première itération}

L'algorithme d'Angluin précise, pour son cas de base, une initialisation de la table $T$ avec les ensembles $R$ et $S$ contenant uniquement $\epsilon$. Le champ $R.\{a,b\}$ ($R.\Sigma$) est rempli via des requête d'appartenance sur les mots $a$ et $b$.

\begin{minipage}{0.5\linewidth}
	\centering
	\begin{tabular}{|c|c|}
		\hline
		$O_0$ & $\epsilon$\\
		\hline
		$\epsilon$ & 0\\
		\hline
		$a$ & 0\\
		$b$ & 0\\
		\hline
	\end{tabular}
\end{minipage}
\begin{minipage}{0.5\linewidth}
	\centering
	\begin{figure}[H]
		\centering
		\begin{tikzpicture}[->,>=stealth',shorten >=1pt,auto,node distance=3cm, semithick, bend angle=10]
		\tikzstyle{every state}=[circle]
		\node[initial, state] (A) {$[[\epsilon]]$};
		\path (A) edge [loop above] node {a,b} (A);
		\end{tikzpicture}
		\caption*{Automate $O_0$}
	\end{figure}
\end{minipage}


\vspace{1cm}
L'étape suivante consiste à vérifier la \emph{closure} de la table d'observation $O_0$. Mathématiquement :

$$ \forall u \in R, \forall a \in \Sigma, \exists v \in R, ua R_L v$$

Intuitivement, pour chaque symbole (ici, $\{a,b\}$, et ce sera vrai jusqu'à la dernière itération), tout mot candidat (dans $R$, la partie supérieure de la table) doit se retrouver, complété de ce symbole, dans une classe d'équivalence d'un autre candidat de $R$. Ici, de toute évidence, les mots $a$ et $b$ sont dans la même classe d'équivalence que $\epsilon$. Dès lors, la propriété de \emph{closure} est respectée.

Si la \emph{closure} est respectée, alors la question de la \emph{consistence} (cohérence) se pose. Mathématiquement : 

$$ \forall u,v \in R, u R_L v \Rightarrow \forall a \in \Sigma, ua R_L va$$

Intuitivement, si deux candidats semblent être dans la même classe d'équivalence (leur lignes dans la table supérieure sont identiques), alors pour n'importe quel symbole, les deux nouveaux mots sont également dans une même classe d'équivalence (leur lignes, potentiellement dans la partie inférieure de la table, sont identiques). N'ayant qu'un seul candidat, cette propriété est forcément respectée ($R_L$ est réflexive).

Les deux propriétés étant respectées, les classes d'équivalences sont calculées (trivialement ici), et un automate $O_0$ est proposé à l'enseignant pour vérification.

Sur cette itération, un automate initial a été proposé, et aucun état final ne pouvant être atteint avec un seul symbole, la version est minime.

\subsubsection{Seconde itération}

L'enseignant répond que non, les automates ne sont pas équivalents. Il fourni le contre-exemple $ba$. Comme il est rejeté par $O_0$, cela signifie qu'il est accepté par $A_4$. Une nouvelle table est alors construite, en ajoutant $ba$ et ses préfixes (ici, juste $b$) à $R$. $R.\Sigma$ est calculé et les mots n'ayant pas encore reçu une valeur dans $T$ sont soumis à l'enseignant pour un test d'appartenance.
\vspace{1cm}

\begin{minipage}{0.25\linewidth}
	\centering
	\begin{tabular}{|c|c|}
		\hline
		$O_1$ & $\epsilon$\\
		\hline
		$\epsilon$ & 0\\
		\textcolor{red}{$b$} & \textcolor{red}{0}\\
		\textcolor{red}{$ba$} & \textcolor{red}{1}\\
		\hline
		$a$ & 0\\
		\textcolor{red}{$bb$} & \textcolor{red}{0}\\
		\textcolor{red}{$baa$} & \textcolor{red}{1}\\
		\textcolor{red}{$bab$} & \textcolor{red}{1}\\
		\hline
	\end{tabular}
\end{minipage}
\begin{minipage}{0.25\linewidth}
	\centering
	\begin{tabular}{|c|cc|}
		\hline
		$O_2$ & $\epsilon$ & \textcolor{red}{$a$}\\
		\hline
		$\epsilon$ & 0& \textcolor{red}{0}\\
		$b$ & 0&\textcolor{red}{1}\\
		$ba$ & 1&\textcolor{red}{1}\\
		\hline
		$a$ & 0&\textcolor{red}{0}\\
		$bb$ & 0&\textcolor{red}{1}\\
		$baa$ & 1&\textcolor{red}{1}\\
		$bab$ & 1&\textcolor{red}{1}\\
		\hline
	\end{tabular}
\end{minipage}
\begin{minipage}{0.5\linewidth}
	\centering
	\begin{figure}[H]
		\centering
		\begin{tikzpicture}[->,>=stealth',shorten >=1pt,auto,node distance=3cm, semithick, bend angle=10]
		\tikzstyle{every state}=[circle]
		
		\node[initial, state] (A) {$[[\epsilon]]$};
		\node[state] (B) [right of=A] {$[[b]]$};
		\node[accepting, state] [right of=B] (C) {$[[ba]]$};
		
		\path
		(A) edge [loop above] node {a} (A)
		(A) edge node {b} (B)
		(B) edge node {a} (C)
		(B) edge [loop above] node {b} (B)
		(C) edge [loop above] node {a,b} (C);
		
		
		\end{tikzpicture}
		\caption*{Automate $O_2$}
	\end{figure}
\end{minipage}

\vspace{1cm}
Comme pour la première itération, la \emph{fermeture} est testée, suivie de la \emph{cohérence}. Celle-ci n'est pas respectée : si on considère les mots $\epsilon$ et $b$, qui ont la même ligne dans la table $T$ ($\epsilon R_O b$), le symbole $a$, on obtient les mots $a$ et $ba$ qui n'ont pas la même ligne : ($\not a R_O ba$). Le symbole $a$ est alors ajouté à $S$ et une nouvelle table $O_2$ est calculée.

La fermeture étant déjà vérifiée, la question de la cohérence est reposée, et cette fois-ci elle est vérifiée ; l'automate est construit et proposé à l'enseignant.

Sur cette itération, l'algorithme a reçu le mot $ba$ comme étant accepté. Il a du ajouter $a$ à $S$ pour permettre de différencier certains états. L'automate se voit ajouter les états $[[b]]$ et $[[ba]]$.

\subsubsection{Troisième itération}

Suivant toujours l'algorithme de comparaison d'automates détaillé dans la section \ref{sec:algorithmes}, l'enseignant découvre qu'ils sont différents. 

Il sort le contre-exemple $aaba$. Si c'est un contre-exemple et qu'il est accepté par $O_2$, c'est qu'il ne l'est pas (0) par $A_4$. Une nouvelle table $O_3$ doit être construite.

\begin{minipage}{0.33\linewidth}
	\centering
	\begin{tabular}{|c|cc|}
		\hline
		$O_3$ & $\epsilon$ & $a$\\
		\hline
		$\epsilon$ & 0 &0\\
		\textcolor{red}{$a$}&\textcolor{red}{0}&\textcolor{red}{0}\\
		$b$&0&1\\
		\textcolor{red}{$aa$}&\textcolor{red}{0}&\textcolor{red}{0}\\
		$ba$&1&1\\
		\textcolor{red}{$aab$}&\textcolor{red}{0}&\textcolor{red}{0}\\
		\textcolor{red}{$aaba$}&\textcolor{red}{0}&\textcolor{red}{0}\\
		\hline
		\textcolor{red}{$ab$}&\textcolor{red}{0}&\textcolor{red}{1}\\
		$bb$&0&1\\
		\textcolor{red}{$aaa$}&\textcolor{red}{0}&\textcolor{red}{0}\\
		$baa$&1&1\\
		$bab$&1&1\\
		\textcolor{red}{$aabb$}&\textcolor{red}{0}&\textcolor{red}{0}\\
		\textcolor{red}{$aabaa$}&\textcolor{red}{0}&\textcolor{red}{0}\\
		\textcolor{red}{$aabab$}&\textcolor{red}{0}&\textcolor{red}{0}\\
		\hline
	\end{tabular}
\end{minipage}
\begin{minipage}{0.33\linewidth}
	\centering
	\begin{tabular}{|c|cc|}
		\hline
		$O_4$ & $\epsilon$ & $a$\\
		\hline
		$\epsilon$ & 0 &0\\
		$a$&0&0\\
		$b$&0&1\\
		$aa$&0&0\\
		\textcolor{red}{$ab$}&\textcolor{red}{0}&\textcolor{red}{1}\\
		$ba$&1&1\\
		$aab$&0&0\\
		$aaba$&0&0\\
		\hline
		$bb$&0&1\\
		$aaa$&0&0\\
		\textcolor{red}{$aba$}&\textcolor{red}{1}&\textcolor{red}{1}\\
		\textcolor{red}{$abb$}&\textcolor{red}{0}&\textcolor{red}{1}\\
		$baa$&1&1\\
		$bab$&1&1\\
		$aabb$&0&0\\
		$aabaa$&0&0\\
		$aabab$&0&0\\
		\hline
	\end{tabular}
\end{minipage}
\begin{minipage}{0.33\linewidth}
	\centering
	\begin{tabular}{|c|cc|}
		\hline
		$O_5$ & $\epsilon$ & $a$\\
		\hline
		$\epsilon$ & 0 &0\\
		$a$&0&0\\
		$b$&0&1\\
		$aa$&0&0\\
		$ab$&0&1\\
		$ba$&1&1\\
		$aab$&0&0\\
		\textcolor{red}{$aba$}&\textcolor{red}{1}&\textcolor{red}{1}\\
		$aaba$&0&0\\
		\hline
		$bb$&0&1\\
		$aaa$&0&0\\
		$abb$&0&1\\
		$baa$&1&1\\
		$bab$&1&1\\
		$aabb$&0&0\\
		\textcolor{red}{$abaa$}&\textcolor{red}{1}&\textcolor{red}{1}\\
		\textcolor{red}{$abab$}&\textcolor{red}{1}&\textcolor{red}{1}\\
		$aabaa$&0&0\\
		$aabab$&0&0\\
		\hline
	\end{tabular}
\end{minipage}



	\begin{figure}[H]
		\centering
		\begin{tikzpicture}[->,>=stealth',shorten >=1pt,auto,node distance=3cm, semithick, bend angle=10]
		\tikzstyle{every state}=[circle]
		
		\node[initial, state] (A) {$[[\epsilon]]$};
		\node[state] (B) [above right of=A] {$[[a]]$};
		\node [state] (E) [right of=B] {$[[aa]]$};
		\node[state] (C) [below right of =A] {$[[b]]$};
		\node[accepting, state] [right of=C] (D) {$[[ba]]$};
		
		\path
		(A) edge node {a} (B)
		(A) edge node {b} (C)
		(B) edge node {b} (C)
		(B) edge node {a} (E)
		(C) edge [loop below] node {b} (C)
		(C) edge node {a} (D)
		(D) edge [loop above] node {a,b} (D)
		(E) edge [loop below] node {a,b} (E);
		
		\end{tikzpicture}
		\caption*{Automate $O_5$}
	\end{figure}


Ayant reçu $aaba$, ce mot et tous ses préfixes sont ajoutés à la table. L'extension $R.\Sigma$ est recalculée et la table $O_3$ est construite.

Ensuite, la question de la \emph{fermeture} est posée. Un manquement est détecté : le mot $a$. En effet, en lui ajoutant le symbole $b$, on obtient $ab$ qui n'est ni dans $R$ ni en relation $R_O$ avec $a$. $ab$ est alors ajouté à $R$, et $R.\Sigma$ est étendu. La nouvelle table, $O_4$ est de nouveau testée.

$O_4$ ne respecte pas la fermeture : le mot $ab$, agrémenté du symbole $a$ donne le mot $aba$, qui n'est ni dans $R$ ni en relation avec $ab$. Le mot est ajouté à $R$, et la table est étendue. La nouvelle table, $O_5$ est à la fois fermée et cohérente.

L'automate $O_5$ est alors proposé à l'enseignant pour vérification. Celui-ci est accepté (isomorphe à $A_4$). L'algorithme s'arrête et un automate minimal pour le langage a été construit. 

\subsection{Algorithme}\label{ss:a_algo}

\subsection{Preuve}\label{ss:a_proof}

\subsection{Complexité}\label{ss:a_comp}

\section{Automates à files}\label{fifo}L'article de Vardhan \cite{Vardhan04} se concentre sur un automate plus général : l'automate FIFO. Celui-ci est Turing Complete. De la sorte, leur équipe propose une réponse pour un ensemble plus large de langage. Toutefois, tous les langages ne sont pas éligibles à l'algorithme. En effet, c'est un problème indécidable de façon générale. Cette section décrit les automates FIFO et lesquels sont éligibles à l'algorithme $L^*$ modifié.



% ██████  ███████ ███████
% ██   ██ ██      ██
% ██   ██ █████   █████
% ██   ██ ██      ██
% ██████  ███████ ██

\subsection{Définitions}\label{ss:fifodef}

\begin{definition}
  Un \emph{automate FIFO} \fifo est défini comme suit :
  \begin{itemize}
    \item $Q$ est un ensemble fini d'\emph{états de contrôle}
    \item $C$ est un ensemble fini de \emph{noms de canaux}
    \item $\Sigma$ est un alphabet
    \item $q_0 \in Q$ est l'\emph{état de contrôle initial}
    \item $\Theta$ est un ensemble fini de \emph{noms de transitions}
    \item $\delta$ est la \emph{fonction nommante}. $\delta : \Theta \rightarrow Q \times ((C \times \{?,!\} \times \Sigma) \bigcup \{\tau\}) \times Q$. Un nom de transition $\theta$ correspond à une transition de la forme $\delta(\theta)=(p,\text{"action"},q)$. Cette action a une des trois formes suivantes :
    \begin{itemize}
      \item $c!m$ : C'est une action d'envoi. Le symbole $m$ est ajouté en fin de canal $c$.
      \item $c?m$ : C'est une action de réception. Le symbole $m$ est consommé en début de canal $c$.
      \item $\tau$ : C'est une action interne. Aucun canal n'est manipulé.
    \end{itemize}
  \end{itemize}
\end{definition}

Un automate $F$ défini un \emph{système de transitions} \tsys. $\mathcal{T}$ est l'objet qui permet de passer d'un \emph{état} à un autre.

En effet, il existe les états de contrôles $q\in Q$, mais les états au sens d'un automate FIFO sont de forme $s \in S=Q\times(\Sigma^*)^C$. En particulier, un état $s=(q,w)$ avec $q\in Q$ un état de contrôle et $w\in (\Sigma^*)^C$ est un vecteur qui fait correspondre à chaque canal $c\in C$ un mot $w[c] \in \Sigma^*$ représentant le contenu de ce canal.

Dès lors, un état $s$ peut être compris comme étant composé d'un état et du contenu des différents canaux.

De plus, la \emph{fonction de transition} $\rightarrow:S\times\Theta\rightarrow S$ associe un état $s$ et un nom de transition $\theta$ à un état $s'$.

$\mathcal{T}$ respecte trois règles, correspondants chacune à un des types d'actions mentionnés précédemment. En plus de la notation $w[c]$, celles-ci utilisent la notation $w[c\mapsto c']$ signifiant $w$ à l'exception du canal $c$ dont le contenu a été remplacé par le mot $c'$.
\begin{itemize}
  \item Si $\delta(\theta)=(p,c?m,q)$ alors $(p,w)\xrightarrow{\theta}(q,w')$ si et seulement si $w=w'[c\mapsto mw'[c]]$
  \item Si $\delta(\theta)=(p,c!m,q)$ alors $(p,w)\xrightarrow{\theta}(q,w')$ si et seulement si $w'=w[c\mapsto mw[c]]$
  \item Si $\delta(\theta)=(p,\tau,q)$ alors $(p,w)\xrightarrow{\theta}(q,w')$ si et seulement si $w=w'$
\end{itemize}




\begin{example}
  Soit un automate FIFO $F$ tel que sont système de transitions corresponde à la figure \ref{fig:fifo1}. Chaque état de l'automate correspondant à un couple état de contrôle/mot, il est imprécis de référer au système de transitions comme étant l'automate. Cependant, par abus de langage, ceux-ci seront souvent confondus dans ce document.

  \begin{figure}[H]
    \centering
    \begin{tikzpicture}[->,>=stealth',shorten >=1pt,auto,node distance=2.5cm, semithick, bend angle=10]

      \tikzstyle{every state}=[circle]

      \node[initial,state] (A)                    {$q_0$};
      \node[state]         (B) [above right= 1cm and 3 cm of A] {$q_1$};
      \node[state]         (C) [below right= 1cm and 3 cm of A] {$q_2$};
      \node[state]         (D) [above right= 1cm and 3 cm of C] {$q_3$};

      \path
      (A) edge node {$\theta_1(a!0)$} (B)
      (A) edge node[below left] {$\theta_2(a!1)$} (C)
      (B) edge node {$\theta_4(a?0)$} (D)
      (B) edge[loop above] node {$\theta_3(b!1)$} (B)
      (C) edge node[below right] {$\theta_6(a?1)$} (D)
      (C) edge [loop below] node{$\theta_5(b!0)$} (C)
      (D) edge node[above] {$\theta_7(b?0)$} (A)
      ;
    \end{tikzpicture}
    \caption{Système de transitions de l'automate FIFO $F$}\label{fig:fifo1}
  \end{figure}

  On retrouve bien la définition d'un automate fifo \fifo avec :
  \begin{itemize}
    \item $Q=\{q_0,q_1,q_2,q_3\}$
    \item $C=\{a,b\}$
    \item $\Sigma=\{0,1\}$
    \item $q_0\in Q$
    \item $\Theta=\{\theta_1, \theta_2, \theta_3, \theta_4, \theta_5, \theta_6\}$
    \item $\delta$ associant à chaque $\theta_i$ un triplet état/action/état. Celui-ci est représenté entre parenthèses à côté du nom de transition associé
  \end{itemize}

  De plus, on peut déduire le système de transition $\mathcal{T}$ défini par $F$. Considérons le mot $w=[\epsilon,\epsilon]$ où le premier élément du vecteur est le contenu du canal $a$ et le second celui du canal $b$.
  Dans cet exemple, comme $\delta(\theta_1)=(q_0,a!0,q_1)$, alors $(q_0,w)\xrightarrow{\theta_1}(q_1,w')$. Dans ce cas, $w'=[0,\epsilon]$. A ce moment, on a bien $w'=w[a\mapsto 0w[a]]$.
  En utilisant ce nouveau mot $w'$, un nouvel état est atteignable : $q_3$. En effet, comme $\delta(\theta_4)=(q_1,a?0,q_3)$, alors $(q_1,w')\xrightarrow{\theta_4}(q_4,w'')$. Dans ce cas, $w''=[\epsilon,\epsilon]$. A ce moment, on a bien $w'=w''[a\mapsto 0w''[a]]$.

  Intuitivement, la première transition $\theta_1$ ajoute le symbole $0$ en tête du canal $a$ en passant de l'état $q_0$ à l'état $q_1$. La transition $\theta_4$, elle, permet de passer de l'état $q_1$ à $q_3$ en consommant $0$ en tête du canal $a$.

\end{example}


% ██       █████  ███    ██  ██████
% ██      ██   ██ ████   ██ ██
% ██      ███████ ██ ██  ██ ██   ███
% ██      ██   ██ ██  ██ ██ ██    ██
% ███████ ██   ██ ██   ████  ██████

\subsection{Langage tracé}

Une façon de définir un langage à partir d'un automate FIFO est de s'intéresser aux noms des transitions suivies lors de l'exécution. Cette section défini les éléments permettant d'arriver à la construction d'un tel langage.

Dans un système de transitions \tsys, la fonction de transition $\rightarrow:S\times\Theta\rightarrow S$ permet de définir le passage d'un état à un autre.

La \emph{fonction de transition étendue} $\xrightarrow{*}$ est la fermeture transitive et réflexive de $\rightarrow$.

Pour une suite de noms de transitions $\sigma=\theta_1\theta_2 ...\theta_n\in\Theta^*$, on note $(p,w)\xrightarrow{\sigma}(q,w')$ si il existe des états $(p_1,w_1)(p_2,w_2)...(p_{n-1},w_{n-1})$ tels que $(p,w)\xrightarrow{\theta_1}(p_1,w_1)\xrightarrow{\theta_2}...\xrightarrow{\theta_{n-1}}(p_{n-1},w_{n-1})\xrightarrow{\theta_n}(q,w')$. Dans ce cas, $\sigma$ est une \emph{trace de chemin}.

\begin{definition} Soit un automate FIFO $F$ et l'état initial $s_0=(q_0, \epsilon^C)$. Celui-ci est le couple état de contrôle initial $q_0$ ainsi que des mots $w[c]=\epsilon$ pour tout canal $c\in C$.

  Le \emph{langage de trace} d'un automate $F$ est

  $$
  L(F)=\{\sigma\in\Theta^*|\exists s=(p,w) \text{ tel quel } s_0\xrightarrow{\sigma}s\}
  $$
\end{definition}

\begin{example}
  Considérons l'automate FIFO $F$ de la figure \ref{fig:fifo1}.

  Pour celui-ci, $\sigma=\theta_1\theta_4\theta_7$ n'est pas un chemin. En effet,
  $$
  (q_0,[\epsilon,\epsilon])\xrightarrow{\theta_1}(q_1,[0,\epsilon])\xrightarrow{\theta_4}(q_3,[\epsilon,\epsilon])
  $$

  Mais, il n'existe pas d'état $s$ tel que $(q_3,[\epsilon,\epsilon])\xrightarrow{\theta_7}s$. En effet, pour appliquer cette transition, il aurait fallu que le canal $b$ contienne un symbole $0$. Ce n'est pas le cas.


  Par contre, $\sigma=\theta_2\theta_5\theta_5\theta_6\theta_7\theta_1\theta_4\theta_7$ est un chemin dans $F$ :
  \begin{equation*}
    \begin{gathered}
      (q_0,[\epsilon,\epsilon])\xrightarrow{\theta_2}
      (q_0,[1,\epsilon])\xrightarrow{\theta_5}
      (q_0,[1,0])\xrightarrow{\theta_5}
      (q_0,[1,00])\xrightarrow{\theta_6}
      (q_0,[\epsilon,00])\xrightarrow{\theta_7}\\
      (q_0,[\epsilon,0])\xrightarrow{\theta_1}
      (q_0,[0,0])\xrightarrow{\theta_4}
      (q_0,[\epsilon,0])\xrightarrow{\theta_7}
      (q_0,[\epsilon,\epsilon])
    \end{gathered}
  \end{equation*}

  On a bien un état $s$ (ici $s=(q_0,[\epsilon,\epsilon])=s_0$) tel que $s_0\xrightarrow{\sigma}s$.

\end{example}

\subsection{Produit cartésien}\label{ss:cartesien}

Par soucis de simplicité, un automate FIFO (et son système de transitions servant à le représenter) peut être représenté comme plusieurs systèmes de transitions utilisant les mêmes canaux. Le \emph{produit cartésien} entre deux automates FIFO $A$ et $B$ retourne un nouvel automate FIFO $F=A \times B$. Dès lors, il est possible de représenter un automate FIFO en se concentrant sur ses parties et en les isolant. Ce produit cartésien fonctionne comme suit.

Soient les automates FIFO \fifoA et \fifoB. Alors le système de transitions \tsys de l'automate FIFO $F=A\times B$ est composé de :
\begin{itemize}
  \item $S \subseteq (Q_A\times Q_B)\times (\Sigma^*)^C$ composé d'un couple d'états de contrôle de $Q_A$ et $Q_B$ et du contenu des différents canaux.
  \item $\Theta = \Theta_A \bigcup \Theta_B$
  \item $\rightarrow$ est construit comme suit. Soit un état $((q_A,q_B), w)\in S$. Soit un triplet $(p,a,q)$ avec $p,q \in (Q_A \bigcup Q_B)$ et $a \in ((C \times \{?,!\} \times \Sigma) \bigcup \{\tau\})$.
  $((q_A,q_B),w)\xrightarrow{\theta}((q_{A'},q_{B'}),w')$ si et seulement si l'une des trois conditions suivantes est remplie

  \begin{itemize}
    \item $\exists \theta_A\in\Theta_A, \delta_A(\theta_A)=(q_A,a,q_{A'})$ et  $(q_A,w)\xrightarrow{\theta_A}(q_{A'},w')$ dans l'automate $A$ et\\ $\exists \theta_B\in\Theta_B, \delta_B(\theta_B)=(q_B,a,q_{B'})$ et $(q_B,w)\xrightarrow{\theta_B}(q_{B'},w')$ dans l'automate $B$
    \item $\exists \theta_A\in\Theta_A, \delta_A(\theta_A)=(q_A,a,q_{A'})$ et  $(q_A,w)\xrightarrow{\theta_A}(q_{A'},w')$ dans l'automate $A$,\\
    $\forall \theta_B\in\Theta_B,\forall q \in Q_B,\delta_B(\theta_B)\neq(q_B,a,q)$ dans l'automate $B$ et $q_{B'}=q_B$
    \item $\forall \theta_A\in\Theta_A,\forall q \in Q_A,\delta_A(\theta_A)\neq(q_A,a,q)$ dans l'automate $A$ et $q_{A'}=q_A$,\\
    $\exists \theta_B\in\Theta_B, \delta_B(\theta_B)=(q_B,a,q_{B'})$ et  $(q_B,w)\xrightarrow{\theta_B}(q_{B'},w')$ dans l'automate $B$
  \end{itemize}
\end{itemize}


Dès lors, l'état initial est (($q_0,q_0$), [$\epsilon$,...,$\epsilon$]). \cite{Suresh20}


Alors, on peut construire l'automate FIFO \fifo grâce à l'algorithme de produit cartésien \ref{alg:crossfifo} :

\begin{algo}[Produit cartésien entre deux automates]\label{alg:crossfifo}
 \begin{algorithmic}[1]
   \REQUIRE Les automates FIFO $A=(Q_A,C,\Sigma,q_{0A}, \Theta_A,\delta_A)$ et $B=(Q_B,C,\Sigma,q_{0B},\Theta_B,\delta_B)$
   \ENSURE Un automate FIFO \fifo

   \STATE $Q=Q_A\times Q_B$
   \STATE $q_0 = (q_{0A},q_{0B})$ \COMMENT{l'état de contrôle à l'intersection de $q_{0A}$ et $q_{0B}$}
   \STATE $\Theta=\Theta_A \times \Theta_B$

   \FORALL {action $a \in (C\times\{?,!\}\times\Sigma)\bigcup\{\tau\}$}
    \FORALL {$\theta_A \in \Theta_A$ tels que $\delta_A(\theta_A)=(q_A,a,q_{A'})$ avec $q_A,q_{A'}\in Q_A$}
      \FORALL {$\theta_B \in \Theta_B$ tels que $\delta_B(\theta_B)=(q_B,a,q_{B'})$ avec $q_B,q_{B'}\in Q_B$}
        \STATE $\delta((\theta_A, \theta_B))=((q_A,q_B),a,(q_{A'},q_{B'}))$
      \ENDFOR
    \ENDFOR
    \FORALL {$\theta \in \Theta_A$ tels que $\delta_A(\theta)=(q_A,a,q_{A'})$ avec $q_A,q_{A'}\in Q_A$ qui n'ont pas été utilisés dans la boucle 6}
      \FORALL {$q_B\in Q_B$}
        \STATE  $\delta((\theta,))=((q_A,q_B),a,(q_{A'},_B))$
      \ENDFOR
    \ENDFOR

    \FORALL {$\theta \in \Theta_B$ tels que $\delta_B(\theta)=(q_B,a,q_{B'})$ avec $q_B,q_{B'}\in Q_B$ qui n'ont pas été utilisés dans la boucle 6}
      \FORALL {$q_A\in Q_A$}
        \STATE  $\delta((,\theta))=((q_A,q_B),a,(q_A,_{B'}))$
      \ENDFOR
    \ENDFOR

   \ENDFOR


   \RETURN \fifo
 \end{algorithmic}
\end{algo}

L'automate produit par cet algorithme \ref{alg:crossfifo} est différents des deux autres, il n'est alors pas pertinent de prouver une égalité. Il s'agit juste d'un autre mode de représentation.


\begin{example}
  Soient deux automates FIFO $A$ et $B$ tels que représentés par leur systèmes de transitions donnés par la figure \ref{fig:fifoAB}. L'automate FIFO $AB=A \times B$ est représenté par son sytème de transitions à la figure \ref{fig:fifocross}.

  \begin{figure}[H]
    \centering
    \begin{subfigure}{0.5\textwidth}
      \centering
      \begin{tikzpicture}[->,>=stealth',shorten >=1pt,auto,node distance=1.5cm, semithick, bend angle=10]
        \tikzstyle{every state}=[circle]

        \node[initial,state] (A)  {$q_{0}$};
        \node[state]         (B) [right=of A]  {$q_{1}$};
        \node[state]         (C) [right=of B]  {$q_{2}$};
        \path
        (A) edge node {$\theta_1(a!1)$} (B)
        (B) edge node {$\theta_2(a?1)$} (C)
        (C) edge[bend left=40] node {$\theta_3(a!0)$} (A)
        ;
      \end{tikzpicture}
      \caption{Automate FIFO A}
      \label{fig:fifoA}
      \end{subfigure}%
      \begin{subfigure}{0.5\textwidth}
        \centering
        \begin{tikzpicture}[->,>=stealth',shorten >=1pt,auto,node distance=1.5cm, semithick, bend angle=10]
          \tikzstyle{every state}=[circle]

          \node[initial,state] (A)  {$q_{A}$};
          \node[state]         (B) [right=of A]  {$q_{B}$};
          \path
          (A) edge[bend left=20] node {$\theta_5(a?0)$} (B)
          (B) edge[bend left=20] node {$\theta_6(a!0)$} (A)
          ;
        \end{tikzpicture}
        \caption{Automate FIFO B}
        \label{fig:fifoB}
      \end{subfigure}
      \caption{Automates FIFO A et B représentés par leur système de transitions}
      \label{fig:fifoAB}
    \end{figure}



    \begin{figure}[H]
      \centering
      \begin{tikzpicture}[->,>=stealth',shorten >=1pt,auto,node distance=1.5cm, semithick, bend angle=15]
        \tikzstyle{every state}=[circle]

        \node[initial,state] (A)  {$q_{0A}$};
        \node[state]         (B) [below=of A]  {$q_{1A}$};
        \node[state]         (C) [below=of B]  {$q_{2A}$};

        \node[state]         (D) [right=of A] {$q_{0B}$};
        \node[state]         (E) [below=of D]  {$q_{1B}$};
        \node[state]         (F) [below=of E]  {$q_{2B}$};

        \path
        (A) edge node {$\theta(a!1)$} (B)
        (A) edge[bend left] node {$\theta(a?0)$} (D)

        (B) edge node {$\theta(a?1)$} (C)
        (B) edge[bend left] node {$\theta{a?0}$} (E)

        (C) edge[bend left=35] node {$\theta(a!0)$} (A)
        (C) edge node {$\theta(a?0)$} (F)

        (D) edge[bend left] node {$\theta{a!0}$} (A)
        (D) edge node {$\theta{a!1}$} (E)

        (E) edge[bend left] node {$\theta{a!0}$} (B)
        (E) edge node {$\theta{a?1}$} (F)

        ;

        \draw [->] (F) ..  controls  ($(E)+(5cm,2cm)$) and
  ($(D)+(-0.5cm,3cm)$).. node[right] {$\theta(a!0)$} (A);
      \end{tikzpicture}
      \caption{Automate FIFO AB résultant du produit cartésien $A\times B$}
      \label{fig:fifocross}
    \end{figure}

  \end{example}



  %  █████  ██████  ██████
  % ██   ██ ██   ██ ██   ██
  % ███████ ██████  ██████
  % ██   ██ ██   ██ ██
  % ██   ██ ██████  ██

  \subsection{Alternating Bit Protocol}\label{ss:abp}

  \begin{figure}[H]
    \centering
    \begin{tikzpicture}[->,>=stealth',shorten >=1pt,auto,node distance=3.5cm, semithick, bend angle=10]

      \tikzstyle{every state}=[circle]

      \node[initial,state] (A)                    {$q_0$};
      \node[state]         (B) [right of=A] {$q_1$};
      \node[state]         (C) [below of=B] {$q_2$};
      \node[state]         (D) [left of=C] {$q_3$};

      \node[state,draw=none]         (i1) [right=0cm of B]      {};
      \node[state,draw=none]         (i2) [right=3.5cm of i1]      {};
      \node[state,draw=none]         (i3) [right=0cm of C]      {};
      \node[state,draw=none]         (i4) [right=3.5cm of i3]      {};

      \node[state] (E) [right=0cm of i2]               {$q_{0'}$};
      \node[state]         (F) [right of=E] {$q_{1'}$};
      \node[state]         (G) [below of=F] {$q_{2'}$};
      \node[state]         (H) [left of=G] {$q_{3'}$};



      \path
      (A) edge node {$\theta_1(A!0)$} (B)
      (B) edge node {$\theta_4(B?ACK0)$} (C)
      (B) edge[loop above] node {$\theta_2(A!0),\theta_3(B?ACK1)$} (B)
      (C) edge node {$\theta_5(A!1)$} (D)
      (D) edge node {$\theta_8(B?ACK1)$} (A)
      (D) edge[loop below] node {$\theta_6(A!1),\theta_7(B?ACK0)$} (D)


      (E) edge node {$\theta_{11}(A?0)$} (F)
      (E) edge[loop above] node {$\theta_9(B!ACK1),\theta_{10}(A?1)$} (E)
      (F) edge node {$\theta_{12}(B!ACK0)$} (G)
      (G) edge node {$\theta_{15}(A?1)$} (H)
      (G) edge[loop below] node {$\theta_{13}(B!ACK0),\theta_{14}(A?0)$} (G)
      (H) edge node {$\theta_{16}(B!ACK1)$} (E)
      ;

      \draw[<-] (E) -- node[above left] {start} ++(-1cm,-1cm);

      \draw[double,->] (i1) -- node[above] {canal A} (i2);
      \draw[double,->] (i4) -- node[below] {canal B} (i3);
    \end{tikzpicture}
    \caption{Automate Fifo du ABP (\cite{Finkel03}, Fig.1.)}\label{fig:fifoabp}
  \end{figure}

  \href{https://scanftree.com/automata/dfa-cross-product-property}{Produit cartésien de deux automates}. Grâce à ça, on peut représenter ce système comme un seul automate, mais en avoir deux sur un graphique, ce qui simplifie la compréhension.

