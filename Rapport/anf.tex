\subsection{Définition}

Une automate non-déterministe fini est une variété d'automate similaire aux ADF, moyennant quelques modifications. Un automate non-déterministe fini s'écrit:
	$$
	A=(Q,\Sigma, \delta, q_0, F)
	$$
avec : 

\begin{itemize}
	\item $Q$ un ensemble fini d'états
	\item $\Sigma$ un alphabet
	\item $q_0$ l'état initial
	\item $F\subseteq Q$ l'ensemble des états acceptants
	\item $\delta : Q \times \Sigma \cup \{\epsilon\} \rightarrow 2^Q$ où $2^Q$ est \emph{l'ensemble des parties} de $Q$. Cela signifie qui la fonction $\delta$ retourne un ensemble d'états de $Q$
\end{itemize}

Les automates non-déterministes finis peuvent être classés en deux groupes :
\begin{enumerate}
	\item Ceux pour lequel au moins une transition de $\delta$ est définie pour $\epsilon$, notés $\epsilon$\emph{-ANF}.
	\item Ceux pour lequel aucune transition n'est définie pour $\epsilon$. En pratique, la définition de delta devient $\delta : Q \times \Sigma \rightarrow 2^Q$. Ils sont notés \emph{ANF}
\end{enumerate}

\begin{exemple}\label{ex:anf}
	 De la même façon que pour l'exemple \ref{ex:adf} de la section \ref{sub:dfa}, considérons un automate \automaton défini comme suit :
	
\begin{itemize}
	\item $Q=\{q_0,q_1,q_2\}$
	\item $\Sigma=\{a,b,c\}$
	\item $q_0$ est l'état du même nom
	\item $\delta$ est donnée par la table \ref{fig:eanfdelta}. 
	\item $F=\{q_2\}$
\end{itemize}

$A$ est un NFA ; une colonne supplémentaire sert à représenter la transition sur $\epsilon$.
	
\begin{figure}[H]
	\centering
	\begin{tabular}{|r||c|c|c|c|}
		\hline
		&$\epsilon$&a&b&c\\
		\hline\hline
		$\rightarrow q_0$&$\{q_1,q_2\}$&$\emptyset$&$\{q_1\}$&$\{q_2\}$\\\hline
		$q_1$&$\emptyset$&$\{q_0\}$&$\{q_2\}$&$\{q_0,q_1\}$\\\hline
		$q_2^*$&$\emptyset$&$\emptyset$&$\emptyset$&$\emptyset$\\\hline
	\end{tabular}
	\caption{$\delta$}
	\label{fig:eanfdelta}
\end{figure}

De plus, $A$ peut être représenté par un graphe suivant la même méthodologie que dans la sous-section \ref{ss:grapheadf} pour les ADF. Additionnellement, $\epsilon$ peut servir d'étiquette même s'il n'appartient pas à $\Sigma$.

\begin{figure}[H]
	\centering
	\begin{tikzpicture}[->,>=stealth',shorten >=1pt,auto,node distance=2.5cm, semithick, bend angle=10]
	
	\tikzstyle{every state}=[circle]
	
	\node[initial,state] (A)                    {$q_0$};
	\node[state]         (B) [above right of=A] {$q_1$};
	\node[accepting, state]         (C) [below right of=A] {$q_2$};
	
	\path 	
	(A) edge [bend left] node{$\epsilon$,b} (B)
	(A) edge node{$\epsilon$,c} (C)
	(B) edge [bend left] node{a,c} (A)	
	(B) edge [loop right] node{c} (B)
	(B) edge node{b} (C);
	\end{tikzpicture}
	\caption{Automate $A$}\label{fig:eanf}
\end{figure}
\end{exemple}

\subsection{Fermeture sur $\epsilon$}

Pour chaque état $q$ d'un ANF, un ensemble d'état peut être atteint sans lire de symbole. Il s'agit de l'état en question et de tous ceux pouvant être atteint uniquement par des transitions sur $\epsilon$. Cet ensemble s'appelle la \emph{fermeture sur epsilon}, ECLOSE($q$), et peut être construit récursivement.

Soit un automate \automaton. Soit $q$ un état dans $Q$.

\paragraph{Cas de base} $q$ est dans ECLOSE($q$)

\paragraph{Pas de récurrence} Si $p$ est dans ECLOSE($q$) et qu'il existe un état $r$ tel quel $r\in\delta(p,\epsilon)$, alors $r$ est dans ECLOSE($q$)

\begin{exemple}\label{ex:anfclosure} Considérons l'automate $A$ de l'exemple \ref{ex:anf}. Les différentes fermetures peuvent être calculées :
	\begin{itemize}
		\item ECLOSE($q_0$) = $\{q_0,q_1,q_2\}$. En effet, $q_0$ appartient à sa fermeture, selon le cas de base. Aussi, $q_1,q_2\in\delta(q_0, \epsilon)$
		\item ECLOSE($q_1$)=$\{q_1\}$ par le cas de base.
		\item ECLOSE($q_2$)=$\{q_2\}$ par le cas de base.
	\end{itemize}
\end{exemple}


\todo{chemin anf avec epsilon OUI : indispensable pour que la preuve fonctionen}

\subsection{Transformation d'$\epsilon$-ANF à ADF}\label{ss:eanfadf}

Cette section présente une méthode permettant de créer un ADF à partir d'un $\epsilon$-ANF. Ceci fait, l'automate résultant peut être requalifié comme ANF.

Soit un ANF $E=(Q_E, \Sigma, \delta_E, q_0, F_E)$. Alors l'ADF équivalent
$$
D=(Q_D, \Sigma, \delta_D, q_D, F_D)
$$
est défini par :
\begin{itemize}
	\item $Q_D = \{S | S \subseteq Q_E \text{ et } S \text{ est \emph{fermé sur epsilon}}\}$. Concrètement, $Q_D$ est l'ensemble des partie des $Q_E$ fermées sur $\epsilon$. Cette fermeture s'écrit $S$=ECLOSE(S), ce qui signifie que chaque transition sur $\epsilon$ depuis un état de $S$ mène à un état également dans $S$. L'ensemble $\emptyset$ est fermé sur $\epsilon$.
	\item $q_D$=ECLOSE($q_0$). L'état initial de $D$ est l'ensemble des états dans la fermeture sur $\epsilon$ des états de $E$.
	\item $F_D= \{S|S \in Q_D \text{ et } S \cap F_E \neq \emptyset\}$ contient les ensembles dont au moins un état est acceptant pour $E$.
	\item $\delta_D(S,a)$ est construit, $\forall a \in \Sigma, \forall S \in Q_D$ par :
		\begin{enumerate}
			\item Soit $S=\{p_1, p_2,\dots,p_k\}$.
			\item Calculer $\bigcup_{i=1}^k\delta_E(p_i,a)$. Renommer cet ensemble en $\{r_1, r_2, \dots, r_m\}$.
			\item Alors $\delta_D(S,a)=\bigcup_{j=1}^m\text{ECLOSE(}r_j\text{)}$.
		\end{enumerate}
\end{itemize}


\begin{exemple} Considérons l'automate \automaton de l'exemple \ref{ex:anf} et les fermetures calculées dans l'exemple \ref{ex:anfclosure}.

Alors, l'automate $D=(Q_D, \Sigma, \delta_D, q_D, F_D)$ est donné par :

\begin{itemize}
	\item $Q_D=\{\emptyset, \{q_1\}, \{q_2\}, \{q_1,q_2\}, \{q_0,q_1,q_2\}\}$. Les ensembles $\{q_0,q_1\}$ et $\{q_0,q_2\}$ sont des sous-ensembles de $Q$ mais ne sont pas fermé sur $\epsilon$.
	\item $q_D=\{q_0,q_1,q_2\}=\text{ECLOSE(}q_0\text{)}$.
	\item $F_D=\{\{q_2\}, \{q_1,q_2\}, \{q_0,q_1,q_2\}\}$, les ensembles contenant $q_2$, étant acceptant de $A$.
	\item $\delta_D$ est exprimé sur le graphe de la figure \ref{fig:dndf}.
\end{itemize}
	

\begin{figure}[H]
	\centering
	\begin{tikzpicture}[->,>=stealth',shorten >=1pt,auto,node distance=4cm, semithick, bend angle=10]
	
	\tikzstyle{every state}=[circle]
	
	\node[accepting, state]         	(B) 				   {$\{q_0\}$};
	\node[state]		 	(A) [above of=B] {$\emptyset$};
	\node[state] (C) [right of=B] {$\{q_1\}$};
	\node[state]         	(D) [below of=C] {$\{q_2\}$};
	\node[accepting,state]         	(E) [right of=C] {$\{q_0,q_1\}$};
	\node[accepting,state]         	(F) [above of=C] {$\{q_0,q_2\}$};
	\node[state]         	(G) [below right of=E] {$\{q_1,q_2\}$};
	\node[initial,accepting, state]  	(H) [above right of=E] {$\{q_0,q_1,q_2\}$};
	
	\path 
	(A) edge [loop left] node{a,b,c} (A)
	
	(B) edge  node{a} (A)
	(B) edge [bend left, near start] node{b} (C)
	(B) edge node{c} (D)
	
	(C) edge [bend left, near start] node{a} (B)
	(C) edge node{b} (D)
	(C) edge node{c} (E)
	
	(D) edge [near end,above right] node{a,b,c} (A)
	
	(E) edge [bend left=20,near start] node{a} (B)
	(E) edge [bend left] node{b} (G)
	(E) edge node{c} (H)
	
	(F) edge node{a,b} (A)
	(F) edge [bend left=20,near end] node{c} (D)
	
	(G) edge [bend left=80] node{a} (B)
	(G) edge node{b} (D)
	(G) edge [bend left] node{c} (E)
	
	(H) edge [bend right=20, near start] node{a} (B)
	(H) edge node{b} (G)
	(H) edge [loop above] node{c} (H)	
	;
	\end{tikzpicture}
	\caption{Automate $D$. De par la construction par les parties de $Q_E$, le nombre de partie est exprimé en exponentiel, d'où la complexité du graphe. Ici, $\{q_0,q_2\}$ n'est pas atteignable et peut être supprimé.}\label{fig:dndf}
\end{figure}
\end{exemple}



\begin{theorem}
	Un langage $L$ peut être représenté par un $\epsilon$-ANF si et seulement si il peut l'être par un ADF.
\end{theorem}

\begin{proof}
	 Soit $L$ un langage. Cette preuve étant une double implication, chacune peut être prouvée séparément.
	
	($\Leftarrow$) $L$ peut être représenté par un ADF $\implies$ $L$ peut être représenté par un $\epsilon$-ANF. Supposons qu'un automate $D=(Q_D, \Sigma, \delta_D, q_D, F_D)$ représente $L$ : $L(D)=L$. 
	L'$\epsilon$-ANF $E=(Q_E, \Sigma, \delta_E, q_E, F_E)$ correspondant est construit comme suit :
	
	\begin{itemize}
		\item $Q_E=\{\{q\}|q\in Q_D\}$
		\item $\delta_E$ contient les transitions de $D$ modifiée. Les objets retournés deviennent des ensembles d'états, c'est-à-dire, si $\delta_D(q,a)=p$ alors $\delta_E(q,a)=\{p\}$. De plus, pour chaque état $q\in Q_D$, $\delta_E(q,\epsilon)=\emptyset$. 
		\item $q_E=\{q_0\}$
		\item $F_E=\{\{q\}| q\in F\}$
	\end{itemize}
	 
	 Dès lors, les transitions sont les mêmes entre $D$ et $E$, mais $E$ précise explicitement qu'il n'y a pas de transition sur $\epsilon$. Comme $E$ représente le même langage, un $\epsilon$-ANF représente $L$.
	
	
	($\Rightarrow$) $L$ peut être représenté par un $\epsilon$-ANF $\implies$ $L$ peut être représenté par un ADF. Soit l'automate $E=(Q_E, \Sigma, \delta_E, q_0, F_E)$. Supposons qu'il représente $L$ ($L=L(E)$). Considérons l'automate obtenu par la transformation détaillée à la section précédente \ref{ss:eanfadf} :
	$$
	D=(Q_D, \Sigma, \delta_D, q_D, F_D)
	$$
	Montrons que $L(D)=L(E)$. Pour ce faire, montrons que les fonctions de transition étendues sont équivalentes. Auquel cas, les chemins sont équivalents et donc les langages également.
	Montrons que $\hdelta_E(q_0,w)=\hdelta_D(q_D,w)$ pour tout mot $w$, par récurrence sur $w$.
	
	\paragraph{Cas de base} Si $|w|=0$, $w=\epsilon$. $\hdelta_E(q_0,\epsilon)=$ ECLOSE($q$), par définition de la fonction de transition étendue. $q_D$=ECLOSE($q_0$) par la construction de $q_D$. Finalement, pour un ADF (ici, $D$), $\hdelta(p,\epsilon)=p$, pour tout état $p$. Par conséquent, $\hdelta_D(q_D,\epsilon)=q_D=\text{ECLOSE(}q_0\text{)}=\hdelta_E(q_0,\epsilon)$.
	
	\paragraph{Pas de récurrence} Supposons $w=xa$ avec $a$ le dernier symbole de $w$. Notre hypothèse de récurrence est que $\hdelta_D(q_D,x)=\hdelta_E(q_0,x)$. Notons cet ensemble comme $\{p_1,p_2, \dots, p_k\}$. Par définition de $\hdelta$ pour un $\epsilon$-NFA, $\hdelta_E(q_0,w)$ est obtenu en :
	
	\begin{enumerate}
		contenu...
	\end{enumerate}
	
	
	
	
	
\end{proof}


