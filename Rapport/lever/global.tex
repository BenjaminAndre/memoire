Ce chapitre a décrit l'adaptation de l'algorithme d'Angluin à l'étude de la sécurité d'automates à file. Grâce à $\mathcal{W}(L)$ et $\mathcal{F}(L)$, des oracles ont pu être construits pour couvrir certains cas. De par les garanties del'algorithme d'Angluin\cite{Angluin87} et la construction utilisée \cite{Vardhan04}, certaines propriétés peuvent être prouvées, énoncées dans le théorème \ref{thm:fifolever}.

\begin{theorem}\label{thm:fifolever}
  Pour vérifier la propriété de safety des automates FIFO, l'algorithme LeVer respecte les propriétés suivantes :
  \begin{enumerate}
    \item Si l'algorithme retourne une réponse, celle-ci est correcte
    \item Si $AL(F)$ est régulier, la procédure s'arrête.
    \item Le nombre de test d'appartenance et d'éuivalence dépend principalement de l'algorithme d'Angluin. Le temps total est borné en temps polynomial du nombre d'états de l'automate minimal pour $AL(F)$ et linéaire en le temps pris pour une requête d'appartenance à $AL(F)$
  \end{enumerate}
\end{theorem}

Ce théorème rappelle la limite de la méthode. Il existe toutes une classe d'automates à files pour lesquels $AL(F)$ n'est pas régulier. À ce moment, aucune garantie ne peut-être fournie quand à l'exécution.

La preuve est disponible en annexe du travail de Vardhan et al.\cite{Vardhan04}.
