Les propriété de fermeture (\ref{ss:ferm}) et de cohérence (\ref{ss:coh}) qui sont définies ici sont à respecter à chaque itération de l'algorithme d'Angluin.


\subsection{Fermeture}\label{ss:ferm}
La propriété de \emph{fermeture} s'exprime mathématiquement par

$$ \forall u \in R, \forall a \in \Sigma, \exists v \in R, ua R_O v$$

Cette propriété peut être vérifiée par cet algorithme, expliqué de façon visuelle sur la table O :

\begin{algorithm}[H]
	\begin{algorithmic}[1]
		\ENSURE si la fermeture est respectée ou non

		\FORALL {élément $w$ de la section $R$}
		\FORALL {symbole $a$ dans $\Sigma$}
			\IF {$wa$ est dans $R$}
				\STATE continuer
			\ELSE
				\STATE \COMMENT{$wa$ est dans $R.\Sigma$ par construction}
				\IF {La ligne de $wa$ dans $T$ est différente de celle de $w$}
					\RETURN Faux
				\ENDIF
			\ENDIF
		\ENDFOR
		\ENDFOR
		\RETURN Vrai
	\end{algorithmic}
	\caption{Vérification de la fermeture}\label{alg:fermeture}
\end{algorithm}

\subsection{Cohérence}\label{ss:coh}

La propriété de \emph{cohérence} se définit mathématiquement comme

$$ \forall u,v \in R, u R_O v \Rightarrow \forall a \in \Sigma, ua R_O va$$

Concrètement, il s'agit de prendre deux mots ($u,v$) dans $R$ ayant la même ligne dans $T$ et vérifier, pour chaque symbole ($a$), s'ils ($ua,va$) ont la même ligne dans $T$.

\begin{example}
	Soit la table d'observation O de la table \ref{tab:Oex} :

	\begin{table}[H]
		\centering
	\begin{tabular}{|c|c|c|}
		\hline
		$O$ & $\epsilon$ & a\\
		\hline
		$\epsilon$ & 0 & 0\\
		$a$ & 0 & 0\\
		\hline
		$aa$ & 0 & 1\\
		\hline
	\end{tabular}\caption{Table d'observation O}\label{tab:Oex}
\end{table}
	\vspace{0.5cm}
	Cette table n'est pas cohérente. En effet, $\epsilon R_O a$ mais en ajoutant le symbole $a\in \Sigma$, on obtient $\neg (a R_O aa)$. Les lignes ont les mêmes valeurs, mais les lignes obtenues par la concaténation du symbole $a$ ont des valeurs différentes.
\end{example}
