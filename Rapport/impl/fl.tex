Soit un langage $L$ et un automate $A$ tel que $L=L(A)$. Calculer $\mathcal{F}(L)$ revient, par définition, à calculer $\bigcup_{\theta\in\Theta}Post(L,\theta)\bigcup \{t_{q_0}\}$.

Pour rappel, voici la définition de $Post(L, \theta)$ telle que donnée dans la section \ref{trace:extension} :

$Post(L,\theta)=\bigcup_{\gamma\in L} Post(\gamma,\theta)$ avec

$$
Post(\gamma,\theta) = \left\{ \begin{array}{ll}
    \emptyset & \text{si } \gamma \text{ n'est pas correctement formaté ou si } \mathcal{C}(\gamma)\neq source(\theta)\\
    \{\mathcal{T}(\gamma)t_{cible(\theta)}\} & \text{sinon si }\delta(\theta)=(p,\tau,q) \text{ ou } \delta(\theta)=(p,c_i!a_j,q) \text{ avec }p,q\in Q\\
    \{deriv(\mathcal{T}(\gamma),\theta)t_{cible(\theta)}\}& \text{sinon si } \delta(\theta)=(p,c_i?a_j,q) \text{ avec }p,q\in Q \\
    \end{array} \right.
$$

Des automates $A_\theta$ peuvent être construits pour $Post(L,\theta)$ pour chaque $\theta$ à partir de $A$. L'union de ces automates $A'=\bigcup_{\theta \in \Theta A_\theta}$ accepte $\bigcup_{\theta\in\Theta}Post(L,\theta)$. L'union de ce nouvel automate avec un automate pour le language $L=\{t_{q_0}\}$ accepte $\mathcal{F}(L)$.

La suite de cette section s'intéresse donc au calcul de $L'= Post(L,\theta)$ où $L$ est accepté par $A$ et $L'$ par un ADF $A'$ construit à partir de $A$.\\

L'automate $A'$ peut être non-déterministe. Ce n'est pas un problème car, comme l'explique la section \ref{adf:anfadf}, il existe un ADF équivalent. Dans le cadre de l'apprentissage avec Angluin, c'est cet ADF qui devient $A'$ et est utilisé pour les autres opérations.


Soit $\theta\in\Theta$ tel que $\delta(\theta)=(p,action,q)$. On explique ci-après les différents cas à traiter pour calculer $A_\theta$.

\subsection{Si $action$ est $\tau$ ou de la forme $c!a$}

Dans ce cas, construire un automate pour $L'=L\bigcap\Sigma^*t_p$ avec $t_p$ correspondant au $p$ de $\delta(\theta)$ et $\Sigma^*t_p$ étant représenté par l'ANF de la figure \ref{fig:sigmatq}.

\begin{figure}[H]
    \centering
    \begin{tikzpicture}[->,>=stealth',shorten >=1pt,auto,node distance=3cm, semithick, bend angle=10]
    \tikzstyle{every state}=[circle,scale=0.5]
    \node[state] (0) {};
    \node[state,accepting] (1)   [right of=0] {};

    \path
    (0) edge[loop above] node {$\Sigma$} (0)
    (0) edge node {$t_p$} (1)
    ;
    \end{tikzpicture}
    \caption{ANF pour $L=\Sigma^*t_p$}\label{fig:sigmatq}
\end{figure}


Dans cet automate $L'$, seuls les états et transitions menant à l'état final par $t_p$ sont considérés.

Construire un automate pour $L\bigcap\Sigma^*t_p=L'$. Si $L'$ est vide, c'est qu'il n'y a pas de mot finissant concerné par $\theta$ et donc que $Post(L,\theta)=\emptyset$.

Sinon, si $L'\neq\emptyset$, supposons d'abord que $action$ est de la forme $c!a$ ou $\tau$. Il faut alors appliquer une transformation à l'automate correspondant à $Post(L',\theta)$.
Si $action$ est de forme $c!a$, la transition sur $\theta$ peut être ajoutée avant les transitions sur $t_p$ et un nouveau $t_q$ signalant le changement d'état dans l'automate à files $F$.

\begin{figure}[H]
    \centering
    \begin{subfigure}{0.5\linewidth}
        \centering
        \begin{tikzpicture}[->,>=stealth',shorten >=1pt,auto,node distance=3cm, semithick, bend angle=10]
            \tikzstyle{every state}=[circle,scale=0.5]
            \node[state] (0) {};
            \node[state,accepting] (1)   [right of=0] {};
            \node[ellipse] [left of=0] {};
            \path
            (0) edge node {$t_p$} (1)
            ;
            \draw (-1,0) ellipse[x radius=1.5,y radius=1];
        \end{tikzpicture}
        \caption{Automate de $L'$}
    \end{subfigure}\hfill
    \begin{subfigure}{0.5\linewidth}
        \centering
        \begin{tikzpicture}[->,>=stealth',shorten >=1pt,auto,node distance=3cm, semithick, bend angle=10]
            \tikzstyle{every state}=[circle,scale=0.5]
            \node[state] (0) {};
            \node[state] (1) [right of=0] {};
            \node[state,accepting] (2)   [right of=1] {};

            \path
            (0) edge node {$\theta$} (1)
            (1) edge node {$t_q$} (2)
            ;
            \draw (-1,0) ellipse[x radius=1.5,y radius=1];
        \end{tikzpicture}
        \caption{Automate de $Post(L',\theta)$}
    \end{subfigure}
    \caption{Application de $Post(L',\theta)$ sur un automate si l'action est de forme $c!a$}
\end{figure}

À présent, supposons que l'action de $\theta$ est de forme $\tau$, $\theta\notin\Phi$. Alors, $t_p$ est remplacé par $t_q$. La transition est alors sous-entendue.

\begin{figure}[H]
    \centering
    \begin{subfigure}{0.5\linewidth}
        \centering
        \begin{tikzpicture}[->,>=stealth',shorten >=1pt,auto,node distance=3cm, semithick, bend angle=10]
            \tikzstyle{every state}=[circle,scale=0.5]
            \node[state] (0) {};
            \node[state,accepting] (1)   [right of=0] {};
            \node[ellipse] [left of=0] {};
            \path
            (0) edge node {$t_p$} (1)
            ;
            \draw (-1,0) ellipse[x radius=1.5,y radius=1];
        \end{tikzpicture}
        \caption{Automate de $L'$}
    \end{subfigure}\hfill
    \begin{subfigure}{0.5\linewidth}
        \centering
        \begin{tikzpicture}[->,>=stealth',shorten >=1pt,auto,node distance=3cm, semithick, bend angle=10]
            \tikzstyle{every state}=[circle,scale=0.5]
            \node[state] (0) {};
            \node[state,accepting] (1)   [right of=0] {};
            \node[ellipse] [left of=0] {};
            \path
            (0) edge node {$t_q$} (1)
            ;
            \draw (-1,0) ellipse[x radius=1.5,y radius=1];
        \end{tikzpicture}
        \caption{Automate de $Post(L',\theta)$}
    \end{subfigure}
    \caption{Application de $Post(L',\theta)$ sur un automate si l'action est $\tau$}
\end{figure}


\subsection{Si $action$ est de la forme $c?a$}

Pour qu'une réception $c?a$ d'un symbole soit possible, il faut qu'il existe un envoi qui lui soit associé. Cet envoi est noté $\theta_s$ tel que $\delta(\theta_s)=(r,c!a,s)$. De plus, il faut, comme dans la section précédente, tester si $Post(L,\theta)$ est vide ou non grâce au $t_p$ correspondant au $p$ de $\delta(\theta)$.

En utilisant $\theta_s$, on construit $L'=L\bigcap\Sigma^*\Theta_s\Sigma^*t_p$ où $\Theta_s$ est l'ensemble des $\theta_s$ associés à $c?a$. Un automate acceptant ce langage est donné à la figure \ref{fig:lseconde}.

\begin{figure}[H]
    \centering
    \begin{tikzpicture}[->,>=stealth',shorten >=1pt,auto,node distance=3cm, semithick, bend angle=10]
        \tikzstyle{every state}=[circle,scale=0.5]
        \node[state] (0) {};
        \node[state] (1) [right of=0] {};
        \node[state,accepting] (2)   [right of=1] {};

        \path
        (0) edge[loop above] node {$\Sigma$} (0)
        (0) edge node {$\Theta_s$} (1)
        (1) edge[loop above] node{$\Sigma$} (1)
        (1) edge node {$t_p$} (2)
        ;
    \end{tikzpicture}
    \caption{ANF pour $L=\Sigma^*\Theta_s\Sigma^*t_p$}\label{fig:lseconde}
\end{figure}

Si $L'=\emptyset$, il n'existe pas de $\theta_s$ dans les chemins qui nous concernent. Il n'y aura donc pas de possibilité de consommer un symbole : $Post(L,\theta)=\emptyset$.

Sinon si $L'$ est non vide, il faut alors, à partir de l'automate $A'$ acceptant $L'$, construire un automate $A_\theta$ acceptant le langage $Post(L',\theta)$.

\begin{figure}[H]
    \centering
    \begin{tikzpicture}[->,>=stealth',shorten >=1pt,auto,node distance=3cm, semithick, bend angle=10]
        \tikzstyle{every state}=[circle,scale=0.5]
        \node[state] (0) {$r$};
        \node[state] (1) [right of=0] {$s$};

        \node[state] (2) [right=4.5cm of 0] {$r'$};
        \node[state] (3) [right of=2] {$s'$};

        \path
        (0) edge[bend right=30] node {$\bt_s$} (3)
        (2) edge node {$\theta_s$} (3)
        ;
        \path [dashed]
        (0) edge node {$\theta_s$} (1)
        ;
        \draw (-1,-1.5) rectangle ++(3,3);
        \draw (4,-1.5) rectangle ++(3,3);

        \node at(0.5,-2) {$A'$};
        \node at(5.5,-2) {$A'_{copie}$};

    \end{tikzpicture}
    \caption{Première étape de construction de l'automate $P_A$ pour $Post(L'',\theta)$}\label{fig:aaprime}
\end{figure}

La figure \ref{fig:aaprime} décrit comment l'automate $A_\theta$ est construit à partir de $A'$. Premièrement, une copie de $A'$ est construite : $A'_{copie}$.
Dans $A'$, les états finaux sont considérés comme des états non finaux. Dans $A'_{copie}$, l'état initial est considéré comme un état non initial. Finalement, dans $A'$, toute transition $\theta_s$ allant de $r$ à $s$ est remplacée par une transition sur $\bt_s$ allant de $r$ à $s'$.

De cette façon, pour qu'un mot soit valide, il faut qu'il commence dans $A'$. La première occurence de $\theta_s$ a été systématiquement remplacée pour $\bt_s$ menant à $A'_{copie}$. Dans $A'_{copie}$, les autres transitions ne sont pas modifiées, laissant intact le reste du chemin. Cependant, la figure \ref{fig:aaprime} à elle seule ne suffit pas à construire $A_\theta$. Pour rester en accord avec la définition de $Post(L,\theta)$, il faut adapter le dernier symbole, ici $t_p$ en $t_q$ (figure \ref{fig:pq}).


\begin{figure}[H]
    \centering
    \begin{subfigure}{0.5\linewidth}
        \centering
        \begin{tikzpicture}[->,>=stealth',shorten >=1pt,auto,node distance=3cm, semithick, bend angle=10]
            \tikzstyle{every state}=[circle,scale=0.5]
            \node[state] (0) {};
            \node[state,accepting] (1)   [right of=0] {};
            \path
            (0) edge node {$t_p$} (1)
            ;
            \draw (-2,-1) rectangle ++(4,2);
        \end{tikzpicture}
        \caption{Automate de la figure \ref{fig:aaprime}}
    \end{subfigure}\hfill
    \begin{subfigure}{0.5\linewidth}
        \centering
        \begin{tikzpicture}[->,>=stealth',shorten >=1pt,auto,node distance=3cm, semithick, bend angle=10]
            \tikzstyle{every state}=[circle,scale=0.5]
            \node[state] (0) {};
            \node[state,accepting] (1)   [right of=0] {};

            \path
            (0) edge node {$t_q$} (1)
            ;
            \draw (-2,-1) rectangle ++(4,2);
        \end{tikzpicture}
        \caption{Automate $A_\theta$ de $Post(L',\theta)$}
    \end{subfigure}
    \caption{Deuxième étape de construction de l'automate $A_\theta$ pour $Post(L',\theta)$}\label{fig:pq}
\end{figure}



\subsection{Exemple}

Dans cette section, on illustre la construction de $\mathcal{F}(L)$ donnée dans la section précédente. Cet exemple se base sur l'automate à files $F$ de la figure \ref{fig:fex} ainsi que l'ADF $A$ de la figure \ref{fig:lfex} acceptant le langage de traces annotées $L$.

\paragraph{$F$ et $L(F)$ candidat à $AL(F)$}
\begin{figure}[H]
  \begin{subfigure}[H]{0.5\linewidth}
    \centering
    \begin{tikzpicture}[->,>=stealth',shorten >=1pt,auto,node distance=5cm, semithick, bend angle=10]

    \tikzstyle{every state}=[circle]


    \node[initial, state]   (0)                         {$q_0$};
    \node[state]            (1) [above right of=0]      {$q_1$};
    \node[state]            (2) [below right of=0]      {$q_2$};

    \path
    (0) edge [bend left=10] node {$\theta_1(a!0)$} (1)
    (0) edge [bend left=10] node {$\theta_2(a!1)$} (2)
    (1) edge [bend left=10] node {$\theta_3(a?0)$} (0)
    (1) edge node {$\theta_4(a!1)$} (2)
    (2) edge [bend left=10] node {$\theta_5(a?1)$} (0)
    ;
    \end{tikzpicture}
    \caption{Automate à files $F$}\label{fig:fex}
\end{subfigure}
\begin{subfigure}[H]{0.5\linewidth}
    \centering
    \begin{tikzpicture}[->,>=stealth',shorten >=1pt,auto,node distance=3cm, semithick, bend angle=10]
    \tikzstyle{every state}=[circle]

    \node[initial, state]   (0)                         {$0$};
    \node[state]            (1) [above right of=0]      {$1$};
    \node[state]            (2) [below right of=0]      {$2$};
    \node[state, accepting] (3) [below right of=1]      {$3$};

    \path
    (0) edge [loop above] node {$\bt_1$} (0)
    (0) edge [loop below] node {$\bt_2$} (0)
    (0) edge [bend left] node {$\theta_1$} (1)
    (0) edge node {$\theta_2$} (2)
    (0) edge node [below left] {$t_{q_0}$} (3)

    (1) edge node {$t_{q_1}$} (3)
    (1) edge [bend left] node {$\bt_4$} (0)
    (1) edge node [pos=0.2,above right] {$\theta_4$} (2)

    (2) edge node {$t_{q_2}$} (3)
    ;
    \end{tikzpicture}
    \caption{Automate $A$ acceptant $L$}\label{fig:lfex}
  \end{subfigure}
\end{figure}

Pour simplifier la lecture de cet exemple, les états de $A$ pour lesquels il existe une transition $t_p$ vers un état final sont appelés \emph{états pré-finaux} sur $t_p$.

\begin{itemize}
  \item Soit $\theta_1$ avec $\delta(\theta_1)=(q_0,a!0,q_1)$ menant au calcul de $A_{\theta_1}$. Les états pré-finaux sur $q_0$ sont $\{0\}$. Un automate est construit pour $L'=L\bigcap \Sigma^*t_{q_0}$ : celui de la figure \ref{fig:tq0}.

\begin{figure}[H]
    \centering
    \begin{tikzpicture}[->,>=stealth',shorten >=1pt,auto,node distance=3cm, semithick, bend angle=10]
    \tikzstyle{every state}=[circle]

    \node[initial, state]   (0)                         {$0$};
    \node[state]            (1) [above right of=0]      {$1$};
    \node[state, accepting] (3) [below right of=1]      {$3$};

    \path
    (0) edge [loop above] node {$\bar{\theta_1}$} (0)
    (0) edge [loop below] node {$\bar{\theta_2}$} (0)
    (0) edge [bend left] node {$\theta_1$} (1)
    (1) edge [bend left] node {$\bt_4$} (0)
    (0) edge node {$t_{q_0}$} (3)
    ;
  \end{tikzpicture}\caption{Un ANF représentant $L'$}\label{fig:tq0}
\end{figure}

En suivant la règle pour les $\theta$ dont l'action est de type $c!a$, ce qui est le cas pour $\theta_1$, un nouvel état est créé entre $0$ et $3$ car il y a une transition $t_{q_0}$ à cet endroit. Cela donne l'ANF de la figure \ref{fig:tq0b}

\begin{figure}[H]
    \centering
    \begin{tikzpicture}[->,>=stealth',shorten >=1pt,auto,node distance=3cm, semithick, bend angle=10]
    \tikzstyle{every state}=[circle]

    \node[initial, state]   (0)                         {$0$};
    \node[state]            (1) [above right of=0]      {$1$};
    \node[state]            (4) [below right of=1]      {$4$};
    \node[state, accepting] (3) [right of=4]      {$3$};

    \path
    (0) edge [loop above] node {$\bar{\theta_1}$} (0)
    (0) edge [loop below] node {$\bar{\theta_2}$} (0)
    (0) edge [bend left] node {$\theta_1$} (1)
    (1) edge [bend left] node {$\bt_4$} (0)
    (0) edge node {$\theta_1$} (4)
    (4) edge node {$t_{q_1}$} (3)
    ;
  \end{tikzpicture}\caption{L'ANF $A_{\theta_1}$ acceptant $Post(L,\theta_1)$}\label{fig:tq0b}
\end{figure}

Cet automate \ref{fig:tq0b} n'est pas déterministe mais il existe un équivalent déterministe qui peut être utilisé pour l'algorithme d'Angluin.


\item Soit $\theta_3$ avec $\delta(\theta_3)=(q_1,a?0,q_3)$ menant au calcul de $A_{\theta_3}$. Les états pré-finaux sur $q_1$ sont $\{1\}$. Contrairement à $\theta_1$, $\theta_3$ est une action de réception.  Ici le seul symbole $\theta_s$ d'action $c!a$ correspondant à $\theta_3$ ($a!0$) est $\theta_1$.

Dès lors, l'ANF pour $L'=L\bigcap \Sigma^*.\{\theta_3\}.\Sigma^*.t{q_1}$ est calculé.

\begin{figure}[H]
    \centering
    \begin{tikzpicture}[->,>=stealth',shorten >=1pt,auto,node distance=3cm, semithick, bend angle=10]
    \tikzstyle{every state}=[circle]

    \node[initial, state]   (0)                         {$0$};
    \node[state]            (1) [above right of=0]      {$1$};
    \node[state, accepting] (3) [below right of=1]      {$3$};

    \path
    (0) edge [loop above] node {$\bar{\theta_1}$} (0)
    (0) edge [loop below] node {$\bar{\theta_2}$} (0)
    (0) edge [bend left] node {$\theta_1$} (1)
    (1) edge [bend left] node {$\bt_4$} (0)
    (1) edge node {$t_{q_1}$} (3)
    ;
  \end{tikzpicture}\caption{Un ANF représentant $L'$}\label{fig:tq1}
\end{figure}

Cet automate sert à construire $Post(L,\theta_3)$ comme décrit pour les $\theta$ dont l'action est de forme $c?a$. $Post'L,\theta_3)$ est représenté par la figure \ref{fig:tq1b}.

\begin{figure}[H]
    \centering
    \begin{tikzpicture}[->,>=stealth',shorten >=1pt,auto,node distance=3cm, semithick, bend angle=10]
        \tikzstyle{every state}=[circle,scale=0.8]

        \node[initial, state]   (0)                         {$0$};
        \node[state]            (1) [above right of=0]      {$1$};
        \node[state] (3) [below right of=1]      {$3$};

        \node[state]   (4) [right=4.5cm of 3]            {$0'$};
        \node[state]            (5) [above right of=4]      {$1'$};
        \node[state, accepting] (6) [below right of=5]      {$3'$};

        \path
        (0) edge [loop above] node {$\bar{\theta_1}$} (0)
        (0) edge [loop below] node {$\bar{\theta_2}$} (0)
        (0) edge [bend right=20] node [below right] {$\bt_1$} (4)
        (1) edge [bend left] node {$\bt_4$} (0)
        (1) edge node {$t_{q_1}$} (3)

        (4) edge [loop above] node {$\bar{\theta_1}$} (4)
        (4) edge [loop below] node {$\bar{\theta_2}$} (4)
        (4) edge [bend left] node {$\theta_1$} (5)
        (5) edge [bend left] node {$\bt_4$} (4)
        (5) edge node {$t_{q_0}$} (6)
        ;

        \draw (-2.5,-2) rectangle ++(7,6);
        \draw (7,-2) rectangle ++(6,6);


    \end{tikzpicture}
    \caption{Un ANF représentant $Post(L,\theta_3)$}\label{fig:tq1b}
\end{figure}
\end{itemize}


En répétant ces opérations pour les symboles $\{ \theta_2, \theta_4, \theta_5\}$, il est possible de calculer l'automate $A'$ représentant $L'=\bigcup_{\theta\in\Theta}Post(L,\theta)$. L'union de cet automate $A'$ avec un automate représentant $\{t_{q_0}\}$ résulte en un automate $A''$ représentant $\mathcal{F}(L)$.
