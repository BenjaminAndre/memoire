L'oracle d'appartenance est l'un des deux oracles utilisés par l'algorithme d'Angluin découvert dans la section \ref{angluin}. La section \ref{mem} mentionne le comportement de cet oracle. Cependant, elle manque de précisions permettant d'exécuter l'automate à files de façon efficace.

L'algorithme d'appartenance \ref{alg:membership} répond à cette problématique.

\begin{algo}[Appartenance]\label{alg:membership}
  \begin{algorithmic}[1]
    \REQUIRE une trace correctement annotée $\gamma\in\Phi^*$ , un automate à files \fifo et un état de contrôle actif $q_a$
    \ENSURE si $\gamma\in AL(F)$ ou non
    \STATE $a,\gamma'\leftarrow$ premier symbole de $\gamma$ et le reste de la trace annotée

    \IF{$a\in T_Q$}
      \IF {$a$ correspond à $q_a$ et $|\gamma'|=0$}
        \RETURN vrai
      \ENDIF
    \ELSIF{$a\in$\barTheta}
      \STATE $a'\leftarrow a$ sans la barre
      \STATE $(p, c!m, q)\leftarrow\delta(a')$ \COMMENT{comme $a\in\bar{\Theta}$, il s'agit forcément d'une action d'envoi}
      \IF{$p=q_a$}
        \FORALL{$\theta_r$ tel que $\delta(\theta_r)=(r,c?m,s)$c}
          \FORALL{position $i$ dans $\gamma'$ sauf après le symbole terminal $t\in T_Q$}
            \STATE $\gamma''\leftarrow\gamma'$ où $\theta_r$ a été inséré à la position $i$
            \IF{Appartenance($\gamma''$, $F$, $q$)}
              \RETURN vrai
            \ENDIF
          \ENDFOR
        \ENDFOR
      \ENDIF
    \ELSE
      \STATE $(p,action,q)\leftarrow\delta(a)$ \COMMENT{action est obligatoirement de type envoi sinon la trace ne serait pas correctement formattée}
      \IF{$p=q_a$}
        \RETURN Appartenance($\gamma'$, $F$, $q$)
      \ENDIF
    \ENDIF
    \RETURN faux
  \end{algorithmic}
\end{algo}

\paragraph{Exemple d'application de l'algorithme d'appartenance}

Soit l'automate à files $F$ dans la figure \ref{fig:fifo1}.
Soit le mot correctement formatté $\gamma=\bt_2\bt_5 t_{q_0} \in \Phi^*$.

Voici le déroulement de $appartenance(\gamma, F, q_0)$.

  $q_a=q_0$, $a=\bt_2$ et $\gamma'=\bt_5 t_{q_0}$. $a\in$\barTheta et $source(a)=q_0$. Les dérivés avec $\theta_r$ sont produits. Ici le seul $\theta_r$ compatible est $\theta_6$
  \begin{itemize}
    \item $Appartenance(\theta_6\bt_5 t_{q_0}, F, q_2)$
    \begin{enumerate}
      \item Ici, $source(\theta_6)=q_2$.
      \item Un appel récursif est fait sur $Appartenance(\bt_5 t_{q_0}, F, q_3)$
      \item $source(\bt_5)\neq q_3=dest(\theta_6))$. Cet appel récursif retourne faux.
    \end{enumerate}
    \item $Appartenance(\bt_5\theta_6 t_{q_0}, F, q_2)$
    \begin{enumerate}
      \item $a=\bt_5\in$\barTheta et $\gamma'=\theta_6 t_{q_0}$
      \item Les dérivés avec $\theta_r$ sont produits. Le seul $\theta_r$ compatible avec $\bt_5$ est $\theta_7$.
      \item $dest(\theta_5)=q_2$
      \begin{itemize}
        \item $Appartenance(\theta_7\theta_6 t_{q_0}, F, q_2)$
        \begin{enumerate}
          \item $source(\theta_7)=q_3\neq q_2$
          \item Cet appel récursif retourne faux.
        \end{enumerate}
        \item $Appartenance(\theta_6\theta_7 t_{q_0}, F, q_2)$
        \begin{enumerate}
          \item $source(\theta_6)=q_2$
          \item $dest(\theta_6)=q3$
          \item Appel récursif sur $Appartenance(\theta_7 t_{q_0}, F, q_3)$
          \item $source(\theta_7)=q_3$ et $dest(\theta_7)=q_0$
          \item Cet appel lance lui même un appel récursif sur $Appartenance(t_{q_0}, F, q_0)$ qui est vrai
        \end{enumerate}
      \end{itemize}
    \end{enumerate}
  \end{itemize}
