\subsection{Théorème de Myhill-Nerode}
	
	\begin{theorem}
		Les 3 énoncés suivants sont équivalents :
		\begin{enumerate}
			\item Un langage $L\subseteq\Sigma^*$ est accepté par un DFA
			\item $L$ est l'union de certaines classes d'équivalence d'index fini respectant une relation d'équivalence et de congruence à droite
			\item Soit la relation d'équivalence $R_L : xR_Ly \Leftrightarrow \forall z \in \Sigma^*, xz \in L \Leftrightarrow yz \in L$. $R_L$ est d'index fini.
		\end{enumerate}
	\end{theorem}
	
	\begin{proof}La preuve d'équivalence se fait en prouvant chaque implication de façon cyclique :\\
		
		$(1)\rightarrow(2)$ 
		
		
		$(2)\rightarrow(3)$ toute relation E de 2 est un refinement de RL du coup chaque c.eq est completement contenue dans une c.Eq de RL. on part de xRMy, cong droite
		
		$(3)\rightarrow(1)$ Mq RL cong droite xRLy, utiliser définitions 
	\end{proof}


	\begin{corollary}
		Possibilité de créer l'automate canonique...
	\end{corollary}