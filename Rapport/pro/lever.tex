Ce chapitre a décrit l'adaptation de l'algorithme d'Angluin à l'étude de la sécurité d'automates à file. Grâce à $\mathcal{W}(L)$ et $\mathcal{F}(L)$, des oracles ont pu être construits pour couvrir certains cas. De par les garanties del'algorithme d'Angluin\cite{Angluin87} et la construction utilisée \cite{Vardhan04}, certaines propriétés peuvent être prouvées, énoncées dans le théorème \ref{thm:fifolever}.

\begin{theorem}\label{thm:fifolever}
  Pour vérifier la propriété de safety des automates FIFO, l'algorithme LeVer respecte les propriétés suivantes :
  \begin{enumerate}
    \item Si l'algorithme retourne une réponse, celle-ci est correcte
    \item Si $AL(F)$ est régulier, la procédure s'arrête.
    \item Le nombre de test d'appartenance et d'éuivalence dépend principalement de l'algorithme d'Angluin. Le temps total est borné en temps polynomial du nombre d'états de l'automate minimal pour $AL(F)$ et linéaire en le temps pris pour une requête d'appartenance à $AL(F)$
  \end{enumerate}
\end{theorem}

Ce théorème rappelle la limite de la méthode. Il existe toutes une classe d'automates à files pour lesquels $AL(F)$ n'est pas régulier. À ce moment, aucune garantie ne peut-être fournie quand à l'exécution.

La preuve est disponible en annexe du travail de Vardhan et al.\cite{Vardhan04}.


Dans les chapitres précédants, les bases théoriques sur les automates et langages ont été posées. Celles-ci ont été suivies de concepts plus étroitement liés à LeVer tels que les automates à files, les traces, traces annotées et langages associés. Un dernier élément présenté dans la section \ref{sec:unsafe} est la sécurité d'un état ou d'un automate.

En appliquant l'algorithme d'Angluin \cite{Angluin87} selon la méthode LeVer \cite{Vardhan04}, il est possible de se prononcer sur la sécurité pour toute une classe d'automates à files : ceux pour lesquels le language de traces annotées est régulier.

L'objectif dès lors n'est pas d'apprendre le language de l'automate à file mais le language de trace associé, et d'adapter la méthode pour répondre à la question de sécurité. En formulant les bonnes propriétés, il est également possible d'interrompre l'algorithme d'apprentissage avant terme si l'on peut se prononcer sur la sécurité de façon certaine.


Contrairement à la version proposée dans la section \ref{sec:angluin}, ici il n'est pas possible de tester directement l'équivalence entre $L$ et le language à apprendre. En effet, ce dernier peut ne pas être régulier et de façon générale le professeur ne connaît justement pas l'automate permettant une telle comparaison si celle-ci existe.

Ici, le professeur passe par trois questions mais à aucun moment il sait dire si $L$ est exactement le language à apprendre. Il peut soit se prononcer sur la sécurité, soit dire avec certitude qu'il existe une meilleure approximation.

\begin{itemize}
	\item \emph{L est-il un point fixe ?} Cette question fait référence à l'application de la fonction d'extension de trace de la section \ref{ss:extension}.
	\item \emph{L intersecte-t-il une région à risque ?} Si $L$ est un point fixe de $\mathcal{F}$, il est soit le language à apprendre soit un point fixe le contenant. Si une surapproximation n'a pas d'intersection avec une région à risque, $L$ n'en a pas non plus.
	\item \emph{Le chemin vers l'état à risque est-il valide ?} Si un \emph{chemin à risque} (menant à un état à risque) existe dans $L$, est-il valide ? C'est-à-dire, appartient-il au language visé et représente-t-il bien un chemin dans l'automate à files étudié ?
\end{itemize}

Les prochaines sections décrivent ce nouvel oracle d'appartenance et celui d'équivalence avec les trois questions posées avant de revenir sur une vue d'ensemble. De la sorte, la section \ref{sec:global} résume quand et pourquoi cette méthode fonctionne.
