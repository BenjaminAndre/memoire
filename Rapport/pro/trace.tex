Cette section propose un nouveau langage : les traces annotées. Il se base sur les traces et permet de retrouver celles-ci. Si ce langage de traces annotées est régulier, il peut être appris conformément à \cite{Vardhan04}. Se basant sur le langage de traces défini à la section \ref{fifo}, un nouvel alphabet est proposé en \ref{trace:alpha}. Grâce à celui-ci, un algorithme est défini en \ref{trace:annot} qui permet de construire une trace annotée à partir d'une trace. Finalement, la sous-section \ref{trace:extension} propose une fonction d'extension de trace et en propose une propriété intéressante qui permet de faciliter la réponse à l'oracle d'équivalence par le professeur de l'algorithme d'Angluin défini dans la section \ref{angluin}.


  % ████████ ██   ██ ███████ ████████  █████
  %    ██    ██   ██ ██         ██    ██   ██
  %    ██    ███████ █████      ██    ███████
  %    ██    ██   ██ ██         ██    ██   ██
  %    ██    ██   ██ ███████    ██    ██   ██



\subsection{Alphabet d'annotation}\label{trace:alpha}

Le langage de traces en tant que tel n'apporte pas de simplification à l'automate. C'est une autre façon d'écrire un chemin. Pour permettre l'apprentissage par l'algorithme d'Angluin, il faut construire un nouveau langage, si possible régulier, qui permette de reconstruire le langage de trace. Pour ce faire, ce nouveau langage devrait pouvoir représenter tout état atteignable ainsi qu'un ou plusieurs chemins ou mots témoins permettant d'atteindre ceux-ci.

Pour ce faire, pour chaque nom de transition correspondant à une action d'envoi, un \emph{co-nom} est défini :
$$
\bar{\Theta}=\{\bar{\theta}|\theta\in\Theta\wedge\exists p,q \in Q, c\in C, a\in\Sigma,\text{tels que } \delta(\theta)=(p,c!a,q)\}
$$

De plus, un \emph{symbole de contrôle} est créé pour chaque état de contrôle : $T_Q = \{t_q | q\in Q\}$.

En combinant les noms de transitions, les co-noms et les symboles de contrôle, un nouvel alphabet peut être défini, l'\emph{alphabet d'annotation} : $\Phi=(\Theta-\Theta_r)\bigcup\bar{\Theta}\bigcup T_Q$. Dans cet alphabet, on a $\Theta_r=\{\theta|\theta\in\Theta\wedge\exists p,q \in Q, c\in C, a\in\Sigma,\text{tels que } \delta(\theta)=(p,c?a,q)\}$, similaire à $\bar{\Theta}$ mais avec un nom pour chaque transition pour les actions de réception.


%  █████  ███    ██ ███    ██
% ██   ██ ████   ██ ████   ██
% ███████ ██ ██  ██ ██ ██  ██
% ██   ██ ██  ██ ██ ██  ██ ██
% ██   ██ ██   ████ ██   ████

\subsection{Trace annotée}\label{trace:annot}

Soit $\mathcal{A}:L(F)\rightarrow\Phi^*$ une fonction associant une trace annotée $\gamma\in\Phi^*$ à une trace d'automate à file $\sigma\in\Theta^*$. Cette \emph{trace annotée} est un mot sur l'alphabet d'annotation $\Phi$ tel qu'il représente au moins la trace $\sigma$. Il y a une perte d'information lors de cette transformation. $\mathcal{A}$ est la \emph{fonction d'annotation}.

Intuitivement, $\mathcal{A}$ compacte une trace en remplaçant tout $\theta$ correspondant à une action d'envoi par son équivalent $\bar{\theta}$ si le symbole envoyé est consommé plus loin dans la trace, sur le même canal. Si une trace valide est annotée, il ne peut y avoir d'action de réception orpheline : le symbole a dû être envoyé avant d'être consommé. Dès lors, la trace annotée ne comporte que des symboles correspondant à des actions d'envoi, avec une barre ou non. De plus, un symbole appartenant à $T_Q$ est ajouté à la fin pour signifier l'état de contrôle atteint en suivant le chemin défini par la trace.

Cette fonction est reprise algorithmiquement en l'algorithme \ref{alg:A}.

\begin{algorithm}[H]
  	\begin{algorithmic}[1]
    \REQUIRE un automate à files \fifo , une trace $\sigma\in L(F)$
		\ENSURE une trace annotée $\gamma\in\Phi^*$ représentant $\sigma$

    \STATE $\gamma\leftarrow\epsilon$
    \FORALL {nom de transition $\theta\in\sigma$}
      \IF {$\delta(\theta)$ est une action de réception}
        \STATE trouver $\theta_s\in\Theta$ correspondant à une action d'envoi antécédant dans $\sigma$ telle que les actions s'appliquent sur le même canal et le même symbole
        \STATE $\gamma\leftarrow$ $\gamma$ où $\theta_s$ est remplacé par $\bar{\theta_s}\in\bar{\Theta}$ \COMMENT {$\theta$ n'est pas ajouté à $\gamma$}
      \ELSIF {$\delta(\theta)$ est une action d'envoi}
        \STATE $\gamma\leftarrow\gamma\theta$
      \ENDIF
    \ENDFOR
    \STATE trouver $q$ l'état de contrôle tel que $s_O\xrightarrow{\sigma}s=(q,w)$ pour un certain $w\in(\Sigma^*)^c$
    \STATE $\gamma\leftarrow\gamma t_q$ avec $t_q\in T_Q$ le symbole de contrôle associé à $q$
		\RETURN $\gamma$
	\end{algorithmic}
	\caption{$\mathcal{A}:\Theta^*\rightarrow\Phi^*$}\label{alg:A}
\end{algorithm}

Soit $AL(F)=\{\mathcal{A}(\sigma)|\sigma \in L(F)\}$ le \emph{langage de traces annotées} de l'automate $F$. $AL(F)$ est un ensemble de traces annotées correspondant à des traces valides pour l'automate $F$. Intuitivement, $AL(F)$ contient l'ensemble des états atteignables par $F$ ainsi que les traces annotées servant de témoins de cette atteignabilité.

Soit un mot $\gamma \in \Phi^*$. $\gamma$ est \emph{correctement formaté} si il finit par un symbole de $T_Q$ et aucune autre occurence d'un symbole de $T_Q$ n'apparaît dans le mot. Soit un langage arbitraire $L$. $L$ est \emph{correctement formaté} si tous les mots $\gamma\in L$ le sont.


\begin{example}
Soit l'automate $F$ représenté par la figure \ref{fig:fifo1}. Soient les traces $\sigma_1=\theta_2\theta_8$ et $\sigma_2=\theta_1\theta_3\theta_5\theta_2$. Alors, les traces annotées de ces traces sont : $\mathcal{A}(\sigma_1)=\theta_2\theta_8t_{(q_1,q_B)}=\gamma_1$ et $\mathcal{A}(\sigma_2)=\bar{\theta_1}\bar{\theta_5}t_{(q_0,q_B)}=\gamma_2$.
Bien qu'elles soient toutes deux correctement formatées, $\gamma_1$ ne correspond à aucune exécution valide de $F$. Dès lors, $\gamma_1$ n'appartient pas au langage de traces annotées de $AL(F)$ contrairement à $\gamma_2$.

Soit la trace annotée $\gamma_3=t_{q_0,q_A}\theta_2 t_{q_0,q_B} \in \Phi^*$. $\gamma_3$ n'est pas correctement formatée : il est impossible que cette trace annotée appartienne à $AL(F)$.
\end{example}


% ███████ ██   ██ ████████
% ██       ██ ██     ██
% █████     ███      ██
% ██       ██ ██     ██
% ███████ ██   ██    ██

\subsection{Fonction d'extension de trace}\label{trace:extension}

Cette section définit \fl pour un langage arbitraire $L$ et démontre que $AL(F)$ en est un point fixe minimum. De la sorte, tout langage qui n'est pas un point fixe minimum de \fl ne peut pas être $AL(F)$. Si c'est le cas, la question d'équivalence est répondue : les langages ne sont pas égaux. Il reste alors à générer un contre-exemple.

La \emph{fonction d'extension} $Post(L)$ permet d'étendre une trace annotée $\gamma$ avec le symbole $\theta$. Si $\gamma$ est correctement formatée, $source(\theta)$ et $cible(\theta)$ donnent respectivement la source et la cible d'une transition $\delta$.

$$
Post(\gamma,\theta) = \left\{ \begin{array}{ll}
    \emptyset & \text{si } \gamma \text{ n'est pas correctement formaté ou si } \mathcal{C}(\gamma)\neq source(\theta)\\
    \{\mathcal{T}(\gamma)t_{cible(\theta)}\} & \text{sinon si }\delta(\theta)=(p,\tau,q) \text{ ou } \delta(\theta)=(p,c_i!a_j,q) \text{ avec }p,q\in Q\\
    \{deriv(\mathcal{T}(\gamma),\theta)t_{cible(\theta)}\}& \text{sinon si } \delta(\theta)=(p,c_i?a_j,q) \text{ avec }p,q\in Q \\
    \end{array} \right.
$$

Sachant que $deriv(\mathcal{T}(\gamma),\theta)$ fonctionne comme l'algorithme $\mathcal{A}$ si $\theta$ est une action de réception. Elle le fait en remplaçant un $\theta_e \in \Theta$ associé à une action d'envoi et le remplace par $\bar{\theta_e} \in \bar{\Theta}$ si l'action porte sur le même canal et le même symbole que $\theta$.

Posons $Post(\gamma)=\bigcup_{\theta\in\Theta}Post(\gamma,\theta)$ et $Post(L)=\bigcup_{\gamma\in L}Post(\gamma)$.


\begin{theorem}\label{thm:fl}
  Soit \fl$=Post(L)\cup\{t_{q_0}\}$ où $q_0$ est l'état de contrôle initial. \fl est une opération monotone sur les ensembles c'est-à-dire qu'elle préserve l'inclusion d'ensembles. De plus, $AL(F)$ est le plus petit point fixe de \fl.
\end{theorem}

La preuve de ce théorème est disponible en annexe de \cite{Vardhan04}. Un algorithme détaillant comment calculer $\mathcal{F}(L)$ est détaillé dans en section \ref{sec:fl}.
