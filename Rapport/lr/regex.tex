
Certains langages peuvent être exprimés par une \emph{expression régulière}.

Une expression régulière est un mot utilisant les symboles à représenter ainsi que les symboles (,),*,| qui sont réservés pour différentes opérations. Une expression régulière est construite à partir d'élements atomiques (les symboles du langage à représenter) et assemblés pour obtenir des langages plus complexes. Un language qui peut-être représenté par une expression régulière est dit \emph{langage régulier}.

Si plusieurs expressions régulières peuvent être composées en une expression plus complexe, une expression régulière peut aussi être décomposée en ses différents composants.

\textbf{Cas de base}
Certains langages peuvent être construits directement sans passer par l'induction:

\begin{itemize}
	\item $\epsilon$ est une expression régulière. Elle décrit le langage $L(\epsilon)=\{\epsilon\}$
	\item $\emptyset$ est une expression régulière décrivant $L(\emptyset)=\emptyset$
	\item Si $a$ est un symbole, alors $a$ est une expression régulière décrivant le langage $L(a) = \{a\}$.
\end{itemize}


\textbf{Induction}
Les autres langages réguliers sont construits suivant différentes règles d'induction présentées par ordre décroissant de priorité :

\begin{itemize}
	\item Si $E$ est une expression régulière, alors $(E)$ est une expression régulière et $L((E)) = L(E)$.
	\item Si $E$ est une expression régulière, alors $E^*$ est une expression régulière représentant la fermeture de $L(E)$, à savoir $L(E^*) = L(E)^*$.
	\item Si $E$ et $F$ sont des expressions régulières, alors $EF$ est une expression régulière décrivant la concaténation des deux langages représentés, à savoir $L(EF)=L(E)L(F)$.
	\item Si $E$ et $F$ sont des expressions régulières, alors $E+F$ est une expression régulière donnant l'union des deux langages représentés, à savoir $L(E+F)=L(E)\cup L(F)$. Ici encore, l'opération est associative et la priorité est à gauche.
\end{itemize}

\begin{example}[Expressions régulières]
	Soit l'expression $E = (b+ab)b^*a(a+b)^*$ qui décrit le langage $L=L(E)$.\\
	\begin{itemize}
		\item Le mot $ba$ fait partie de $L$. En effet, $ba=b\epsilon a \epsilon=(b)b^0a(a+b)^0$, ce qui respecte bien la définition de $E$.
		\item Le mot $ababbab$ fait partie de $L$. A nouveau, $ababbab=ab\epsilon a (a+b)(a+b)(a+b)(a+b)=(ab)b^0a(a+b)^4$.
		\item Le mot $aa$ ne fait \textbf{pas} partie de $L$. Supposons par l'absurde que $aa \in L$. Alors il existerait une façon de décomposer $E$ en $aa$. Or, les premiers symboles doivent être soit $b$, soit $ab$. Il y a contradiction. Donc, $aa \notin L$.
	\end{itemize}
	\label{ex:regex}
\end{example}

Un autre exemple d'expression régulière est $E=01^*0$. $E$ décrit le langage $L(E)$ constitué de tous les mots commençant et finissant par $0$ avec uniquement des $1$ entre les deux.
