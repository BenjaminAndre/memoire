\begin{theorem}
  Soit une fonction monotone $F$ portant sur des ensembles et $P$ l'ensemble de points préfixes de $F$. Alors $\mu=\bigcap_{p\in P}p$ est le point fixe minimal de $F$.
\end{theorem}

Il s'agit en réalité d'une reformulation adaptée aux ensembles et outils utilisés.

\begin{proof}
  Soit une fonction monotone $F$ portant sur des ensembles et $P$ l'ensemble de points préfixes de $F$.

  Posons $\mu=\bigcap_{p\in P}p$.

  Par définition, $\forall p\in P,\mu\subseteq p$. Comme la fonction $F$ est monotone, elle peut être appliquée aux deux ensembles : $\forall p\in P,F(\mu)\subseteq F(p)$.

  Comme $F(p)\subseteq p$ par définition, on obtient par transitivité de l'inclusion que $\forall p\in P,F(\mu)\subseteq p$.
  Si $F(\mu)$ est inclus à tout ensemble $p$, c'est qu'aucun élément de $F(\mu)$ ne fait pas partie d'un $p$ en particulier : cela revient à la définition de l'intersection. Donc, $F(\mu)\subseteq\bigcap_{p\in P}=\mu$. Cela prouve que $\mu$ est un point préfixe ($F(\mu)\subseteq\mu$). Comme $F$ est monotone, $F(F(\mu))\subseteq F(\mu)$ et $F(\mu)$ est également un point préfixe.

  Or, comme $\mu\subseteq p$ est vrai pour tout $p$ et en particulier pour $p=F(\mu)$ puisque $F(\mu)$ est un point préfixe et appartient alors à $P$, on obtient $\mu\subseteq(F(\mu))$. Dès lors, $\mu=F(\mu)$; $\mu$ est le point fixe minimal (la minimalité venant du fait que $\mu$ est déjà un point préfixe minimal).

  \hfill$\square$
\end{proof}
