Soit un langage $L$ et un automate $A$ tel que $L=L(A)$. Calculer $\mathcal{F}(L)$ revient, par définition, à calculer $\bigcup_{\theta\in\Theta}Post(L,\theta)\bigcup \{t_{q_0}\}$.


\section{Construction par union}

Des automates équivalents correspondent à des langages équivalents (ref?). Si des automates $\{A', A", ...\}$ peuvent êtres construits pour $Post(L,\theta)$ pour chaque $\theta$ à partir de $A$, l'union de ces automates $A'\bigcup A"\bigcup\dots$ donne $\bigcup_{\theta\in\Theta}Post(L,\theta)$. L'union de ce nouvel automate avec un automate pour le language $L=\{t_{q_0}\}$ donne $\mathcal{F}(L)$.

La suite de cette section s'intéresse au calcul de $L'= Post(L,\theta)$ où $L$ est donné par $A$ et $L'$ par un ADF $A'$ construit à partir de $A$.\\ 



Soit $\theta=\in\Theta$ tel que $\delta(\theta)=(p,action,q)$.

\section{Si $action$ est $\tau$ ou de forme $c!a$}

Dans ce cas, construire un automate pour $L'=L\bigcap\Sigma^*t_p$ avec $t_p$ correspondant au $p$ de $\delta(\theta)$ et $\Sigma^*t_p$ étant représenté par l'automate de la figure \ref{fig:sigmatq}.

\begin{figure}[H]
    \centering
    \begin{tikzpicture}[->,>=stealth',shorten >=1pt,auto,node distance=3cm, semithick, bend angle=10]
    \tikzstyle{every state}=[circle,scale=0.5]
    \node[state] (0) {};
    \node[state,accepting] (1)   [right of=0] {};

    \path
    (0) edge[loop above] node {$\Sigma$} (0)
    (0) edge node {$t_p$} (1)
    ;
    \end{tikzpicture}
    \caption{Automate pour $L=\Sigma^*t_p$}\label{fig:sigmatq}
\end{figure}

Dans cet automate $L'$, seuls les états et transitions menant à l'état final par $t_q$ sont considérés.

Construire un automate pour $L\bigcap\Sigma^*t_p=L'$. Si $L'$ est vide, c'est qu'il n'y a pas de mot finissant concerné par $\theta$ et donc que $Post(L,\theta)=\emptyset$.

Dans l'autre cas, il faut appliquer une transformation à l'automate correspondant à $Post(L',\theta)$. Comme $action$ est de forme $\tau$ ou $c!a$, la transition sur $\theta$ peut être ajoutée avant les transitions sur $t_p$ et un nouveau $t_q$ signalant le changement d'état dans l'automate à files $F$.

\begin{figure}[H]
    \centering
    \begin{subfigure}{0.5\linewidth}
        \centering
        \begin{tikzpicture}[->,>=stealth',shorten >=1pt,auto,node distance=3cm, semithick, bend angle=10]
            \tikzstyle{every state}=[circle,scale=0.5]
            \node[state] (0) {};
            \node[state,accepting] (1)   [right of=0] {};
            \node[ellipse] [left of=0] {};
            \path
            (0) edge node {$t_p$} (1)
            ;
            \draw (-1,0) ellipse[x radius=1.5,y radius=1];
        \end{tikzpicture}
        \caption{Automate de $L'$}
    \end{subfigure}\hfill
    \begin{subfigure}{0.5\linewidth}
        \centering
        \begin{tikzpicture}[->,>=stealth',shorten >=1pt,auto,node distance=3cm, semithick, bend angle=10]
            \tikzstyle{every state}=[circle,scale=0.5]
            \node[state] (0) {};
            \node[state] (1) [right of=0] {};
            \node[state,accepting] (2)   [right of=1] {};

            \path
            (0) edge node {$\theta$} (1)
            (1) edge node {$t_q$} (2)
            ;
            \draw (-1,0) ellipse[x radius=1.5,y radius=1];
        \end{tikzpicture}
        \caption{Automate de $Post(L',\theta)$}
    \end{subfigure}
    \caption{Application de $Post(L',\theta)$ sur un automate}
\end{figure}





\section{Si $action$ est de forme $c?a$}

Dans ce cas, comme dans la section précédente, construire $L'$ pour ne pas travailler si $Post(L,\theta)=\emptyset$ car il n'y a pas d'état finissant sur $p$ dans $F$.

De plus, il doit exister un envoi pour permettre une réception. Cet envoi est noté $\theta_s=(c!a)$.

Grâce à ce $\theta_s$, construire $L''=L'\bigcap\Sigma^*\theta_s\Sigma^*t_p$. Si $L''=\emptyset$, il n'existe pas de $\theta_s$ dans les chemins qui nous concerne. Il n'y aura donc pas de possibilité de consommer un symbole : $Post(L,\theta)=\emptyset$. 


\begin{figure}[H]
    \centering
    \begin{tikzpicture}[->,>=stealth',shorten >=1pt,auto,node distance=3cm, semithick, bend angle=10]
        \tikzstyle{every state}=[circle,scale=0.5]
        \node[state] (0) {};
        \node[state] (1) [right of=0] {};
        \node[state,accepting] (2)   [right of=1] {};

        \path
        (0) edge[loop above] node {$\Sigma$} (0)
        (0) edge node {$\theta_s$} (1)
        (1) edge[loop above] node{$\Sigma$} (1)
        (1) edge node {$t_p$} (2)
        ;
    \end{tikzpicture}
    \caption{Automate pour $L=\Sigma^*\theta_s\Sigma^*t_p$}\label{fig:lseconde}
\end{figure}


Si $L''$ est non vide, il faut alors, à partir de l'automate $A$ de $L''$, construire un automate $P_A$ dont le langage représenté est $Post(L'',\theta)$.

\begin{figure}[H]
    \centering
    \begin{tikzpicture}[->,>=stealth',shorten >=1pt,auto,node distance=3cm, semithick, bend angle=10]
        \tikzstyle{every state}=[circle,scale=0.5]
        \node[state] (0) {$r$};
        \node[state] (1) [right of=0] {$s$};

        \node[state] (2) [right=4.5cm of 0] {$r'$};
        \node[state] (3) [right of=2] {$s'$};

        \path
        (0) edge[bend right=30] node {$\bt_s$} (3)
        (2) edge node {$\theta_s$} (3)
        ;
        \path [dashed]
        (0) edge node {$\theta_s$} (1)
        ;
        \draw (-1,-1.5) rectangle ++(3,3);
        \draw (4,-1.5) rectangle ++(3,3);

        \node at(0.5,-2) {$A$};
        \node at(5.5,-2) {$A'$};

    \end{tikzpicture}
    \caption{Première étape de construction de l'automate $P_A$ pour $Post(L'',\theta)$}\label{fig:aaprime}
\end{figure}

La figure \ref{fig:aaprime} décrit comment l'automate $P_A$ est construit à partir de $A$. Premièrement, une copie de $A$ est construite : $A'$.
Dans $A$, les états finaux sont considérés comme des états normaux. Dans $A'$, l'état initial est considéré comme un état normal. Finalement, dans $A$, la toute transition $\theta_s$ allant de $r$ à $s$ est remplacée par une transition sur $\bt_s$ allant de $r$ à $s'$. Même si $r=s$, il n'existe qu'une transition sortante et l'automate résultant reste déterministe.

De cette façon, pour qu'un mot soit valide, il faut qu'il commence dans $A$. La première occurence de $\theta_s$ a été systématiquement remplacée pour $\bt_s$ menant à $A'$. Dans $A'$, les autres transitions ne sont pas modifié, laissant intact le reste du chemin. Cependant, la figure \ref{fig:aaprime} a elle seule ne suffit pas à construire $P_A$. Pour rester en accord avec l'automate à file $F$ sous-jacent, il faut adapter le dernier symbole, ici $t_p$ en $t_q$ (figure \ref{fig:pq}). Cela permet de respecter le fait qu'une transition menant de $p$ à $q$ a été suivie dans $F$.


\begin{figure}[H]
    \centering
    \begin{subfigure}{0.5\linewidth}
        \centering
        \begin{tikzpicture}[->,>=stealth',shorten >=1pt,auto,node distance=3cm, semithick, bend angle=10]
            \tikzstyle{every state}=[circle,scale=0.5]
            \node[state] (0) {};
            \node[state,accepting] (1)   [right of=0] {};
            \path
            (0) edge node {$t_p$} (1)
            ;
            \draw (-2,-1) rectangle ++(4,2);
        \end{tikzpicture}
        \caption{Automate de obtenu par \ref{fig:aaprime}}
    \end{subfigure}\hfill
    \begin{subfigure}{0.5\linewidth}
        \centering
        \begin{tikzpicture}[->,>=stealth',shorten >=1pt,auto,node distance=3cm, semithick, bend angle=10]
            \tikzstyle{every state}=[circle,scale=0.5]
            \node[state] (0) {};
            \node[state,accepting] (1)   [right of=0] {};

            \path
            (0) edge node {$t_q$} (1)
            ;
            \draw (-2,-1) rectangle ++(4,2);
        \end{tikzpicture}
        \caption{Automate $P_A$ de $Post(L'',\theta)$}
    \end{subfigure}
    \caption{Application de $Post(L'',\theta)$ sur un automate}
\end{figure}


































$\mathcal{F}(L)$ est l'union de la transition $t_{q0}$ et des automates $Post(l, \theta)$ pour chaque $\theta\in\Theta$.
Comme une union d'automate correspond à une union de langage, il est possible de construire un automate par $\theta$.

\section{Méthode de construction}

La construction se fait pour chaque $\theta$. De plus, comme chaque état $q$ correspond à une classe de mots $[[q]]$, il est possible de s'intéresser à tout mot $l\in L$ en s'intéressant à chaque état $q$.

La question devient alors : "Pour un état $q$ et un symbole $\theta$, comment modifier l'automate pour qu'il corresponde à l'application de la fonction Post ?".

Cela dépend du type d'action de $\theta$.
Il est important de garder à l'esprit que par définition, les mots définis par $AL(F)$ finissent par une transition sur un élément de $T_Q$. 
L'état initial de ces transition est appellé le avant-derniers. 
Il y a donc à chaque fois un état précédant l'état final qui n'a pas de transition sortante.
Si un automate ne respecte pas ces règles, il ne peut pas être AL(F).

\paragraph{$\theta=(p, \tau, q)$}

Soient les avant-derniers états. Ceux qui ne mènent pas à $p$ n'ont plus de transition finale (Donnant $\emptyset$ comme ensemble de mot acceptés, en accord avec la définition).

Ceux menant à $p$ se voient modifiés pour pointer sur $q$ à la place, suivis d'une transition $t_q$ à la place.


\begin{figure}[H]
    \centering
    \begin{tikzpicture}[->,>=stealth',shorten >=1pt,auto,node distance=3cm, semithick, bend angle=10]
    \tikzstyle{every state}=[circle]
    \node[state] (0) {$0$};
    \node[state]   (1)   [right of=0]                {$1$};
    \node[state, accepting] (2) [right of=1]      {$F$};

    \path
    (0) edge (1)
    (1) edge node {$t_p$} (2)
    ;
    \end{tikzpicture}
\end{figure}

Devient :

\begin{figure}[H]
    \centering
    \begin{tikzpicture}[->,>=stealth',shorten >=1pt,auto,node distance=3cm, semithick, bend angle=10]
    \tikzstyle{every state}=[circle]
    \node[state] (0) {$0$};
    \node[state]   (1)   [right of=0]                {$1'$};
    \node[state, accepting] (2) [right of=1]      {$F$};

    \path
    (0) edge (1)
    (1) edge node {$t_q$} (2)
    ;
    \end{tikzpicture}
\end{figure}


\paragraph{$\theta=(p,c!a,q)$}

De la même façon que pour $\tau$, seuls les états correspondant à $p$ permettent de créer un automate aidant à construire $\mathcal{F}(L)$

Cependant, le mot s'est en fait allongé. Il faut donc une transition en plus avant d'arriver à l'état final.


\begin{figure}[H]
    \centering
    \begin{tikzpicture}[->,>=stealth',shorten >=1pt,auto,node distance=3cm, semithick, bend angle=10]
    \tikzstyle{every state}=[circle]
    \node[state] (0) {$0$};
    \node[state]   (1)   [right of=0]                {$1$};
    \node[state, accepting] (2) [right of=1]      {$F$};

    \path
    (0) edge (1)
    (1) edge node {$t_p$} (2)
    ;
    \end{tikzpicture}
\end{figure}

Devient :

\begin{figure}[H]
    \centering
    \begin{tikzpicture}[->,>=stealth',shorten >=1pt,auto,node distance=3cm, semithick, bend angle=10]
    \tikzstyle{every state}=[circle]
    \node[state] (0) {$0$};
    \node[state]   (1)   [right of=0]                {$1$};
    \node[state]   (2)   [right of=1]                {$2$};
    \node[state, accepting] (3) [right of=2]      {$F$};

    \path
    (0) edge (1)
    (1) edge node {$\theta$} (2)
    (2) edge node {$t_q$} (3)
    ;
    \end{tikzpicture}
\end{figure}

\paragraph{$\theta=(p,c?a,q)$}

Encore une fois, les états avant-derniers ne correspondant pas à $p$ ne participent pas à la création de l'automate de $\mathcal{F}(L)$.

Cependant, celui-ci est plus compliqué : pour chaque mot, il faut barrer la première occurence de theta non barrée portant sur $c$ et $a$.

Parmis l'ensemble des transitions de L (de taille $n$), il en est un sous-ensemble qui porte sur $c$ et $a$. Une telle transition est notée $\theta_{ca}$.

A partir de l'état avant-dernier considéré, il est possible de recréer un arbre des différentes éxécutions y ayant mené (depuis l'état d'origine).

Heureusement, comme le $\theta_{ca}$ a trouvé est le premier, les boucles peuvent être ignorées, le premier passage étant suffisant.

C'est une forme d'exploration sans répétition, notant à chaque fois si le noeud a déjà été exploré pour éviter les répétions.

Une fois ces chemins d'éxécutions trouvés, ceux contenant un $\theta_{ca}$ sont conservés. Le premier $\theta_{ca}$ de ces chemins devient alors $\bar{\theta_{ca}}$.


\begin{figure}[H]
    \centering
    \begin{tikzpicture}[->,>=stealth',shorten >=1pt,auto,node distance=2cm, semithick, bend angle=10]
    \tikzstyle{every state}=[circle]
    \node[initial, state] (0) {$0$};
    \node[state] (1)[right of=0] {$...$};
    \node[state] (2)[right of=1] {$1$};
    \node[state] (3)[right of=2] {$2$};
    \node[state] (4)[right of=3] {$...$};
    \node[state] (5)[right of=4]{$n$};
    \node[state, accepting] (6) [right of=5]      {$F$};

    \path
    (0) edge (1)
    (1) edge (2)
    (2) edge node {$\theta_{ca}$} (3)
    (3) edge (4)
    (4) edge (5)
    (5) edge node {$t_p$} (6)
    ;
    \end{tikzpicture}
\end{figure}

Devient :

\begin{figure}[H]
    \centering
    \begin{tikzpicture}[->,>=stealth',shorten >=1pt,auto,node distance=2cm, semithick, bend angle=10]
    \tikzstyle{every state}=[circle]
    \node[initial, state] (0) {$0$};
    \node[state] (1)[right of=0] {$...$};
    \node[state] (2)[right of=1] {$1$};
    \node[state] (3)[right of=2] {$2$};
    \node[state] (4)[right of=3] {$...$};
    \node[state] (5)[right of=4]{$n'$};
    \node[state, accepting] (6) [right of=5]      {$F$};

    \path
    (0) edge (1)
    (1) edge (2)
    (2) edge node {$\bar{\theta_{ca}}$} (3)
    (3) edge (4)
    (4) edge (5)
    (5) edge node {$t_q$} (6)
    ;
    \end{tikzpicture}
\end{figure}

\section{Exemple}
\paragraph{$F$ et $L(F)$ candidat à $AL(F)$}
\begin{figure}[H]
    \centering
    \begin{tikzpicture}[->,>=stealth',shorten >=1pt,auto,node distance=3cm, semithick, bend angle=10]
   
    \tikzstyle{every state}=[circle]
   

    \node[initial, state]   (0)                         {$q_0$};
    \node[state]            (1) [above right of=0]      {$q_1$};
    \node[state]            (2) [below right of=0]      {$q_2$};
    \node[state]            (3) [below right of=1]      {$q_3$};

    \path
    (0) edge node {$\theta_1(a!0)$} (1)    
    (0) edge node [below left] {$\theta_2(a!1)$} (2)    
    (1) edge node {$\theta_3(a?0)$} (3)    
    (2) edge node [below right] {$\theta_4(a?1)$} (3)    
    (3) edge node {$\theta_5(\tau)$} (0)    
    ;
    \end{tikzpicture}
    \caption{Automate à Files $F$}\label{fig:fex}
\end{figure}

Regex pour AL(F) : $(\bar{\theta_1}|\bar{\theta_2})^*(\theta_1|\theta_2)?$

\begin{figure}[H]
    \centering
    \begin{tikzpicture}[->,>=stealth',shorten >=1pt,auto,node distance=3cm, semithick, bend angle=10]
    \tikzstyle{every state}=[circle]

    \node[initial, state]   (0)                         {$0$};
    \node[state]            (1) [above right of=0]      {$1$};
    \node[state]            (2) [below right of=0]      {$2$};
    \node[state, accepting] (3) [below right of=1]      {$F$};

    \path
    (0) edge [loop above] node {$\bar{\theta_1}$} (0)
    (0) edge [loop below] node {$\bar{\theta_2}$} (0)
    (0) edge node {$\theta_1$} (1)
    (0) edge node {$\theta_2$} (2)
    (0) edge node {$t_{q_0},t_{q_3}$} (3)
    (1) edge node {$t_{q_1}$} (3)
    (2) edge node {$t_{q_2}$} (3)
    ;
    \end{tikzpicture}
    \caption{Candidat L(F) pour AL(F)}\label{fig:lfex}
\end{figure}
\paragraph{$t_{q0}$} N'appartient pas à l'ensemble $\Theta$ mais fait partie de la définition de $\mathcal{F}(L)$.

\begin{figure}[H]
    \centering
    \begin{tikzpicture}[->,>=stealth',shorten >=1pt,auto,node distance=3cm, semithick, bend angle=10]
    \tikzstyle{every state}=[circle]
    \node[initial, state]   (0)                   {$0$};
    \node[state, accepting] (1) [right of=0]      {$F$};

    \path
    (0) edge node {$t_{q0}$} (1)
    ;
    \end{tikzpicture}
\end{figure}

\paragraph{$\theta_1=(q_0,a!0,q_1)$}

États avant-derniers sur $q_0$ : ${0}$.

\begin{figure}[H]
    \centering
    \begin{tikzpicture}[->,>=stealth',shorten >=1pt,auto,node distance=3cm, semithick, bend angle=10]
    \tikzstyle{every state}=[circle]

    \node[initial, state]   (0)                   {$0$};
    \node[state]            (1) [right of=0]      {$X$};
    \node[state, accepting] (2) [right of=1]      {$F$};

    \path
    (0) edge [loop above] node {$\bar{\theta_1}$} (0)
    (0) edge [loop below] node {$\bar{\theta_2}$} (0)
    (0) edge node {$\theta_1$} (1)
    (1) edge node {$t_{q_1}$} (2)
    ;
    \end{tikzpicture}
\end{figure}

\paragraph{$\theta_2=(q_0,a!1,q_2)$}

États avant-derniers sur $q_0$ : ${0}$.

\begin{figure}[H]
    \centering
    \begin{tikzpicture}[->,>=stealth',shorten >=1pt,auto,node distance=3cm, semithick, bend angle=10]
    \tikzstyle{every state}=[circle]

    \node[initial, state]   (0)                   {$0$};
    \node[state]            (1) [right of=0]      {$X$};
    \node[state, accepting] (2) [right of=1]      {$F$};

    \path
    (0) edge [loop above] node {$\bar{\theta_1}$} (0)
    (0) edge [loop below] node {$\bar{\theta_2}$} (0)
    (0) edge node {$\theta_2$} (1)
    (1) edge node {$t_{q_2}$} (2)
    ;
    \end{tikzpicture}
\end{figure}



\paragraph{$\theta_3=(q_1,a?0,q_3)$}

États avant-derniers sur $q_1$ : ${1}$.

\begin{figure}[H]
    \centering
    \begin{tikzpicture}[->,>=stealth',shorten >=1pt,auto,node distance=3cm, semithick, bend angle=10]
    \tikzstyle{every state}=[circle]

    \node[initial, state]   (0)                         {$0$};
    \node[state]            (1) [above right of=0]      {$1'$};
    \node[state, accepting] (3) [below right of=1]      {$F$};

    \path
    (0) edge [loop above] node {$\bar{\theta_1}$} (0)
    (0) edge [loop below] node {$\bar{\theta_2}$} (0)
    (0) edge node {$\bar{\theta_1}$} (1)
    (1) edge node {$t_{q_3}$} (3)
    ;
    \end{tikzpicture}
\end{figure}

\paragraph{$\theta_4={(q_2,a?1,q_3)}$}

États avant-derniers sur $q_2$ : ${2}$.

\begin{figure}[H]
    \centering
    \begin{tikzpicture}[->,>=stealth',shorten >=1pt,auto,node distance=3cm, semithick, bend angle=10]
    \tikzstyle{every state}=[circle]

    \node[initial, state]   (0)                         {$0$};
    \node[state]            (2) [below right of=0]      {$2'$};
    \node[state, accepting] (3) [below right of=1]      {$F$};

    \path
    (0) edge [loop above] node {$\bar{\theta_1}$} (0)
    (0) edge [loop below] node {$\bar{\theta_2}$} (0)
    (0) edge node {$\bar{\theta_2}$} (2)
    (2) edge node {$t_{q_3}$} (3)
    ;
    \end{tikzpicture}
\end{figure}

\paragraph{$\theta_5=(q_3,\tau,q_0)$}

États avant-derniers sur $q_3$ : $\emptyset$.


\paragraph{Résultat}

\begin{figure}[H]
    \centering
    \begin{tikzpicture}[->,>=stealth',shorten >=1pt,auto,node distance=3cm, semithick, bend angle=10]
    \tikzstyle{every state}=[circle]

    \node[initial, state]   (0)                         {$0$};
    \node[state]            (1) [above right of=0]      {$1$};
    \node[state]            (2) [below right of=0]      {$2$};
    \node[state, accepting] (3) [below right of=1]      {$F$};

    \path
    (0) edge [loop above] node {$\bar{\theta_1}$} (0)
    (0) edge [loop below] node {$\bar{\theta_2}$} (0)
    (0) edge node {$\theta_1$} (1)
    (0) edge node {$\theta_2$} (2)
    (0) edge node {$t_{q_0},t_{q_3}$} (3)
    (1) edge node {$t_{q_1}$} (3)
    (2) edge node {$t_{q_2}$} (3)
    ;
    \end{tikzpicture}
    \caption{Auomtate pour $\mathcal{F}(L)=L$}
\end{figure}

L'automate obtenu par la concaténation de ces sous-automates est équivalent à $L(F)$. On a bien un candidat pour $AL(F)$.

