Soit un langage $L$ et un automate $A$ tel que $L=L(A)$. Calculer $\mathcal{F}(L)$ revient, par définition, à calculer $\bigcup_{\theta\in\Theta}Post(L,\theta)\bigcup \{t_{q_0}\}$.


\section{Construction par union}

Des automates équivalents correspondent à des langages équivalents \todo{ref?}. Si des automates $\{A', A", ...\}$ peuvent êtres construits pour $Post(L,\theta)$ pour chaque $\theta$ à partir de $A$, l'union de ces automates $A'\bigcup A"\bigcup\dots$ donne $\bigcup_{\theta\in\Theta}Post(L,\theta)$. L'union de ce nouvel automate avec un automate pour le language $L=\{t_{q_0}\}$ donne $\mathcal{F}(L)$.

La suite de cette section s'intéresse au calcul de $L'= Post(L,\theta)$ où $L$ est donné par $A$ et $L'$ par un ADF $A'$ construit à partir de $A$.\\



Soit $\theta=\in\Theta$ tel que $\delta(\theta)=(p,action,q)$.

\section{Si $action$ est $\tau$ ou de forme $c!a$}

Dans ce cas, construire un automate pour $L'=L\bigcap\Sigma^*t_p$ avec $t_p$ correspondant au $p$ de $\delta(\theta)$ et $\Sigma^*t_p$ étant représenté par l'ANF de la figure \ref{fig:sigmatq}. Un automate déterministe éuivalent existe, comme prouvé dans le chapitre \ref{adf:anfadf}.

\begin{figure}[H]
    \centering
    \begin{tikzpicture}[->,>=stealth',shorten >=1pt,auto,node distance=3cm, semithick, bend angle=10]
    \tikzstyle{every state}=[circle,scale=0.5]
    \node[state] (0) {};
    \node[state,accepting] (1)   [right of=0] {};

    \path
    (0) edge[loop above] node {$\Sigma$} (0)
    (0) edge node {$t_p$} (1)
    ;
    \end{tikzpicture}
    \caption{ANF pour $L=\Sigma^*t_p$}\label{fig:sigmatq}
\end{figure}


Dans cet automate $L'$, seuls les états et transitions menant à l'état final par $t_q$ sont considérés.

Construire un automate pour $L\bigcap\Sigma^*t_p=L'$. Si $L'$ est vide, c'est qu'il n'y a pas de mot finissant concerné par $\theta$ et donc que $Post(L,\theta)=\emptyset$.

Dans l'autre cas, il faut appliquer une transformation à l'automate correspondant à $Post(L',\theta)$.
Si $action$ est de forme $c!a$, la transition sur $\theta$ peut être ajoutée avant les transitions sur $t_p$ et un nouveau $t_q$ signalant le changement d'état dans l'automate à files $F$.

\begin{figure}[H]
    \centering
    \begin{subfigure}{0.5\linewidth}
        \centering
        \begin{tikzpicture}[->,>=stealth',shorten >=1pt,auto,node distance=3cm, semithick, bend angle=10]
            \tikzstyle{every state}=[circle,scale=0.5]
            \node[state] (0) {};
            \node[state,accepting] (1)   [right of=0] {};
            \node[ellipse] [left of=0] {};
            \path
            (0) edge node {$t_p$} (1)
            ;
            \draw (-1,0) ellipse[x radius=1.5,y radius=1];
        \end{tikzpicture}
        \caption{Automate de $L'$}
    \end{subfigure}\hfill
    \begin{subfigure}{0.5\linewidth}
        \centering
        \begin{tikzpicture}[->,>=stealth',shorten >=1pt,auto,node distance=3cm, semithick, bend angle=10]
            \tikzstyle{every state}=[circle,scale=0.5]
            \node[state] (0) {};
            \node[state] (1) [right of=0] {};
            \node[state,accepting] (2)   [right of=1] {};

            \path
            (0) edge node {$\theta$} (1)
            (1) edge node {$t_q$} (2)
            ;
            \draw (-1,0) ellipse[x radius=1.5,y radius=1];
        \end{tikzpicture}
        \caption{Automate de $Post(L',\theta)$}
    \end{subfigure}
    \caption{Application de $Post(L',\theta)$ sur un automate si l'action est de forme $c!a$}
\end{figure}

Cependant, si l'action est de forme $\tau$, $\theta\notin\Phi$. Ce n'est pas un problème : $t_p$ est remplacé par $t_q$. La transition est alors sous-entendue.

\begin{figure}[H]
    \centering
    \begin{subfigure}{0.5\linewidth}
        \centering
        \begin{tikzpicture}[->,>=stealth',shorten >=1pt,auto,node distance=3cm, semithick, bend angle=10]
            \tikzstyle{every state}=[circle,scale=0.5]
            \node[state] (0) {};
            \node[state,accepting] (1)   [right of=0] {};
            \node[ellipse] [left of=0] {};
            \path
            (0) edge node {$t_p$} (1)
            ;
            \draw (-1,0) ellipse[x radius=1.5,y radius=1];
        \end{tikzpicture}
        \caption{Automate de $L'$}
    \end{subfigure}\hfill
    \begin{subfigure}{0.5\linewidth}
        \centering
        \begin{tikzpicture}[->,>=stealth',shorten >=1pt,auto,node distance=3cm, semithick, bend angle=10]
            \tikzstyle{every state}=[circle,scale=0.5]
            \node[state] (0) {};
            \node[state,accepting] (1)   [right of=0] {};
            \node[ellipse] [left of=0] {};
            \path
            (0) edge node {$t_q$} (1)
            ;
            \draw (-1,0) ellipse[x radius=1.5,y radius=1];
        \end{tikzpicture}
        \caption{Automate de $Post(L',\theta)$}
    \end{subfigure}
    \caption{Application de $Post(L',\theta)$ sur un automate si l'action est $\tau$}
\end{figure}


\section{Si $action$ est de forme $c?a$}

Dans ce cas, comme dans la section précédente, construire $L'$ pour ne pas travailler si $Post(L,\theta)=\emptyset$ car il n'y a pas d'état finissant sur $p$ dans $F$.

De plus, il doit exister un envoi pour permettre une réception. Cet envoi est noté $\theta_s=(c!a)$.

Grâce à ce $\theta_s$, construire $L''=L'\bigcap\Sigma^*\theta_s\Sigma^*t_p$. Si $L''=\emptyset$, il n'existe pas de $\theta_s$ dans les chemins qui nous concerne. Il n'y aura donc pas de possibilité de consommer un symbole : $Post(L,\theta)=\emptyset$.


\begin{figure}[H]
    \centering
    \begin{tikzpicture}[->,>=stealth',shorten >=1pt,auto,node distance=3cm, semithick, bend angle=10]
        \tikzstyle{every state}=[circle,scale=0.5]
        \node[state] (0) {};
        \node[state] (1) [right of=0] {};
        \node[state,accepting] (2)   [right of=1] {};

        \path
        (0) edge[loop above] node {$\Sigma$} (0)
        (0) edge node {$\theta_s$} (1)
        (1) edge[loop above] node{$\Sigma$} (1)
        (1) edge node {$t_p$} (2)
        ;
    \end{tikzpicture}
    \caption{ANF pour $L=\Sigma^*\theta_s\Sigma^*t_p$}\label{fig:lseconde}
\end{figure}

Comme rappellé à la section précédente, il existe un ADF équivalent à \ref{fig:lseconde} comme prouvé dans la section \ref{adf:anfadf}.

Si $L''$ est non vide, il faut alors, à partir de l'automate $A$ de $L''$, construire un automate $P_A$ dont le langage représenté est $Post(L'',\theta)$.

\begin{figure}[H]
    \centering
    \begin{tikzpicture}[->,>=stealth',shorten >=1pt,auto,node distance=3cm, semithick, bend angle=10]
        \tikzstyle{every state}=[circle,scale=0.5]
        \node[state] (0) {$r$};
        \node[state] (1) [right of=0] {$s$};

        \node[state] (2) [right=4.5cm of 0] {$r'$};
        \node[state] (3) [right of=2] {$s'$};

        \path
        (0) edge[bend right=30] node {$\bt_s$} (3)
        (2) edge node {$\theta_s$} (3)
        ;
        \path [dashed]
        (0) edge node {$\theta_s$} (1)
        ;
        \draw (-1,-1.5) rectangle ++(3,3);
        \draw (4,-1.5) rectangle ++(3,3);

        \node at(0.5,-2) {$A$};
        \node at(5.5,-2) {$A'$};

    \end{tikzpicture}
    \caption{Première étape de construction de l'automate $P_A$ pour $Post(L'',\theta)$}\label{fig:aaprime}
\end{figure}

Cette construction est potentiellement indéterministe : rien n'interdit un transition $\bt_s$ dont la source est $r$ dans $A$. À nouveau, un ADF équivalent peut être calculé par \ref{adf:anfadf}.

La figure \ref{fig:aaprime} décrit comment l'automate $P_A$ est construit à partir de $A$. Premièrement, une copie de $A$ est construite : $A'$.
Dans $A$, les états finaux sont considérés comme des états normaux. Dans $A'$, l'état initial est considéré comme un état normal. Finalement, dans $A$, la toute transition $\theta_s$ allant de $r$ à $s$ est remplacée par une transition sur $\bt_s$ allant de $r$ à $s'$. Même si $r=s$, il n'existe qu'une transition sortante et l'automate résultant reste déterministe.

De cette façon, pour qu'un mot soit valide, il faut qu'il commence dans $A$. La première occurence de $\theta_s$ a été systématiquement remplacée pour $\bt_s$ menant à $A'$. Dans $A'$, les autres transitions ne sont pas modifié, laissant intact le reste du chemin. Cependant, la figure \ref{fig:aaprime} a elle seule ne suffit pas à construire $P_A$. Pour rester en accord avec l'automate à file $F$ sous-jacent, il faut adapter le dernier symbole, ici $t_p$ en $t_q$ (figure \ref{fig:pq}). Cela permet de respecter le fait qu'une transition menant de $p$ à $q$ a été suivie dans $F$.


\begin{figure}[H]
    \centering
    \begin{subfigure}{0.5\linewidth}
        \centering
        \begin{tikzpicture}[->,>=stealth',shorten >=1pt,auto,node distance=3cm, semithick, bend angle=10]
            \tikzstyle{every state}=[circle,scale=0.5]
            \node[state] (0) {};
            \node[state,accepting] (1)   [right of=0] {};
            \path
            (0) edge node {$t_p$} (1)
            ;
            \draw (-2,-1) rectangle ++(4,2);
        \end{tikzpicture}
        \caption{Automate de obtenu par \ref{fig:aaprime}}
    \end{subfigure}\hfill
    \begin{subfigure}{0.5\linewidth}
        \centering
        \begin{tikzpicture}[->,>=stealth',shorten >=1pt,auto,node distance=3cm, semithick, bend angle=10]
            \tikzstyle{every state}=[circle,scale=0.5]
            \node[state] (0) {};
            \node[state,accepting] (1)   [right of=0] {};

            \path
            (0) edge node {$t_q$} (1)
            ;
            \draw (-2,-1) rectangle ++(4,2);
        \end{tikzpicture}
        \caption{Automate $P_A$ de $Post(L'',\theta)$}
    \end{subfigure}
    \caption{Application de $Post(L'',\theta)$ sur un automate}\label{fig:pq}
\end{figure}















\section{Exemple}
\paragraph{$F$ et $L(F)$ candidat à $AL(F)$}
\begin{figure}[H]
    \centering
    \begin{tikzpicture}[->,>=stealth',shorten >=1pt,auto,node distance=5cm, semithick, bend angle=10]

    \tikzstyle{every state}=[circle]


    \node[initial, state]   (0)                         {$q_0$};
    \node[state]            (1) [above right of=0]      {$q_1$};
    \node[state]            (2) [below right of=0]      {$q_2$};

    \path
    (0) edge [bend left=10] node {$\theta_1(a!0)$} (1)
    (0) edge [bend left=10] node {$\theta_2(a!1)$} (2)
    (1) edge [bend left=10] node {$\theta_3(a?0)$} (0)
    (1) edge node {$\theta_4(a!1)$} (2)
    (2) edge [bend left=10] node {$\theta_5(a?1)$} (0)
    ;
    \end{tikzpicture}
    \caption{Automate à Files $F$}\label{fig:fex}
\end{figure}

Regex pour AL(F) : $(\bar{\theta_1}|\bar{\theta_2})^*(\theta_1|\theta_2)?$

\begin{figure}[H]
    \centering
    \begin{tikzpicture}[->,>=stealth',shorten >=1pt,auto,node distance=3cm, semithick, bend angle=10]
    \tikzstyle{every state}=[circle]

    \node[initial, state]   (0)                         {$0$};
    \node[state]            (1) [above right of=0]      {$1$};
    \node[state]            (2) [below right of=0]      {$2$};
    \node[state, accepting] (3) [below right of=1]      {$3$};

    \path
    (0) edge [loop above] node {$\bt_1$} (0)
    (0) edge [loop below] node {$\bt_2$} (0)
    (0) edge [bend left] node {$\theta_1$} (1)
    (0) edge node {$\theta_2$} (2)
    (0) edge node [below left] {$t_{q_0}$} (3)

    (1) edge node {$t_{q_1}$} (3)
    (1) edge [bend left] node {$\bt_4$} (0)
    (1) edge node [pos=0.2,above right] {$\theta_4$} (2)

    (2) edge node {$t_{q_2}$} (3)
    ;
    \end{tikzpicture}
    \caption{Candidat L(F) pour AL(F)}\label{fig:lfex}
\end{figure}

Pour simplifier la lecture de cet exemple, les états pour lesquels il existe une transition $t_p$ vers un état final sont appelés \emph{états pré-finaux} sur $t_p$.

\paragraph{$t_{q0}$} N'appartient pas à l'ensemble $\Theta$ mais fait partie de la définition de $\mathcal{F}(L)$.

\begin{figure}[H]
    \centering
    \begin{tikzpicture}[->,>=stealth',shorten >=1pt,auto,node distance=3cm, semithick, bend angle=10]
    \tikzstyle{every state}=[circle]
    \node[initial, state]   (0)                   {$0$};
    \node[state, accepting] (1) [right of=0]      {$3$};

    \path
    (0) edge node {$t_{q0}$} (1)
    ;
    \end{tikzpicture}
\end{figure}

\paragraph{$\theta_1=(q_0,a!0,q_1)$}

États pré-finaux sur $q_0$ : ${0}$. Un automate est construit pour $L'=L\bigcap \Sigma^*t_q0$ : \ref{fig:tq0}.

\begin{figure}[H]
    \centering
    \begin{tikzpicture}[->,>=stealth',shorten >=1pt,auto,node distance=3cm, semithick, bend angle=10]
    \tikzstyle{every state}=[circle]

    \node[initial, state]   (0)                         {$0$};
    \node[state]            (1) [above right of=0]      {$1$};
    \node[state, accepting] (3) [below right of=1]      {$3$};

    \path
    (0) edge [loop above] node {$\bar{\theta_1}$} (0)
    (0) edge [loop below] node {$\bar{\theta_2}$} (0)
    (0) edge [bend left] node {$\theta_1$} (1)
    (1) edge [bend left] node {$\bt_4$} (0)
    (0) edge node {$t_{q_0}$} (3)
    ;
  \end{tikzpicture}\caption{Un ANF représentant $L'$}\label{fig:tq0}
\end{figure}

En suivant la règle pour les $\theta$ dont l'action est de type $c!a$, ce qui est le cas pour $\theta_1$, un nouvel état est créé entre $0$ et $3$ car il y a une trasition $t_{q_0}$ à cet endroit. Cela donne l'ANF \ref{fig:tq0b}

\begin{figure}[H]
    \centering
    \begin{tikzpicture}[->,>=stealth',shorten >=1pt,auto,node distance=3cm, semithick, bend angle=10]
    \tikzstyle{every state}=[circle]

    \node[initial, state]   (0)                         {$0$};
    \node[state]            (1) [above right of=0]      {$1$};
    \node[state]            (4) [below right of=1]      {$4$};
    \node[state, accepting] (3) [right of=4]      {$3$};

    \path
    (0) edge [loop above] node {$\bar{\theta_1}$} (0)
    (0) edge [loop below] node {$\bar{\theta_2}$} (0)
    (0) edge [bend left] node {$\theta_1$} (1)
    (1) edge [bend left] node {$\bt_4$} (0)
    (0) edge node {$\theta_1$} (4)
    (4) edge node {$t_{q_1}$} (3)
    ;
  \end{tikzpicture}\caption{Un ANF représentant $Post(L,\theta_1)$}\label{fig:tq0b}
\end{figure}

Cet automate \ref{fig:tq0b} n'est pas déterministe mais il existe un équivalent déterministe qui peut être utilisé pour l'algorithme d'Angluin.

\paragraph{$\theta_2=(q_0,a!1,q_2)$}

États pré-finaux sur $q_0$ : ${0}$. De la même façon que pour $\theta_1$, un automate est calculé.

\begin{figure}[H]
    \centering
    \begin{tikzpicture}[->,>=stealth',shorten >=1pt,auto,node distance=3cm, semithick, bend angle=10]
    \tikzstyle{every state}=[circle]

    \node[initial, state]   (0)                         {$0$};
    \node[state]            (1) [above right of=0]      {$1$};
    \node[state]            (4) [below right of=1]      {$4$};
    \node[state, accepting] (3) [right of=4]      {$3$};

    \path
    (0) edge [loop above] node {$\bar{\theta_1}$} (0)
    (0) edge [loop below] node {$\bar{\theta_2}$} (0)
    (0) edge [bend left] node {$\theta_1$} (1)
    (1) edge [bend left] node {$\bt_4$} (0)
    (0) edge node {$\theta_2$} (4)
    (4) edge node {$t_{q_2}$} (3)
    ;
  \end{tikzpicture}\caption{Un ANF représentant $Post(L,\theta_2)$}
\end{figure}

\paragraph{$\theta_3=(q_1,a?0,q_3)$}

États avant-derniers sur $q_1$ : ${1}$. Contrairement à $\theta_1$ et $\theta_2$, $\theta_3$ est une action de réception.

Premièrement, l'ANF pour $L'=L\bigcap \Sigma^*.\theta_3.\Sigma^*.tq_1$ est calculé.

\begin{figure}[H]
    \centering
    \begin{tikzpicture}[->,>=stealth',shorten >=1pt,auto,node distance=3cm, semithick, bend angle=10]
    \tikzstyle{every state}=[circle]

    \node[initial, state]   (0)                         {$0$};
    \node[state]            (1) [above right of=0]      {$1$};
    \node[state, accepting] (3) [below right of=1]      {$3$};

    \path
    (0) edge [loop above] node {$\bar{\theta_1}$} (0)
    (0) edge [loop below] node {$\bar{\theta_2}$} (0)
    (0) edge [bend left] node {$\theta_1$} (1)
    (1) edge [bend left] node {$\bt_4$} (0)
    (1) edge node {$t_{q_1}$} (3)
    ;
  \end{tikzpicture}\caption{Un ANF représentant $L'$}\label{fig:tq1}
\end{figure}

Cet automate sert à construire $Post(L,\theta_3)$ comme décrit pour les $\theta$ dont l'action est de forme $c?a$ : \ref{fig:tq1b}. Ici le seule symbole $\theta_s$ d'action $c!a$ correspondant à $\theta_3$ ($a!0$) est $\theta_1$.

\begin{figure}[H]
    \centering
    \begin{tikzpicture}[->,>=stealth',shorten >=1pt,auto,node distance=3cm, semithick, bend angle=10]
        \tikzstyle{every state}=[circle,scale=0.8]

        \node[initial, state]   (0)                         {$0$};
        \node[state]            (1) [above right of=0]      {$1$};
        \node[state] (3) [below right of=1]      {$3$};

        \node[state]   (4) [right=4.5cm of 3]            {$0'$};
        \node[state]            (5) [above right of=4]      {$1'$};
        \node[state, accepting] (6) [below right of=5]      {$3'$};

        \path
        (0) edge [loop above] node {$\bar{\theta_1}$} (0)
        (0) edge [loop below] node {$\bar{\theta_2}$} (0)
        (0) edge [bend right=20] node [below right] {$\bt_1$} (4)
        (1) edge [bend left] node {$\bt_4$} (0)
        (1) edge node {$t_{q_1}$} (3)

        (4) edge [loop above] node {$\bar{\theta_1}$} (4)
        (4) edge [loop below] node {$\bar{\theta_2}$} (4)
        (4) edge [bend left] node {$\theta_1$} (5)
        (5) edge [bend left] node {$\bt_4$} (4)
        (5) edge node {$t_{q_1}$} (6)
        ;

        \draw (-2.5,-2) rectangle ++(7,6);
        \draw (7,-2) rectangle ++(6,6);


    \end{tikzpicture}
    \caption{Un ANF représentant $Post(L,\theta_3)$}\label{fig:tq1b}
\end{figure}

Dans ce cas particulier, $Post(L,\theta_3)=L'$.

Les mêmes opérations sont effectuées pour $\theta_4$ et $\theta_5$. Finalement, ces 6 automates (5 obtenus par les $Post(L,\theta)$ et $t_{q_0}$) sont regroupés en un seul par union.

Dans cet exemple particulier, le résultat de cette union est équivalent à \ref{fig:lfex}. Le candidat $L(F)$ est probablement équivalent à $AL(F)$. Aucun contre-exemple ne peut-être trouvé à ce niveau.
