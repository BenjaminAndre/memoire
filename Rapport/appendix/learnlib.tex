Cette annexe est une liste exhaustive des modifications apportées aux librairies Automatalib et Learnlib pour leur permettre de supporter l'apprentissage automatique de la sécurité d'automates à files.

\section{Ajouts et modifications à Automatalib}

Automatalib est, comme son nom l'indique, la librairie responsable de toutes les opérations sur les automates.

Voici une liste exhaustive des ajouts et modifications apportés :
\begin{itemize}
  \item Une énumeration Action a été développée, listant les types d'action ($?,!,\tau$)
  \item Une classe Theta a été créée. Celle-ci permet d'abstraire une transition $(p,action,q)$ en un nom de transition $\theta$
  \item Une nouvelle méthode dans Alphabets, permettant de calculer l'ensemble $\Theta$ en fonction des canaux et symboles utilisés.
  \item Une classe PhiChar, symbolisant les symboles $\phi\in\Phi$
  \item Une interface MutableFIFOA, définissant le contrat d'un automate à files. Il doit entre autre permettre d'obtenir son alphabet d'annotation $\Phi$.
  \item Une interface FIFOA, continuant le contrat à la validation de trace annotée; vérifiant au passage si elles sont correctement annotées.
  \item Une nouvelle méthode permettant d'insérer un symbole dans un mot Word à une place arbitraire, utile pour tester différentes permutations.
  \item Des méthodes permettant d'obtenir les transitions entrantes pour un symbole donné, dans CompactDFA et CompactNFA. Ces méthodes sont utilisées lors du calcul de $\mathcal{F}(L)$
  \item Une classe CompactFIFOA, remplissant les deux contrats FIFOA et MutableFIFOA. De plus, elle offre une méthode utilitaire permettant de retrouver l'ensemble de $\theta_s$ associés à une action $c!m$ précise. Cette classe hérite de CompactDFA, représentant l'automate à file comme un ADF dont les symbole de transition sont des $\theta$.
  \item Des nouvelles méthodes dans AbstractCompact et AbstractCompactSimpleDeterministic simplifiant la manipulation bas niveau requise par les différents algorithmes.
  \item Une classe FIFOBuilder récupérant le système d'annotation et de syntaxe de la librairie pour générer les classes offrant une syntaxe simple pour la création d'un automate à files. Voici l'expression régulière définissant les opérations acceptées pour construire un automate à files \emph{(from (((on (write|read))|pass) (loop|to))+)* withInitial create}
  \item Une classe FIFOAs offrant les principaux algorithmes détaillés dans le chapitre \ref{impl}. On y trouve :
    \begin{itemize}
      \item applyFL pour $\mathcal{F}(L)$.
      \item sigmaStarTP créant un automate représentant $L=\Sigma^*t_p$.
      \item sigmaStarThetaSSigmaStarTP créant un automate représentant $L=\Sigma^*\Theta_s\Sigma^*t_p$.
      \item aPlusAPrime gérant le cas où l'action est de la forme $c?m$ lors du calcul de \fl.
      \item reverseFL implémentant l'algorithme du même nom.
      \item reversehcj implémentant la verison de $\mathcal{W}(L)$ limitée au test du passage par une liste d'états de contrôle.
    \end{itemize}
  \item Une méthode dans DFAs pour fusionner de nombreux ADF. C'est utile pour calculer $\bigcup_{\theta\in\Theta}A_\theta$.
  \item Une méthode dans NFAs pour convertir un ADF en ANF équivalent.
\end{itemize}


\section{Ajouts et modifications à Learnlib}

De façon similaire, Learnlib contient entre autres la logique liée à l'apprentissage actif.

Voici une liste exhaustive des modifications et ajouts proposés pour Learnlib.

\begin{itemize}
  \item Une exception SafeException permettant d'interrompre l'apprentissage (algorithme d'Angluin) s'il est possible de déclarer l'automate à files comme étant sûr.
  \item Une exception UnsafeException remplissant le même rôle dans le cas où l'automate à files est déclaré comme étant à risque.
  \item Une classe FIFOTraceSimulatorOracle permettant de répondre à l'oracle d'appartenance en simulant l'exécution d'une trace annotée dans l'automate à files.
  \item Une interface FIFOAEquivalenceOracle décrivant le contrat d'un oracle d'équivalence pour un automate à files. En pratique, c'est le même contrat que pour un automate fini, mais le typage oblige à créer une interface supplémentaire.
  \item Une classe ALFEQOracle implémentant le contrat précédant. Pour ce faire, différentes méthodes internes sont ajoutées, cherchant un contre-exemple à $L=AL(F)$, proposant un mot dans $\mathcal{W}(L)$ et vérifiant si celui-ci est valide. En pratique, cette classe utilise principalement la logique offerte par FIFOAs dans la librairie Automatalib.
  \item Une classe LeverExperiment faisant le lien entre les différentes classes implémentées et le moteur logique de Learnlib pour l'algorithme $L^*$.
\end{itemize}


\section{Projet de démonstration}

En plus des librairies Automatalib et Learnlib, un projet de démonstration est proposé. Celui-ci reprend les exemples cités dans la section \ref{sec:exps}. De plus, il propose une série de tests unitaires.

Voici la liste des ajouts du projet de démonstration :
\begin{itemize}
  \item Une classe TestFIFOA fournissant un exemple complet de l'apprentissage automatique de la sûreté d'un automate à files. Aide aussi à mettre en évidence le processus par lequel un lecteur peut faire appel aux différents algorithmes pour étudier l'automate à files de son choix.
  \item Une classe TestAutomatons listant les ADF et automates à files utilisés pour des tests unitaires.
  \item Une class AlphabetTest composée de tests unitaires de l'utilisation de $\Phi$
  \item Une classe FIFOATest composée de différents tests unitaires :
    \begin{itemize}
      \item Des tests sur les traces. Sont-elles correctes ? Sont-elles valides ?
      \item Des tests sur les différentes méthodes utilitaires.
    \end{itemize}
  \item Finalement, une classe FLTest testant les différents segments servant à calculer \fl :
    \begin{itemize}
      \item Des tests sur les automates pour $L=Sigma^*.t_p$.
      \item Des tests sur les automates pour $L=\Sigma^*.\Theta_s.\Sigma^*.t_p$.
      \item Des tests pour aplusAPrime.
      \item Des tests pour reverseFL.
    \end{itemize}
\end{itemize}
