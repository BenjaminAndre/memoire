Cette section pose les bases théoriques et les conventions nécessaires à la compréhension des sections suivantes. De plus, certaines propriétés importantes sont énoncées et démontrées.

	\subsection{Alphabet}
	
	Un alphabet, nommé par convention $\Sigma$ est un ensemble fini et non vide de symboles.
	Voici certains exemples d'alphabets :
	\begin{itemize}
		\item $\Sigma = \{0,1,2,3,4,5,6,7,8,9\}$, l'alphabet des chiffres
		\item $\Sigma = \{a,b,c,...,z,A,B,C,...,Z\}$, l'alphabet latin
		\item $\Sigma = \{0,1\}$ l'alphabet binaire
	\end{itemize}

	\subsection{Mots}
	
	Comme $\Sigma$ est un ensemble, on peut définir $\Sigma^k$, qui donne des k-uples de symboles, appartenant tous à $\Sigma$.
	
	Un mot $w$ de taille $|w|=k$ est un ensemble de symboles provenant de $\Sigma^k$. Dans le cas particulier où $k=0$, on note le mot vide (sans symbole) $w=\epsilon$.
	
	De façon générale, $w$ est un mot sur $\Sigma$ si il existe $k$ tel que $w \in \Sigma^k$. Par convention, les mots sont nommés par une lettre minuscule, souvent $w,x,y,z$. 
	
	L'ensemble de tous ces mots possible sur $\Sigma$ est noté $\Sigma^*$. Cet ensemble est infini.
	

	
	\subsection{Langage}
	
	Un langage $L$ est défini sur un alphabet $\Sigma$. $L$ est un ensemble de mots sur cet alphabet : $L \subseteq \Sigma^*$. Comme $\Sigma^*$ contient une infinité de mots, $L$ est susceptible de ne pas être fini non plus.
	
	Par exemple, par rapport à tous les mots disponibles avec les lettres de l'alphabet latin ($\Sigma^*$), seulement certains font partie de la langue française ($L$).
	
	$L$ peut être défini  :
	\begin{itemize}
		\item en énumérant les mots en faisant partie : $L=\{12,35,42,7,0\}$
		\item via un notation ensembliste : $L=\{0^k1^j|k+j=7\}$ ou $L=\{w|w \text{ est un mot français}\}$
	\end{itemize}
	Dans ces deux cas, $\Sigma$ est souvent implicite.
	
	\subsubsection{Problème}
	Une notion liée aux langages est celle de problème. Ce que nous appelons couramment problème peut être exprimé en terme d'appartenance d'un mot $w$ à un langage $L$ sur un alphabet $\Sigma$.
	
	Par exemple, prenons l'alphabet $\Sigma=\{0,1,2,3,4,5,6,7,8,9\}$. Considérons ensuite le langage $L = \{w | \text{le nombre représenté par } w \text{ est pair}\}$.
	Demander si un mot $w$ appartient à $L$ revient à résoudre le problème sur le test de parité de $w$. (Le cas échéant, soit $w$ ne représente pas un nombre, soit il représente un nombre impair).
		
	\subsection{Expression régulière}
	
	Une expression régulière est une façon efficace de définir un langage. La construction de l'expression se fait de façon inductive.
	
	Le cas de base est l'expression $e = a, a \in \Sigma$. Dès lors $L_e = {a}$ où $L_e$ est le langage donné par l'expression $e$.
	
	Les autres règles sont inductives. Par ordre de priorité :
	\begin{itemize}
		\item $e = (e_0)$ La mise en évidence. Ici, $L_e = \{w|w \in L_{e_0}\}$
		\item $e = e_0a, a \in \Sigma$. La concaténation. Ici, $L_e = \{wa|w \in L_{e_0}\}$
		\item $e = e_0+e_1$. L'union. Ici, $L_e = L_{e_0} \cup L_{e_1}$
		\item $e = e_0^*$. La fermeture. Intuitivement, il s'agit de tous les mots qui peuvent être formé par une concaténation de mots définis par $e_0$, éventuellement aucun. Ici, $L_e = \{w_0w_1...w_k | k \in \mathbb{N}, w_0,w_1,...,w_k \in L_{e_0}\}$
		\item $e = e_0^+$. La fermeture non nulle. Il s'agit de la fermeture mais avec toujours au moins un mot venant du langage défini par $e_0$. Ici, $L_e = \{w_0w_1...w_k | k \in \mathbb{N}^0, w_0,w_1,...,w_k \in L_{e_0}\}$
	\end{itemize}
	
	Par exemple, on peut écrire l'expression $e_B = (1+01)1^*0(0+1)^*$.\\
	
	\begin{minipage}{0.5\linewidth}
		Les mots suivant en feraient partie :
		\begin{itemize}
			\item 10
			\item 010
			\item 0110
			\item 0111110
			\item 0101101
		\end{itemize}
	\end{minipage}
	\begin{minipage}{0.5\linewidth}
		Ceux-ci n'en feraient pas partie
		\begin{itemize}
			\item 00
			\item 1
			\item 01
			\item 0101
			\item 11
		\end{itemize}
	\end{minipage}

	Un langage pouvant être écrit sous la forme d'une expression régulière est appelé langage régulier. Un des conséquences de cette propriété est qu'il peut être représenté par un automate déterministe fini. \todo{a prouver}
	
	
	\subsection{Automate déterministe fini}\label{sub:dfa}
	Soit un ensemble de symboles $\Sigma$. Soient $\Sigma^* = \{ a_1a_2a_3...a_n | a_1,a_2,a_3,...,a_n \in \Sigma \}$, l'ensemble des mots de taille arbitraire qu'il est possible de former à partir de $\Sigma$ et $|w|, w \in \Sigma$ la longueur de $w$, le nombre de symboles utilisés. Si $|w|=0$, on note $w=\epsilon$.
	
	
	Un automate est défini par $A = (Q, \Sigma, q_0, \delta, F)$ où
	\begin{itemize}
		\item $Q$ est un ensemble d'états, différenciés par leur indice $q_1, q_2, ..., q_n$ ou $n = |Q|$.
		\item $\Sigma$ est un ensemble de symboles
		\item $q_0 \in Q$ est l'état initial
		\item $\delta : Q \times \Sigma \rightarrow Q$ est la fonction de transition. A partir d'un état de $Q$, en fonction d'un symbole, elle retourne un nouvel état faisant partie de $Q$.
		\item $F \subseteq Q$ est un ensemble d'état finaux.
	\end{itemize}
	 
	 
	 Exemple :
	 Soient 
	 \begin{itemize}
	 	\item $\Sigma=\{0,1\}$
	 	\item $Q=\{q_a,q_b,q_c,q_d,q_e,q_f,q_g\}$
	 	\item $q_0=q_a$
	 	\item $F=\{q_d\}$
	 \end{itemize}
 	
 	Pour obtenir un automate, il manque la description de $\delta$. Une façon efficace de noter cette fonction de transition est à l'aide d'une table dont les lignes reprennent des éléments de $Q$, les colonnes des symboles de $\Sigma$, et les cases des éléments de $Q$. Ainsi, on peut lire la relation $\delta : Q \times \Sigma \rightarrow Q$ en prenant une ligne et une colonne.
 	De plus, via cette notation, $Q$ et $\Sigma$ sont explicites. En dénotant l'état initial par $\rightarrow$ et les états acceptants par $*$, on obtient une définition complète d'un automate.
 	
 	\begin{figure}[H]
 	\centering
 	\begin{tabular}{|r||c|c|}
 		\hline
 		&0&1\\
 		\hline\hline
 		$\rightarrow q_a$&$q_c$&$q_b$\\\hline
 		$q_b$&$q_d$&$q_f$\\\hline
 		$q_c$&$q_e$&$q_f$\\\hline
 		$q_d$&$q_d$&$q_d$\\\hline
 		$q_e$&$q_e$&$q_e$\\\hline
 		$q_f$&$q_d$&$q_b$\\\hline
 		$q_g$&$q_e$&$q_f$\\\hline
 	\end{tabular}
	\caption{La table de transitions $\delta_B$}
 	\end{figure}
	 L'automate peut aussi être représenté graphiquement : 
	 
	 \begin{figure}[H]
	 	\centering
	 	\begin{tikzpicture}[->,>=stealth',shorten >=1pt,auto,node distance=3cm, semithick, bend angle=10]
	 	
	 	\tikzstyle{every state}=[circle]
	 	
	 	\node[initial,state] (A)                    {$q_a$};
	 	\node[state]         (B) [below right of=A] {$q_b$};
	 	\node[state]         (C) [below left of=A] {$q_c$};
	 	\node[accepting, state]         (D) [below right of=B] {$q_d$};
	 	\node[state]         (E) [below left of=C]       {$q_e$};
	 	\node[state]         (G) [below right of=E]       {$q_g$};
	 	\node[state]         (F) [above right of=G]       {$q_f$};
	 	
	 	\path 	(A) 	edge              node {0} (C)
	 	edge              node {1} (B)
	 	(B) 	edge              node {0} (D)
	 	edge [bend left]  node {1} (F)
	 	(C) 	edge              node {0} (E)
	 	edge              node {1} (F)
	 	(D) 	edge [loop above] node {0,1} (D)
	 	(E) 	edge [loop above] node {0,1} (E)
	 	(F) 	edge              node {0} (D)
	 	edge [bend left]  node {1} (B)
	 	(G) 	edge              node {0} (E)
	 	edge              node {1} (F);
	 	\end{tikzpicture}
	 	\caption{Automate $A_B$, exemple personnel}\label{fig:ab}
	 \end{figure}
	 
	 Cette représentation d'un automate peut sembler plus naturelle pour un humain alors que la table de transitions est plus proche d'un langage informatique. De plus, dans la représentation par graphe, les ensembles $Q$ et $\Sigma$ sont implicites et doivent être définis ou déduits à part. $q_0$ et $F$ sont respectivement représenté par la flèche entrante et le double cercle.
	 
	 \subsubsection{Chemin}
	 
	 Pour faire le lien avec la notion de langage, il doit exister une façon pour l'automate de représenter quels mots sont acceptés ou non.
	 
	 Pour ce faire, la fonction $\delta$ peut être étendue à un chemin $w$:
	 
	 \begin{itemize}
	 	\item Si $|w| \leq 1$, $\hat{\delta}(x, w) = \delta(x, w)$
	 	\item Sinon, c'est que $w$ peut s'écrire $au, u \in \Sigma^*$. Alors, $\hat{\delta}(x,w) = \hat{\delta}(x,au) = \hat{\delta}(\delta(x,a),u)$
	 \end{itemize}
	 
	 \subsubsection{Langage défini par un automate}
	 
	 Le langage représenté par un automate $A$ peut alors se définir comme les mots menant de l'état initial a un état acceptant :
	 $$
	 \{w \in \Sigma^* | \hat{\delta}(q_O,w) \in F_A\}
	 $$
	 Ainsi, pour tester l'appartenance d'un mot $w$ à un langage $L$ défini par l'automate $A$, il suffit de test $\hat{\delta}(q_O,w) \in F_A$.
	 
	 \subsubsection{La relation $R_M$}\label{ss:rm}
	 
	 Soit un automate $M$. Définissons la relation $R_M$ entre deux états : $$xR_My \iff (\forall w \in \Sigma^*, \hat{\delta}(x,w) \in F \iff \hat{\delta}(y,w) \in F)$$
	 
	 Intuitivement, ces deux états sont en relation si tout mot lu à partir de celui-ci mène à des états étant simultanément acceptants ou non. Il s'agit d'une relation d'équivalence. En effet, cette relation est :
	 
	 \begin{itemize}
	 	\item \textbf{Réflexive :} Soient un état $x \in Q_M$ et $w \in \Sigma^*$. Alors, $\hat{\delta}(x,w) \in F \iff \hat{\delta}(x,w) \in F$ et par définition, $xR_Mx$.
	 	\item \textbf{Transitive :} Soient les états $x,y,z \in Q_M$ tels que $xR_My$ et $yR_Mz$ ainsi que $w \in \Sigma^*$. Par hypothèse, $\hat{\delta}(x,w) \in F \iff \hat{\delta}(y,w)\in F$ et $\hat{\delta}(y,w) \in F\iff \hat{\delta}(z,w) \in F$. Par transitivité de l'implication, on obtient $\hat{\delta}(x,w) \in F \iff \hat{\delta}(z,w)\in F$. On a donc $xR_Mz$.
	 	\item \textbf{Symétrique : } Soient les états $x,y \in Q_M$ tels que $xR_My$ et un mot $w \in \Sigma^*$. Par hypothèse, $\hat{\delta}(x, w)\in F \iff \hat{\delta}(y, w)\in F$. En lisant la double implication depuis la droite, on a bien $\hat{\delta}(y, w) \in F\iff \hat{\delta}(x, w)\in F$ et donc $yR_Mx$.
	 	\item De plus, la relation est \textbf{congruente à droite :} si la relation est vraie pour deux état, elle reste valable pour les états atteints par la lecture d'un symbole quelconque. Soient les états $x,y \in Q_M$ tels que $xR_My$. Soit un symbole $a \in \Sigma$. Par hypothèse, 
	 	$$\forall w \in \Sigma^*, \hat{\delta}(x, w) \in F \iff \hat{\delta}(y, w) \in F$$
	 	C'est donc vrai en particulier pour $w = au, u \in \Sigma*$. Dès lors,
	 	$$\hat{\delta}(x, au) \in F\iff \hat{\delta}(y, au)\in F$$
	 	$$\hat{\delta}(\delta(x,a),u) \in F\iff\hat{\delta}(\delta(y,a),u)\in F$$
	 	$$\hat{\delta}(p,u) \in F\iff \hat{\delta}(q,u)\in F$$
	 \end{itemize}
	 
	 Deux informations importantes découlent des ces caractéristiques : 
	 \begin{enumerate}
	 	\item Les états forment des classes d'équivalence.
	 	\item Tout état dans une classe d'équivalence mène, pour un même symbole, à une même classe d'équivalence.
	 \end{enumerate}
	 
	 