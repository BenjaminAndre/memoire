% ██████  ███████ ███████ ██ ███    ██ ██ ████████ ██  ██████  ███    ██
% ██   ██ ██      ██      ██ ████   ██ ██    ██    ██ ██    ██ ████   ██
% ██   ██ █████   █████   ██ ██ ██  ██ ██    ██    ██ ██    ██ ██ ██  ██
% ██   ██ ██      ██      ██ ██  ██ ██ ██    ██    ██ ██    ██ ██  ██ ██
% ██████  ███████ ██      ██ ██   ████ ██    ██    ██  ██████  ██   ████


Un \emph{automate fini} \automaton est défini comme suit :
\begin{itemize}
  \item $Q$ est un ensemble fini d'\emph{états}
  \item $\Sigma$ est un alphabet
  \item $q_0 \in Q$ est l'\emph{état initial}
  \item $\delta$ est la \emph{fonction de transition}
  \item $F \subseteq Q$ est un ensemble d'\emph{états acceptants}.
\end{itemize}

La fonction de transition $\delta$ est définie différemment en fonction du type d'automate souhaité :
\begin{itemize}
  \item \textbf{Automate Déterministe Fini (ADF)} $\delta : Q \times \Sigma \rightarrow Q$. Soit un état $q$ et un symbole $a$. Alors la \emph{transition} $\delta(q,a)$ retourne un état $p$. $\delta(q,a)$ doit être définie pour tout état et tout symbole.
  \item \textbf{Automate Non-déterministe Fini (ANF)} $\delta : Q \times \Sigma \rightarrow 2^Q$. Soit un état $q$ et un symbole $a$. Alors la transition $\delta(q,a)$ retourne un ensemble d'états $P=\{p_1,p_2,\dots,p_n\}\subseteq Q$.
  \item \textbf{Automate Non-déterministe Fini avec des transitions sur $\epsilon$ ($\epsilon$-ANF)} $\delta : Q \times \Sigma \cup \{\epsilon\} \rightarrow 2^Q$. Pareil que précédemment mais une transition peut exister sans symbole : elle se fait alors sur $\epsilon$.
\end{itemize}

Lorsqu'un automate est mentionné dans ce document, il s'agit implicitement d'un $\epsilon$-ANF, sauf mention contraire. En effet, c'est la forme la plus générale. Cependant, ces trois types d'automates ont la même puissance expressive, ce qui est prouvé dans la section \ref{ss:eqadfanf}.

Soit la transition $\delta(q,a)=p$ (dans un ADF). Pour $q$, c'est une \emph{transition sortante sur a}. Pour $p$, c'est une \emph{transition entrante sur a}.

Si $\delta(q,a)=P=\{p_1,p_2,\dots,p_n\}$ dans un ANF, alors les états $\{p_1,p_2,\dots,p_n\}$ auront une transition entrante sur $a$.

Dans le cas des ANFs et $\epsilon$-ANFs, il peut être pratique d'utiliser $\delta$ sur un ensemble d'états $S$. A ce moment, $\delta(S,a)=\bigcup_{q\in S}\delta(q,a)$ avec $a\in \Sigma$.


% ███████ ██   ██ ███████ ███    ███ ██████  ██      ███████ ███████
% ██       ██ ██  ██      ████  ████ ██   ██ ██      ██      ██
% █████     ███   █████   ██ ████ ██ ██████  ██      █████   ███████
% ██       ██ ██  ██      ██  ██  ██ ██      ██      ██           ██
% ███████ ██   ██ ███████ ██      ██ ██      ███████ ███████ ███████


\begin{example}[Automate déterministe fini]\label{ex:adf}
  On considère l'automate \automaton défini comme suit :
  \begin{itemize}
    \item $Q=\{q_0,q_1,q_2,q_3,q_4,q_5,q_6\}$
    \item $\Sigma=\{a,b\}$
    \item $q_0$ est l'état du même nom.
    \item La fonction de transition $\delta$ est décrite par la table \ref{table:transdelta}.
    \item $F=\{q_3\}$
  \end{itemize}

  \begin{table}[H]
    \centering
    \begin{tabular}{|r||c|c|}
      \hline
      &a&b\\
      \hline\hline
      $\rightarrow q_0$&$q_2$&$q_1$\\\hline
      $q_1$&$q_3$&$q_5$\\\hline
      $q_2$&$q_4$&$q_5$\\\hline
      $q_3^*$&$q_3$&$q_3$\\\hline
      $q_4$&$q_4$&$q_4$\\\hline
      $q_5$&$q_3$&$q_1$\\\hline
      $q_6$&$q_4$&$q_5$\\\hline
    \end{tabular}
    \caption{La \emph{table de transitions} $\delta$ d'un ANF}
    \label{table:transdelta}
  \end{table}
\end{example}

Cette table de transitions est construite comme suit :
\begin{itemize}
  \item Les en-têtes de colonnes sont des symboles $a\in\Sigma$.
  \item Les en-têtes de lignes sont des états $q \in Q$.
  \item Un cellule à la croisée de la ligne $q$ et du symbole $a$ contient un état $p$ avec $p=\delta(q,a)$.
\end{itemize}

Via la notation de la table \ref{table:transdelta}, $Q$ et $\Sigma$ sont explicites. En dénotant l'état initial par $\rightarrow$ et les états acceptants par $*$ en exposant, on obtient une définition complète d'un automate : $(Q,\Sigma, q_0, \delta, F)$.

\begin{example}[$\epsilon$-ANF]\label{ex:anf}
	 De la même façon que pour l'exemple précédent, considérons un automate \automaton défini comme suit :

\begin{itemize}
	\item $Q=\{q_0,q_1,q_2\}$
	\item $\Sigma=\{a,b,c\}$
	\item $q_0$ est l'état du même nom
	\item $\delta$ est donnée par la table \ref{table:eanfdelta}.
	\item $F=\{q_2\}$
\end{itemize}

$A$ est un $\epsilon$-ANF ; une colonne supplémentaire sert à représenter la transition sur $\epsilon$.

\begin{table}[H]
	\centering
	\begin{tabular}{|r||c|c|c|c|}
		\hline
		&$\epsilon$&a&b&c\\
		\hline\hline
		$\rightarrow q_0$&$\{q_1,q_2\}$&$\emptyset$&$\{q_1\}$&$\{q_2\}$\\\hline
		$q_1$&$\emptyset$&$\{q_0\}$&$\{q_2\}$&$\{q_0,q_1\}$\\\hline
		$q_2^*$&$\emptyset$&$\emptyset$&$\emptyset$&$\emptyset$\\\hline
	\end{tabular}
	\caption{La table de transitions $\delta$ d'un $\Sigma$-ANF}
	\label{table:eanfdelta}
\end{table}

Une table similaire sans la colonne $\epsilon$ représenterait un $ANF$ au sens strict. Celui-ci ne serait pas pour autant équivalent à l'$\epsilon-ANF$ de la table \ref{table:eanfdelta}.

\end{example}
