Cette section décrit les automates déterministes finis (ADF), fait le lien avec la notion de langage régulier et propose une représentation visuelle de ces automates. La notation suivie dans cette section s'appuye principalement sur \cite{Hopcroft79}, \cite{Hopcroft00} et \cite{Kozen97}.

Cette section décrit les automates déterministes finis (ADF) dans la section \ref{adf:def} et en propose une représentation dans la section \ref{adf:repr}. Par après, le langage correspondant est défini à la section \ref{adf:lang}.

En plus des automates déterministes finis, les automates non-déterministes finis avec ou sans transition sur $\epsilon$ sont également définis. Leur unique utilité dans le cadre de ce mémoire est de permettre de prouver l'équivalence entre un automate déterministe fini et une expression régulière à la section \ref{adf:eq}. Cette dernière section permet dès lors de passer d'une expression régulière à un automate et vice-versa, ce qui est utile lors de l'apprentissage d'un langage.

Pour facilité la construction du langage et de l'algorithme d'équivalence, une fonction eclose est définie dans la section \ref{adf:eclose}.


\section{Définition}\label{adf:def}% ██████  ███████ ███████ ██ ███    ██ ██ ████████ ██  ██████  ███    ██
% ██   ██ ██      ██      ██ ████   ██ ██    ██    ██ ██    ██ ████   ██
% ██   ██ █████   █████   ██ ██ ██  ██ ██    ██    ██ ██    ██ ██ ██  ██
% ██   ██ ██      ██      ██ ██  ██ ██ ██    ██    ██ ██    ██ ██  ██ ██
% ██████  ███████ ██      ██ ██   ████ ██    ██    ██  ██████  ██   ████


Un \emph{automate fini} \automaton est défini comme suit :
\begin{itemize}
  \item $Q$ est un ensemble fini d'\emph{états}
  \item $\Sigma$ est un alphabet
  \item $q_0 \in Q$ est l'\emph{état initial}
  \item $\delta$ est la \emph{fonction de transition}
  \item $F \subseteq Q$ est un ensemble d'\emph{états acceptants}.
\end{itemize}

La fonction de transition $\delta$ est définie différemment en fonction du type d'automate souhaité :
\begin{itemize}
  \item \textbf{Automate Déterministe Fini (ADF)} $\delta : Q \times \Sigma \rightarrow Q$. Soit un état $q$ et un symbole $a$. Alors la \emph{transition} $\delta(q,a)$ retourne un état $p$. $\delta(q,a)$ doit être définie pour tout état et tout symbole.
  \item \textbf{Automate Non-déterministe Fini (ANF)} $\delta : Q \times \Sigma \rightarrow 2^Q$. Soit un état $q$ et un symbole $a$. Alors la transition $\delta(q,a)$ retourne un ensemble d'états $P=\{p_1,p_2,\dots,p_n\}\subseteq Q$.
  \item \textbf{Automate Non-déterministe Fini avec des transitions sur $\epsilon$ ($\epsilon$-ANF)} $\delta : Q \times \Sigma \cup \{\epsilon\} \rightarrow 2^Q$. Pareil que précédemment mais une transition peut exister sans symbole : elle se fait alors sur $\epsilon$.
\end{itemize}

Lorsqu'un automate est mentionné dans ce document, il s'agit implicitement d'un $\epsilon$-ANF, sauf mention contraire. En effet, c'est la forme la plus générale. Cependant, ces trois types d'automates ont la même puissance expressive, ce qui est prouvé dans la section \ref{ss:eqadfanf}.

Soit la transition $\delta(q,a)=p$ (dans un ADF). Pour $q$, c'est une \emph{transition sortante sur a}. Pour $p$, c'est une \emph{transition entrante sur a}.

Si $\delta(q,a)=P=\{p_1,p_2,\dots,p_n\}$ dans un ANF, alors les états $\{p_1,p_2,\dots,p_n\}$ auront une transition entrante sur $a$.

Dans le cas des ANFs et $\epsilon$-ANFs, il peut être pratique d'utiliser $\delta$ sur un ensemble d'états $S$. A ce moment, $\delta(S,a)=\bigcup_{q\in S}\delta(q,a)$ avec $a\in \Sigma$.


% ███████ ██   ██ ███████ ███    ███ ██████  ██      ███████ ███████
% ██       ██ ██  ██      ████  ████ ██   ██ ██      ██      ██
% █████     ███   █████   ██ ████ ██ ██████  ██      █████   ███████
% ██       ██ ██  ██      ██  ██  ██ ██      ██      ██           ██
% ███████ ██   ██ ███████ ██      ██ ██      ███████ ███████ ███████


\begin{example}[Automate déterministe fini]\label{ex:adf}
  On considère l'automate \automaton défini comme suit :
  \begin{itemize}
    \item $Q=\{q_0,q_1,q_2,q_3,q_4,q_5,q_6\}$
    \item $\Sigma=\{a,b\}$
    \item $q_0$ est l'état du même nom.
    \item La fonction de transition $\delta$ est décrite par la table \ref{table:transdelta}.
    \item $F=\{q_3\}$
  \end{itemize}

  \begin{table}[H]
    \centering
    \begin{tabular}{|r||c|c|}
      \hline
      &a&b\\
      \hline\hline
      $\rightarrow q_0$&$q_2$&$q_1$\\\hline
      $q_1$&$q_3$&$q_5$\\\hline
      $q_2$&$q_4$&$q_5$\\\hline
      $q_3^*$&$q_3$&$q_3$\\\hline
      $q_4$&$q_4$&$q_4$\\\hline
      $q_5$&$q_3$&$q_1$\\\hline
      $q_6$&$q_4$&$q_5$\\\hline
    \end{tabular}
    \caption{La \emph{table de transitions} $\delta$ d'un ANF}
    \label{table:transdelta}
  \end{table}
\end{example}

Cette table de transitions est construite comme suit :
\begin{itemize}
  \item Les en-têtes de colonnes sont des symboles $a\in\Sigma$.
  \item Les en-têtes de lignes sont des états $q \in Q$.
  \item Un cellule à la croisée de la ligne $q$ et du symbole $a$ contient un état $p$ avec $p=\delta(q,a)$.
\end{itemize}

Via la notation de la table \ref{table:transdelta}, $Q$ et $\Sigma$ sont explicites. En dénotant l'état initial par $\rightarrow$ et les états acceptants par $*$ en exposant, on obtient une définition complète d'un automate : $(Q,\Sigma, q_0, \delta, F)$.

\begin{example}[$\epsilon$-ANF]\label{ex:anf}
	 De la même façon que pour l'exemple précédent, considérons un automate \automaton défini comme suit :

\begin{itemize}
	\item $Q=\{q_0,q_1,q_2\}$
	\item $\Sigma=\{a,b,c\}$
	\item $q_0$ est l'état du même nom
	\item $\delta$ est donnée par la table \ref{table:eanfdelta}.
	\item $F=\{q_2\}$
\end{itemize}

$A$ est un $\epsilon$-ANF ; une colonne supplémentaire sert à représenter la transition sur $\epsilon$.

\begin{table}[H]
	\centering
	\begin{tabular}{|r||c|c|c|c|}
		\hline
		&$\epsilon$&a&b&c\\
		\hline\hline
		$\rightarrow q_0$&$\{q_1,q_2\}$&$\emptyset$&$\{q_1\}$&$\{q_2\}$\\\hline
		$q_1$&$\emptyset$&$\{q_0\}$&$\{q_2\}$&$\{q_0,q_1\}$\\\hline
		$q_2^*$&$\emptyset$&$\emptyset$&$\emptyset$&$\emptyset$\\\hline
	\end{tabular}
	\caption{La table de transitions $\delta$ d'un $\Sigma$-ANF}
	\label{table:eanfdelta}
\end{table}

Une table similaire sans la colonne $\epsilon$ représenterait un $ANF$ au sens strict. Celui-ci ne serait pas pour autant équivalent à l'$\epsilon-ANF$ de la table \ref{table:eanfdelta}.

\end{example}

\section{Représentation}\label{adf:repr}Le \emph{graphe d'un automate fini} \automaton est un graphe dirigé construit comme suit :

\begin{itemize}
  \item Chaque état de $Q$ est représenté par un nœud.
  \item Chaque transition $\delta(q,a)$ est représenté par un arc étiqueté $a$. Dans le cas d'un automate non-déterministe, un arc existe pour chacun des états obtenus en suivant la transition. Si il y a plusieurs transitions sortant d'un même état et entrant dans un même autre état, les arcs peuvent être fusionnés en listant les étiquettes.
  \item L'état initial est mis en évidence par une flèche entrante.
  \item Les états acceptants sont représentés par un double cercle, en opposition au simple cercle des autres nœuds.
\end{itemize}

\begin{example}[Graphe d'automate]
 Voici les graphes représentant les automates définis dans les tables \ref{table:transdelta} et \ref{table:eanfdelta} :

 \begin{minipage}[t]{0.5\textwidth}
   \begin{figure}[H]
    \centering
    \begin{tikzpicture}[->,>=stealth',shorten >=1pt,auto,node distance=2.5cm, semithick, bend angle=10]

    \tikzstyle{every state}=[circle]

    \node[initial,state] (A)                    {$q_0$};
    \node[state]         (B) [below right of=A] {$q_1$};
    \node[state]         (C) [below left of=A] {$q_2$};
    \node[accepting, state]         (D) [below right of=B] {$q_3$};
    \node[state]         (E) [below left of=C]       {$q_4$};
    \node[state]         (G) [below right of=E]       {$q_6$};
    \node[state]         (F) [above right of=G]       {$q_5$};

    \path 	(A) 	edge              node {a} (C)
    edge              node {b} (B)
    (B) 	edge              node {a} (D)
    edge [bend left]  node {b} (F)
    (C) 	edge              node {a} (E)
    edge              node {b} (F)
    (D) 	edge [loop above] node {a,b} (D)
    (E) 	edge [loop above] node {a,b} (E)
    (F) 	edge              node {a} (D)
    edge [bend left]  node {b} (B)
    (G) 	edge              node {a} (E)
    edge              node {b} (F);
    \end{tikzpicture}
    \caption{Graphe de l'ADF de la table \ref{table:transdelta}}\label{fig:a1}
   \end{figure}
 \end{minipage}
 \begin{minipage}[t]{0.5\textwidth}
   \begin{figure}[H]
   	\centering
   	\begin{tikzpicture}[->,>=stealth',shorten >=1pt,auto,node distance=2.5cm, semithick, bend angle=10]

   	\tikzstyle{every state}=[circle]

   	\node[initial,state] (A)                    {$q_0$};
   	\node[state]         (B) [above right of=A] {$q_1$};
   	\node[accepting, state]         (C) [below right of=A] {$q_2$};

   	\path
   	(A) edge [bend left] node{$\epsilon$,b} (B)
   	(A) edge node{$\epsilon$,c} (C)
   	(B) edge [bend left] node{a,c} (A)
   	(B) edge [loop right] node{c} (B)
   	(B) edge node{b} (C);
   	\end{tikzpicture}
   	\caption{Graphe du $\epsilon$-ANF de la table \ref{table:eanfdelta}}\label{fig:eanf}
   \end{figure}
\end{minipage}
\end{example}

Cette représentation a l'avantage d'être plus visuelle, alors que la table de transition est plus structurée.

\section{ECLOSE}\label{adf:eclose}Cette sous-section introduit l'algorithme ECLOSE. Cet algorithme concerne les $\epsilon$-ANFs. Il permet, à partir d'un état spécifique, de calculer l'ensemble des états atteignables uniquement par des transitions sur $\epsilon$. Ce calcul sert notemment au test d'appartenance d'un mot à un langage défini par un $\epsilon$-ANF comme présenté à la section \ref{ss:lang}.

Soit un $\epsilon$-ANF \automaton. Il est possible de construire un fonction retournant l'ensemble des états atteints uniquement en suivant des transitions sur $\epsilon$ pour un état $q$ donné. Cette fonction est la \emph{fermeture sur epsilon} $ECLOSE : Q \rightarrow 2^Q$. Sa définition est inductive.\\

Soit $q$ un état dans $Q$.\\
\textbf{Cas de base} $q$ est dans ECLOSE($q$)\\
\textbf{Pas de récurrence} Si $p$ est dans ECLOSE($q$) et qu'il existe un état $r$ tel quel $r\in\delta(p,\epsilon)$, alors $r$ est dans ECLOSE($q$).\\

ECLOSE peut être utilisé indifféremment sur un ensemble d'états S ($ECLOSE : 2^Q \rightarrow 2^Q$). Alors, $ECLOSE(S)=\bigcup_{q\in S}ECLOSE(q)$.

\begin{example}[ECLOSE]\label{ex:anfclosure} Considérons l'automate $A$ de l'exemple \ref{ex:anf}. Les différentes fermetures peuvent être calculées :
	\begin{itemize}
		\item ECLOSE($q_0$) = $\{q_0,q_1,q_2\}$. En effet, $q_0$ appartient à sa fermeture, selon le cas de base. Aussi, $q_1,q_2\in\delta(q_0, \epsilon)$.
		\item ECLOSE($q_1$)=$\{q_1\}$ par le cas de base.
		\item ECLOSE($q_2$)=$\{q_2\}$ par le cas de base.
	\end{itemize}
\end{example}

\section{Langage d'un ADF}\label{adf:lang}Cette section s'intéresse à la notion de langage. Langages qui peuvent éventuellement être associés à un automate comme énoncé dans les sections \ref{adf:lang} pour un automate déterministe fini et \ref{fifo:lang} pour un automate à file. La section \ref{lr:def} en donne une définition. La section \ref{lr:regex} définit les expression régulières en faisant le lien avec les langages.

Un \emph{alphabet} $\Sigma$ est un ensemble fini et non vide de \emph{symboles}. Un \emph{mot} sur cet alphabet $\Sigma$ est une suite finie de $k$ éléments de $\Sigma$ notée $ w = a_1a_2\dots a_k$ où $k$ est un nombre naturel. $k$ est la \emph{longueur} de ce mot aussi notée $|w|=k$. Le \emph{mot vide} est un mot de taille $k=0$ noté $w=\epsilon$.

La \emph{concaténation} de deux mots $w=a_1a_2\dots a_k$ et $x=b_1b_2\dots b_j$ est l'opération consistant à créer un nouveau mot $wx=a_1a_2\dots a_kb_1b_2\dots b_j$ de longueur $i=k+j$.

\begin{proposition}[$\epsilon$ et la concaténation]\emph{
$\epsilon$ est \emph{l'identité pour la concaténation}, à savoir pour tout mot $w$, $w\epsilon = \epsilon w = w$.}
\end{proposition}

Cette proposition est triviale par la définition de la concaténation.

\emph{L'exponentiation} d'un symbole $a$ à la puissance $k$, notée $a^k$, retourne un mot de longueur $k$ obtenu par la concaténation de $k$ copies du symbole $a$. Noter que $a^0=\epsilon$. $\Sigma^k$ est \emph{l'ensemble des mots sur $\Sigma$} de longueur $k$. L'ensemble de \emph{tous les mots possibles sur $\Sigma$} est noté $\Sigma^* = \bigcup_{k=0}^{\infty}\Sigma^k$.


Un ensemble quelconque de mots sur $\Sigma$ est un \emph{langage}, noté $L \subseteq \Sigma^*$. Étant donné que $\Sigma^*$ est infini, $L$ peut l'être également.

\begin{example}[Langages] Voici des exemples utilisant plusieurs modes de définition. $\Sigma$ y est implicite mais peut être donné explicitement.
	\begin{itemize}
		\item $L=\{12,35,42,7,0\}$, un langage défini explicitement
		\item $L=\{0^k1^j|k+j=7\}$, les mots de 7 symboles sur $\Sigma=\{0,1\}$ commençant par zéro, un ou plusieurs $0$ et finissant par zéro, un ou plusieurs $1$. Ici, $L$ est donné par notation ensembliste
		\item $L$ contient tous les noms de villes belges. Ici $L$ est défini en langage courant.
		\item $\emptyset$ est un langage sur tout alphabet.
		\item $L=\{\epsilon\}$ ne contient que le mot vide, et est un langage sur tout alphabet.
	\end{itemize}
\end{example}

Soient $L$ et $M$ deux langages. Le langage $L \cup M = \{w | w \in L\vee w \in M\}$ est l'\emph{union} de ces deux langages. Il est composé des mots venant d'un des deux langages.

Le langage composé de tous les mots produits par la concaténation d'un mot de $L$ avec un mot de $M$ est une \emph{concaténation} de ces deux langages et s'écrit $LM = \{vw | v \in L \wedge w \in M\}$.

La \emph{fermeture} de $L$ est un langage constitué de tous les mots qui peuvent être construits par un concaténation d'un nombre arbitraire de mots de $L$, noté $L^*=\{w_1w_2\dots w_n|n\in \mathbb{N},\forall i \in \{1,2,\dots,n\}, w_i \in L\}$.

\section{Équivalence avec une expression régulière}\label{adf:eq}\input{adf/eq}
