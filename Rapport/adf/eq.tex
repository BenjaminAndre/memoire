

% ███████       ██     ██        █████
% ██           ██       ██      ██   ██
% █████       ██  █████  ██     ███████
% ██           ██       ██      ██   ██
% ███████       ██     ██       ██   ██

\subsection{Équivalence entre expression régulière et automate}\label{ss:e-a}

Certains problèmes peuvent être exprimés sous forme d'appartenance à un langage régulier, et par extension à une expression régulière.
Pouvoir convertir une expression réguliere en automate permet d'exécuter cet automate sur une machine pour tester l'appartenance. Ainsi, grâce à cette méthode, une grande classe de problèmes peut être résolue.


\begin{theorem}[ADF et expression régulière]
	Un langage peut être exprimé par un automate déterministe fini si et seulement si il peut être décrit par une expression régulière.
\end{theorem}

Ce théorème étant une double implication, il est vrai si les deux implications le sont. Étudions celles-ci séparemment.


\begin{theorem}[ADF $\implies$ expression régulière]
	Soit un langage $L$. Il existe un automate déterministe $A$ tel que $L(A)=L$ $\implies$ il existe une expression régulière $E$ telle que $L(E)=L$.
\end{theorem}
\begin{proof}
	Soit un langage $L$. Supposons qu'il existe un ADF \automaton tel que $L(A)=L$. $Q$ étant un ensemble fini, on peut définir sa cardinalité : $|Q|=n$. Supposons que ses états soient nommés $\{1,2,\dots,n\}$. Il est possible de construire des expressions régulières par induction sur le nombre d'états considérés. De plus, un tel automate est aisément représenté dans un ordre séquentiel, de gauche à droite. Ceci permet de séparer visuellement les $k$ premiers états du reste.

  Soient $i,j,k \in Q$, tous équivalents à des nombres inférieurs ou égaux à $n$. Définissons $E_{i,j}^k$ comme étant l'expression régulière décrivant un langage constitué des mots $w$ tels que $\hdelta(i,w)=j$ et qu'aucun état intermédiaire n'ait un nombre strictement supérieur à $k$.

	\begin{figure}[H]
		\centering
		\begin{tikzpicture}[->,>=stealth',shorten >=1pt,auto,node distance=2cm, semithick, bend angle=10]

		\tikzstyle{every state}=[circle]

		\node[initial,state] (A)                    {$1$};
		\node[state]         (B) [right of=A] {$2$};
		\node[state]         (C)  [right of=B] {$3$};
		\node[accepting, state]         (D)  [right of=C] {$4$};
		\node[state]         (E)  [right of=D]       {$5$};

		\path
		(A) edge [bend left=45] node{a} (C)
		(A) edge  node{b} (B)

		(B) edge [bend left=45] node{a} (A)
		(B) edge node{b} (C)

		(C) edge [bend left=45,dashed] node{a} (E)
		(C) edge [loop below] node{b} (C)

		(D) edge [loop below,dashed] node{a,b} (D)
		(E) edge [,dashed] node{a,b} (D)

		;
		\end{tikzpicture}
		\caption{Exemple : automate mettant $E_{1,3}^3$ en évidence. Les mots représentés par un chemin passant par un arc discontinu n'appartiennent par à $E_{1,3}^3$ : un des états intermédiaires est dénommé par un nombre supérieur à $k=3$.}\label{fig:proofeijk}
	\end{figure}


	L'exemple ci-dessus illustre ce fait qu'aucun état supérieur à $k$ ne peut faire partie des états intermédiaires. Intuitivement, il s'agit d'un automate auquel on a enlevé les transitions :
  \begin{itemize}
    \item Allant de $i$ à un nombre supérieur à $k$ (sauf $j$)
    \item Entre deux nombres supérieurs à $k$
    \item Allant d'un nombre supérieur à $k$ (sauf $i$) à $j$
  \end{itemize}
  En pratiques, celles-ci ne sont juste pas considérées lors de l'application de $\delta$.

	\paragraph{Cas de base} $k=0$. Comme tout état est numéroté $1$ ou plus, aucun intermédiaire n'est accepté.

  Une possibilité est $i \neq j$. Alors les chemins possibles ne se composent que d'un arc allant directement de $i$ à $j$. Pour les construire :

	Pour chaque paire $i$, $j$ :
	\begin{itemize}
		\item Il n'existe pas de symbole $a$ tel que $\delta(i,a)=j$. Alors, $E_{ij}^0=\emptyset$
		\item Il existe un unique symbole $a$ tel que  $\delta(i,a)=j$. Alors, $E_{ij}^0=a$
		\item Il existe des symboles $a_1,a_2,\dots,a_k$ tels que $\forall l \in \{1,\dots, k\}, \delta(i,a_l)=j$. Alors, $E_{ij}^0=a_1+a_2+\dots+a_k$
	\end{itemize}


  Une autre possibilité est $i=j$ et indique un chemin de longueur $0$. Auquel cas l'expression régulière représentant un chemin sans symbole est $\epsilon$. Ce chemin doit être ajouté au langage décrit par $E^k_{i,j}$ si $i=j$.


	\paragraph{Pas de récurrence} Supposons qu'il existe un chemin allant de $i$ à $j$ ne passant par aucun état ayant un numéro supérieur à $k$.	La première possibilité est que le-dit chemin ne passe pas par $k$. Alors, le mot représenté par ce chemin fait partie du langage de $E_{ij}^{k-1}$. Seconde possibilité, le chemin passe par $k$ une ou plusieurs fois comme représenté à la figure \ref{fig:ikjpath}.

	\begin{figure}[H]\centering
	\begin{tikzpicture}[->,>=stealth',shorten >=1pt,auto,node distance=2cm, semithick, bend angle=10]

	\tikzstyle{every state}=[circle]

	\node[state] (A) {$i$};
	\node[state] (B) [right of=A] {$k$};
	\node[state] (C) [right of=B] {$k$};
	\node[state] (D) [right of=C] {$k$};
	\node[state] (E) [right of=D] {$k$};
	\node[state] (F) [right of=E] {$j$};

	\path
	(A) edge [snake] (B)
	(B) edge [snake] (C)
	(C) edge [snake] (D)
	(D) edge [snake] (E)
	(E) edge [snake] (F)
	;

	\draw[-] (0.3,-0.8) arc (-100:-80:4.5) ;
	\draw[-] (2.24,-0.8) arc (-100:-80:16) ;
	\draw[-] (8.3,-0.8) arc (-100:-80:4.5) ;


	\node[draw=none] at (1.25,-1.4) {Dans $E_{ik}^{k-1}$};
	\node[draw=none] at (5.25,-1.4) {Mots dans $E_{kk}^{k-1}$};
	\node[draw=none] at (9.25,-1.4) {Dans $E_{kj}^{k-1}$};

	\end{tikzpicture}
	\caption{Un chemin de $i$ à $j$ peut être découpé en différent segment en fonction de $k$}\label{fig:ikjpath}
	\end{figure}

	Auquel cas, ces chemins sont composés d'un sous-chemin donnant un mot dans $E_{ik}^{k-1}$, suivi d'un sous-chemin donnant un ou plusieurs mots dans $E_{kk}^{k-1}$ et finalement un mot dans $E_{kj}^{k-1}$.

	En combinant les expressions des deux types, on obtient :
	$$
	E_{ij}^k = E_{ij}^{k-1}+E_{ik}^{k-1}(E_{kk}^{k-1})^*E_{kj}^{k-1}
	$$

	En commençant cette construction sur $E_{ij}^n$, comme l'appel se fait toujours à des chaînes plus courtes, éventuellement on retombe sur le cas de base. Si l'état initial est numéroté 1, alors l'expression régulière $E$ exprimant $L$ est l'union ($+$) des $E_{1j}^n$ tels que $j$ est un état acceptant.

	\hfill$\square$
\end{proof}
\stepcounter{algo}
\begin{complexity}
	Soit un ADF \automaton comportant $n$ états. Pour connaître la complexité totale de cet algorithme, il faut connaître le nombre total d'expressions régulières construites et la longueur de chacune de celles-ci.

	A chacune des $n$ itérations (ajoutant progressivement des nouveaux états admis pour état intermédiaire), la longueur de l'expression peut quadrupler : elle est exprimée par 4 facteurs. Ainsi, après $n$ étapes, cette expression peut être de taille $\mathcal{O}(4^n)$.

	Le nombre d'expressions à construire, lui, est décomposable en deux facteurs également : le nombre d'itérations et celui de paires $i,j$ possibles. Le premier facteur est $n$, quand aux paires, leur nombre s'exprime par $n^2$. $n^3$ expressions sont construites.

	En regroupant ces deux facteurs, on obtient $n^3\mathcal{O}(4^n)=\mathcal{O}(n^34^n)=\mathcal{O}(2^n)$. Comme $n$ correspond au nombre d'états, si la transformation se fait depuis un ANF, via un ADF, vers une expression régulière, la complexité devient doublement exponentielle. La première transformation étant elle-même exponentielle en le nombre d'états de l'ANF.
\end{complexity}


\begin{example}[Construction d'une expression régulière]
	Construction d'une expression régulière à partir de l'automate de la figure suivante :

	\begin{figure}[H]\centering
		\begin{tikzpicture}[->,>=stealth',shorten >=1pt,auto,node distance=2cm, semithick, bend angle=10]

		\tikzstyle{every state}=[circle]

		\node[initial,state] (A) {$1$};
		\node[accepting,state] (B) [right of=A] {$2$};

		\path
		(A) edge [bend left=20] node{0} (B)
		(B) edge [bend left=20] node{0} (A)

		(A) edge [loop above] node{1} (A)
		(B) edge [loop above] node{1} (B)
		;

		\end{tikzpicture}
		\caption{Un automate acceptant tout mot ayant un nombre impair de $0$}
	\end{figure}

	La construction par récurrence commençant avec $k=0$ le processus peut être représenté par des tableaux correspondant à différents $k$ de façon croissante.

	\paragraph{Première itération} Dans la première itération, chaque expression se résume à un des trois cas de base, avec éventuellement $\epsilon$ si $i=j$ pour l'expression analysée. Ici, $k=0$.

	\begin{figure}[H]
		\centering
		\begin{tabular}{|l|c|}
			\hline
			 & Cas de base\\
			\hline
			$E_{11}^0$& $1+\epsilon$\\
			$E_{12}^0$& $0$\\
			$E_{21}^0$& $0$\\
			$E_{22}^0$& $1+\epsilon$\\
			\hline
		\end{tabular}
	\end{figure}

	\paragraph{Seconde itération} Ensuite, l'état $1$ est autorisé comme état intermédiaire : $k=1$. Ayant potentiellement un état intermédiaire, la formule de récurrence est utilisée.

	\begin{figure}[H]
		\centering
		\begin{tabular}{|l|c|c|c|}
			\hline
			 & Formule de récurrence & Détail & Simplification\\
			\hline
			$E_{11}^1$& $E_{11}^0 + E_{11}^0(E_{11}^0)^*E_{11}^0$&
			$(1+\epsilon)+(1+\epsilon)(1+\epsilon)^*(1+\epsilon)$ & $1^*$\\
			$E_{12}^1$& $E_{12}^0 + E_{11}^0(E_{11}^0)^*E_{12}^0$&
			$0+(1+\epsilon)(1+\epsilon)^*0$ & $1^*0$ \\
			$E_{21}^1$& $E_{21}^0 + E_{21}^0(E_{11}^0)^*E_{11}^0$&
			$0+0(1+\epsilon)^*(1+\epsilon)$& $01^*$\\
			$E_{22}^1$& $E_{22}^0 + E_{21}^0(E_{11}^0)^*E_{12}^0$&
			$(1+\epsilon)+0(1+\epsilon)^*0$ & $\epsilon+1+01^*0$\\
			\hline
		\end{tabular}
	\end{figure}


	\paragraph{Troisième itération} A la troisième itération, l'état $2$ est autorisé comme état intermédiaire($k=2$).

	\begin{figure}[H]
    \centering
		\hspace{-5mm}\begin{tabular}{|l|c|c|}
			\hline
			 & Résultat\\
			\hline
			$E_{11}^2$&$1^*+1^*0(1+01^*0)^*01^*$\\
			$E_{12}^2$&$1^*0(1+01^*0)^*$\\
			$E_{21}^2$&$(1+01^*0)^*01^*$\\
			$E_{22}^2$&$(1+01^*0)^*$\\
			\hline
		\end{tabular}
	\end{figure}

	Pour obtenir une expression régulière correspondant à l'automate, on s'intéresse à celle qui décrit un chemin entre l'état initial ($1$) et les états acceptants (uniquement $2$ ici). Dès lors,  $E^2_{12}=1^*0(1+01^*0)^*=L$.

	Cette expression régulière $1^*0(1+01^*0)^*$ décrit bien un nombre impair de $0$. Il en faut absolument un, et tout ajout supplémentaire se fait par paire.

\end{example}



\begin{theorem}[Expression régulière $\implies$ ADF]
	($\Leftarrow$) Soit un langage $L$. Il existe une expression régulière $E$ telle que $L(E)=L$ $\implies$  il existe un automate déterministe $A$ tel que $L(A)=L$.
\end{theorem}


\begin{proof}
	Comme tout $\epsilon$-ANF a un ADF équivalent (théorème \ref{anf-dnf}), montrer qu'une expression régulière $E$ a un ANF équivalent est suffisant pour obtenir cet ADF.

	Soit un langage $L$. Soit $E$ une expression régulière telle que $L(E)=L$. On peut construire l'automate récursivement sur la définition des expressions régulières à la section \ref{ss:regex}. Cette preuve par récurrence repose sur trois invariants portant sur chaque ANF construit :
	\begin{enumerate}
		\item Il y a un unique état acceptant
		\item Aucune transition ne mène à l'état initial
		\item Aucune transition ne part de l'état acceptant
	\end{enumerate}
  Intuitivement, ces invariants servent à assurer que l'automate ainsi créé est lu de gauche à droite tel une expression régulière.

	\paragraph{Cas de base}	Les ANFs de la figure \ref{fig:regexadfbase} représentent les automates correspondant aux trois cas de base.

	\begin{figure}[H]

	\begin{subfigure}{.33\textwidth}\centering
		\begin{tikzpicture}[->,>=stealth',shorten >=1pt,auto,node distance=4cm, semithick, bend angle=10,initial text= ]

		\tikzstyle{every state}=[circle]

		\node[initial,state,scale=0.5] (A) {};
		\node[accepting,state,scale=0.5] (B) [right of=A] {};

		\path
		(A) edge  node{$\epsilon$} (B)
		;
		\node[draw, fit=(A) (B)] {};

		\end{tikzpicture}
		\caption{$L=\{\epsilon\}$}
	\end{subfigure}
	\begin{subfigure}{.33\textwidth}\centering
		\begin{tikzpicture}[->,>=stealth',shorten >=1pt,auto,node distance=4cm, semithick, bend angle=10,initial text= ]

		\tikzstyle{every state}=[circle]

		\node[initial,state,scale=0.5] (A) {};
		\node[accepting,state,scale=0.5] (B) [right of=A] {};

		\node[draw, fit=(A) (B)] {};

		\end{tikzpicture}
		\caption{$L=\emptyset$}
	\end{subfigure}
	\begin{subfigure}{.33\textwidth}\centering
		\begin{tikzpicture}[->,>=stealth',shorten >=1pt,auto,node distance=4cm, semithick, bend angle=10,initial text= ]

		\tikzstyle{every state}=[circle]

		\node[initial,state,scale=0.5] (A) {};
		\node[accepting,state,scale=0.5] (B) [right of=A] {};

		\path
		(A) edge  node{$a$} (B)
		;
		\node[draw, fit=(A) (B)] {};

		\end{tikzpicture}
		\caption{$L=\{a\}$}
	\end{subfigure}

	\caption{Blocs de base pour la construction d'un automate à partir d'une expression régulière}
	\label{fig:regexadfbase}
	\end{figure}


	En effet, l'automate (a) représente le langage $\{\epsilon\}$ égal à $L(E)$ : le seul arc de l'état initial à un état final est $\epsilon$. L'automate (b) ne propose pas d'arc atteignant l'état final. Aucun mot n'appartient au langage représenté par cet automate qui vaut donc $\emptyset=L(\emptyset)$. Finalement, (c) propose un arc pour $a$. Dès lors, il existe un unique chemin de longueur 1 correspondant au mot $a$. Ainsi, le langage exprimé par cet automate $\{a\}$ est bien égal à $L(E)=L(a)$. De plus, ces automates respectent bien l'invariant de récurrence proposé.

	\paragraph{Pas de récurrence} Les ANF \emph{ abstraits} de la figure \ref{fig:regexadfrec} représentent la façon dont un automate peut être construit récursivement en fonction des règles de récurrence des expressions régulières. Ces ANF sont abstraits car le contenu d'un bloc $E$ ou $F$ n'est pas représenté explicitement. Cependant, celui-ci respecte les invariants de récurrence.

	\begin{figure}[H]

			\hspace{0.2\textwidth}\begin{subfigure}{.6\textwidth}\centering
				\begin{tikzpicture}[->,>=stealth',shorten >=1pt,auto,node distance=4cm, semithick, bend angle=10,initial text= ]

				\tikzstyle{every state}=[circle]

				\node[initial,state,scale=0.5] (A) {};
				\node[state,scale=0.5] (B) [right of=A] {};
				\node[state,scale=0.5] (C) [right of=B] {};
				\node[accepting,state,scale=0.5] (D) [right of=C] {};

				\node[draw=none] (K) [right= 0.5cm of B] {$E$};

				\path
				(A) edge  node{$\epsilon$} (B)
				(C) edge  node{$\epsilon$} (D)

				(C) edge[bend right=90]  node{$\epsilon$} (B)
				(A) edge[bend right=40]  node{$\epsilon$} (D)

				;


				\node[draw, fit=(B) (C)] {};

				\end{tikzpicture}
				\caption{$L=L(E)^*$}
			\end{subfigure}\newline\vspace{1cm}


			\hspace{0.2\textwidth}\begin{subfigure}{.6\textwidth}\centering
				\begin{tikzpicture}[->,>=stealth',shorten >=1pt,auto,node distance=4cm, semithick, bend angle=10,initial text= ]

				\tikzstyle{every state}=[circle]

				\node[initial,state,scale=0.5] (A) {};
				\node[state, scale=0.5] (B) [right of=A] {};
				\node[state, scale=0.5] (C) [right of=B] {};
				\node[accepting,state,scale=0.5] (D) [right of=C] {};


				\node[draw=none] (K) [right= 0.5cm of A] {$E$};
				\node[draw=none] (L) [right= 0.5cm of C] {$F$};

				\path
				(B) edge  node{$\epsilon$} (C)
				;


				\node[draw, fit=(A) (B)] {};
				\node[draw, fit=(C) (D)] {};



				\end{tikzpicture}
				\caption{$L=L(E)L(F)$}
			\end{subfigure}\newline\vspace{1cm}


			\hspace{0.2\textwidth}\begin{subfigure}{.6\textwidth}\centering
			\begin{tikzpicture}[->,>=stealth',shorten >=1pt,auto,node distance=4cm, semithick, bend angle=10,initial text= ]

			\tikzstyle{every state}=[circle]

			\node[initial,state,scale=0.5] (A) {};

			\node[state,scale=0.5] (B) [above right of=A] {};
			\node[state,scale=0.5] (C) [below right of=A] {};

			\node[draw=none] (K) [right= 0.5cm of B] {$E$};
			\node[draw=none] (L) [right= 0.5cm of C] {$F$};

			\node[state,scale=0.5] (D) [right of=B] {};
			\node[state,scale=0.5] (E) [right of=C] {};

			\node[accepting,state,scale=0.5] (F) [below right of=D] {};

			\path
			(A) edge  node{$\epsilon$} (B)
			(A) edge  node{$\epsilon$} (C)

			(D) edge  node{$\epsilon$} (F)
			(E) edge  node{$\epsilon$} (F)
			;
			\node[draw, fit=(B)(D)] {};
			\node[draw, fit=(C)(E)] {};

			\end{tikzpicture}
			\caption{$L=L(E)+L(F)$}
		\end{subfigure}





		\caption{Enchaînement de blocs pour une construction récursive}
		\label{fig:regexadfrec}
	\end{figure}

	Les quatre règles de récurrence sur une expression régulière permettent de construire les automates:
	\begin{itemize}
    \item $L((E))=L(E)$ ne necéssite pas de construction supplémentaire.
    \item $L(E^*)=L(E)^*$ est construit comme en (a). En effet, l'arc revenant au début de $E$ permet d'exprimer $E^1$,$E^2$,$E^3$,...
    \item $L(EF)=L(E)L(F)$ est construit comme en (b). En effet, tout mot de cet automate est de la forme $v\epsilon w$ avec $v\in L(E)$ et $w\in L(F)$.
    \item $L(E+F)=L(E)\bigcup L(F)$ est construit comme en (c). En effet, tout mot de cet automate est de la forme $\epsilon v\epsilon$ ou $\epsilon w \epsilon$ avec $v\in L(E)$ et $w\in L(F)$, en accord avec la définition de l'union ensembliste.
	\end{itemize}

	Les automates (a), (b) et (c) respectent bien l'invariant de récurrence : pas de transition vers l'état initial, un seul état acceptant n'ayant pas de transition sortante. Chaque automate abstrait pour $E$ ou $F$ peut lui même être construit récursivement jusqu'au cas de base.

	\hfill$\square$
\end{proof}
\stepcounter{algo}
\begin{complexity}
	Soit une expression régulière $E$ contenant $n$ symboles (alphabet et opérations comprises) représentant un langage $L=L(E)$.
  La construction d'un ANF pour $L$ peut se faire en $\mathcal{O}(n)$. En effet :
	\begin{itemize}
		\item Cas de base : Au plus $n$ ANFs sont créés. Chacun correspond à un symbole (opérations non comprises). Chaque ANF a un état, rendant la création en temps constant. La création de tous ces ANFs est alors en $\mathcal{O}(n)$
		\item Récurrence : Il reste au plus $n$ symboles correspondant à des opérations, ce qui implique au plus $n$ opérations. Chaque opération se base sur les 4 règles de récurrences définies précédemment. Dans les cas necéssitant une construction, celle-ci peut se faire en temps constant (ajout d'au plus deux états et quatre transitions). Chaque opération n'est à effectuer qu'une seule fois, consommant le symbole, et que chacune se fait en temps constant. Dès lors, la totalité des étapes de récurrence se fait au plus en $\mathcal{O}(n)$
	\end{itemize}

	La complexité totale de cette conversion est en $\mathcal{O}(n)$ vers un ANF. La conversion vers un ADF, comme mentionné dans la section ci-après peut quand à elle être exponentielle.

\end{complexity}


%  █████  ██████  ███████       ██     ██        █████  ███    ██ ███████
% ██   ██ ██   ██ ██           ██       ██      ██   ██ ████   ██ ██
% ███████ ██   ██ █████       ██  █████  ██     ███████ ██ ██  ██ █████
% ██   ██ ██   ██ ██           ██       ██      ██   ██ ██  ██ ██ ██
% ██   ██ ██████  ██            ██     ██       ██   ██ ██   ████ ██


\subsection{Équivalence entre un automate déterministe fini et un automate non-déterministe fini}\label{ss:eqadfanf}
Cette section présente une méthode permettant de créer un ADF à partir d'un ANF et réciproquement. Ici, ANF est considéré au sens-large et peut tout aussi bien être un ANF normal qu'un $\epsilon$-ANF. Ceci permet de justifier l'abstraction faite entre les différents types d'automates finis et l'utilisation du même terme pour tous ceux-ci.

Soit un ANF \automaton. Alors l'ADF équivalent
$$
D=(Q_D, \Sigma, \delta_D, q_D, F_D)
$$
est défini par :
\begin{itemize}
	\item $Q_D = \{T | T=\text{ECLOSE(}S\text{)} \text{ et } S \subseteq Q\}$. Concrètement, $Q_D$ est l'ensemble des parties des $Q$ fermées sur $\epsilon$. Ceci qui signifie que chaque transition sur $\epsilon$ depuis un état de $T$ mène à un état également dans $T$. L'ensemble $\emptyset$ est fermé sur $\epsilon$.
	\item $q_D$=ECLOSE($q_0$). L'état initial de $D$ est l'ensemble des états dans la fermeture sur $\epsilon$ des états de $A$.
	\item $F_D= \{T|T \in Q_D \text{ et } T \cap F \neq \emptyset\}$ contient les ensembles dont au moins un état est acceptant pour $A$.
	\item $\delta_D(T,a)$ est construit, $\forall a \in \Sigma, \forall T \in Q_D$ par :
		\begin{enumerate}
			\item Soit $T=\{p_1, p_2,\dots,p_k\}$.
			\item Calculer $\bigcup_{i=1}^k\delta(p_i,a)$. Renommer cet ensemble en $\{r_1, r_2, \dots, r_m\}$.
			\item Alors $\delta_D(T,a)=\bigcup_{j=1}^m\text{ECLOSE(}r_j\text{)}$.
		\end{enumerate}
\end{itemize}


\begin{example}[Transformation ANF vers ADF]\label{ex:anfadf} Considérons l'automate \automaton de l'exemple \ref{ex:anf} et les fermetures calculées dans l'exemple \ref{ex:anfclosure}.

Alors, l'automate $D=(Q_D, \Sigma, \delta_D, q_D, F_D)$ est donné par :

\begin{itemize}
	\item $Q_D=\{\emptyset, \{q_1\}, \{q_2\}, \{q_1,q_2\}, \{q_0,q_1,q_2\}\}$. Les ensembles $\{q_0,q_1\}$ et $\{q_0,q_2\}$ sont des sous-ensembles de $Q$ mais ne sont pas fermés sur $\epsilon$.
	\item $q_D=\{q_0,q_1,q_2\}=\text{ECLOSE(}q_0\text{)}$.
	\item $F_D=\{\{q_2\}, \{q_1,q_2\}, \{q_0,q_1,q_2\}\}$, les ensembles contenant $q_2$, étant acceptant de $A$.
	\item $\delta_D$ est exprimé sur le graphe de la figure \ref{fig:dndf}.
\end{itemize}


\begin{figure}[H]
	\centering
	\begin{tikzpicture}[->,>=stealth',shorten >=1pt,auto,node distance=4cm, semithick, bend angle=10]

	\tikzstyle{every state}=[circle]

  \node[initial,state,accepting] (A) {$\{q_0,q_1,q_2\}$};
  \node[state,accepting] (B) [right of=A] {$\{q_0,q_1\}$};
  \node[state,accepting] (C) [right of=B] {$\{q_2\}$};
  \node[state] (D) [below of=B] {$\{q_1\}$};
  \node[state] (E) [below of=C] {$\emptyset$};


	\path
  (A) edge [loop above] node{a,c} (A)
  (A) edge [bend left] node{b} (B)

  (B) edge [bend left] node{a,c} (A)
  (B) edge node{b} (C)

  (C) edge node{a,b,c} (E)

  (D) edge node{a,c} (A)
  (D) edge node{b} (C)

  (E) edge [loop below] node{a,b,c} (E)
	;
	\end{tikzpicture}
  \begin{tikzpicture}[->,>=stealth',shorten >=1pt,auto,node distance=4cm, semithick, bend angle=10]

	\tikzstyle{every state}=[circle]

  \node[initial,state,accepting] (A) {$A$};
  \node[state,accepting] (B) [right of=A] {$B$};
  \node[state,accepting] (C) [right of=B] {$C$};


	\path
  (A) edge [loop above] node{a,c} (A)
  (A) edge [bend left] node{b} (B)

  (B) edge [bend left] node{a,c} (A)
  (B) edge node{b} (C)
	;
	\end{tikzpicture}
	\caption{Automate $D$. De par la construction de $Q$, le nombre d'états de $D$ est exponentiel. Les états inatteignables et $\emptyset$ sont souvent omis pour clarifier la représentation.}\label{fig:dndf}
\end{figure}
\end{example}



\begin{theorem}[ANF $\Leftrightarrow$ ADF]\label{anf-dnf}
	Un langage $L$ peut être représenté par un ANF si et seulement si il peut l'être par un ADF.
\end{theorem}

\begin{proof}
	 Soit $L$ un langage. Cette preuve étant une double implication, chacune peut être prouvée séparément.

	($\Leftarrow$) $L$ peut être représenté par un ADF $\implies$ $L$ peut être représenté par un ANF. Supposons qu'un automate $D=(Q_D, \Sigma, \delta_D, q_D, F_D)$ représente $L$ : $L(D)=L$.
	L'ANF \automaton correspondant est construit comme suit :

	\begin{itemize}
		\item $Q=\{\{q\}|q\in Q_D\}\bigcup\emptyset$
		\item $\delta$ contient les transitions de $D$ modifiées. Les objets retournés deviennent des ensembles d'états. C'est-à-dire, si $\delta_D(q,a)=p$ alors $\delta(\{q\},a)=\{p\}$. De plus, pour chaque état $q\in Q_D$, $\delta(\{q\},\epsilon)=\emptyset$.
		\item $q_0=\{q_D\}$
		\item $F=\{\{q\}| q\in F_D\}$
	\end{itemize}

	 Dès lors, les transitions sont les mêmes entre $D$ et $A$, mais $A$ précise explicitement qu'il n'y a pas de transition sur $\epsilon$. Comme $A$ représente le même langage, il existe donc bien un ANF qui représente $L$.


	($\Rightarrow$) $L$ peut être représenté par un ANF $\implies$ $L$ peut être représenté par un ADF. Soit l'automate \automaton. Supposons qu'il représente $L=L(A)$. Considérons l'automate obtenu par la transformation détaillée à la section précédente (page \pageref{ex:anfadf}) :
	$$
	D=(Q_D, \Sigma, \delta_D, q_D, F_D)
	$$
	Montrons que $L(D)=L(A)$. Pour ce faire, montrons que les fonctions de transition étendues sont équivalentes. Auquel cas, les chemins sont équivalents et donc les langages sont égaux.
	Montrons que $\hdelta(q_0,w)=\hdelta_D(q_D,w)$ pour tout mot $w$, par récurrence sur $w$.

	\paragraph{Cas de base} Si $|w|=0$, alors $w=\epsilon$. $\hdelta(q_0,\epsilon)=$ ECLOSE($q_0$), par définition de la fonction de transition étendue. $q_D$=ECLOSE($q_0$) par la construction de $q_D$. Pour un ADF (ici, $D$), $\hdelta(p,\epsilon)=p$, pour tout état $p$. Par conséquent, $\hdelta_D(q_D,\epsilon)=q_D=\text{ECLOSE(}q_0\text{)}=\hdelta(q_0,\epsilon)$.

	\paragraph{Pas de récurrence} Supposons $w=xa$ avec $a$ le dernier symbole de $w$. Notre hypothèse de récurrence est que $\hdelta_D(q_D,x)=\hdelta(q_0,x)$. Ce sont bien les mêmes objets car $\hdelta_D$ retourne un état de $D$ qui correspond à un ensemble d'états de $A$. Notons celui-ci $\{p_1,p_2, \dots, p_k\}$. Par définition de $\hdelta$ pour un ANF, $\hdelta(q_0,w)$ est obtenu en :

	\begin{enumerate}
		\item Construisant $\{r_1,r_2,\dots, r_m\}=\bigcup_{i=1}^k \delta(p_i,a)$. Cet ensemble correspond aux états obtenus par la lecture du symbole $a$ à partir de $\{p_1,p_2,\dots,p_k\}$.
		\item Calculant $\hdelta(q_0,w)=\bigcup_{j=1}^m$ECLOSE($r_j$). Un état atteint par la lecture de $a$ l'est aussi par $a\epsilon$.
	\end{enumerate}

	$D$ a été construit avec ces deux mêmes étapes pour $\delta_D(\{p_1,p_2,\dots, p_k\},a)$. Dès lors, $\hdelta_D(q_D,w)=\delta_D(\{p_1,p_2,\dots, p_k\},a)=\bigcup_{j=1}^k$ECLOSE($p_j$)$=\hdelta(q_0,w)$.

	On a bien $\hdelta_D(q_D,w)=\hdelta(q_0,w)$.

  Les langages sont donc bien égaux.

	\hfill$\square$
\end{proof}

\stepcounter{algo}%TODO because we don't exactly have an algorithm here

\begin{complexity}[Conversion ANF vers ADF]

	La complexité d'une conversion ANF vers ADF peut être exprimée en fonction de $n$ le nombre d'états de l'ANF. La taille de l'alphabet $\Sigma$ est ici comptée comme une constante $k$. Elle est ignorée dans l'analyse grand-O. L'algorithme de conversion se fait en deux étapes. Le calcul de ECLOSE et la construction à proprement parler. Ici, l'automate est stocké sous forme d'une table de transitions. Cette solution est plus facile à manipuler mais peut engendrer un surcoût en mémoire, qui n'est pas analysé ici.

	\begin{itemize}
		\item ECLOSE : Chacun des $n$ états ayant une entrée pour $\epsilon$ dans la fonction $\delta$, le temps de calcul sur chaque nœud ajouté est en temps constant. Chacune des $n$ fermetures pour chacun des $n$ états $q \in Q$ pouvant au plus compter les $n$ états, le temps total de cette opération est en $n\mathcal{O}(n)=\mathcal{O}(n^2)$.

		\item Construction : Posons $s$ le nombre d'états dans l'ADF (qui, dans le pire des cas vaut $s=2^n$ par la construction des sous-ensembles). La création d'un état de l'ADF est en $\mathcal{O}(n)$. En effet, il faut garder des références vers les états de l'ANF concernés. Ceux-ci sont au plus $n$.

    Comme il y a $s$ états dans l'ADF, il y a $ks$ transitions. Chacune de celle-ci peut être construite en $\mathcal{O}(n)$. En effet, chaque état de l'ADF étant constitué d'au plus $n$ états de l'ANF, il y a au plus $n$ transitions à suivre pour obtenir l'ensemble d'états résultant dans l'ANF. Cet ensemble correspond alors à un état de l'ADF obtenu. Les transitions sont construites en $\mathcal{O}(nks)=\mathcal{O}(nks)$. $k$ est toujours considéré comme une constante.

	\end{itemize}

	La complexité dans le pire des cas est $\mathcal{O}(n^2) + \mathcal{O}(sn) + \mathcal{O}(sn) = \mathcal{O}(sn)=\mathcal{O}(n2^n)$.
	Le détail est donné sur $s$ car, comme prouvé dans \cite{Hopcroft00}, en pratique le nombre d'états dans l'ADF obtenu est rarement de l'ordre de $2^n$, typiquement de l'ordre de $n$. Dans ce cas là, la complexité devient $\mathcal{O}(n^2)$.

\end{complexity}


\begin{complexity}[Conversion ADF vers ANF]
	La conversion d'un ADF \automaton vers un ANF consiste au remplacement de $n$ états par $n$ ensembles d'un seul état. Chaque copie individuelle étant en temps constant, cette opération est en $\mathcal{O}(n)$.
  Ensuite, une nouvelle table de transition doit être créée. Si l'alphabet $\Sigma$ est de taille $k$, celle-ci a toujours $n$ lignes mais $k+1$ colonnes. En effet, un colonne est ajoutée pour $\epsilon$. La création de cette nouvelle table se fait alors en $\mathcal{O}(kn)$.
	La complexité totale d'une conversion d'un ADF vers un ANF est en $\mathcal{O}(kn)$.
\end{complexity}
