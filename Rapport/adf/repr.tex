Le \emph{graphe d'un automate fini} \automaton est un graphe dirigé construit comme suit :

\begin{itemize}
  \item Chaque état de $Q$ est représenté par un nœud.
  \item Chaque transition $\delta(q,a)$ est représenté par un arc étiqueté $a$. Dans le cas d'un automate non-déterministe, un arc existe pour chacun des états obtenus en suivant la transition. Si il y a plusieurs transitions sortant d'un même état et entrant dans un même autre état, les arcs peuvent être fusionnés en listant les étiquettes.
  \item L'état initial est mis en évidence par une flèche entrante.
  \item Les états acceptants sont représentés par un double cercle, en opposition au simple cercle des autres nœuds.
\end{itemize}

\begin{example}[Graphe d'automate]
 Voici les graphes représentant les automates définis dans les tables \ref{table:transdelta} et \ref{table:eanfdelta} :

 \begin{minipage}[t]{0.5\textwidth}
   \begin{figure}[H]
    \centering
    \begin{tikzpicture}[->,>=stealth',shorten >=1pt,auto,node distance=2.5cm, semithick, bend angle=10]

    \tikzstyle{every state}=[circle]

    \node[initial,state] (A)                    {$q_0$};
    \node[state]         (B) [below right of=A] {$q_1$};
    \node[state]         (C) [below left of=A] {$q_2$};
    \node[accepting, state]         (D) [below right of=B] {$q_3$};
    \node[state]         (E) [below left of=C]       {$q_4$};
    \node[state]         (G) [below right of=E]       {$q_6$};
    \node[state]         (F) [above right of=G]       {$q_5$};

    \path 	(A) 	edge              node {a} (C)
    edge              node {b} (B)
    (B) 	edge              node {a} (D)
    edge [bend left]  node {b} (F)
    (C) 	edge              node {a} (E)
    edge              node {b} (F)
    (D) 	edge [loop above] node {a,b} (D)
    (E) 	edge [loop above] node {a,b} (E)
    (F) 	edge              node {a} (D)
    edge [bend left]  node {b} (B)
    (G) 	edge              node {a} (E)
    edge              node {b} (F);
    \end{tikzpicture}
    \caption{Graphe de l'ADF de la table \ref{table:transdelta}}\label{fig:a1}
   \end{figure}
 \end{minipage}
 \begin{minipage}[t]{0.5\textwidth}
   \begin{figure}[H]
   	\centering
   	\begin{tikzpicture}[->,>=stealth',shorten >=1pt,auto,node distance=2.5cm, semithick, bend angle=10]

   	\tikzstyle{every state}=[circle]

   	\node[initial,state] (A)                    {$q_0$};
   	\node[state]         (B) [above right of=A] {$q_1$};
   	\node[accepting, state]         (C) [below right of=A] {$q_2$};

   	\path
   	(A) edge [bend left] node{$\epsilon$,b} (B)
   	(A) edge node{$\epsilon$,c} (C)
   	(B) edge [bend left] node{a,c} (A)
   	(B) edge [loop right] node{c} (B)
   	(B) edge node{b} (C);
   	\end{tikzpicture}
   	\caption{Graphe du $\epsilon$-ANF de la table \ref{table:eanfdelta}}\label{fig:eanf}
   \end{figure}
\end{minipage}
\end{example}

Cette représentation a l'avantage d'être plus visuelle, alors que la table de transition est plus structurée.
