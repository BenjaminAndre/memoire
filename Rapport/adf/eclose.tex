Cette sous-section introduit l'algorithme ECLOSE. Cet algorithme concerne les $\epsilon$-ANFs. Il permet, à partir d'un état spécifique, de calculer l'ensemble des états atteignables uniquement par des transitions sur $\epsilon$. Ce calcul sert notemment au test d'appartenance d'un mot à un langage défini par un $\epsilon$-ANF comme présenté à la section \ref{ss:lang}.

Soit un $\epsilon$-ANF \automaton. Il est possible de construire un fonction retournant l'ensemble des états atteints uniquement en suivant des transitions sur $\epsilon$ pour un état $q$ donné. Cette fonction est la \emph{fermeture sur epsilon} $ECLOSE : Q \rightarrow 2^Q$. Sa définition est inductive.\\

Soit $q$ un état dans $Q$.\\
\textbf{Cas de base} $q$ est dans ECLOSE($q$)\\
\textbf{Pas de récurrence} Si $p$ est dans ECLOSE($q$) et qu'il existe un état $r$ tel quel $r\in\delta(p,\epsilon)$, alors $r$ est dans ECLOSE($q$).\\

ECLOSE peut être utilisé indifféremment sur un ensemble d'états S ($ECLOSE : 2^Q \rightarrow 2^Q$). Alors, $ECLOSE(S)=\bigcup_{q\in S}ECLOSE(q)$.

\begin{example}[ECLOSE]\label{ex:anfclosure} Considérons l'automate $A$ de l'exemple \ref{ex:anf}. Les différentes fermetures peuvent être calculées :
	\begin{itemize}
		\item ECLOSE($q_0$) = $\{q_0,q_1,q_2\}$. En effet, $q_0$ appartient à sa fermeture, selon le cas de base. Aussi, $q_1,q_2\in\delta(q_0, \epsilon)$.
		\item ECLOSE($q_1$)=$\{q_1\}$ par le cas de base.
		\item ECLOSE($q_2$)=$\{q_2\}$ par le cas de base.
	\end{itemize}
\end{example}
