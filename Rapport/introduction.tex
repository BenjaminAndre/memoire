Quatrième version du document.
\todo{Construire une introduction}

\subsection*{Contenu du mail introductif}

\paragraph{Objectif}

Le but ultime du mémoire serait de comprendre l’approche de l’article "Actively learning to verify safety for FIFO automata" \cite{Vardhan04} et de reproduire une partie des expérimentations. 

Les différentes étapes seraient
\begin{itemize}
	\item te rappeler la notion d’automate fini (voir cours de compilation et de calculabilité et complexité)
	\item comprendre la notion d’automate fini minimal
	\item comprendre l’algorithme L* d’Angluin pour l’apprentissage d’un automate
	\item implémenter l'algorithme de Vardhan pour tester les résultats
\end{itemize}


\paragraph{Contexte}

Un professeur connaît un automate A. Un élève veut l'apprendre en posant deux types de questions, celles de \emph{membership} et d'\emph{equivalence}. Si on a un automate équivalent, on peut utiliser un algorithme pour obtenir l'automate minimal qui calcule le même langage que A.

Cet algorithme A* (ou variantes) a plein d'applications, notamment décrites dans l'article \cite{Vardhan04}. On y considères des automates FIFO, plus généraux. (\emph{de ce que j'ai compris, ils sont équivalent à des automates avec un nombre d'état infini}). Pour ceux-ci, on voudrait savoir si les configurations calculées à partir de l’état initial évitent certaines mauvaises configurations.

Vu la puissance des automates FIFO, il est algorithmiquement impossible de calculer toutes les configurations atteignables à partir de l’état initial. (\emph{ce qui serait cohérent avec ma compréhension})

L’idée est alors de calculer (par apprentissage) un automate fini qui est une sur-approximation de cet ensemble de configurations et ensuite de tester si oui ou non on peut éviter les mauvaises configurations.


\subsection{Reformulation du but général}

Il existe les automates FIFO, qui ont des canaux pouvant contenir une infinité de symboles. Ceux-ci sont équivalents à des automates sans canaux mais avec un nombre infini d'états. Ceux-ci peuvent être utilisés pour faire du model checking, c'est-à-dire s'assurer de ne pas atteindre certains états peut importe la suite d'instructions.

Comme il n'est pas possible de résoudre 


