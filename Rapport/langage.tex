Cette section pose différents concepts et notations pour arriver à la notion de langage. Celle-ci reprennent les notations proposées par Hopcroft et al. \cite{Hopcroft00}.

\subsection{Alphabet}

Un \emph{alphabet}, nommé $\Sigma$ par convention, est un ensemble fini et non vide de \emph{symboles}.

\begin{example}[Alphabets] Voici trois alphabets : 
	\begin{itemize}
		\item $\Sigma = \{0,1,2,3,4,5,6,7,8,9\}$, l'alphabet des chiffres
		\item $\Sigma = \{a,b,c,...,z,A,B,C,...,Z\}$, l'alphabet latin
		\item $\Sigma = \{0,1\}$, l'alphabet binaire
	\end{itemize}
\end{example}


\subsection{Mots}

Soit l'alphabet $\Sigma$ et un entier naturel $k$. Un \emph{mot} sur $\Sigma$ est une suite finie de $k$ éléments de $\Sigma$ notée $ w = a_1a_2\dots a_k $.

L'entier $k$ est la \emph{longueur} de ce mot aussi notée $|w|=k$.

\begin{example}[Mot]
	$w=01110010$ est un mot sur $\Sigma=\{0,1\}$
\end{example}


Le \emph{mot vide} est un mot de taille $k=0$ noté $w=\epsilon$.

$\Sigma^k$ est l'ensemble des mots sur $\Sigma$ de longueur $k$.
	
L'ensemble de tous les mots possibles sur $\Sigma$ est noté $\Sigma^* = \bigcup_{k=0}^{\infty}\Sigma^k$.

La \emph{concaténation} de deux mots $w=a_1a_2\dots a_k$ et $x=b_1b_2\dots b_j$ est l'opération consistant à créer un nouveau mot $wx=a_1a_2\dots a_kb_1b_2\dots b_j$ de longueur $i=k+j$.

\begin{example}[Concaténation]
	Soient les mots $x=41$ et $y=31$. Alors $xy=4131$ et $yx=3141$.
\end{example}

\begin{lemma}[$\epsilon$ et la concaténation]
	$\epsilon$ est \emph{l'identité pour la concaténation}, à savoir pour tout mot $w$, $w\epsilon = \epsilon w = w$.
\end{lemma}

\begin{proof}[$\epsilon$ et la concaténation]
	Par définition de la concaténation, tout mot concaténé avec $\epsilon$ retourne le même mot. 
	\hfill$\square$
\end{proof}

\emph{L'exponentiation} d'un symbole $a$ à la puissance $k$, notée $a^k$, retourne un mot de longueur $k$ obtenu par la concaténation de copies du symbole $a$. Noter que $a^0=\epsilon$.

\subsection{Langage}

Un ensemble de mots sur $\Sigma$ est un \emph{langage} \cite{Hopcroft00}, noté $L$.$L \subseteq \Sigma^*$. Étant donné que $\Sigma^*$ est infini, $L$ peut l'être également.

\begin{example}[Définition d'un langage] Voici des exemples utilisant plusieurs modes de définition. $\Sigma$ y est implicite mais peut être donné explicitement.
	\begin{itemize}
		\item $L=\{12,35,42,7,0\}$, un ensemble défini explicitement
		\item $L=\{0^k1^j|k+j=7\}$, les mots de 7 symboles sur $\Sigma=\{0,1\}$ commençant par zéro, un ou plusieurs $0$ et finissant par zéro,un ou plusieurs $1$. Ici, $L$ est donné par notation ensembliste
		\item L est "Tous les noms de villes belges". Ici $L$ est défini en français.
		\item $\emptyset$ est un langage pour tout alphabet.
		\item $L=\{\epsilon\}$ ne contient que le mot vide, et est un langage sur tout alphabet.
	\end{itemize}
\end{example}


\subsubsection*{Opérations sur les langages}

Soient $L$ et $M$ deux langages. Le langage $L \cup M = \{w | w \in L\vee w \in M\}$ est l'\emph{union} de ces deux langages. Il est composé des mots venant d'un des deux langages.

Le langage composé de tous les mots produit par la concaténation d'un mot de $L$ avec un mot de $M$ est une \emph{concaténation} de ces deux langages et s'écrit $LM$.

La \emph{fermeture} de $L$ est notée $L^*$ et donne un langage constitué de tous les mots qui peuvent être construits par un concaténation d'un nombre arbitraire de mots de $L$.

\subsection{Expression régulière}\label{ss:regex}

Certains langages peuvent être exprimés par une \emph{expression régulière}. Un exemple de celles-ci est $01*0$ qui décrit la langage constitué de tous les mots commençant et finissant par $0$ avec uniquement des $1$ entre les deux.

Les expressions régulières suivent un algèbre avec ses opérations et leur priorités. Le langage décrit par une expression est construit de façon inductive par ces différentes opérations. Pour une expression régulière $E$, le langage exprimé est noté $L(E)$. Un langage qui peut être exprimé par une expression régulière est dit \emph{langage régulier}.


\textbf{Cas de base}
Certains langages peuvent être construits directement sans passer par l'induction:

\begin{itemize}
	\item $\epsilon$ est une expression régulière. Elle exprime le langage $L(\epsilon)=\{\epsilon\}$
	\item $\emptyset$ est une expression régulière décrivant $L(\emptyset)=\emptyset$
	\item Si $a$ est un symbole, alors \textbf{a} est une expression régulière composée uniquement de $a$. $L(a) = \{a\}$.
	\item Une variable, souvent en majuscule et italique, représente un langage quelconque, par exemple $L$.
\end{itemize}


\textbf{Induction}
Les autres langages réguliers sont construits suivant différentes règles d'induction présentées par ordre décroissant de priorité :

\begin{itemize}
	\item Si $E$ est une expression régulière, $(E)$ est une expression régulière et $L((E)) = L(E)$.
	\item Si $E$ est une expression régulière, $E^*$ est une expression régulière représentant la fermeture de $L(E)$, à savoir $L(E^*) = L(E)^*$.
	\item Si $E$ et $F$ sont des expressions régulières, $EF$ est une expression régulière décrivant la concaténation des deux langages représentés, à savoir $L(EF)=L(E)L(F)$. La concaténation étant commutative, l'ordre de groupement n'est pas important, mais par convention, la priorité est à gauche.
	\item Si $E$ et $F$ sont des expressions régulières, $E+F$ est une expression régulière donnant l'union des deux langages représentés, à savoir $L(E+F)=L(E)\cup L(F)$. Ici encore, l'opération est commutative et la priorité est à gauche.
\end{itemize}
	
\begin{example}[Expression régulière]
	Soit l'expression $E = (b+ab)b^*a(a+b)^*$ qui représente le langage $L$.\\
	\begin{itemize}
		\item \textbf{ba} fait partie de $L$. En effet, en développant $E$ avec des choix sur les unions et le degré d'une fermeture, on obtient  $E= (b)b^0a(a+b)^0 = b\epsilon a \epsilon = ba$.
		\item \textbf{ababbab} fait partie de $L$. En développant à nouveau $E$ en posant des choix sur les unions et fermetures, on obtient $E=(ab)b^0a(a+b)^4 = ab\epsilon a (a+b)(a+b)(a+b)(a+b) = ababbab$.
		\item \textbf{aa} ne fait \textbf{pas} partie de $L$. Supposons par l'absurde que $aa \in L$. Alors il existerait une façon de décomposer $E$ en $aa$. Or, les premiers symboles doivent être soit $b$, soit $ab$. Il y a contradiction : $E$ ne peut pas être décomposé. Comme $aa$ ne peut pas être construit par $E$, $aa \notin L$.
	\end{itemize}
	\label{ex:regex}
\end{example}