Cette section apporte le détail sur la relation de Myhill-Nérode, en prouve les propriétés avant d'en faire l'usage dans le théorème du même nom.




\subsection{Relation de Myhill-Nérode}

Soit un langage $L$ sur un alphabet $\Sigma$.

Soit la relation $R_L$ portant sur deux mots (ne faisant pas nécessairement partie de $L$). Deux mots $x$ et $y$ respectent la relation de Myhill-Nérode $R_L$ si 

$$\forall z \in \Sigma^*, xz \in L \Leftrightarrow yz \in L$$

Intuitivement, deux mots sont en relation si pour tout mot qu'on leur concatène, les deux mots résultants sont tous deux dans le langage ou non.

\begin{lemma}
	Cette relation est une relation d'équivalence. De plus, elle respecte la congruence à droite. C'est à dire que si $xR_Ly$, alors pour tout symbole $a \in \Sigma$, $xaR_Lya$
\end{lemma}

\begin{proof}[Equivalence et Congruence à droite]
	Dire d'une relation qu'elle décrit une équivalence, revient à dire qu'elle est réflexive, transitive et symétrique
\begin{itemize}
		\item $R_L$ est réflexive. Soit $x \in \Sigma^*$. Soit $z \in \Sigma^*$. Montrer que $xR_Lx$ est vrai revient à montrer que $ xz \in L \Leftrightarrow xz \in L$ est vrai. $R_L$ est donc réflexive.
		\item $R_L$ est symétrique. Soient $x, y \in \Sigma^*$ tels que $xR_Ly$. Soit $w \in \Sigma^*$. Montrer que $yR_Lx$ revient à montrer que $ yw \in L \Leftrightarrow xw \in L$. Or, par hypothèse, $ xz \in L \Leftrightarrow yz \in L$, qui peut s'écrire aussi $ yz \in L \Leftrightarrow xz \in L$ pour tout $z \in \Sigma^*$, et en particulier $z=w$.
		\item $R_L$ est transitive. Soient $x,y,u \in \Sigma^*$ tels que $xR_Ly$ et $yR_Lz$. Soit $w \in \Sigma^*$. Comme $ xz \in L \Leftrightarrow yz \in L$ et $ yz \in L \Leftrightarrow uz \in L$ pour tout $z \in \Sigma^*$ (par hypothèse), c'est vrai en particulier pour $z=w$. Dès lors,  $ xw \in L \Leftrightarrow yw \in L$et $ yw \in L \Leftrightarrow uw \in L$. Par transitivité de l'implication, on obtient $ xw \in L \Leftrightarrow uw \in L$, à savoir $xR_Lu$.
		\item $R_L$ est congruente à droite. Soient $x,y \in \Sigma^*$ tels que $xR_Ly$. Soit $a \in \Sigma$. Par hypothèse, $ xz \in L \Leftrightarrow yz \in L$ pour tout $z \in \Sigma^*$. Cela doit donc être vrai en particulier pour le mot $z=aw$ avec $w$ quelconque. En remplaçant dans l'hypothèse, on obtient  $ xaw \in L \Leftrightarrow yaw \in L$. Ce qui montre que $xaR_Lya$.
	\end{itemize}

\hfill$\square$
\end{proof}


\subsection{Théorème de Myhill-Nerode}
	
	\begin{theorem}
		Les 3 énoncés suivants sont équivalents :
		\begin{enumerate}
			\item Un langage $L\subseteq\Sigma^*$ est accepté par un DFA
			\item $L$ est l'union de certaines classes d'équivalence d'index fini respectant une relation d'équivalence et de congruence à droite
			\item Soit la relation d'équivalence $R_L : xR_Ly \Leftrightarrow \forall z \in \Sigma^*, xz \in L \Leftrightarrow yz \in L$ (la relation de Myhill-Nérode définie précédemment). $R_L$ est d'index fini.
		\end{enumerate}
	\end{theorem}
	
	\begin{proof}La preuve d'équivalence se fait en prouvant chaque implication de façon cyclique :\\
		
		$(1)\rightarrow(2)$ Supposons que (1) soit vrai, c'est à dire que le langage $L$ est accepté par un automate déterministe fini $A$. Considérons la relation d'équivalence $R_M$ étant vraie pour les mots $x,y$ si $\hat{\delta}(q_0,x)\in F \iff \hat{\delta}(q_0,y)\in F$. Elle a été définie en \ref{ss:rm}. Il y est prouvé qu'elle est congruente à droite. Comme il y a au plus une classe d'équivalence pour $R_M$ par état de $A$. Comme ce nombre d'états est fini, $R_M$ est d'index fini. De plus, $L$ est l'union de classes contenant un mot $w$ tel que $\hat{\delta}(q_0,w) \in F$, (or, ce chemin retourne un état. Il s'agit donc d'une union des classes correspondant aux états acceptants).
		
		$(2)\rightarrow(3)$ Montrons que pour toute relation $E$ satisfaisant (2), chaque classe est intégralement contenue dans une seule classe de $R_L$. Ces classes étant d'index fini, c'est un argument suffisant pour déduire que $R_L$ est d'index fini. Considérons $x,y$ tels que $xEy$. Comme $E$ est congruente à droite, pour tout mot $z \in \Sigma^*$, on sait que $xzEyz$. Comme $L$ est un union de ces classes d'équivalence, $xzEyz$ implique que $xz \in L \Leftrightarrow yz \in L$, ce qui revient à $xR_Ly$. Cela signifie que tout mot dans la classe d'équivalence de $x$ définie par $E$ se retrouve dans la même classe d'équivalence que $x$ par $R_L$. Ce qui permet de conclure que chaque classe d'équivalence de $E$ est contenue dans une classe d'équivalence de $R_L$. 
	
		
		$(3)\rightarrow(1)$ Considérons la relation $R_L$ définie précédemment, et déduisons-en $Q'$ les classes d'équivalence sur $L$ et $[[x]]$ l'élément(la classe) de $Q'$ qui contient $x$. Puisque $R_L$ a été démontré comme congruent à droite, on peut définir des transitions : $\delta'([[x]],a) = [[xa]]$. En choisissant un élément $y$ dans $[[x]]$ (ce qui signifie que $xR_Ly$), on obtient $\delta'([[x]],a)=[[ya]]$. Sauf que par définition, $xR_Ly$ signifie qu'en y ajoutant n'importe quel mot $z$, $xz$ et $yz$ appartiennent tous deux où non à $L$. C'est vrai en particulier pour $z=az'$. Ainsi, $xaz'$ et $yaz'$ appartiennent tous deux à $L$ ou non. Ce qui signifie que $xaR_Lya$ et donc $[[xa]]=[[ya]]$. Posons $q_0'=[[\epsilon]]$ et $F' = \{[[x]]|x \in L\}$. Tous ces éléments forment l'automate $M'=(Q', \Sigma, \delta', q_0', F')$. Il est déterministe par la définition de $\delta'$, fini car $Q'$ est fini par construction (le nombre de classes d'équivalence est fini). De plus, il accepte $L$ puisque $\delta'(q_0',x)=[[x]]$, ce qui signifie que $x \in L(M')$ si et seulement si $[[x]] \in F'$, qui a été défini comme tel.
		
		
		\hfill$\square$
	\end{proof}


	\begin{corollary}
		Grâce à la preuve du théorème de Myhill-Nérode, en particulier la justification partant de la relation d'équivalence $R_L$ pour montrer que la langage $L\subseteq\Sigma^*$ est accepté par un DFA, on a une méthode pour construire un automate à partir d'un langage.
	\end{corollary}