	\subsection{Définition}\label{sub:dfa}
	
	Un \emph{automate déterministe fini (ADF)} \automaton est défini comme suit :
	\begin{itemize}
		\item $Q$ est un ensemble fini d'\emph{états}
		\item $\Sigma$ est un alphabet
		\item $q_0 \in Q$ est l'\emph{état initial}
		\item $\delta : Q \times \Sigma \rightarrow Q$ est la \emph{fonction de transition}. A partir d'un état $q$ de $Q$, en fonction d'un symbole $a$, elle retourne un état de $Q$ : $\delta(q,a)$. Cette transition est dite \emph{transition sur $a$}.
		\item $F \subseteq Q$ est un ensemble d'\emph{états acceptants}.
	\end{itemize}
	
	\begin{example}[Automate déterministe fini]\label{ex:adf}
		On considère l'automate \automaton défini comme suit :
		\begin{itemize}
			\item $Q=\{q_0,q_1,q_2,q_3,q_4,q_5,q_6\}$
			\item $\Sigma=\{0,1\}$
			\item $q_0$ est l'état du même nom
			\item La fonction de transition $\delta$ est décrite par la table \ref{fig:transdelta}. L'intersection d'une ligne reprenant un élément $q \in Q$ et d'une colonne $a \in \Sigma$ donne l'état $\delta(q,a)$.
			\item $F=\{q_d\}$
		\end{itemize}
	
		\begin{figure}[H]
			\centering
			\begin{tabular}{|r||c|c|}
				\hline
				&a&b\\
				\hline\hline
				$\rightarrow q_0$&$q_2$&$q_1$\\\hline
				$q_1$&$q_3$&$q_5$\\\hline
				$q_2$&$q_4$&$q_5$\\\hline
				$q_3^*$&$q_3$&$q_3$\\\hline
				$q_4$&$q_4$&$q_4$\\\hline
				$q_5$&$q_3$&$q_1$\\\hline
				$q_6$&$q_4$&$q_5$\\\hline
			\end{tabular}
			\caption{La table de transitions $\delta$}
			\label{fig:transdelta}
		\end{figure}
	\end{example}
	 
	Via cette notation, $Q$ et $\Sigma$ sont explicites. En dénotant l'état initial par $\rightarrow$ et les états acceptants par $*$ en exposant, on obtient une définition complète d'un automate : $(Q,\Sigma, q_0, \delta, F)$.
 	
 	
 	\subsection{Graphe d'automate déterministe fini}\label{ss:grapheadf}
 	
 	Le \emph{graphe d'un automate déterministe fini} \automaton est un graphe dirigé construit comme suit :
 	
 	\begin{itemize}
 		\item Chaque nœud du graphe correspond à un état de $Q$
 		\item Chaque arc a un symbole de $\Sigma$ comme étiquette. Un arc relie un état $q_0$ à un état $q_1$. Cet arc défini $\delta(q_0,a)=q_1$, un transition de la fonction de transition. Si plusieurs symboles causent une même transition de $q_0$ à $q_1$, il n'y a qu'une seule étiquette sur l'arc, listant ces différents symboles.
 		\item L'état initial est mis en évidence par une flèche entrante.
 		\item Les états acceptants sont représentés par un double cercle, en opposition au simple cercle des autres nœuds.
 	\end{itemize}
 	
 	\begin{example}[Graphe d'automate]
	 Voici le graphe représentant l'automate défini par la table \ref{fig:transdelta}
	 \begin{figure}[H]
	 	\centering
	 	\begin{tikzpicture}[->,>=stealth',shorten >=1pt,auto,node distance=3cm, semithick, bend angle=10]
	 	
	 	\tikzstyle{every state}=[circle]
	 	
	 	\node[initial,state] (A)                    {$q_0$};
	 	\node[state]         (B) [below right of=A] {$q_1$};
	 	\node[state]         (C) [below left of=A] {$q_2$};
	 	\node[accepting, state]         (D) [below right of=B] {$q_3$};
	 	\node[state]         (E) [below left of=C]       {$q_4$};
	 	\node[state]         (G) [below right of=E]       {$q_6$};
	 	\node[state]         (F) [above right of=G]       {$q_5$};
	 	
	 	\path 	(A) 	edge              node {a} (C)
	 	edge              node {b} (B)
	 	(B) 	edge              node {a} (D)
	 	edge [bend left]  node {b} (F)
	 	(C) 	edge              node {a} (E)
	 	edge              node {b} (F)
	 	(D) 	edge [loop above] node {a,b} (D)
	 	(E) 	edge [loop above] node {a,b} (E)
	 	(F) 	edge              node {a} (D)
	 	edge [bend left]  node {b} (B)
	 	(G) 	edge              node {a} (E)
	 	edge              node {b} (F);
	 	\end{tikzpicture}
	 	\caption{Automate $A_1$}\label{fig:a1}
	 \end{figure}
	 
 	\end{example}

	 Cette représentation d'un automate peut sembler plus naturelle pour un humain alors que la table de transitions est plus proche d'un langage informatique. De plus, dans la représentation par graphe, les ensembles $Q$ et $\Sigma$ sont implicites et doivent être définis ou déduits à part.
	 
	 \subsection{Chemin}
	 
	 La \emph{fonction de transition étendue} 
	 $$\hdelta : Q \times \Sigma^* \rightarrow Q$$
	 prend en entrée un état de $Q$ et un mot $w$ sur $\Sigma$ et retourne un état de $Q$.
	 
	 $\hdelta$ est définie de façon récursive par sur $w$:
	 \paragraph{Cas de base} Il y a deux cas de base:
	 \begin{itemize}
	 	\item $w$ est un mot vide : $\hdelta(q, \epsilon) = q$
	 	\item $w$ est un symbole : $\hdelta(q, w)$ avec $w=a\in \Sigma$. Alors, le chemin utilise la fonction de transition : $\hdelta(q,a)=\delta(q,a)$.
	 \end{itemize}
	 \paragraph{Pas de récurrence} Si $|w|>1$, alors $w=xa$ avec $x$ un mot sur $\Sigma$ et $a$ un symbole de $\Sigma$. Les chemins sur des mots de longueur strictement supérieure à 1 sont définis comme $\hdelta(q,w) = \hdelta(q,xa)= \delta(\hdelta(q,x),a)$.
	 
	 Il se peut que $\delta$ ne soit pas définie pour une paire d'arguments. Auquel cas, $\hdelta$ ne l'est pas non plus.\\
	 
	 Un \emph{chemin} est une application de cette fonction sur un état et un mot.
	 
	 \begin{example}[Chemin]
	 	Considérons l'automate $A$ de la figure \ref{fig:a1}. Il existe un chemin de $q_0$ à $q_5$ : $\hdelta(q_0, ab) = \delta(\hdelta(q_0,a),b) = \delta(\delta(q_0,a),b) = \delta(q_2, b)=q_5$.
	 \end{example}
	 
	 \subsection{Langage défini par un automate}
	 
	 Le langage représenté par un automate \automaton peut alors se définir comme les mots qui, par l'application de $\hdelta$ sur l'état initial, donnent un état acceptant :
	 $$
	 L(A)= \{w \in \Sigma^* | \hdelta(q_0,w) \in F\}
	 $$
	 Ainsi, un mot $w$ appartient à un langage $L$ défini par l'automate $A$ si $\hdelta(q_0,w) \in F$. L'algorithme \ref{alg:adfmembership} représente cette appartenance pour un mot.
	 
	 
	 \begin{algo}[Appartenance d'un mot à un langage défini par un automate]\label{alg:adfmembership}
	 	\begin{algorithmic}[1]
	 		\REQUIRE un mot $w$, un automate \automaton représentant $L$
	 		\ENSURE si $w$ appartient à $L$
	 		
	 		\STATE $q_{c} \leftarrow q_0$ \COMMENT{$q_c$ est l'état courant}
	 		
	 		\WHILE {$|w| > 0$}
	 		\STATE décomposer $w$ en $ax$ avec $a\in\Sigma$ et $x$ le reste du mot
	 		\STATE $q_c \leftarrow \delta(q_c,a)$ \COMMENT{passage à l'état suivant}
	 		\STATE $w \leftarrow x$
	 		\ENDWHILE
	 		
	 		\RETURN si $q_c$ appartient à $F$
	 	\end{algorithmic}
 	\end{algo}
 
 	\begin{complexity}[Lecture d'un mot par un automate]
 		Si $|w|=n$, l'algorithme \ref{alg:adfmembership} est en $\mathcal{O}(n)$. En effet, les étapes 1 et 7 sont en temps constant. La boucle de l'étape 2 est parcourue $n$ fois (la taille étant diminuée de 1 exactement à chaque itération). Le test de 2 et les opérations de 3 et 5 peuvent être faites en temps constant (par exemple, en voyant w comme une queue). L'étape 4, déterminante, peut être effectuée en temps constant également, par exemple avec l'utilisation d'un tableau de transition.
	
	\end{complexity}
	 
	 
	 
	 
	 \subsection{La relation $R_M$}\label{ss:rm}
	 
	 Soit un automate \automaton. Définissons la relation $R_M$ entre deux états : 
	 $$xR_My \iff (\forall w \in \Sigma^*,\hdelta(x,w) \in F \iff \hdelta(y,w) \in F)$$
	 
	 Intuitivement, ces deux états sont en relation si tout mot lu à partir de celui-ci mène à des états étant simultanément acceptants ou non. 
	 
	 \begin{proposition}[$R_M$]
	 	\rm est une relation d'équivalence.
	 \end{proposition}
	
	 \begin{proof}[$R_M$ est une relation d'équivalence] Montrer que \rm est une relation d'équivalence revient à montrer qu'elle est réflexive, transitive et symétrique.
	 	\begin{itemize}
	 		\item \textbf{Réflexive :} Soient un état $x \in Q_M$ et $w \in \Sigma^*$. Alors, $\hat{\delta}(x,w) \in F \iff \hat{\delta}(x,w) \in F$ et par définition, $xR_Mx$.
	 		\item \textbf{Transitive :} Soient les états $x,y,z \in Q_M$ tels que $xR_My$ et $yR_Mz$ ainsi que $w \in \Sigma^*$. Par hypothèse, $\hat{\delta}(x,w) \in F \iff \hat{\delta}(y,w)\in F$ et $\hat{\delta}(y,w) \in F\iff \hat{\delta}(z,w) \in F$. Par transitivité de l'implication, on obtient $\hat{\delta}(x,w) \in F \iff \hat{\delta}(z,w)\in F$. On a donc $xR_Mz$.
	 		\item \textbf{Symétrique : } Soient les états $x,y \in Q_M$ tels que $xR_My$ et un mot $w \in \Sigma^*$. Par hypothèse, $\hat{\delta}(x, w)\in F \iff \hat{\delta}(y, w)\in F$. En lisant la double implication depuis la droite, on a bien $\hat{\delta}(y, w) \in F\iff \hat{\delta}(x, w)\in F$ et donc $yR_Mx$.
	 	\end{itemize}
 		\hfill$\square$
	 \end{proof}
	 
	 \begin{corollary}
	 	\rm sépare les états de $Q$ en classes d'équivalence.
	 \end{corollary}
	 
	 La classe d'équivalence de tous les états en relation \rm avec $q$ (qui sert alors de \emph{représentant}) se note $[[q]]$ ou par une lettre majuscule, typiquement $S$ ou $T$.
	 
	 La \emph{congruence à droite} d'une relation $R$ entre des mots sur un alphabet $\Sigma$ est définie comme :
	 $$
	 \forall x,y \in \Sigma^*, xRy \Rightarrow \forall a \in \Sigma, xaRya 
	 $$  
	 
	 \begin{proposition}[Congruence de \rm]
	 	\rm est congruente à droite.
	 \end{proposition}
	 
	 \begin{proof}[Congruence de \rm]
	 	Si la relation est vraie pour deux état, elle reste valable pour les états atteints par la lecture d'un symbole quelconque. Soient les états $x,y \in Q_M$ tels que $xR_My$. Soit un symbole $a \in \Sigma$. Par hypothèse, 
	 	$$\forall w \in \Sigma^*, \hat{\delta}(x, w) \in F \iff \hat{\delta}(y, w) \in F$$
	 	C'est donc vrai en particulier pour $w = au, u \in \Sigma*$. Dès lors,
	 	$$\hat{\delta}(x, au) \in F\iff \hat{\delta}(y, au)\in F$$
	 	$$\hat{\delta}(\delta(x,a),u) \in F\iff\hat{\delta}(\delta(y,a),u)\in F$$
	 	$$\hat{\delta}(p,u) \in F\iff \hat{\delta}(q,u)\in F$$

	\hfill$\square$	 
 \end{proof}
 
 	\begin{corollary}\label{col:st}
 		Pour chaque symbole, toutes les transitions sortant d'une classe d'équivalence mènent à une même classe d'équivalence :
 		$\forall a \in \Sigma, \exists T, \forall q \in S, \delta(q,a)\in T$ avec $T$ une classe d'équivalence.
 	\end{corollary}
	 
	 
	 \subsection{Automate et problème de décision}
	 
	 Une notion liée aux langages est celle de \emph{problème}. Une forme de problème est celle dite de \emph{décision} : une question à laquelle la réponse est oui ou non.
	 
	 Ces problèmes de décision peuvent être exprimés en terme d'appartenance d'un mot à un langage.
	 
	 Par exemple, prenons l'alphabet des chiffres $\Sigma=\{0,1,2,3,4,5,6,7,8,9\}$. Considérons ensuite le langage $L = \{w | \text{le nombre représenté par } w \text{ est pair}\}$.
	 
	 Demander si un nombre est pair peut alors être traduit par l'appartenance d'un mot le représentant à $L$. Si le langage peut être représenté par un automate déterministe fini, la réponse peut être trouvée par l'exécution de celui-ci.
	 
	 
	 
	 