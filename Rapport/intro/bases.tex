Dans la section \ref{sec:langage}, les notions de langage sont posées. Elles sont ensuite utilisées dans la section \ref{sec:automaton} sur les automates. Le "Table Filling Algorithm" de la section \ref{sec:tfa} se base sur ces automates et permet de les minimiser et de répondre à la requête d'équivalence. Cette requête d'équivalence est une des deux requêtes necéssaire au fonctionnement de l'algorithme d'angluin de la section \ref{sec:angluin}.

Tous ces éléments combinés permettent la compréhension des notions plus spécifiques utilisées dans l'article \cite{Vardhan04} ainsi que celles qui y sont construites. Ces notions se retrouvent dans le chapitre \ref{ch:specific}.
