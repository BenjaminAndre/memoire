Huitième version du document.


Le but de ce mémoire est de comprendre et implémenter l'article "Actively learning to verify safety for FIFO automata" \cite{Vardhan04}.

Pour ce faire, plusieurs étapes sont necéssaires. Dans le chapitre \ref{ch:bases}, les bases théoriques et leurs notations sont rappelées. Celles-ci concernent les automates, les langages et l'algorithme d'Angluin.

Dans le chapitre \ref{ch:specific}, des notions plus spécifiques telles que les automates à files ou les langages de trace (développés dans l'article) sont abordées.

Ensuite, le chapitre \ref{ch:lever} s'appuie sur toutes ces notions et explique l'algorithme LeVer (Learning to Verify) qui permet justement l'apprentissage actif d'automates à files pour vérifier leur sécurité.

Le chapitre \ref{ch:impl} mentionne les choix techniques faits pour l'implémentation avant de présenter et discuter les résultats obtenus.

Finalement, le chapitre \ref{ch:ccl} résume l'apport et les conséquences de ce travail avant de proposer des pistes d'amélioration.
