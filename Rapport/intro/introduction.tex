Septième version du document.


Le but de ce mémoire est de comprendre et implémenter l'article "Actively learning to verify safety for FIFO automata" \cite{Vardhan04}.

Pour ce faire, plusieurs étapes sont necéssaires. Dans le chapitre \ref{ch:bases}, les bases théoriques et leurs notations sont rappelées. Celles-ci concernent les automates, les langages et l'algorithme d'Angluin.

Dans le chapitre \ref{ch:learning}, l'article en question est discuté. La technique choisie est expliquée et les concepts spécifiques et ceux introduits par l'article sont définis. Ceux-ci concernent les automates FIFO, la notion de sécurité, et des notations propres à cet article. La section se fini par une explication de l'algorithme obtenu.

Ensuite, le chapitre \ref{ch:impl} mentionne les choix techniques faits pour l'implémentation avant de présenter et discuter les résultats obtenus.

Finalement, le chapitre \ref{ch:ccl} résume l'apport et les conséquences de ce travail avant de proposer des pistes d'amélioration.
