\documentclass[french,letterpaper, 12pt]{article}

\usepackage[top = 1.6cm, left = 2cm, right = 2cm ]{geometry}
\usepackage[pdftex]{graphicx}
\usepackage{soulutf8}
\usepackage{amsmath}
\usepackage{tikz}
\usepackage[utf8]{inputenc}
\usepackage{longtable}
\usepackage[T1]{fontenc}
\usepackage{epigraph}
\usepackage{fancyhdr}
\usepackage{float}
\usepackage{xcolor}
\usepackage{eurosym}
\usepackage{calc}
\usepackage{hyperref}
\usepackage{multirow}
\usepackage{caption}
\usepackage[Algorithme]{algorithm}
\usepackage{algorithmic}
\usepackage{enumerate}
\usepackage[french]{babel}
\usepackage{tcolorbox}
\usepackage{multicol}
\usepackage{etoolbox,refcount}
\usepackage{listings}
\usepackage{amssymb}
\usepackage{subcaption}
\usepackage[standard,thref, framed, hyperref,standard,thmmarks]{ntheorem}
\usepackage{lmodern}
\usepackage{thmtools}
\usepackage{tikz-cd}

%
%%%%%%% Sub librairies

\usetikzlibrary{arrows,automata}
\usetikzlibrary{decorations.pathreplacing,shapes,arrows,positioning}
\usetikzlibrary{calc}
\usetikzlibrary{positioning}
\usetikzlibrary{snakes}
\usetikzlibrary{shapes,fit}
\usetikzlibrary{positioning}
\usetikzlibrary{automata,arrows,trees,positioning,shapes,calc}

%
%%%%% Custom commands
%



\lstset{
	keywordstyle=\color{red},
	basicstyle=\scriptsize\ttfamily,
	commentstyle=\ttfamily\itshape\color{gray},
	stringstyle=\ttfamily,
	showstringspaces=false,
	breaklines=true,
	frameround=ffff,
	rulecolor=\color{black}
}

%
\def\changemargin#1#2{\list{}{\rightmargin#2\leftmargin#1}\item[]}
\let\endchangemargin=\endlist
%
\newcommand{\newlinealinea}{
	~\\ \hspace*{0.5cm}}
%
\newcommand{\alinea}{
	\hspace*{0.5cm}}
%
\newcommand{\alinealong}{
	\hspace*{1.1cm}}
%
\newcommand{\alignparagraph}{
	\hspace*{0.6cm}}
%
\newcommand{\red}[1]{
	\textcolor{red}{#1}}
%
\newcommand{\green}[1]{
	\textcolor{green}{#1}}
%
\newcommand{\point}{$\bullet\ $}
%
\makeatletter
\newcommand*{\whiten}[1]{\llap{\textcolor{white}{{\the\SOUL@token}}\hspace{#1pt}}}
\newcommand{\myul}[1]{
	\underline{\smash{#1}}
}
\makeatother
%
\setlength{\fboxsep}{2pt}
%
\DeclareMathOperator*{\argmax}{\arg\!\max}
%
%
%%%%% Custom text
%
%
\makeatletter
\@addtoreset{section}{part}
\makeatother
%
\renewcommand*\sfdefault{phv}
\renewcommand*\rmdefault{ppl}
%
\renewcommand\epigraphflush{flushright}
\renewcommand\epigraphsize{\normalsize}
\setlength\epigraphwidth{0.7\textwidth}
%
\definecolor{titlepagecolor}{cmyk}{0.24,0.92,0.78,0.25}
\definecolor{red}{cmyk}{0, 0.91, 0.91, 0.20}
%
\DeclareFixedFont{\titlefont}{T1}{phv}{\seriesdefault}{n}{0.375in}
%
%
%%%%% Header
%
%
\pagestyle{fancy}
\lhead{\student}
\rhead{\grade}
\cfoot{\thepage}
%
%
%%%%% Title page. The following code is borrowed from:
%%%%%       http://tex.stackexchange.com/a/86310/10898
%
%
\newcommand\titlepagedecoration{%
	\begin{tikzpicture}[remember picture,overlay,shorten >= -10pt]

	\coordinate (aux1) at ([yshift=-70pt]current page.north east);
	\coordinate (aux2) at ([yshift=-460pt]current page.north east);
	\coordinate (aux3) at ([xshift=-6cm]current page.north east);
	\coordinate (aux4) at ([yshift=-150pt]current page.north east);

	\begin{scope}[titlepagecolor!40,line width=12pt,rounded corners=12pt]
	\draw
	(aux1) -- coordinate (a)
	++(225:5) --
	++(-45:5.1) coordinate (b);
	\draw[shorten <= -10pt]
	(aux3) --
	(a) --
	(aux1);
	\draw[opacity=0.6,titlepagecolor,shorten <= -10pt]
	(b) --
	++(225:2.2) --
	++(-45:2.2);
	\end{scope}
	\draw[titlepagecolor,line width=8pt,rounded corners=8pt,shorten <= -10pt]
	(aux4) --
	++(225:0.8) --
	++(-45:0.8);
	\begin{scope}[titlepagecolor!70,line width=6pt,rounded corners=8pt]
	\draw[shorten <= -10pt]
	(aux2) --
	++(225:3) coordinate[pos=0.45] (c) --
	++(-45:3.1);
	\draw
	(aux2) --
	(c) --
	++(135:2.5) --
	++(45:2.5) --
	++(-45:2.5) coordinate[pos=0.3] (d);
	\draw
	(d) -- +(45:1);
	\end{scope}
	\end{tikzpicture}%
}


%  █████  ██       ██████   ██████
% ██   ██ ██      ██       ██    ██
% ███████ ██      ██   ███ ██    ██
% ██   ██ ██      ██    ██ ██    ██
% ██   ██ ███████  ██████   ██████


\renewcommand{\algorithmicrequire}{\textbf{Requis:}}
\renewcommand{\algorithmicensure}{\textbf{Promet:}}
\renewcommand{\algorithmicend}{\textbf{fin}}
\renewcommand{\algorithmicif}{\textbf{si}}
\renewcommand{\algorithmicthen}{\textbf{alors}}
\renewcommand{\algorithmicelse}{\textbf{sinon}}
\renewcommand{\algorithmicelsif}{\algorithmicelse\ \algorithmicif}
\renewcommand{\algorithmicendif}{\algorithmicend\ \algorithmicif}
\renewcommand{\algorithmicfor}{\textbf{pour}}
\renewcommand{\algorithmicforall}{\textbf{pour chaque}}
\renewcommand{\algorithmicdo}{\textbf{faire}}
\renewcommand{\algorithmicendfor}{\algorithmicend\ \algorithmicfor}
\renewcommand{\algorithmicwhile}{\textbf{tant que}}
\renewcommand{\algorithmicendwhile}{\algorithmicend\ \algorithmicwhile}
\renewcommand{\algorithmicloop}{\textbf{boucle}}
\renewcommand{\algorithmicendloop}{\algorithmicend\ \algorithmicloop}
\renewcommand{\algorithmicrepeat}{\textbf{répéter}}
\renewcommand{\algorithmicuntil}{\textbf{jusqu'à}}
\renewcommand{\algorithmicprint}{\textbf{afficher}}
\renewcommand{\algorithmicreturn}{\textbf{retourner}}
\renewcommand{\algorithmictrue}{\textbf{vrai}}
\renewcommand{\algorithmicfalse}{\textbf{faux}}

\renewcommand{\algorithmicand}{\textbf{et}}
\renewcommand{\algorithmicor}{\textbf{ou}}
%%%%%%%%%%%%%%%%%%%%%%%%%%%%%%%%%%%%%%%%%%%%%%%%%%%

\newcommand{\todo}[1]{\textcolor{red}{\emph{\textbf{TODO} : #1}}}
%%%%%%%%%%%%%%%%%%%%%%%%%%%%%%%%%%%%%%%%%%%%%%%%%%%%


\newcounter{countitems}
\newcounter{nextitemizecount}
\newcommand{\setupcountitems}{%
	\stepcounter{nextitemizecount}%
	\setcounter{countitems}{0}%
	\preto\item{\stepcounter{countitems}}%
}
\makeatletter
\newcommand{\computecountitems}{%
	\edef\@currentlabel{\number\c@countitems}%
	\label{countitems@\number\numexpr\value{nextitemizecount}-1\relax}%
}
\newcommand{\nextitemizecount}{%
	\getrefnumber{countitems@\number\c@nextitemizecount}%
}
\newcommand{\previtemizecount}{%
	\getrefnumber{countitems@\number\numexpr\value{nextitemizecount}-1\relax}%
}
\makeatother
\newenvironment{AutoMultiColItemize}{%
	\ifnumcomp{\nextitemizecount}{>}{3}{\begin{multicols}{3}}{}%
		\setupcountitems\begin{itemize}}%
		{\end{itemize}%
		\unskip\computecountitems\ifnumcomp{\previtemizecount}{>}{3}{\end{multicols}}{}}


%%%%%%%%%%%%%%%%%%%%%%%%%%%%%%%%%%%%%%%%%%%%%%%%%%%%

% ████████ ██   ██ ███    ███
%    ██    ██   ██ ████  ████
%    ██    ███████ ██ ████ ██
%    ██    ██   ██ ██  ██  ██
%    ██    ██   ██ ██      ██



\theoremstyle{plain}


\renewtheorem{theorem}{Théorème}[section]
\renewtheorem{lemma}[theorem]{Lemme}
\renewtheorem{proposition}[theorem]{Proposition}


\theoremstyle{break}

\newtheorem{algo}{Algorithme}[section]

\theoremheaderfont{\normalfont\smallskip\normalsize\itshape\bfseries}
\theorembodyfont{\normalfont\normalsize}



\renewtheorem{proof}{Preuve}[theorem]
\renewtheorem{example}{Exemple}
\renewtheorem{corollary}{Corollaire}[theorem]


\newtheorem{algoproof}{Preuve}[algo]
\newtheorem{complexity}[algoproof]{Complexité}


% ██   ██ ███████ ██      ██████
% ██   ██ ██      ██      ██   ██
% ███████ █████   ██      ██████
% ██   ██ ██      ██      ██
% ██   ██ ███████ ███████ ██



\newcommand{\hdelta}{\hat{\delta}}
\newcommand{\automaton}{$A=(Q,\Sigma, q_0, \delta, F)$\ }
\newcommand{\automatonbis}{$B=(Q_B,\Sigma_b, q_b, \delta_b, F_b)$\ }
\newcommand{\fifo}{$F=(Q,C, \Sigma, q_0, \Theta, \delta)$\ }
\newcommand{\fifoA}{$A=(Q_A,C, \Sigma, q_{0A}, \Theta_A, \delta_A)$\ }
\newcommand{\fifoB}{$B=(Q_B,C, \Sigma, q_{0B}, \Theta_B, \delta_B)$\ }
\newcommand{\tsys}{$\mathcal{T}=(S,\Theta, \rightarrow)$\ }

\newcommand{\re}{$R_E$\ }
\newcommand{\rb}{$R_B$\ }
\newcommand{\rl}{$R_L$\ }
\newcommand{\ro}{$R_O$\ }

%%%%%%%%%%%%%%%%%%%%%%%%%%%%%%%%%%%%%%%%%%%%%%%%%%%%


\newcommand{\student}{Benjamin André}
\newcommand{\grade}{MAB2 Sciences Informatiques}
\newcommand{\director}{Véronique Bruyère}
\renewcommand{\title}{Apprentissage actif d'automates}
\renewcommand{\date}{\today}

\begin{document}

	\begin{titlepage}
		%
		\noindent
		%
		\newgeometry{bottom = 2cm, top = 2.5cm}
		\begin{center}
			\includegraphics[scale=1.2]{res/UMONS}\\
			\vspace*{0.3cm}
			\includegraphics[scale=0.23]{res/FS_Logo}\\
			\vspace*{2.5cm}
			%
			\titlefont \title \par
			%
		\end{center}
		\vspace*{3cm}
		\hfill
		%
		\begin{minipage}{0.18\linewidth}
			\begin{flushright}
				\rule{0.5pt}{50pt}
			\end{flushright}
		\end{minipage}
		%
		\begin{minipage}{0.8\linewidth}
			\begin{flushleft}
				\textsf{\textbf{Étudiant:}} \student\\
				\textsf{\textbf{Directrice:}}\director\\
				\date
			\end{flushleft}
		\end{minipage}
		%
		\vspace*{\fill}                                                             
		%
		\begin{center}
			Faculté des Sciences $\bullet$ Université de Mons $\bullet$ 
			Place du Parc 20 $\bullet$ B-7000 Mons
		\end{center}
		%
		\titlepagedecoration
		%
	\end{titlepage}
	%
	%
	%%%% Tables des matières
	%
	%
	\newgeometry{top = 3cm, left = 2cm, right = 2cm, bottom=2.5cm}
	\cleardoublepage

	\tableofcontents
	\newpage

	\section*{Environnements théoriques}
	\theoremlisttype{allname}
	\listtheorems{lemma,proposition,theorem,corrolary,proof}

	\section*{Exemples}
	\theoremlisttype{allname}
	\listtheorems{example}

	\section*{Environnements algorithmiques}
	\theoremlisttype{allname}
	\listtheorems{algo,algoproof,complexity}

	\newpage

	\chapter{Introduction}\label{ch:intro}Huitième version du document.


Le but de ce mémoire est de comprendre et implémenter l'article "Actively learning to verify safety for FIFO automata" \cite{Vardhan04}.

Pour ce faire, plusieurs étapes sont necéssaires. Dans le chapitre \ref{ch:bases}, les bases théoriques et leurs notations sont rappelées. Celles-ci concernent les automates, les langages et l'algorithme d'Angluin.

Dans le chapitre \ref{ch:specific}, des notions plus spécifiques telles que les automates à files ou les langages de trace (développés dans l'article) sont abordées.

Ensuite, le chapitre \ref{ch:lever} s'appuie sur toutes ces notions et explique l'algorithme LeVer (Learning to Verify) qui permet justement l'apprentissage actif d'automates à files pour vérifier leur sécurité.

Le chapitre \ref{ch:impl} mentionne les choix techniques faits pour l'implémentation avant de présenter et discuter les résultats obtenus.

Finalement, le chapitre \ref{ch:ccl} résume l'apport et les conséquences de ce travail avant de proposer des pistes d'amélioration.


	\chapter{Bases théoriques}\label{ch:bases}Dans la section \ref{sec:langage}, les notions de langage sont posées. Elles sont ensuite utilisées dans la section \ref{sec:automaton} sur les automates. L'algorithme "Table Filling" de la section \ref{sec:tfa} se base sur ces automates et permet de minimiser et répondre à la requête d'équivalence. Cette requête d'équivalence est une des deux requêtes necéssaire au fonctionnement de l'algorithme d'angluin de la section \ref{sec:angluin}.

Tous ces éléments combinés permettent la compréhension de l'article \cite{Vardhan04}.

	\section{Langage}\label{sec:langage}Cette section pose différents concepts et notations pour arriver à la notion de langage. Celle-ci reprennent les notations proposées par Hopcroft et al. \cite{Hopcroft00}.

\subsection{Alphabet}

Un \emph{alphabet}, nommé $\Sigma$ par convention, est un ensemble fini et non vide de \emph{symboles}.

\begin{exemple} Voici trois alphabets : 
	\begin{itemize}
		\item $\Sigma = \{0,1,2,3,4,5,6,7,8,9\}$, l'alphabet des chiffres
		\item $\Sigma = \{a,b,c,...,z,A,B,C,...,Z\}$, l'alphabet latin
		\item $\Sigma = \{0,1\}$, l'alphabet binaire
	\end{itemize}
\end{exemple}


\subsection{Mots}

Soit l'alphabet $\Sigma$ et un entier naturel $k$. Un \emph{mot} sur $\Sigma$ est une suite finie de $k$ éléments de $\Sigma$ notée $ w = a_1 \dots a_k $.

L'entier $k$ est la \emph{longueur} de ce mot aussi notée $|w|=k$.

\begin{exemple}
	$w=01110010$ est un mot sur $\Sigma=\{0,1\}$
\end{exemple}


Le \emph{mot vide} est un mot de taille $k=0$, noté $w=\epsilon$.

$\Sigma^k$ est l'ensemble des mots sur $\Sigma$ de longueur $k$.
	
L'ensemble de tous les mots possibles sur $\Sigma$ est noté $\Sigma^* = \bigcup_{k=0}^{\infty}\Sigma^k$.

La \emph{concaténation} de deux mots $w=a_1\dots a_k$ et $x=b_1\dots b_j$ est l'opération consistant à créer un nouveau mot $wx$, mot de taille $i=k+j$ s'écrivant $wx=a_1\dots a_kb_1\dots b_j$.

\begin{exemple}
	Soient les mots $x=41$ et $y=31$. Alors $xy=4131$ et $yx=3141$.
\end{exemple}

\begin{lemma}
	$\epsilon$ est \emph{l'identité pour la concaténation}, à savoir pour tout mot $w$, $w\epsilon = \epsilon w = w$. Par définition de la concaténation, tout mot concaténé avec $\epsilon$ retourne le même mot. $\square$
\end{lemma}

\emph{L'exponentiation} d'un symbole $a$ à la puissance $k$, notée $a^k$, retourne un mot de longueur $k$ obtenu par la concaténation de copies du symbole $a$. Noter que $a^0=\epsilon$.

\subsection{Langage}

Un ensemble de mots sur $\Sigma$ est un \emph{langage} \cite{Hopcroft00}, noté $L$. Alternativement, $L \subseteq \Sigma^*$. Étant donné que $\Sigma^*$ est infini, $L$ peut l'être également.

\begin{exemple} Voici des exemples, utilisant plusieurs modes de définition. $\Sigma$ y est implicite, mais il peut être donné explicitement.
	\begin{itemize}
		\item $L=\{12,35,42,7,0\}$, un ensemble défini explicitement
		\item $L=\{0^k1^j|k+j=7\}$, les mots de 7 symboles sur $\Sigma=\{0,1\}$ ne contenant pas $10$. Ici, $L$ est donné par notation ensembliste
		\item L est donné par "Tous les noms de villes belges.". Ici $L$ est défini en français.
		\item $\emptyset$ est un langage sur tout alphabet.
		\item $L=\{\epsilon\}$ ne contient que le mot vide, et est un langage sur tout alphabet.
	\end{itemize}
\end{exemple}


\subsubsection*{Opérations sur les langages}

Soient $L$ et $M$ deux langages. $L \cup M = \{w | w \in L\vee w \in M\}$ est l'\emph{union} de ces deux langages et en donne un nouveau. Ce langage est composé des mots venant d'un des deux langages.

Le langage composé de tous les mots produit par la concaténation d'un mot de $L$ avec un mot de $M$ est une \emph{concaténation} de ces deux langages et s'écrit $LM$.

La \emph{fermeture} de $L$ est notée $L^*$ et donne un langage constitué de tous les mots qui peuvent être construits par un concaténation d'un nombre arbitraire de mots de $L$.

\subsection{Expression régulière}\label{ss:regex}

Certains langages peuvent être exprimés par une \emph{expression régulière}. Un exemple de celles-ci est $01*0$ qui décrit la langage constitué de tous les mots commençant et finissant par $0$ avec uniquement des $1$ entre les deux.

Les expressions régulières suivent un algèbre avec ses opérations et leur priorités. Le langage décrit par une expression est construit de façon inductive par ces différentes opérations. Pour une expression régulière $E$, le langage exprimé est noté $L(E)$. Un langage qui peut être exprimé par une expression régulière est dit \emph{langage régulier}.


\textbf{Cas de base}
Certains langages peuvent être construits directement sans passer par l'induction:

\begin{itemize}
	\item $\epsilon$ est une expression régulière. Elle exprime le langage $L(\epsilon)=\{\epsilon\}$
	\item $\emptyset$ est une expression régulière décrivant $L(\emptyset)=\emptyset$
	\item Si $a$ est un symbole, alors \textbf{a} est une expression régulière composée uniquement de $a$. $L(a) = \{a\}$.
	\item Une variable, souvent en majuscule et italique, représente un langage quelconque, par exemple $L$.
\end{itemize}


\textbf{Induction}
Les autres langages réguliers sont construits suivant différentes règles d'induction présentées par ordre décroissant de priorité :

\begin{itemize}
	\item Si $E$ est une expression régulière, $(E)$ est une expression régulière et $L((E)) = L(E)$.
	\item Si $E$ est une expression régulière, $E^*$ est une expression régulière représentant la fermeture de $L(E)$, à savoir $L(E^*) = L(E)^*$.
	\item Si $E$ et $F$ sont des expressions régulières, $EF$ est une expression régulière décrivant la concaténation des deux langages représentés, à savoir $L(EF)=L(E)L(F)$. La concaténation étant commutative, l'ordre de groupement n'est pas important, mais par convention, la priorité est à gauche.
	\item Si $E$ et $F$ sont des expressions régulières, $E+F$ est une expression régulière donnant l'union des deux langages représentés, à savoir $L(E+F)=L(E)\cup L(F)$. Ici encore, l'opération est commutative et la priorité est à gauche.
\end{itemize}
	
\begin{exemple}
	Soit l'expression $E = (b+ab)b^*a(a+b)^*$ qui représente le langage $L$.\\
	\begin{itemize}
		\item \textbf{ba} fait partie de $L$. En effet, en développant $E$ avec des choix sur les unions et le degré d'une fermeture, on obtient  $E= (b)b^0a(a+b)^0 = b\epsilon a \epsilon = ba$.
		\item \textbf{ababbab} fait partie de $L$. En développant à nouveau $E$ en posant des choix sur les unions et fermetures, on obtient $E=(ab)b^0a(a+b)^4 = ab\epsilon a (a+b)(a+b)(a+b)(a+b) = ababbab$.
		\item \textbf{aa} ne fait \textbf{pas} partie de $L$. Supposons par l'absurde que $aa \in L$. Alors il existerait une façon de décomposer $E$ en $aa$. Or, les premiers symboles doivent être soit $b$, soit $ab$. Il y a contradiction : $E$ ne peut pas être décomposé. Comme $aa$ ne peut pas être construit par $E$, $aa \notin L$.
	\end{itemize}
	\label{ex:regex}
\end{exemple}
	\section{Automate Fini}\label{sec:automaton}Cette section décrit les automates finis, fait le lien avec la notion de langage régulier et propose une représentation visuelle de ces automates.

\TODO{sources}

\subsection{Définitions}\label{ss:autodef}


% ██████  ███████ ███████ ██ ███    ██ ██ ████████ ██  ██████  ███    ██
% ██   ██ ██      ██      ██ ████   ██ ██    ██    ██ ██    ██ ████   ██
% ██   ██ █████   █████   ██ ██ ██  ██ ██    ██    ██ ██    ██ ██ ██  ██
% ██   ██ ██      ██      ██ ██  ██ ██ ██    ██    ██ ██    ██ ██  ██ ██
% ██████  ███████ ██      ██ ██   ████ ██    ██    ██  ██████  ██   ████


Un \emph{automate fini} \automaton est défini comme suit :
\begin{itemize}
  \item $Q$ est un ensemble fini d'\emph{états}
  \item $\Sigma$ est un alphabet
  \item $q_0 \in Q$ est l'\emph{état initial}
  \item $\delta$ est la \emph{fonction de transition}
  \item $F \subseteq Q$ est un ensemble d'\emph{états acceptants}.
\end{itemize}

La fonction de transition $\delta$ est définie différemment en fonction du type d'automate souhaité :
\begin{itemize}
  \item \textbf{Automate Déterministe Fini (ADF)} $\delta : Q \times \Sigma \rightarrow Q$. Soit un état $q$ et un symbole $a$. Alors la \emph{transition} $\delta(q,a)$ retourne un état $p$. $\delta(q,a)$ doit être définie pour tout état et tout symbole.
  \item \textbf{Automate Non-déterministe Fini (ANF)} $\delta : Q \times \Sigma \rightarrow 2^Q$. Soit un état $q$ et un symbole $a$. Alors la transition $\delta(q,a)$ retourne un ensemble d'états $P=\{p_1,p_2,\dots,p_n\}\subseteq Q$.
  \item \textbf{Automate Non-déterministe Fini avec des transitions sur $\epsilon$ ($\epsilon$-ANF)} $\delta : Q \times \Sigma \cup \{\epsilon\} \rightarrow 2^Q$. Pareil que précédemment mais une transition peut exister sans symbole : elle se fait alors sur $\epsilon$.
\end{itemize}

Lorsqu'un automate est mentionné dans ce document, il s'agit implicitement d'un $\epsilon$-ANF, sauf mention contraire. En effet, c'est la forme la plus générale. Cependant, ces trois types d'automates ont la même puissance expressive, ce qui est prouvé dans la section \ref{ss:eqadfanf}.

Soit la transition $\delta(q,a)=p$ (dans un ADF). Pour $q$, c'est une \emph{transition sortante sur a}. Pour $p$, c'est une \emph{transition entrante sur a}. Si $\delta(q,a)=P=\{p_1,p_2,\dots,p_n\}$ dans un ANF, alors les états $\{p_1,p_2,\dots,p_n\}$ auront une transition entrante sur $a$.

Dans le cas de ANF et $\epsilon$-ANF, il peut être pratique d'utiliser $\delta$ sur un ensemble d'états $S$. A ce moment, $\delta(S,a)=\bigcup_{q\in S}\delta(q,a)$ avec $a\in \Sigma$.


% ███████ ██   ██ ███████ ███    ███ ██████  ██      ███████ ███████
% ██       ██ ██  ██      ████  ████ ██   ██ ██      ██      ██
% █████     ███   █████   ██ ████ ██ ██████  ██      █████   ███████
% ██       ██ ██  ██      ██  ██  ██ ██      ██      ██           ██
% ███████ ██   ██ ███████ ██      ██ ██      ███████ ███████ ███████


\begin{example}[Automate déterministe fini]\label{ex:adf}
  On considère l'automate \automaton défini comme suit :
  \begin{itemize}
    \item $Q=\{q_0,q_1,q_2,q_3,q_4,q_5,q_6\}$
    \item $\Sigma=\{a,b\}$
    \item $q_0$ est l'état du même nom
    \item La fonction de transition $\delta$ est décrite par la table \ref{fig:transdelta}. L'intersection d'une ligne reprenant un élément $q \in Q$ et d'une colonne $a \in \Sigma$ donne l'état $\delta(q,a)$.
    \item $F=\{q_3\}$
  \end{itemize}

  \begin{figure}[H]
    \centering
    \begin{tabular}{|r||c|c|}
      \hline
      &a&b\\
      \hline\hline
      $\rightarrow q_0$&$q_2$&$q_1$\\\hline
      $q_1$&$q_3$&$q_5$\\\hline
      $q_2$&$q_4$&$q_5$\\\hline
      $q_3^*$&$q_3$&$q_3$\\\hline
      $q_4$&$q_4$&$q_4$\\\hline
      $q_5$&$q_3$&$q_1$\\\hline
      $q_6$&$q_4$&$q_5$\\\hline
    \end{tabular}
    \caption{La table de transitions $\delta$}
    \label{fig:transdelta}
  \end{figure}
\end{example}

Via cette notation, $Q$ et $\Sigma$ sont explicites. En dénotant l'état initial par $\rightarrow$ et les états acceptants par $*$ en exposant, on obtient une définition complète d'un automate : $(Q,\Sigma, q_0, \delta, F)$.

\begin{example}[Automate non-déterministe fini avec une transition sur $\epsilon$]\label{ex:anf}
	 De la même façon que pour l'exemple précédant, considérons un automate \automaton défini comme suit :

\begin{itemize}
	\item $Q=\{q_0,q_1,q_2\}$
	\item $\Sigma=\{a,b,c\}$
	\item $q_0$ est l'état du même nom
	\item $\delta$ est donnée par la table \ref{fig:eanfdelta}.
	\item $F=\{q_2\}$
\end{itemize}

$A$ est un $\epsilon$-ANF ; une colonne supplémentaire sert à représenter la transition sur $\epsilon$.

\begin{figure}[H]
	\centering
	\begin{tabular}{|r||c|c|c|c|}
		\hline
		&$\epsilon$&a&b&c\\
		\hline\hline
		$\rightarrow q_0$&$\{q_1,q_2\}$&$\emptyset$&$\{q_1\}$&$\{q_2\}$\\\hline
		$q_1$&$\emptyset$&$\{q_0\}$&$\{q_2\}$&$\{q_0,q_1\}$\\\hline
		$q_2^*$&$\emptyset$&$\emptyset$&$\emptyset$&$\emptyset$\\\hline
	\end{tabular}
	\caption{$\delta$}
	\label{fig:eanfdelta}
\end{figure}

Un exemple d'ANF sans transition sur $\epsilon$ serait juste A sans ces dites transitions. Il ne seraient cependant pas équivalents.

\end{example}


%  ██████  ██████   █████  ██████  ██   ██ ███████
% ██       ██   ██ ██   ██ ██   ██ ██   ██ ██
% ██   ███ ██████  ███████ ██████  ███████ █████
% ██    ██ ██   ██ ██   ██ ██      ██   ██ ██
%  ██████  ██   ██ ██   ██ ██      ██   ██ ███████

\subsection{Représentation graphique}

Le \emph{graphe d'un automate fini} \automaton est un graphe dirigé construit comme suit :

\begin{itemize}
  \item Chaque état de $Q$ est représenté par un nœud.
  \item Chaque transition $\delta(q,a)$ est représenté par un arc étiquetté $a$. Dans le cas d'un automate non-déterministe, un arc existe pour chacun des états obtenus en suivant la transition. Si il y a plusieurs transitions sortant d'un même état et entrant dans un même autre état, les arcs peuvent être fusionnés en listant les étiquettes.
  \item L'état initial est mis en évidence par une flèche entrante.
  \item Les états acceptants sont représentés par un double cercle, en opposition au simple cercle des autres nœuds.
\end{itemize}

\begin{example}[Graphe d'automate]
 Voici les graphes représentant les automates définis dans la section précédente :

 \begin{minipage}[t]{0.5\textwidth}
   \begin{figure}[H]
    \centering
    \begin{tikzpicture}[->,>=stealth',shorten >=1pt,auto,node distance=2.5cm, semithick, bend angle=10]

    \tikzstyle{every state}=[circle]

    \node[initial,state] (A)                    {$q_0$};
    \node[state]         (B) [below right of=A] {$q_1$};
    \node[state]         (C) [below left of=A] {$q_2$};
    \node[accepting, state]         (D) [below right of=B] {$q_3$};
    \node[state]         (E) [below left of=C]       {$q_4$};
    \node[state]         (G) [below right of=E]       {$q_6$};
    \node[state]         (F) [above right of=G]       {$q_5$};

    \path 	(A) 	edge              node {a} (C)
    edge              node {b} (B)
    (B) 	edge              node {a} (D)
    edge [bend left]  node {b} (F)
    (C) 	edge              node {a} (E)
    edge              node {b} (F)
    (D) 	edge [loop above] node {a,b} (D)
    (E) 	edge [loop above] node {a,b} (E)
    (F) 	edge              node {a} (D)
    edge [bend left]  node {b} (B)
    (G) 	edge              node {a} (E)
    edge              node {b} (F);
    \end{tikzpicture}
    \caption{Graphe de l'automate de la table \ref{fig:transdelta}}\label{fig:a1}
   \end{figure}
 \end{minipage}
 \begin{minipage}[t]{0.5\textwidth}
   \begin{figure}[H]
   	\centering
   	\begin{tikzpicture}[->,>=stealth',shorten >=1pt,auto,node distance=2.5cm, semithick, bend angle=10]

   	\tikzstyle{every state}=[circle]

   	\node[initial,state] (A)                    {$q_0$};
   	\node[state]         (B) [above right of=A] {$q_1$};
   	\node[accepting, state]         (C) [below right of=A] {$q_2$};

   	\path
   	(A) edge [bend left] node{$\epsilon$,b} (B)
   	(A) edge node{$\epsilon$,c} (C)
   	(B) edge [bend left] node{a,c} (A)
   	(B) edge [loop right] node{c} (B)
   	(B) edge node{b} (C);
   	\end{tikzpicture}
   	\caption{Graphe de l'automate de la table \ref{fig:eanfdelta}}\label{fig:eanf}
   \end{figure}
\end{minipage}
\end{example}

 Cette représentation d'un automate peut sembler plus naturelle pour un humain alors que la table de transitions est plus proche d'un langage informatique. De plus, dans la représentation par graphe, les ensembles $Q$ et $\Sigma$ sont implicites et doivent être définis ou déduits à part.



% ███████  ██████ ██       ██████  ███████ ███████
% ██      ██      ██      ██    ██ ██      ██
% █████   ██      ██      ██    ██ ███████ █████
% ██      ██      ██      ██    ██      ██ ██
% ███████  ██████ ███████  ██████  ███████ ███████


\subsection{ECLOSE}
Soit un $\epsilon$-ANF \automaton. Il est possible de construire un fonction retournant l'ensemble des états atteints uniquement en suivant des transitions sur $\epsilon$ pour un état $q$ donné. Cette fonction est la \emph{fermeture sur epsilon} $ECLOSE : Q \rightarrow 2^Q$. Sa définition est inductive.\\

Soit $q$ un état dans $Q$.\\
\textbf{Cas de base} $q$ est dans ECLOSE($q$)\\
\textbf{Pas de récurrence} Si $p$ est dans ECLOSE($q$) et qu'il existe un état $r$ tel quel $r\in\delta(p,\epsilon)$, alors $r$ est dans ECLOSE($q$)\\

ECLOSE peut être utilisé indifféremment sur un ensemble d'états S ($ECLOSE : 2^Q \rightarrow 2^Q$). Alors, $ECLOSE(S)=\bigcup_{q\in S}ECLOSE(q)$

\begin{example}[ECLOSE]\label{ex:anfclosure} Considérons l'automate $A$ de l'exemple \ref{ex:anf}. Les différentes fermetures peuvent être calculées :
	\begin{itemize}
		\item ECLOSE($q_0$) = $\{q_0,q_1,q_2\}$. En effet, $q_0$ appartient à sa fermeture, selon le cas de base. Aussi, $q_1,q_2\in\delta(q_0, \epsilon)$
		\item ECLOSE($q_1$)=$\{q_1\}$ par le cas de base.
		\item ECLOSE($q_2$)=$\{q_2\}$ par le cas de base.
	\end{itemize}
\end{example}


% ██       █████  ███    ██  ██████   █████   ██████  ███████
% ██      ██   ██ ████   ██ ██       ██   ██ ██       ██
% ██      ███████ ██ ██  ██ ██   ███ ███████ ██   ███ █████
% ██      ██   ██ ██  ██ ██ ██    ██ ██   ██ ██    ██ ██
% ███████ ██   ██ ██   ████  ██████  ██   ██  ██████  ███████


\subsection{Langage}

Un automate représente un langage, tel que défini dans la section \ref{sec:langage}.

\subsubsection*{Fonction de transition étendue}

La \emph{fonction de transition étendue} $\hdelta$ est une extension de la fonction de transition, acceptant plusieurs symboles de façon consécutive. Intuitivement, il s'agit de suivre plusieurs arcs sur le graphe.

Comme $\delta$ est différente en fonction du type d'automate considéré, $\hdelta$ l'est aussi.

\begin{itemize}
  \item \textbf{ADF :} $\hdelta : Q \times \Sigma^* \rightarrow Q$. $\hdelta$ prend en entrée un état de $Q$ et un mot $w$ sur $\Sigma$ et retourne un état de $Q$.
  \item \textbf{ANF et $\epsilon$-ANF :} $\hdelta : Q \times \Sigma^* \rightarrow 2^Q$. $\hdelta$ prend en entrée un état de $Q$ et un mot $w$ sur $\Sigma$ et retourne un ensemble d'états de $Q$.
\end{itemize}

Soit un état $q \in Q$ et un mot $w \in \Sigma^*$. Alors $\hdelta$ est définie par:\\
\textbf{Cas de base} Il y a deux cas de base:
\begin{itemize}
 \item $w$ est un mot vide :
    \begin{itemize}
      \item Pour un ADF ou ANF : $\hdelta(q, \epsilon) = q$.
      \item Pour un $\epsilon$-ANF : $\hdelta(q, \epsilon) = ECLOSE(q)$.
    \end{itemize}
 \item $w$ est un symbole : $\hdelta(q, w)=\hdelta(q,a)$ avec $w=a\in \Sigma$.
    \begin{itemize}
      \item Pour un ADF ou ANF : $\hdelta(q,a)=\delta(q,a)$.
      \item Pour un $\epsilon$-ANF : $\hdelta(q,a)=ECLOSE(\delta(q,a))$.
    \end{itemize}
\end{itemize}
\textbf{Pas de récurrence} Si $|w|>1$, alors $w=xa$ avec $x$ un mot sur $\Sigma$ et $a$ un symbole de $\Sigma$.
\begin{itemize}
  \item Pour un ADF ou ANF : $\hdelta(q,w) = \hdelta(q,xa)= \delta(\hdelta(q,x),a)$
  \item Pour un $\epsilon$-ANF : $\hdelta(q,w) = \hdelta(q,xa)= ECLOSE(\delta(\hdelta(q,x),a))$.
\end{itemize}


Il se peut que la fonction de transition $\delta$ ne soit pas définie pour une paire d'arguments. Auquel cas, $\hdelta$ ne l'est pas non plus.\\

\subsubsection*{Chemin}

Un \emph{chemin} est une application de cette fonction sur un état et un mot.

\begin{example}[Chemin]
 Considérons l'automate $A$ de la figure \ref{fig:a1}. Il existe un chemin de $q_0$ à $q_5$ : $\hdelta(q_0, ab) = \delta(\hdelta(q_0,a),b) = \delta(\delta(q_0,a),b) = \delta(q_2, b)=q_5$.
\end{example}

\subsubsection*{Langage}
Le langage représenté par un automate \automaton peut alors se définir comme les mots qui, par l'application de $\hdelta$ sur l'état initial, donnent un état acceptant. Voici les définitions ensemblistes, respectivement pour un ADF et pour un $(\epsilon)$-ANF
$$
L(A)= \{w \in \Sigma^* | \hdelta(q_0,w) \in F\} \hspace{0.5cm}
L(A)= \{w \in \Sigma^* | \exists q \in \hdelta(q_0,w), q \in F\}
$$
Ainsi, un mot $w$ appartient à un langage $L$ défini par l'automate $A$ si $\hdelta(q_0,w) \in F$. L'algorithme \ref{alg:membership} représente cette appartenance pour un mot.


\begin{algo}[Appartenance d'un mot à un langage défini par un automate]\label{alg:membership}
 \begin{algorithmic}[1]
   \REQUIRE un mot $w$, un automate \automaton représentant $L$
   \ENSURE si $w$ appartient à $L$

   \STATE $C \leftarrow \{q_0\}$ \COMMENT{$C$ sont les états courants}
   \WHILE {$|w| > 0$}
   \STATE décomposer $w$ en $ax$ avec $a\in\Sigma$ et $x$ le reste du mot
   \STATE $C \leftarrow \delta(C,a)$ \COMMENT{passage à l'état suivant}
   \STATE \COMMENT{Si l'automate est un ADF, C ne contient toujours qu'un seul état}
   \STATE $w \leftarrow x$
   \ENDWHILE

   \RETURN s'il existe un état $q \in C$ appartenant à $F$
 \end{algorithmic}
\end{algo}

\begin{complexity}[Lecture d'un mot par un automate]
 Si $|w|=n$, l'algorithme \ref{alg:membership} est en $\mathcal{O}(n)$. En effet, les étapes 1 et 8 sont en temps constant. La boucle de l'étape 2 est parcourue $n$ fois (la taille étant diminuée de 1 exactement à chaque itération). Le test de 2 et les opérations de 3 et 6 peuvent être faites en temps constant (par exemple, en voyant w comme une queue). L'étape 4, déterminante, peut être effectuée en temps constant également, par exemple avec l'utilisation d'un tableau de transition.

\end{complexity}


% ██      ██    █████
% ██       ██  ██   ██
% ██ █████  ██ ███████
% ██       ██  ██   ██
% ███████ ██   ██   ██

\subsection{Construction d'un automate depuis un langage régulier}
Soit le langage $A_N = \{w | w \text{ fini par b et ne contient pas bb}\}$ défini sur $\Sigma_N = {a,b}$.

On peut diviser les mots en 3 ensembles :

\begin{itemize}
 \item $W_0$ le sous-ensemble des mots ne finissant pas le symbole $b$
 \item $W_1$ celui des mots finissant par le symbole $b$ mais ne contenant pas $bb$
 \item $W_2$ celui des mots contenant au moins $bb$
\end{itemize}

Il y a d'autres façons de construire des sous-ensembles, mais celle-ci à l'avantage de rendre la question de l'appartenance à $L_N$ triviale : un mot appartient au second ensemble si et seulement si il fait partie du langage, par définition.

De plus, tous les éléments d'un sous-ensemble respectent la relation $R_L$ entre eux. ($R_L : xR_Ly \Leftrightarrow \forall z \in \Sigma^*, xz \in L \Leftrightarrow yz \in L$). Cela en fait des classes d'équivalence sur cette relation.

Cela peut être démontré pour chaque sous-ensemble :
\begin{itemize}
 \item Soient $x,y \in W_0$. Soit $z \in \Sigma^*$. Dès lors, si $xz \in L_N$, c'est que $z$ fini par $b$ mais ne contient pas $bb$, et donc $yz \in L_N$. Si $yz \in L_N$, le même argument peut être appliqué.
 \item Soient $x,y \in W_1$. Soit $z \in \Sigma^*$. Dès lors, si $xz \in L_N$, c'est que $z$ ne commençait pas le symbole $b$ et ne contenait pas $bb$, $yz$ ne contiendra donc pas $bb$, puisque cette chaîne n'est ni dans $z$ ni dans $y$, ni a cheval sur les deux, $z$ ne commençant pas par $b$. Ainsi, $yz \in L_N$. Si $yz \in L_N$, le même argument peut être appliqué.
 \item Soient $x,y \in W_2$. Soit $z \in \Sigma^*$. Comme $x$ contient déjà $bb$, $x \notin L_N$ et, a fortiori, $xz \notin L_N$. Comme la prémisse est fausse, l'implication $xz \in L \Rightarrow yz \in L$ est vraie. La même logique peut être appliquée à partir de $y$ pour justifier l'implication inverse.
\end{itemize}

De plus, ces sous-ensembles sont disjoints. Cela peut se prouver en invalidant la relation pour certains éléments entre eux, mais dans ce cas-ci, la propriété est assurée par définition.

Ceci revient à démontrer que $W_0,W_1,W_2$ sont des classes d'équivalence. De plus, $R_L$ respecte la congruence à droite, comme démontré dans la preuve du théorème de Myhill-Nérode. Ce même théorème donne une méthode pour construire un automate : prendre un représentant pour chaque classe et en faire un état.

\begin{itemize}
 \item $\Sigma=\{a,b\}$ est connu.
 \item $Q=\{[[\epsilon]]\, [[b]], [[bb]]\} = \{q_\epsilon, q_b, q_{bb}\}$
 \item $q_0 = q_\epsilon$
 \item $F = \{q_b\}$ l'union des classes acceptant
 \item $\delta$ défini en utilisant des exemples tirés des classes d'équivalence.
\end{itemize}

Ce qui donne l'automate de la figure \ref{fig:an}

\begin{figure}[H]
 \centering
 \begin{tikzpicture}[->,>=stealth',shorten >=1pt,auto,node distance=2cm and 2cm, semithick, bend angle=10]

 \tikzstyle{every state}=[circle]

 \node[initial,state]	(A)					{$q_\epsilon$};
 \node[accepting,state]	(B)	[right= of A]	{$q_b$};
 \node[state]			(C) [right= of B]	{$q_{bb}$};

 \path
 (A)	edge	[bend left]		node{b}		(B)
 (A)	edge	[loop above]	node{a}		(A)
 (B) edge	[bend left]		node{a}		(A)
 (B) edge					node{b}		(C)
 (C)	edge	[loop above]	node{a,b}	(C)

 ;
 \end{tikzpicture}
 \caption{Automate $A_N$, exemple d'une thèse\cite{Neider14}}\label{fig:an}
\end{figure}

Cet automate est bien une représentation du langage $L_N$. Seul un mot finissant par $b$ mais ne contenant pas $bb$ se termine à l'état $q_b$.



% ███████       ██     ██        █████
% ██           ██       ██      ██   ██
% █████       ██  █████  ██     ███████
% ██           ██       ██      ██   ██
% ███████       ██     ██       ██   ██

\subsection{Équivalence entre expression régulière et automate}
\begin{proposition}[ADF et expression régulière]
	Un langage peut être exprimé par un automate déterministe fini si et seulement si il peut être exprimé par une expression régulière.
\end{proposition}

Cette proposition étant une double implication, elle est vraie si les deux implications le sont. Soit un langage $L$.


\begin{theorem}[ADF $\implies$ expression régulière]
	Il existe un automate déterministe $A$ tel que $L(A)=L$ $\implies$ il existe une expression régulière $E$ telle que $L(E)=L$.
\end{theorem}
\begin{proof}
	Supposons qu'il existe un ADF \automaton tel que $L(A)=L$. $Q$ étant un ensemble fini, on peut définir sa cardinalité : $|Q|=n$. Supposons que ses états soient nommés $\{1,2,\dots,n\}$. Il est possible de construire des expressions régulières par induction sur le nombre d'états considérés.

	Posons $E_{ij}^k$ l'expression régulière exprimant un langage constitué des mots $w$ tels que $\delta(i,w)=j$ et qu'aucun état intermédiaire n'ait un nombre supérieur à $k$. Il n'y a pas de contrainte sur $i$ et $j$.

	\begin{figure}[H]
		\centering
		\begin{tikzpicture}[->,>=stealth',shorten >=1pt,auto,node distance=2cm, semithick, bend angle=10]

		\tikzstyle{every state}=[circle]

		\node[initial,state] (A)                    {$1$};
		\node[state]         (B) [right of=A] {$2$};
		\node[state]         (C)  [right of=B] {$3$};
		\node[accepting, state]         (D)  [right of=C] {$4$};
		\node[state]         (E)  [right of=D]       {$5$};

		\path
		(A) edge [bend left=45] node{a} (C)
		(A) edge  node{b} (B)

		(B) edge [bend left=45] node{a} (A)
		(B) edge node{b} (C)

		(C) edge [bend left=45,dashed] node{a} (E)
		(C) edge [loop below] node{b} (C)

		(D) edge [loop below,dashed] node{a,b} (D)
		(E) edge [,dashed] node{a,b} (D)

		;
		\end{tikzpicture}
		\caption{Exemple : automate mettant $E_{1,3}^3$ en évidence}\label{fig:proofeijk}
	\end{figure}


	L'exemple ci-dessus illustre ce fait qu'aucun état supérieur à $k$ ne peut faire partie des intermédiaires. Dans cet exemple, $E_{5,4}^3$ tolère la transition de $5$ à $4$ bien que supérieure à $k$ : ce ne sont pas des intermédiaires. Construisons le langage par induction sur les états autorisés.

	\paragraph{Cas de base} $k=0$. Comme tout état est numéroté $1$ ou plus, aucun intermédiaire n'est accepté. La première possibilité est $i=j$ et indique un chemin de longueur $0$. Auquel cas l'expression régulière représentant un chemin sans symbole est $\epsilon$. Ce chemin doit être ajouté aux possibilités si $i=j$.
	La deuxième possibilité est $i \neq j$. Alors les chemins possibles ne se composent que d'un arc allant directement de $i$ à $j$. Pour les construire :

	Pour chaque paire $i$, $j$ :
	\begin{itemize}
		\item Il n'existe pas de symbole $a$ tel que $\delta(i,a)=j$. Alors, $R_{ij}^0=\emptyset (+\epsilon)$
		\item Il existe un unique symbole $a$ tel que  $\delta(i,a)=j$. Alors, $R_{ij}^0=a(+\epsilon)$
		\item Il existe des symboles $a_1,a_2,\dots,a_k$ tels que $\forall l \in \{1,\dots, k\}, \delta(i,a_l)=j$. Alors, $R_{ij}^0=a_1+a_2+\dots+a_k(+\epsilon)$
	\end{itemize}

	\paragraph{Pas de récurrence} Supposons qu'il existe un chemin allant de $i$ à $j$ ne passant par aucun état ayant un numéro supérieur à $k$.	La première possibilité est que le-dit chemin ne passe pas par $k$. Alors, le mot représenté par ce chemin fait partie du langage de $E_{ij}^{k-1}$. Seconde possibilité, le chemin passe par $k$ une ou plusieurs fois comme représenté à la figure \ref{fig:ikjpath}.

	\begin{figure}[H]\centering
	\begin{tikzpicture}[->,>=stealth',shorten >=1pt,auto,node distance=2cm, semithick, bend angle=10]

	\tikzstyle{every state}=[circle]

	\node[state] (A) {$i$};
	\node[state] (B) [right of=A] {$k$};
	\node[state] (C) [right of=B] {$k$};
	\node[state] (D) [right of=C] {$k$};
	\node[state] (E) [right of=D] {$k$};
	\node[state] (F) [right of=E] {$j$};

	\path
	(A) edge [snake] (B)
	(B) edge [snake] (C)
	(C) edge [snake] (D)
	(D) edge [snake] (E)
	(E) edge [snake] (F)
	;

	\draw[-] (0.3,-0.8) arc (-100:-80:4.5) ;
	\draw[-] (2.24,-0.8) arc (-100:-80:16) ;
	\draw[-] (8.3,-0.8) arc (-100:-80:4.5) ;


	\node[draw=none] at (1.25,-1.4) {Dans $E_{ik}^{k-1}$};
	\node[draw=none] at (5.25,-1.4) {Mots dans $E_{kk}^{k-1}$};
	\node[draw=none] at (9.25,-1.4) {Dans $E_{kj}^{k-1}$};

	\end{tikzpicture}
	\caption{Un chemin de $i$ à $j$ peut être découpé en différent segment en fonction de $k$}\label{fig:ikjpath}
	\end{figure}

	Auquel cas, ces chemins sont composés d'une sous-chemin donnant un mot dans $E_{ik}^{k-1}$, suivi d'un sous-chemin donnant un ou plusieurs mots dans $E_{kk}^{k-1}$ et finalement un mot dans $E_{kj}^{k-1}$.

	En combinant les expressions des deux types, on obtient :
	$$
	E_{ij}^k = E_{ij}^{k-1}+E_{ik}^{k-1}(E_{kk}^{k-1})*E_{kj}^{k-1}
	$$

	En commençant cette construction sur $E_{ij}^n$, comme l'appel se fait toujours à des chaînes plus courtes, éventuellement on retombe sur le cas de base. Si l'état initial est numéroté 1, alors l'expression régulière $E$ exprimant $L$ est l'union ($+$) des $E_{1j}^n$ tel que $j$ est un état acceptant.

	\hfill$\square$
\end{proof}
\stepcounter{algo}
\begin{complexity}
	Soit un ADF \automaton comportant $n$ états. La complexité peut se décomposer en deux facteurs : la longueur d'une expression régulière et le nombre de celles-ci.

	A chacune des $n$ itérations (ajoutant progressivement des nouveaux états admis pour état intermédiaire), la longueur de l'expression peut quadrupler : elle est exprimée par 4 facteurs. Ainsi, après $n$ étapes, cette expression peut être de taille $\mathcal{O}(4^n)$.

	Le nombre d'expressions à construire, lui, est décomposable en deux facteurs également : le nombre d'itérations et celui de paires $i,j$ possibles. Le premier facteur est $n$, quand aux paires, leur nombre s'exprime par $n^2$. $n^3$ expressions sont construites.

	En regroupant ces deux facteurs, on obtient $n^3\mathcal{O}(4^n)=\mathcal{O}(n^34^n)$. Comme $n$ correspond au nombre d'états, si la transformation se fait depuis un ANF, via un ADF, vers une expression régulière, la complexité devient doublement exponentielle, la première transformation étant elle-même exponentielle en le nombre d'états de l'ANF.
\end{complexity}


\begin{example}[Construction d'une expression régulière]
	Construction d'une expression régulière à partir de l'automate de la figure suivante :

	\begin{figure}[H]\centering
		\begin{tikzpicture}[->,>=stealth',shorten >=1pt,auto,node distance=2cm, semithick, bend angle=10]

		\tikzstyle{every state}=[circle]

		\node[initial,state] (A) {$1$};
		\node[accepting,state] (B) [right of=A] {$2$};

		\path
		(A) edge [bend left=20] node{0} (B)
		(B) edge [bend left=20] node{0} (A)

		(A) edge [loop above] node{1} (A)
		(B) edge [loop above] node{1} (B)
		;

		\end{tikzpicture}
		\caption{Un automate acceptant tout mot ayant un nombre impair de $0$}
	\end{figure}

	La construction par récurrence commençant avec $k=0$ le processus peut être représenté par des tableaux correspondant à différents $k$ de façon croissante.

	\paragraph{Première itération} Dans la première itération, chaque expression se résume à un des trois cas de base, avec éventuellement $\epsilon$ si $i=j$ pour l'expression analysée.

	\begin{figure}[H]
		\centering
		\begin{tabular}{|l|c|}
			\hline
			 & Cas de base\\
			\hline
			$E_{11}^0$& $1+\epsilon$\\
			$E_{12}^0$& $0$\\
			$E_{21}^0$& $0$\\
			$E_{22}^0$& $1+\epsilon$\\
			\hline
		\end{tabular}
	\end{figure}

	\paragraph{Seconde itération} Ensuite, l'état $1$ est autorisé comme état intermédiaire : $k=1$. Ayant potentiellement un état intermédiaire, la formule de récurrence est utilisée.

	\begin{figure}[H]
		\centering
		\begin{tabular}{|l|c|c|c|}
			\hline
			 & Formule de récurrence & Détail & Simplification\\
			\hline
			$E_{11}^1$& $E_{11}^0 + E_{11}^0(E_{11}^0)^*E_{11}^0$&
			$(1+\epsilon)+(1+\epsilon)(1+\epsilon)^*(1+\epsilon)$ & $1^*$\\
			$E_{12}^1$& $E_{12}^0 + E_{11}^0(E_{11}^0)^*E_{12}^0$&
			$0+(1+\epsilon)(1+\epsilon)^*0$ & $1^*0$ \\
			$E_{21}^1$& $E_{21}^0 + E_{21}^0(E_{11}^0)^*E_{11}^0$&
			$0+0(1+\epsilon)^*(1+\epsilon)$& $01^*$\\
			$E_{22}^1$& $E_{22}^0 + E_{21}^0(E_{11}^0)^*E_{12}^0$&
			$(1+\epsilon)+0(1+\epsilon)^*0$ & $1+01^*0$\\
			\hline
		\end{tabular}
	\end{figure}


	\paragraph{Troisième itération} A la troisième itération, l'état $2$ est autorisé comme état intermédiaire.

	\begin{figure}[H]
		\hspace{-5mm}\begin{tabular}{|l|c|c|c|}
			\hline
			 & Formule de récurrence & Détail & Simplification\\
			\hline
			$E_{11}^2$& $E_{11}^1 + E_{12}^1(E_{22}^1)^*E_{21}^1$&
			$1^*+1^*0(1+01^*0)^*01^*$&$1^*+1^*0(1+01^*0)^*01^*$\\
			$E_{12}^2$& $E_{12}^1 + E_{12}^1(E_{22}^1)^*E_{22}^1$&
			$1^*0+1^*0(1+01^*0)^*(1+01^*0)$&$1^*0(1+01^*0)^*$\\
			$E_{21}^2$& $E_{21}^1 + E_{22}^1(E_{22}^1)^*E_{21}^1$&
			$01^*+(1+01^*0)(1+01^*0)^*01^*$&$(1+01^*0)^*01^*$\\
			$E_{22}^2$& $E_{22}^1 + E_{22}^1(E_{22}^1)^*E_{22}^1$&
			$(1+01^*0)+(1+01^*0)(1+01^*0)^*(1+01^*0)$&$(1+01^*0)^*$\\
			\hline
		\end{tabular}
	\end{figure}

	Pour obtenir une expression régulière correspondant à l'automate, on s'intéresse à celle qui décrit un chemin entre l'état initial ($1$) et les états acceptants (uniquement $2$ ici). Dès lors,  $L(1^*0(1+01^*0)^*)=L$.

	Cette expression régulière $1^*0(1+01^*0)^*$ décrit bien un nombre impair de $0$. Il en faut absolument un, et tout ajout supplémentaire de se fait par paire. Cela correspond bien à un nombre impair.

\end{example}



\begin{theorem}[Expression régulière $\implies$ ADF]
	($\Leftarrow$) Il existe une expression régulière $E$ telle que $L(E)=L$ $\implies$  il existe un automate déterministe $A$ tel que $L(A)=L$.
\end{theorem}


\begin{proof}
	Comme tout ANF a un ADF équivalent (théorème \ref{anf-dnf}), montrer qu'une expression régulière $E$ a un ANF équivalent est suffisant pour obtenir cet ADF.

	Soit $L$. Soit $E$ une expression régulière telle que $L(E)=L$. On peut construire l'automate récursivement sur la définition des expressions régulières à la section \ref{ss:regex}. Cette preuve par récurrence repose sur trois invariants portant sur chaque ANF construit :
	\begin{enumerate}
		\item Il y a un unique état acceptant
		\item Aucune transition ne mène à l'état initial
		\item Aucune transition ne part de l'état acceptant
	\end{enumerate}




	\paragraph{Cas de base}	Les ANF de la figure \ref{fig:regexadfbase} représentent les automates correspondant aux trois cas de base.

	\begin{figure}[H]

	\begin{subfigure}{.33\textwidth}\centering
		\begin{tikzpicture}[->,>=stealth',shorten >=1pt,auto,node distance=4cm, semithick, bend angle=10,initial text= ]

		\tikzstyle{every state}=[circle]

		\node[initial,state,scale=0.5] (A) {};
		\node[accepting,state,scale=0.5] (B) [right of=A] {};

		\path
		(A) edge  node{$\epsilon$} (B)
		;
		\node[draw, fit=(A) (B)] {};

		\end{tikzpicture}
		\caption{$L=\{\epsilon\}$}
	\end{subfigure}
	\begin{subfigure}{.33\textwidth}\centering
		\begin{tikzpicture}[->,>=stealth',shorten >=1pt,auto,node distance=4cm, semithick, bend angle=10,initial text= ]

		\tikzstyle{every state}=[circle]

		\node[initial,state,scale=0.5] (A) {};
		\node[accepting,state,scale=0.5] (B) [right of=A] {};

		\node[draw, fit=(A) (B)] {};

		\end{tikzpicture}
		\caption{$L=\emptyset$}
	\end{subfigure}
	\begin{subfigure}{.33\textwidth}\centering
		\begin{tikzpicture}[->,>=stealth',shorten >=1pt,auto,node distance=4cm, semithick, bend angle=10,initial text= ]

		\tikzstyle{every state}=[circle]

		\node[initial,state,scale=0.5] (A) {};
		\node[accepting,state,scale=0.5] (B) [right of=A] {};

		\path
		(A) edge  node{$a$} (B)
		;
		\node[draw, fit=(A) (B)] {};

		\end{tikzpicture}
		\caption{$L=\{a\}$}
	\end{subfigure}

	\caption{Blocs de base pour la construction d'un automate à partir d'une expression régulière}
	\label{fig:regexadfbase}
	\end{figure}


	En effet, l'automate (a) correspond à l'expression $\epsilon$ : le seul arc de l'état initial à un état final est $\epsilon$. L'automate (b) ne propose pas d'arc atteignant l'état final. Aucun mot n'appartient au langage d'où la construction de $\emptyset$. Finalement, (c) propose un arc pour $a$, donnant le seul mot $a$ comme faisant partie du langage, faisant de $a$ une expression régulière équivalente. De plus, ces automates respectent bien l'invariant de récurrence proposé.

	\paragraph{Pas de récurrence} Les ANF \emph{abstraits} de la figure \ref{fig:regexadfrec} représentent la façon dont un automate peut être construit récursivement en fonction des règles de récurrence des expressions régulières. Ces ANF sont abstraits car le contenu d'un bloc $R$ ou $S$ n'est pas représenté explicitement. Cependant, celui-ci respecte les invariants de récurrence.

	\begin{figure}[H]

			\hspace{0.2\textwidth}\begin{subfigure}{.6\textwidth}\centering
				\begin{tikzpicture}[->,>=stealth',shorten >=1pt,auto,node distance=4cm, semithick, bend angle=10,initial text= ]

				\tikzstyle{every state}=[circle]

				\node[initial,state,scale=0.5] (A) {};
				\node[state,scale=0.5] (B) [right of=A] {};
				\node[state,scale=0.5] (C) [right of=B] {};
				\node[accepting,state,scale=0.5] (D) [right of=C] {};

				\node[draw=none] (K) [right= 0.5cm of B] {$E$};

				\path
				(A) edge  node{$\epsilon$} (B)
				(C) edge  node{$\epsilon$} (D)

				(C) edge[bend right=90]  node{$\epsilon$} (B)
				(A) edge[bend right=40]  node{$\epsilon$} (D)

				;


				\node[draw, fit=(B) (C)] {};

				\end{tikzpicture}
				\caption{$L=L(E)^*$}
			\end{subfigure}\newline\vspace{1cm}


			\hspace{0.2\textwidth}\begin{subfigure}{.6\textwidth}\centering
				\begin{tikzpicture}[->,>=stealth',shorten >=1pt,auto,node distance=4cm, semithick, bend angle=10,initial text= ]

				\tikzstyle{every state}=[circle]

				\node[initial,state,scale=0.5] (A) {};
				\node[state, scale=0.5] (B) [right of=A] {};
				\node[state, scale=0.5] (C) [right of=B] {};
				\node[accepting,state,scale=0.5] (D) [right of=C] {};


				\node[draw=none] (K) [right= 0.5cm of A] {$E$};
				\node[draw=none] (L) [right= 0.5cm of C] {$F$};

				\path
				(B) edge  node{$\epsilon$} (C)
				;


				\node[draw, fit=(A) (B)] {};
				\node[draw, fit=(C) (D)] {};



				\end{tikzpicture}
				\caption{$L=L(E)L(F)$}
			\end{subfigure}\newline\vspace{1cm}


			\hspace{0.2\textwidth}\begin{subfigure}{.6\textwidth}\centering
			\begin{tikzpicture}[->,>=stealth',shorten >=1pt,auto,node distance=4cm, semithick, bend angle=10,initial text= ]

			\tikzstyle{every state}=[circle]

			\node[initial,state,scale=0.5] (A) {};

			\node[state,scale=0.5] (B) [above right of=A] {};
			\node[state,scale=0.5] (C) [below right of=A] {};

			\node[draw=none] (K) [right= 0.5cm of B] {$E$};
			\node[draw=none] (L) [right= 0.5cm of C] {$F$};

			\node[state,scale=0.5] (D) [right of=B] {};
			\node[state,scale=0.5] (E) [right of=C] {};

			\node[accepting,state,scale=0.5] (F) [below right of=D] {};

			\path
			(A) edge  node{$\epsilon$} (B)
			(A) edge  node{$\epsilon$} (C)

			(D) edge  node{$\epsilon$} (F)
			(E) edge  node{$\epsilon$} (F)
			;
			\node[draw, fit=(B)(D)] {};
			\node[draw, fit=(C)(E)] {};

			\end{tikzpicture}
			\caption{$L=L(E)+L(F)$}
		\end{subfigure}





		\caption{Enchaînement de blocs pour une construction récursive}
		\label{fig:regexadfrec}
	\end{figure}

	Les quatre règles de récurrence sur une expression régulière permettent de construire les automates:
	\begin{itemize}
		\item Pour une expression régulière de forme $(E)$, le langage $L(E)$ étant équivalent à $L((E))$, l'automate construit pour $E$ reste valable.
		\item L'expression régulière est de forme $E^*$. Par induction, il existe un automate exprimant le même langage que $E$. L'automate pour $E^*$ est construit comme en (a). Cet automate comprend un arc $\epsilon$ de l'état initial à l'état acceptant pour représenté le cas $E^0$. Ensuite, un arc $\epsilon$ permet de concaténer plus chemins dans $E$, donnant des mots représentés par $E^1$,$E^2$,$E^3$,... Le tout complétant l'ensemble des mots possibles des $L(E)^*$. On a bien $L(E^*)=L(E)^*$.
		\item L'expression régulière est de forme $EF$. Par induction, il existe des automates représentants les mêmes langages que $E$ et $F$ et respectant notre invariant. L'automate abstrait (b) représente cette concaténation. En effet, un mot de cet automate doit se composer d'un mot $v\in L(E)$ et d'un mot $w \in L(F)$. Les mots possibles sont alors de la forme $v\epsilon w$. Donc (b) représente bien, selon la définition d'un expression régulière $L(EF)=L(E)L(F)$.
		\item L'expression régulière est de forme $E+F$. Alors, comme mis en évidence par l'automate abstrait (c), il existe des automates correspondants aux expression $E$ et $F$. Par cette construction, en particulier les transitions sur $\epsilon$, permettent à $c$ de représenter tout mot de $L(E)$ ou $L(F)$. Le langage est alors, en concordance avec la définition d'une expression régulière $L(E+F)=L(E)\cup L(F)$.
	\end{itemize}

	Les automates (a), (b) et (c) respectent bien l'invariant de récurrence : pas de transition vers l'état initial, un seul état acceptant n'ayant pas de transition sortante. Chaque automate abstrait pour $E$ ou $F$ peut lui même être construit récursivement jusqu'au cas de base.

	\hfill$\square$
\end{proof}
\stepcounter{algo}
\begin{complexity}
	Soit une expression régulière $E$ de longueur $n$ représentant un langage $L$. Si un \emph{arbre syntaxique} est créé pour $E$, il est possible de construire un ANF pour $L$ en $\mathcal{O}(n)$. En effet :
	\begin{itemize}
		\item Cas de base : $n$ ANFs sont créés. Cependant, chacun est constitué de 2 états et au plus 1 transition. Ces nombres sont des constantes. Ce cas de base est effectué en $\mathcal{O}(n*3)=\mathcal{O}(n)$
		\item Récurrence : L'arbre syntaxique requiert au plus $n$ lectures d'opération de récurrence pour fusionner les $n$ ANF en un seul. Cependant, chacune de ses opération implique au plus la création de 2 états et 4 arcs. Ces nombres sont des constantes. La récurrence s'effectue en $\mathcal{O}(n*6)=\mathcal{O}(n)$
	\end{itemize}

	La complexité totale de cette conversion est en $\mathcal{O}(n)$ vers un ANF. La conversion vers un ADF, comme mentionné dans la section \ref{sec:anf} peut quand à elle être exponentielle.

\end{complexity}


%  █████  ██████  ███████       ██     ██        █████  ███    ██ ███████
% ██   ██ ██   ██ ██           ██       ██      ██   ██ ████   ██ ██
% ███████ ██   ██ █████       ██  █████  ██     ███████ ██ ██  ██ █████
% ██   ██ ██   ██ ██           ██       ██      ██   ██ ██  ██ ██ ██
% ██   ██ ██████  ██            ██     ██       ██   ██ ██   ████ ██


\subsection{Équivalence entre un automate déterministe fini et un automate non-déterministe fini}\label{ss:eqadfanf}
Cette section présente une méthode permettant de créer un ADF à partir d'un ANF.

Soit un ANF \automaton. Alors l'ADF équivalent
$$
D=(Q_D, \Sigma, \delta_D, q_D, F_D)
$$
est défini par :
\begin{itemize}
	\item $Q_D = \{S | S \subseteq Q \text{ et } S \text{ est \emph{fermé sur epsilon}}\}$. Concrètement, $Q_D$ est l'ensemble des partie des $Q$ fermées sur $\epsilon$. Cette fermeture s'écrit $S$=ECLOSE(S), ce qui signifie que chaque transition sur $\epsilon$ depuis un état de $S$ mène à un état également dans $S$. L'ensemble $\emptyset$ est fermé sur $\epsilon$.
	\item $q_D$=ECLOSE($q_0$). L'état initial de $D$ est l'ensemble des états dans la fermeture sur $\epsilon$ des états de $A$.
	\item $F_D= \{S|S \in Q_D \text{ et } S \cap F \neq \emptyset\}$ contient les ensembles dont au moins un état est acceptant pour $A$.
	\item $\delta_D(S,a)$ est construit, $\forall a \in \Sigma, \forall S \in Q_D$ par :
		\begin{enumerate}
			\item Soit $S=\{p_1, p_2,\dots,p_k\}$.
			\item Calculer $\bigcup_{i=1}^k\delta(p_i,a)$. Renommer cet ensemble en $\{r_1, r_2, \dots, r_m\}$.
			\item Alors $\delta_D(S,a)=\bigcup_{j=1}^m\text{ECLOSE(}r_j\text{)}$.
		\end{enumerate}
\end{itemize}


\begin{example}[Transformation ANF vers ADF] Considérons l'automate \automaton de l'exemple \ref{ex:anf} et les fermetures calculées dans l'exemple \ref{ex:anfclosure}.

Alors, l'automate $D=(Q_D, \Sigma, \delta_D, q_D, F_D)$ est donné par :

\begin{itemize}
	\item $Q_D=\{\emptyset, \{q_1\}, \{q_2\}, \{q_1,q_2\}, \{q_0,q_1,q_2\}\}$. Les ensembles $\{q_0,q_1\}$ et $\{q_0,q_2\}$ sont des sous-ensembles de $Q$ mais ne sont pas fermé sur $\epsilon$.
	\item $q_D=\{q_0,q_1,q_2\}=\text{ECLOSE(}q_0\text{)}$.
	\item $F_D=\{\{q_2\}, \{q_1,q_2\}, \{q_0,q_1,q_2\}\}$, les ensembles contenant $q_2$, étant acceptant de $A$.
	\item $\delta_D$ est exprimé sur le graphe de la figure \ref{fig:dndf}.
\end{itemize}


\begin{figure}[H]
	\centering
	\begin{tikzpicture}[->,>=stealth',shorten >=1pt,auto,node distance=4cm, semithick, bend angle=10]

	\tikzstyle{every state}=[circle]

	\node[accepting, state]         	(B) 				   {$\{q_0\}$};
	\node[state]		 	(A) [above of=B] {$\emptyset$};
	\node[state] (C) [right of=B] {$\{q_1\}$};
	\node[state]         	(D) [below of=C] {$\{q_2\}$};
	\node[accepting,state]         	(E) [right of=C] {$\{q_0,q_1\}$};
	\node[accepting,state]         	(F) [above of=C] {$\{q_0,q_2\}$};
	\node[state]         	(G) [below right of=E] {$\{q_1,q_2\}$};
	\node[initial,accepting, state]  	(H) [above right of=E] {$\{q_0,q_1,q_2\}$};

	\path
	(A) edge [loop left] node{a,b,c} (A)

	(B) edge  node{a} (A)
	(B) edge [bend left, near start] node{b} (C)
	(B) edge node{c} (D)

	(C) edge [bend left, near start] node{a} (B)
	(C) edge node{b} (D)
	(C) edge node{c} (E)

	(D) edge [near end,above right] node{a,b,c} (A)

	(E) edge [bend left=20,near start] node{a} (B)
	(E) edge [bend left] node{b} (G)
	(E) edge node{c} (H)

	(F) edge node{a,b} (A)
	(F) edge [bend left=20,near end] node{c} (D)

	(G) edge [bend left=80] node{a} (B)
	(G) edge node{b} (D)
	(G) edge [bend left] node{c} (E)

	(H) edge [bend right=20, near start] node{a} (B)
	(H) edge node{b} (G)
	(H) edge [loop above] node{c} (H)
	;
	\end{tikzpicture}
	\caption{Automate $D$. De par la construction par les parties de $Q$, le nombre de parties est exprimé en exponentiel, d'où la complexité du graphe. Ici, $\{q_0,q_2\}$ n'est pas atteignable et peut être supprimé. De même $\emptyset$ est souvent omis pour clarifier la représentation.}\label{fig:dndf}
\end{figure}
\end{example}



\begin{theorem}[ANF $\Leftrightarrow$ ADF]\label{anf-dnf}
	Un langage $L$ peut être représenté par un ANF si et seulement si il peut l'être par un ADF.
\end{theorem}

\begin{proof}
	 Soit $L$ un langage. Cette preuve étant une double implication, chacune peut être prouvée séparément.

	($\Leftarrow$) $L$ peut être représenté par un ADF $\implies$ $L$ peut être représenté par un ANF. Supposons qu'un automate $D=(Q_D, \Sigma, \delta_D, q_D, F_D)$ représente $L$ : $L(D)=L$.
	L'ANF \automaton correspondant est construit comme suit :

	\begin{itemize}
		\item $Q=\{\{q\}|q\in Q_D\}$
		\item $\delta$ contient les transitions de $D$ modifiées. Les objets retournés deviennent des ensembles d'états. C'est-à-dire, si $\delta_D(q,a)=p$ alors $\delta(q,a)=\{p\}$. De plus, pour chaque état $q\in Q_D$, $\delta(q,\epsilon)=\emptyset$.
		\item $q_0=\{q_D\}$
		\item $F=\{\{q\}| q\in F_D\}$
	\end{itemize}

	 Dès lors, les transitions sont les mêmes entre $D$ et $A$, mais $A$ précise explicitement qu'il n'y a pas de transition sur $\epsilon$. Comme $A$ représente le même langage, un ANF représente $L$.


	($\Rightarrow$) $L$ peut être représenté par un ANF $\implies$ $L$ peut être représenté par un ADF. Soit l'automate \automaton. Supposons qu'il représente $L$ ($L=L(A)$). Considérons l'automate obtenu par la transformation détaillée à la section précédente \ref{ss:eanfadf} :
	$$
	D=(Q_D, \Sigma, \delta_D, q_D, F_D)
	$$
	Montrons que $L(D)=L(A)$. Pour ce faire, montrons que les fonctions de transition étendues sont équivalentes. Auquel cas, les chemins sont équivalents et donc les langages également.
	Montrons que $\hdelta(q_0,w)=\hdelta_D(q_D,w)$ pour tout mot $w$, par récurrence sur $w$.

	\paragraph{Cas de base} Si $|w|=0$, $w=\epsilon$. $\hdelta(q_0,\epsilon)=$ ECLOSE($q$), par définition de la fonction de transition étendue. $q_D$=ECLOSE($q_0$) par la construction de $q_D$. Finalement, pour un ADF (ici, $D$), $\hdelta(p,\epsilon)=p$, pour tout état $p$. Par conséquent, $\hdelta_D(q_D,\epsilon)=q_D=\text{ECLOSE(}q_0\text{)}=\hdelta(q_0,\epsilon)$.

	\paragraph{Pas de récurrence} Supposons $w=xa$ avec $a$ le dernier symbole de $w$. Notre hypothèse de récurrence est que $\hdelta_D(q_D,x)=\hdelta(q_0,x)$. Ce sont bien les mêmes objets car $\hdelta_D$ retourne un état de $D$ qui correspond à un ensemble d'états de $A$. Notons celui-ci $\{p_1,p_2, \dots, p_k\}$. Par définition de $\hdelta$ pour un ANF, $\hdelta(q_0,w)$ est obtenu en :

	\begin{enumerate}
		\item Soit $\{r_1,r_2,\dots, r_m\}$ donné par $\bigcup_{i=1}^k \delta(p_i,a)$, les états obtenus par la lecture du symbole $a$ à partir de $\{p_1,p_2,\dots,p_k\}$.
		\item Alors $\hdelta(q_0,w)=\bigcup_{j=1}^m$ECLOSE($r_j$). Un état atteint par la lecture de $a$ l'est aussi par $a\epsilon$.
	\end{enumerate}

	$D$ a été construit avec ces deux mêmes étapes pour $\delta_D(\{p_1,p_2,\dots, p_k\},a)$. Dès lors, $\hdelta_D(q_D,w)=\delta_D(\{p_1,p_2,\dots, p_k\},a)=\bigcup_{j=1}^k$ECLOSE($p_j$)$=\hdelta(q_0,w)$.

	On a bien $\hdelta_D(q_D,w)=\hdelta(q_0,w)$. Les langages sont équivalents.

	\hfill$\square$
\end{proof}

\stepcounter{algo}%TODO because we don't exactly have an algorithm here

\begin{complexity}[Conversion ANF vers ADF]

	La complexité d'une conversion ANF vers ADF peut être exprimée en fonction de $n$ le nombre d'états de l'ANF. La taille de l'alphabet $\Sigma$ est ici comptée comme une constante, elle est ignorée dans l'analyse grand-O. L'algorithme de conversion se fait en deux étapes. Le calcul de ECLOSE et la construction à proprement parler.

	\begin{itemize}
		\item ECLOSE : Pour chacun des $n$ états, il y a au plus $n^2$ transitions à suivre sur $\epsilon$ pour construire la fermeture. Ceci représente le cas où tous les états sont reliés avec tous les autres par des transitions sur $\epsilon$. Le coût de cet algorithme est dès lors de $n*\mathcal{O}(n^2)=\mathcal{O}(n^3)$.

		\item Construction : Posons $s$ le nombre d'états dans l'ADF (qui, dans le pire des cas vaut $s=2^n$ par la construction des sous-ensembles). La création d'un état est en temps $\mathcal{O}(n)$, correspondant au plus à $n$ états de l'ANF. Pour ce qui est des transitions, pour chacun des $s$ nouveaux états, ECLOSE contient au plus $n$ éléments. Chacune des $n^2$ transitions de l'ADF sont alors suivies pour chaque symbole $a\in\Sigma$. Le coût de construction d'une transition est alors de $n*\mathcal{O}(n^2)=\mathcal{O}(n^3)$ auquel vient s'ajouter $\mathcal{O}(n^2)$, négligeable, pour l'union de l'ensemble obtenu.
	\end{itemize}

	La complexité totale est $\mathcal{O}(n^3) + s * \mathcal{O}(n^3) = \mathcal{O}(s*n^3) = \mathcal{O}(2^n*n^3)$.
	Le détail est donné sur $s$ car, comme mentionné par Hopcroft et Al. \cite{Hopcroft00}, en pratique le nombre de l'état dans l'ADF obtenu est rarement de l'ordre de $2^n$, typiquement de l'ordre de $n$.

\end{complexity}


\begin{complexity}[Conversion ADF vers ANF]
	La conversion d'un ADF \automaton vers un ANF consiste au remplacement d'états par des ensembles d'états. Si l'ADF contient $n$ états, cette étape est en $\mathcal{O}(n)$. De plus, une colonne pour $\epsilon$ doit être ajoutée à la table de transition (pour la fonction $\delta$), et ce pour chacun des états. Cette étape se fait également en $\mathcal{O}(n)$.

	La complexité totale d'une conversion d'un ADF vers un ANF est en $\mathcal{O}(n)$.
\end{complexity}

	\section{Table Filling Algorithm}\label{sec:tfa}
% ██████  ███████
% ██   ██ ██
% ██████  █████
% ██   ██ ██
% ██   ██ ███████



\subsection{Relation \re}\label{ss:re}

Soit un automate \automaton. Définissons la relation \re entre deux états :
$$xR_Ey \iff (\forall w \in \Sigma^*,\hdelta(x,w) \in F \iff \hdelta(y,w) \in F)$$

Intuitivement, ces deux états sont en relation si tout mot lu à partir de celui-ci mène à des états étant simultanément acceptants ou non.

\begin{proposition}[\re]
 \re est une relation d'équivalence.
\end{proposition}

\begin{proof}[\re est une relation d'équivalence] Montrer que \re est une relation d'équivalence revient à montrer qu'elle est réflexive, transitive et symétrique.
 \begin{itemize}
	 \item \textbf{Réflexive :} Soient un état $x \in Q_M$ et $w \in \Sigma^*$. Alors, $\hat{\delta}(x,w) \in F \iff \hat{\delta}(x,w) \in F$ et par définition, $xR_Ex$.
	 \item \textbf{Transitive :} Soient les états $x,y,z \in Q_M$ tels que $xR_Ey$ et $yR_Ez$ ainsi que $w \in \Sigma^*$. Par hypothèse, $\hat{\delta}(x,w) \in F \iff \hat{\delta}(y,w)\in F$ et $\hat{\delta}(y,w) \in F\iff \hat{\delta}(z,w) \in F$. Par transitivité de l'implication, on obtient $\hat{\delta}(x,w) \in F \iff \hat{\delta}(z,w)\in F$. On a donc $xR_Ez$.
	 \item \textbf{Symétrique : } Soient les états $x,y \in Q_M$ tels que $xR_Ey$ et un mot $w \in \Sigma^*$. Par hypothèse, $\hat{\delta}(x, w)\in F \iff \hat{\delta}(y, w)\in F$. En lisant la double implication depuis la droite, on a bien $\hat{\delta}(y, w) \in F\iff \hat{\delta}(x, w)\in F$ et donc $yR_Ex$.
 \end{itemize}
 \hfill$\square$
\end{proof}

\begin{corollary}
 \re sépare les états de $Q$ en classes d'équivalence.
\end{corollary}

La classe d'équivalence de tous les états en relation \re avec $q$ (qui sert alors de \emph{représentant}) se note $[[q]]$ ou par une lettre majuscule, typiquement $S$ ou $T$.

La \emph{congruence à droite} d'une relation $R$ entre des mots sur un alphabet $\Sigma$ est définie comme :
$$
\forall x,y \in \Sigma^*, xRy \Rightarrow \forall a \in \Sigma, xaRya
$$

\begin{proposition}[Congruence de \re]
 \re est congruente à droite.
\end{proposition}

\begin{proof}[Congruence de \re]\label{proof:rmcongruency}
 Si la relation est vraie pour deux état, elle reste valable pour les états atteints par la lecture d'un symbole quelconque. Soient les états $x,y \in Q_M$ tels que $xR_Ey$. Soit un symbole $a \in \Sigma$. Par hypothèse,
 $$\forall w \in \Sigma^*, \hat{\delta}(x, w) \in F \iff \hat{\delta}(y, w) \in F$$
 C'est donc vrai en particulier pour $w = au, u \in \Sigma*$. Dès lors,
 $$\hat{\delta}(x, au) \in F\iff \hat{\delta}(y, au)\in F$$
 $$\hat{\delta}(\delta(x,a),u) \in F\iff\hat{\delta}(\delta(y,a),u)\in F$$
 $$\hat{\delta}(p,u) \in F\iff \hat{\delta}(q,u)\in F$$

\hfill$\square$
\end{proof}

\begin{corollary}\label{col:st}
 Pour chaque symbole, toutes les transitions sortant d'une classe d'équivalence mènent à une même classe d'équivalence :
 $\forall a \in \Sigma, \exists T, \forall q \in S, \delta(q,a)\in T$ avec $T$ une classe d'équivalence.
\end{corollary}


% ████████ ███████  █████
%    ██    ██      ██   ██
%    ██    █████   ███████
%    ██    ██      ██   ██
%    ██    ██      ██   ██

\subsection{Table Filling Algorithm}
Certains états d'un automate peuvent être \emph{équivalents} selon la relation \re. Celui-ci peut alors être simplifié. Une façon de détecter ces équivalences est de construire un tableau via l'\emph{algorithme de remplissage de tableau}.

Celui-ci détecte les paires \emph{différenciables}, récursivement sur un automate \automaton. Un paire $\{p,q\}$ est différenciable s'il existe un mot $w$ tel qu'un chemin $\hdelta(p,w)$ mène à un état acceptant et $\hdelta(q,w)$ mène à un état non-acceptant ou vice-versa. $w$ sert alors de \emph{mot témoin}.

\textbf{Cas de base :} Si $p$ est un état acceptant et que $q$ ne l'est pas, alors la paire $\{p,q\}$ est différenciable. Le mot témoin est $\epsilon$.

\textbf{Pas de récurrence : } Soient $p,q$ des états de $Q$ et un symbole $a \in \Sigma$ tel que $\delta(p,a)=r$ et $\delta(q,a)=s$. Si $r$ et $s$ sont différenciables, alors $p$ et $q$ le sont aussi. En effet, il existe un mot \emph{témoin} $w$ qui permet de différencier $r$ et $s$. Alors le mot $aw$ est le mot témoin qui permet de différencier $p$ et $q$.

\begin{theorem}[Table d'équivalence]
 Si deux états ne sont pas distingués par l'algorithme de remplissage de tableau, les états sont équivalents (ils respectent la relation \re).
\end{theorem}

\begin{proof}

Considérons un automate déterministe fini quelconque \automaton. Supposons par l'absurde qu'il existe une paire d'états $\{p,q\}$ tels que :
\begin{enumerate}
	 \item $p$ et $q$ ne sont pas distingués par l'algorithme de remplissage de table.
	 \item Les états ne sont pas équivalents, $\not pR_E q$. Par extension, il existe un mot témoin $w$ différenciant $p$ et $q$.
\end{enumerate}

Une telle paire est une \emph{mauvaise paire}. Si il y a des mauvaises paires, chacune distinguée par un mot témoin, il doit exister un paire distinguée par le mot témoin le plus court. Posons $\{p,q\}$ comme étant cette paire et $w=a_1a_2\dots a_n$ le mot témoin le plus court qui les distingue. Dès lors, soit $\hdelta(p,w)$ est acceptant, soit $\hdelta(q,w)$ l'est, mais pas les deux.

Ce mot $w$ ne peut pas être $\epsilon$. Auquel cas, la table aurait été remplie dès l'étape d'induction de l'algorithme. La paire $\{p,q\}$ ne serait pas une mauvaise paire, ne respectant pas l'hypothèse 1.

$w$ n'étant pas $\epsilon$, $ |w| \ge 1$. Considérons les états $r = \delta(p,a_1)$ et $s=\delta(q,a_1)$. Ces états sont différenciés par $a_2a_3\dots a_n$ car $\hdelta(p,w) = \hdelta(r, a_2a_3\dots a_n)$ et $\hdelta(q,w) = \hdelta(s, a_2a_3\dots a_n)$ et $p$ et $q$ sont différenciables.

Cela signifie qu'il existe un mot plus petit que $w$ qui différencie deux états: le mot $a_2a_3\dots a_n$. Comme on a supposé que $w$ est le mot le plus petit qui différencie une mauvaise paire, $r$ et $s$ ne peuvent pas être une mauvaise paire. Donc, l'algorithme a du découvrir qu'ils sont différenciables.

Cependant, le pas de récurrence impose que $\delta(p, a_1)$ et $\delta(q, a_1)$ mènent à deux états différentiables implique que $p$ et $q$ le sont aussi. On a une contradiction de notre hypothèse : $\{p,q\}$ n'est pas une mauvaise paire.

Ainsi, s'il n'existe pas de mauvaise paire, c'est que chaque paire différenciable est reconnue par l'algorithme.

\hfill$\square$
\end{proof}

\begin{example}[Table d'équivalence] Voici une application de cet algorithme sur l'automate $A_2$, version réduite de l'automate $A_1$ de la figure \ref{fig:a1}.

\begin{figure}[H]
 \centering
 \begin{tikzpicture}[->,>=stealth',shorten >=1pt,auto,node distance=3cm, semithick, bend angle=10]

 \tikzstyle{every state}=[circle]

 \node[initial,state] (A)                    {$q_0$};
 \node[state]         (B) [below right of=A] {$q_1$};
 \node[state]         (C) [below left of=A] {$q_2$};
 \node[accepting,state]         (D) [below right of=B] {$q_3$};
 \node[state]         (E) [below left of=C]       {$q_4$};
 \node[state]         (F) [below right of=C]       {$q_5$};

 \path 	(A) 	edge              node {a} (C)
 edge              node {b} (B)
 (B) 	edge              node {a} (D)
 edge [bend left]  node {b} (F)
 (C) 	edge              node {a} (E)
 edge              node {b} (F)
 (D) 	edge [loop above] node {a,b} (D)
 (E) 	edge [loop above] node {a,b} (E)
 (F) 	edge              node {a} (D)
 edge [bend left]  node {b} (B);
 \end{tikzpicture}
 \caption{Automate $A_2$}\label{fig:a2}
\end{figure}

La première étape est de remplir la table avec l'algorithme précédant. Tout état est distinguable de $q_3$ : il est le seul état acceptant. 5 cases peuvent déjà êtres cochées. Le reste de la table est remplie par induction.

\begin{figure}[H]
 \centering
 \begin{tabular}{ccccccc}
	 \cline{2-2}
	 \multicolumn{1}{c|}{$q_1$} & \multicolumn{1}{c|}{x} &&&&\\
	 \cline{2-3}
	 \multicolumn{1}{c|}{$q_2$} & \multicolumn{1}{c|}{x} &\multicolumn{1}{c|}{x}&&&\\
	 \cline{2-4}
	 \multicolumn{1}{c|}{$q_3$} & \multicolumn{1}{c|}{x} &\multicolumn{1}{c|}{x}&\multicolumn{1}{c|}{x}&&\\
	 \cline{2-5}
	 \multicolumn{1}{c|}{$q_4$} & \multicolumn{1}{c|}{x} &\multicolumn{1}{c|}{x}&\multicolumn{1}{c|}{x}&\multicolumn{1}{c|}{x}&\\
	 \cline{2-6}
	 \multicolumn{1}{c|}{$q_5$} & \multicolumn{1}{c|}{x} & \multicolumn{1}{c|}{}&\multicolumn{1}{c|}{x}&\multicolumn{1}{c|}{x}&\multicolumn{1}{c|}{x}\\
	 \cline{2-6}
	 \multicolumn{1}{c}{} & $q_0$&$q_1$&$q_2$&$q_3$&$q_4$\\

 \end{tabular}
 \caption{Table filling pour $A_2$, décelant des équivalences d'états}
 \label{fig:ta2}
\end{figure}
\end{example}
\stepcounter{algo}
\begin{complexity}

Considérons $n$ le nombre d'états d'un automate, et $k$ la taille de l'alphabet $\Sigma$ supporté.

Si il y a $n$ états, il y a $\begin{pmatrix}n\\2\end{pmatrix}$ soit $\frac{n(n-1)}{2}$ paires d'états. A chaque itération (sur l'ensemble de la table), il faut considérer chaque paire, et vérifier si un de leur successeurs est différentiable. Cette étape prend au plus $\mathcal{O}(k)$ pour tester chaque successeurs potentiel (en fonction du symbole lu).  Ainsi, une itération sur la table se fait en $\mathcal{O}(kn^2)$. Si une itération ne découvre pas de nouveaux état différentiable s'arrête. Comme la table a une taille en $\mathcal{O}(n^2)$ et qu'à chaque étape un élément au minimum doit y être coché, la complexité totale de l'algorithme est en $\mathcal{O}(kn^4)$.

Cependant, il existe des pistes d'amélioration. La première est d'avoir, pour chaque paire $\{r,s\}$ une liste des paire $\{p,q\}$ qui, pour un même symbole, mènent à $\{r,s\}$. On dit de ces paires qu'elles sont dépendantes. Si la paire $\{r,s\}$ est marquée comme différenciable, leurs paires dépendantes seront de facto différenciables.

Cette liste peut être construite en considérant chaque symbole $a \in \Sigma$ et ajoutant les paires $\{p,q\}$ à chacune de leur dépendance $\{\delta(p,a),\delta(q,a)\}$. Cette étape prend au plus $k.\mathcal{O}(n^2)=\mathcal{O}(kn^2)$. (Le nombre de symboles multiplié par le nombre de paires à considérer).

Ensuite, il suffit de partir des cas initiaux (se reposant sur le cas de base de l'algorithme), et de marquer tous leurs états dépendants comme différentiables, tout en ajoutant leur propre liste à chaque fois. La complexité de cette exploration est bornée par le nombre d'éléments dans une liste et le nombre de listes. Respectivement, $k$ et $\mathcal{O}(n^2)$, ce qui donne $\mathcal{O}(kn^2)$ pour cette exploration.

La complexité totale revient à $\mathcal{O}(kn^2)$.
\end{complexity}


% ███    ███ ██ ███    ██ ██ ███    ███
% ████  ████ ██ ████   ██ ██ ████  ████
% ██ ████ ██ ██ ██ ██  ██ ██ ██ ████ ██
% ██  ██  ██ ██ ██  ██ ██ ██ ██  ██  ██
% ██      ██ ██ ██   ████ ██ ██      ██

\subsection{Minimisation}
La minimisation d'automate se fait en deux étapes :
\begin{enumerate}
 \item Se débarrasser de tous les états injoignables : ils ne participent pas à la construction du langage représenté
 \item Grâce aux équivalences d'états trouvées grâce à l'algorithme de remplissage de tableau défini au point \ref{ss:tfa}, construire un nouvel automate.
\end{enumerate}

Soit un automate déterministe fini \automaton. Les états non-atteignables peuvent être supprimés de $Q$ et de $\delta$.

Pour minimiser cet automate, il faut :
\begin{enumerate}
 \item Générer la table de différenciation.
 \item Séparer $Q$ en classes d'équivalences
 \item Construire l'automate canonique $C=(Q_C,\Sigma, \delta_C, q_C, F_C)$:
 \begin{itemize}
	 \item Soit $S$ une des classes d'équivalence obtenues par la table de différenciation.
	 \item Ajouter $S$ à $Q_C$ et à $F_C$ si $S$ contient un état acceptant : $q\in S, q\in F$.
	 \item Si $S$ contient $q_0$ l'état initial de $A$, alors $S$ est $q_C$ l'état initial de $C$.
	 \item Pour un symbole $a \in \Sigma$, alors il doit exister une classe d'équivalence $T$ tel que pour chaque état $\forall q \in S,\delta(q,a) \in T$. Si ce n'est pas le cas, c'est que deux états $p$ et $q$ dans $S$ mènent à différentes classes d'équivalences. Or, ces deux états sont différenciables, et ne pourraient pas appartenir tous deux à $S$ par construction. Ce fait est déjà mentionné dans le corollaire \ref{col:st}. On peut écrire $\delta_C(S,a)=T$. Pour rappel, la fonction $\delta$ est définie pour tout état et tout symbole. Rien n'empêche $T=S$.
 \end{itemize}
\end{enumerate}


\begin{example}

 Considérons l'automate $A_1$ représenté à la figure \ref{fig:a1}. En supprimant l'état $q_6$ qui n'est pas atteignable, on obtient l'automate $A_2$ de la figure \ref{fig:a2}.

 Le tableau de la figure \ref{fig:ta2} sert d'exemple pour l'algorithme de remplissage de tableau, sur $A_2$.
 $A_3$.

 En appliquant l'algorithme, qui peut se résumer intuitivement à fusionner les états équivalents, on obtient l'automate $A_3$ de la figure \ref{fig:a3}.

 \begin{figure}[H]
	 \centering
	 \begin{tikzpicture}[->,>=stealth',shorten >=1pt,auto,node distance=3cm, semithick, bend angle=10]

	 \tikzstyle{every state}=[circle]

	 \node[initial,state] (A)                    {$q_0$};
	 \node[state]         (B) [below right of=A] {$q_1$};
	 \node[state]         (C) [below left of=A] {$q_2$};
	 \node[accepting, state]         (D) [below right of=B] {$q_3$};
	 \node[state]         (E) [below left of=C]       {$q_4$};

	 \path
	 (A) 	edge              node {a} (C)
	 edge              node {b} (B)
	 (B) 	edge              node {a} (D)
	 edge [loop above] node {b} (B)
	 (C) 	edge              node {a} (E)
	 edge              node {b} (B)
	 (D) 	edge [loop above] node {a,b} (D)
	 (E) 	edge [loop above] node {a,b} (E);
	 \end{tikzpicture}
	 \caption{Automate $A_3$}\label{fig:a3}
 \end{figure}

 Une expression régulière ($(b+ab)b^*a(a+b)^*$) peut être déduite pour $L$ grâce à cet automate. Cette expression régulière est celle de l'exemple \ref{ex:regex}
\end{example}


\begin{theorem}[Minimalité de l'automate réduit]
 Soit un ADF $A$ et soit $C$ l'automate construit par cet algorithme de minimisation. Aucun automate équivalent à $A$ n'a moins d'états que $C$. De plus, chaque automate ayant autant d'états que $C$ peut être transformé en celui-ci par homomorphisme.
\end{theorem}


\begin{proof}
 Prouvons que l'algorithme de minimisation fourni un automate minimum (il n'en existe aucun comportant moins d'états pour un même langage)
 Soient un ADF $A$ et $C$ l'automate obtenu par l'algorithme de minimisation. Posons que $C$ comporte $k$ états.

 Par l'absurde, supposons qu'il existe $M$ un ADF minimisé équivalent à $A$ mais comptant moins d'états que $C$. Posons qu'il en comporte $l<k$.
 Appliquons l'algorithme de remplissage de table sur $C$ et $M$, comme s'ils étaient un seul ADF, comme proposé dans la section \ref{ss:eqauto}. Les états initiaux sont équivalents (pas différentiables) puisque $L(C)=L(M)$. Dès lors, les successeurs pour chaque symboles sont eux aussi équivalent. Le cas contraire impliquerait que états initiaux sont différentiables, ce qui n'est pas le cas.
 De plus, ni $C$ ni $M$ n'ont un état inaccessible, sinon il pourrait être éliminé, résultant en un automate comportant moins d'états pour un même langage.
 Soit $p$ un état de $C$. Soit un mot $a_1a_2\dots a_i$, qui mène de l'état initial de $C$ à $p$. Alors, il existe un état $q$ de $M$ équivalent à $p$. Puisque les états initiaux sont équivalents, et que par induction, les états obtenus par la lecture d'un symbole le sont aussi, l'état $q$ dans $M$ obtenu par la lecture du mot $a_1a_2\dots a_i$ est équivalent à $p$. Ceci signifie que tout état de $C$ est équivalent à au moins un état de $M$.
 Or, $lk>l$. Cela signifie qu'il doit exister au moins deux états de $C$ équivalents à un même état de $M$ et donc équivalent entre eux. Il y a la contradiction : par construction, les états de $C$ sont tous différentiables les uns des autres. La supposition de l'existence de $M$ est fausse. Il n'existe pas d'automate équivalent à $A$ comportant moins d'états que $C$.

 \hfill$\square$
\end{proof}

\begin{proof}
 Prouvons que tout automate minimal pour un langage est $C$, à un isomorphisme sur les noms des états près.

 Soit $A$ un ADF pour un langage $L$. Soient $C$ un ADF obtenu par l'algorithme de minimisation et $M$ un automate minimal comportant autant d'états que $C$.

 Comme mentionné dans la preuve précédente, il doit y avoir une équivalence 1 à 1 entre chaque état de $C$ et de $M$. (Au minimum 1 et au plus 1). De plus, aucun état de $M$ ne peut être équivalent à 2 états de $C$, selon le même argument.

 Dès lors, l'automate minimisé, dit \emph{canonique} est unique à l'exception du renommage des différents états.

 \hfill$\square$
\end{proof}


% ███████  ██████  ██    ██ ██ ██    ██
% ██      ██    ██ ██    ██ ██ ██    ██
% █████   ██    ██ ██    ██ ██ ██    ██
% ██      ██ ▄▄ ██ ██    ██ ██  ██  ██
% ███████  ██████   ██████  ██   ████
%             ▀▀


\subsection{Appartenance et équivalence}
Considérons les automates $A_H$ et $A_I$ donnés dans les figures \ref{fig:ah} et \ref{fig:ai}

\begin{minipage}{0.4\linewidth}
 \begin{figure}[H]
	 \centering
	 \begin{tikzpicture}[->,>=stealth',shorten >=1pt,auto,node distance=2cm and 5cm, semithick, bend angle=10]

	 \tikzstyle{every state}=[circle]

	 \node[initial,state]	(A)					{$q_0$};
	 \node[state]			(B)	[right= of A]	{$q_1$};
	 \node[accepting,state]	(C) [below of=A]	{$q_2$};
	 \node[accepting,state]	(D)	[below of=B]	{$q_3$};
	 \node[accepting,state]	(E)	[below of=C]	{$q_4$};
	 \node[state]			(F)	[below of=D]	{$q_5$};

	 \path
	 (A)	edge	[bend left]		node{a}		(B)
	 (A)	edge					node{b}		(C)
	 (B) edge	[bend left]		node{a}		(A)
	 (B) edge					node{b}		(D)
	 (C)	edge					node{a}		(E)
	 (C)	edge					node[near start]{b}		(F)
	 (D)	edge					node[near start, above]{a}		(E)
	 (D)	edge					node{b}		(F)
	 (E)	edge	[loop below]	node{a}	(E)
	 (E) edge					node{b} (F)
	 (F)	edge	[loop below]	node{a,b}	(F)

	 ;
	 \end{tikzpicture}
	 \caption{Automate $A_H$, du livre d'Hopcraft et al. de 1979\cite{Hopcroft79} (Fig3.2)}\label{fig:ah}
 \end{figure}
\end{minipage}\hspace{0.2\linewidth}
\begin{minipage}{0.4\linewidth}
 \begin{figure}[H]
	 \centering
	 \begin{tikzpicture}[->,>=stealth',shorten >=1pt,auto,node distance=1cm and 1cm, semithick, bend angle=10]

	 \tikzstyle{every state}=[circle]

	 \node[initial,state]	(A)					{$q_6$};
	 \node[accepting,state]	(B)	[right= of A]	{$q_7$};
	 \node[state]			(C) [right= of B]	{$q_8$};

	 \path
	 (A)	edge					node{b}		(B)
	 (A)	edge	[loop above]	node{a}		(A)
	 (B) edge					node{b}		(C)
	 (B) edge	[loop above]	node{a}		(B)
	 (C)	edge	[loop above]	node{a,b}	(C)

	 ;
	 \end{tikzpicture}
	 \caption{Automate $A_I$, provenant également de \cite{Hopcroft79}. Les états ont été renommés. }\label{fig:ai}
 \end{figure}
\end{minipage}

Il est possible de remplir un tableau via l'algorithme éponyme. Pour ce faire, les deux automates sont considérés comme un seul dont les états sont disjoints.

\begin{figure}[H]
 \centering
 \begin{tabular}{ccccccccc}
	 \cline{2-2}
	 \multicolumn{1}{c|}{$q_1$}&\multicolumn{1}{c|}{} &&&&&&&\\
	 \cline{2-3}
	 \multicolumn{1}{c|}{$q_2$}&\multicolumn{1}{c|}{x} &\multicolumn{1}{c|}{x}&&&&&&\\
	 \cline{2-4}
	 \multicolumn{1}{c|}{$q_3$}&\multicolumn{1}{c|}{x}&\multicolumn{1}{c|}{x}&\multicolumn{1}{c|}{}&&&&&\\
	 \cline{2-5}
	 \multicolumn{1}{c|}{$q_4$}&\multicolumn{1}{c|}{x}&\multicolumn{1}{c|}{x}&\multicolumn{1}{c|}{}&\multicolumn{1}{c|}{}&&&&\\
	 \cline{2-6}
	 \multicolumn{1}{c|}{$q_5$}&\multicolumn{1}{c|}{x}&\multicolumn{1}{c|}{x}&\multicolumn{1}{c|}{x}&\multicolumn{1}{c|}{x}&\multicolumn{1}{c|}{x}&&&\\
	 \cline{2-7}
	 \multicolumn{1}{c|}{$q_6$}&\multicolumn{1}{c|}{}&\multicolumn{1}{c|}{}&\multicolumn{1}{c|}{x}&\multicolumn{1}{c|}{x}&\multicolumn{1}{c|}{x}&\multicolumn{1}{c|}{x}&&\\
	 \cline{2-8}
	 \multicolumn{1}{c|}{$q_7$}&\multicolumn{1}{c|}{x}&\multicolumn{1}{c|}{x}&\multicolumn{1}{c|}{}&\multicolumn{1}{c|}{}&\multicolumn{1}{c|}{}&\multicolumn{1}{c|}{x}&\multicolumn{1}{c|}{x}&\\
	 \cline{2-9}
	 \multicolumn{1}{c|}{$q_8$}&\multicolumn{1}{c|}{x}&\multicolumn{1}{c|}{x}&\multicolumn{1}{c|}{x}&\multicolumn{1}{c|}{x}&\multicolumn{1}{c|}{x}&\multicolumn{1}{c|}{}&\multicolumn{1}{c|}{x}&\multicolumn{1}{c|}{x}\\
	 \cline{2-9}
	 \multicolumn{1}{c}{} & $q_0$& $q_1$ & $q_2$ & $q_3$ & $q_4$ & $q_5$ & $q_6$ & $q_7$\\

 \end{tabular}
 \caption{Tableau généré par l'application de l'algorithme sur $A_H$ et $A_I$}\label{fig:tahi}
\end{figure}

De cette table, toujours grâce aux conclusions précédentes, il est possible d'extraire des classes d'équivalences :
\begin{itemize}
 \item $C_0 = \{q_0, q_1, q_6\}$
 \item $C_1 = \{q_2, q_3, q_4, q_7\}$
 \item $C_2 = \{q_5, q_8\}$
\end{itemize}

En particulier, la classe $C_0$ souligne que les états initiaux sont équivalents. Cela signifie, par définition, que tout mot $w$ lu en partant d'un de ces états sera soit accepté dans les deux automates, soit refusé dans les deux. $A_H$ et $A_I$ définissent donc le même langage.
\stepcounter{algo}
\begin{complexity}
 Reposant sur la construction de la table d'équivalence d'états, la complexité est en $\mathcal{O}(kn^2)$, avec $k$ la taille de l'alphabet et $n$ le nombre d'états. L'étape supplémentaire, la lecture de cette table, est en temps constant et n'impacte pas la complexité.
\end{complexity}


Les différentes notions liées à l'égalité : les propriétés de réflexivité, transitivité et symétrie ont été démontrées dans la section \ref{ss:rm}.

	\section{Algorithme d'Angluin}\label{sec:angluin}
L'algorithme d'Angluin repose, en plus des éléments précédents sur quatre concepts :

\begin{itemize}
	\item Une table d'observation
	\item La relation $R_O$, se basant sur la table d'observation et semblable à la relation $R_L$
	\item La propriété de fermeture (closure en anglais)
	\item La propriété de cohérence (consistence en anglais)
\end{itemize}

Cette section commence par décrire cette table en \ref{ss:a_tblo}, la relation $R_O$ en \ref{ss:a_ro}, la fermeture en \ref{ss:a_fermeture}, la cohérence en \ref{ss:a_coherence}.

Une fois toutes ces bases posées, une exécution de l'algorithme sur un exemple est proposée en \ref{ss:a_exemple}, suivie du fonctionnement formel de l'algorithme et des preuves sur son exactitude et sa complexité en \ref{ss:a_algo}, \ref{ss:a_proof} et \ref{ss:a_comp}.


\subsection{Table d'observation}\label{ss:a_tblo}

\subsection{Relation $R_O$}\label{ss:a_ro}

\subsection{Fermeture}\label{ss:a_fermeture}

La propriété de fermeture (closure) s'exprime mathématiquement par 

$$ \forall u \in R, \forall a \in \Sigma, \exists v \in R, ua R_O v$$

En pratique, pour vérifier cette propriété, il suffit de de suivre cet algorithme, expliqué de façon visuelle sur la table O :

\begin{algorithm}[H]
	\begin{algorithmic}[1]
		\ENSURE si la fermeture est respectée ou non
		
		\FORALL {élément $w$ de la section $R$}
		\FORALL {symbole $a$ dans $\Sigma$}		
			\IF {$wa$ est dans $R$} 
				\STATE continuer
			\ELSE
				\STATE \COMMENT{$wa$ est dans $R.\Sigma$ par construction}
				\IF {La ligne de $wa$ dans $T$ est différente de celle de $w$}
					\RETURN Faux
				\ENDIF
			\ENDIF
		\ENDFOR
		\ENDFOR
		\RETURN Vrai
	\end{algorithmic}
	\caption{Vérification de la fermeture}\label{alg:closure}
\end{algorithm}

\subsection{Cohérence}\label{ss:a_coherence}

La propriété de cohérence (consistence) se définit mathématiquement comme 

$$ \forall u,v \in R, u R_O v \Rightarrow \forall a \in \Sigma, ua R_O va$$

Concrètement, il s'agit de prendre deux mots ($u,v$) dans $R$ ayant la même ligne dans $T$ et vérifier, pour chaque symbole ($a$), s'ils ($ua,va$) ont la même ligne dans $T$.



\subsection{Exemple}\label{ss:a_exemple}
Soit l'automate $A_3$ construit à la section \ref{ss:miniauto} sur la minimisation. L'automate $A_4$ recopié ici n'est qu'une isomorphie : les symboles $\{0,1\}$ ont été remplacés par $\{a,b\}$ pour plus de lisibilité dans les tables d'observation.
\begin{figure}[H]
	\centering
	\begin{tikzpicture}[->,>=stealth',shorten >=1pt,auto,node distance=3cm, semithick, bend angle=10]
	
	\tikzstyle{every state}=[circle]
	
	\node[initial,state] (A)                    {$q_a$};
	\node[state]         (B) [below right of=A] {$q_b$};
	\node[state]         (C) [below left of=A] {$q_c$};
	\node[accepting, state]         (D) [below right of=B] {$q_d$};
	\node[state]         (E) [below left of=C]       {$q_e$};
	
	\path 	
	(A) 	edge              node {a} (C)
	edge              node {b} (B)
	(B) 	edge              node {a} (D)
	edge [loop above] node {b} (B)
	(C) 	edge              node {a} (E)
	edge              node {b} (B)
	(D) 	edge [loop above] node {a,b} (D)
	(E) 	edge [loop above] node {a,b} (E);
	\end{tikzpicture}
	\caption{Automate $A_4$}
\end{figure}

\todo{Marquer la différence entre $R_L$ et $R_O$}

\subsubsection{Première itération}

L'algorithme d'Angluin précise, pour son cas de base, une initialisation de la table $T$ avec les ensembles $R$ et $S$ contenant uniquement $\epsilon$. Le champ $R.\{a,b\}$ ($R.\Sigma$) est rempli via des requête d'appartenance sur les mots $a$ et $b$.

\begin{minipage}{0.5\linewidth}
	\centering
	\begin{tabular}{|c|c|}
		\hline
		$O_0$ & $\epsilon$\\
		\hline
		$\epsilon$ & 0\\
		\hline
		$a$ & 0\\
		$b$ & 0\\
		\hline
	\end{tabular}
\end{minipage}
\begin{minipage}{0.5\linewidth}
	\centering
	\begin{figure}[H]
		\centering
		\begin{tikzpicture}[->,>=stealth',shorten >=1pt,auto,node distance=3cm, semithick, bend angle=10]
		\tikzstyle{every state}=[circle]
		\node[initial, state] (A) {$[[\epsilon]]$};
		\path (A) edge [loop above] node {a,b} (A);
		\end{tikzpicture}
		\caption*{Automate $O_0$}
	\end{figure}
\end{minipage}


\vspace{1cm}
L'étape suivante consiste à vérifier la \emph{closure} de la table d'observation $O_0$. Mathématiquement :

$$ \forall u \in R, \forall a \in \Sigma, \exists v \in R, ua R_L v$$

Intuitivement, pour chaque symbole (ici, $\{a,b\}$, et ce sera vrai jusqu'à la dernière itération), tout mot candidat (dans $R$, la partie supérieure de la table) doit se retrouver, complété de ce symbole, dans une classe d'équivalence d'un autre candidat de $R$. Ici, de toute évidence, les mots $a$ et $b$ sont dans la même classe d'équivalence que $\epsilon$. Dès lors, la propriété de \emph{closure} est respectée.

Si la \emph{closure} est respectée, alors la question de la \emph{consistence} (cohérence) se pose. Mathématiquement : 

$$ \forall u,v \in R, u R_L v \Rightarrow \forall a \in \Sigma, ua R_L va$$

Intuitivement, si deux candidats semblent être dans la même classe d'équivalence (leur lignes dans la table supérieure sont identiques), alors pour n'importe quel symbole, les deux nouveaux mots sont également dans une même classe d'équivalence (leur lignes, potentiellement dans la partie inférieure de la table, sont identiques). N'ayant qu'un seul candidat, cette propriété est forcément respectée ($R_L$ est réflexive).

Les deux propriétés étant respectées, les classes d'équivalences sont calculées (trivialement ici), et un automate $O_0$ est proposé à l'enseignant pour vérification.

Sur cette itération, un automate initial a été proposé, et aucun état final ne pouvant être atteint avec un seul symbole, la version est minime.

\subsubsection{Seconde itération}

L'enseignant répond que non, les automates ne sont pas équivalents. Il fourni le contre-exemple $ba$. Comme il est rejeté par $O_0$, cela signifie qu'il est accepté par $A_4$. Une nouvelle table est alors construite, en ajoutant $ba$ et ses préfixes (ici, juste $b$) à $R$. $R.\Sigma$ est calculé et les mots n'ayant pas encore reçu une valeur dans $T$ sont soumis à l'enseignant pour un test d'appartenance.
\vspace{1cm}

\begin{minipage}{0.25\linewidth}
	\centering
	\begin{tabular}{|c|c|}
		\hline
		$O_1$ & $\epsilon$\\
		\hline
		$\epsilon$ & 0\\
		\textcolor{red}{$b$} & \textcolor{red}{0}\\
		\textcolor{red}{$ba$} & \textcolor{red}{1}\\
		\hline
		$a$ & 0\\
		\textcolor{red}{$bb$} & \textcolor{red}{0}\\
		\textcolor{red}{$baa$} & \textcolor{red}{1}\\
		\textcolor{red}{$bab$} & \textcolor{red}{1}\\
		\hline
	\end{tabular}
\end{minipage}
\begin{minipage}{0.25\linewidth}
	\centering
	\begin{tabular}{|c|cc|}
		\hline
		$O_2$ & $\epsilon$ & \textcolor{red}{$a$}\\
		\hline
		$\epsilon$ & 0& \textcolor{red}{0}\\
		$b$ & 0&\textcolor{red}{1}\\
		$ba$ & 1&\textcolor{red}{1}\\
		\hline
		$a$ & 0&\textcolor{red}{0}\\
		$bb$ & 0&\textcolor{red}{1}\\
		$baa$ & 1&\textcolor{red}{1}\\
		$bab$ & 1&\textcolor{red}{1}\\
		\hline
	\end{tabular}
\end{minipage}
\begin{minipage}{0.5\linewidth}
	\centering
	\begin{figure}[H]
		\centering
		\begin{tikzpicture}[->,>=stealth',shorten >=1pt,auto,node distance=3cm, semithick, bend angle=10]
		\tikzstyle{every state}=[circle]
		
		\node[initial, state] (A) {$[[\epsilon]]$};
		\node[state] (B) [right of=A] {$[[b]]$};
		\node[accepting, state] [right of=B] (C) {$[[ba]]$};
		
		\path
		(A) edge [loop above] node {a} (A)
		(A) edge node {b} (B)
		(B) edge node {a} (C)
		(B) edge [loop above] node {b} (B)
		(C) edge [loop above] node {a,b} (C);
		
		
		\end{tikzpicture}
		\caption*{Automate $O_2$}
	\end{figure}
\end{minipage}

\vspace{1cm}
Comme pour la première itération, la \emph{fermeture} est testée, suivie de la \emph{cohérence}. Celle-ci n'est pas respectée : si on considère les mots $\epsilon$ et $b$, qui ont la même ligne dans la table $T$ ($\epsilon R_O b$), le symbole $a$, on obtient les mots $a$ et $ba$ qui n'ont pas la même ligne : ($\not a R_O ba$). Le symbole $a$ est alors ajouté à $S$ et une nouvelle table $O_2$ est calculée.

La fermeture étant déjà vérifiée, la question de la cohérence est reposée, et cette fois-ci elle est vérifiée ; l'automate est construit et proposé à l'enseignant.

Sur cette itération, l'algorithme a reçu le mot $ba$ comme étant accepté. Il a du ajouter $a$ à $S$ pour permettre de différencier certains états. L'automate se voit ajouter les états $[[b]]$ et $[[ba]]$.

\subsubsection{Troisième itération}

Suivant toujours l'algorithme de comparaison d'automates détaillé dans la section \ref{sec:algorithmes}, l'enseignant découvre qu'ils sont différents. 

Il sort le contre-exemple $aaba$. Si c'est un contre-exemple et qu'il est accepté par $O_2$, c'est qu'il ne l'est pas (0) par $A_4$. Une nouvelle table $O_3$ doit être construite.

\begin{minipage}{0.33\linewidth}
	\centering
	\begin{tabular}{|c|cc|}
		\hline
		$O_3$ & $\epsilon$ & $a$\\
		\hline
		$\epsilon$ & 0 &0\\
		\textcolor{red}{$a$}&\textcolor{red}{0}&\textcolor{red}{0}\\
		$b$&0&1\\
		\textcolor{red}{$aa$}&\textcolor{red}{0}&\textcolor{red}{0}\\
		$ba$&1&1\\
		\textcolor{red}{$aab$}&\textcolor{red}{0}&\textcolor{red}{0}\\
		\textcolor{red}{$aaba$}&\textcolor{red}{0}&\textcolor{red}{0}\\
		\hline
		\textcolor{red}{$ab$}&\textcolor{red}{0}&\textcolor{red}{1}\\
		$bb$&0&1\\
		\textcolor{red}{$aaa$}&\textcolor{red}{0}&\textcolor{red}{0}\\
		$baa$&1&1\\
		$bab$&1&1\\
		\textcolor{red}{$aabb$}&\textcolor{red}{0}&\textcolor{red}{0}\\
		\textcolor{red}{$aabaa$}&\textcolor{red}{0}&\textcolor{red}{0}\\
		\textcolor{red}{$aabab$}&\textcolor{red}{0}&\textcolor{red}{0}\\
		\hline
	\end{tabular}
\end{minipage}
\begin{minipage}{0.33\linewidth}
	\centering
	\begin{tabular}{|c|cc|}
		\hline
		$O_4$ & $\epsilon$ & $a$\\
		\hline
		$\epsilon$ & 0 &0\\
		$a$&0&0\\
		$b$&0&1\\
		$aa$&0&0\\
		\textcolor{red}{$ab$}&\textcolor{red}{0}&\textcolor{red}{1}\\
		$ba$&1&1\\
		$aab$&0&0\\
		$aaba$&0&0\\
		\hline
		$bb$&0&1\\
		$aaa$&0&0\\
		\textcolor{red}{$aba$}&\textcolor{red}{1}&\textcolor{red}{1}\\
		\textcolor{red}{$abb$}&\textcolor{red}{0}&\textcolor{red}{1}\\
		$baa$&1&1\\
		$bab$&1&1\\
		$aabb$&0&0\\
		$aabaa$&0&0\\
		$aabab$&0&0\\
		\hline
	\end{tabular}
\end{minipage}
\begin{minipage}{0.33\linewidth}
	\centering
	\begin{tabular}{|c|cc|}
		\hline
		$O_5$ & $\epsilon$ & $a$\\
		\hline
		$\epsilon$ & 0 &0\\
		$a$&0&0\\
		$b$&0&1\\
		$aa$&0&0\\
		$ab$&0&1\\
		$ba$&1&1\\
		$aab$&0&0\\
		\textcolor{red}{$aba$}&\textcolor{red}{1}&\textcolor{red}{1}\\
		$aaba$&0&0\\
		\hline
		$bb$&0&1\\
		$aaa$&0&0\\
		$abb$&0&1\\
		$baa$&1&1\\
		$bab$&1&1\\
		$aabb$&0&0\\
		\textcolor{red}{$abaa$}&\textcolor{red}{1}&\textcolor{red}{1}\\
		\textcolor{red}{$abab$}&\textcolor{red}{1}&\textcolor{red}{1}\\
		$aabaa$&0&0\\
		$aabab$&0&0\\
		\hline
	\end{tabular}
\end{minipage}



	\begin{figure}[H]
		\centering
		\begin{tikzpicture}[->,>=stealth',shorten >=1pt,auto,node distance=3cm, semithick, bend angle=10]
		\tikzstyle{every state}=[circle]
		
		\node[initial, state] (A) {$[[\epsilon]]$};
		\node[state] (B) [above right of=A] {$[[a]]$};
		\node [state] (E) [right of=B] {$[[aa]]$};
		\node[state] (C) [below right of =A] {$[[b]]$};
		\node[accepting, state] [right of=C] (D) {$[[ba]]$};
		
		\path
		(A) edge node {a} (B)
		(A) edge node {b} (C)
		(B) edge node {b} (C)
		(B) edge node {a} (E)
		(C) edge [loop below] node {b} (C)
		(C) edge node {a} (D)
		(D) edge [loop above] node {a,b} (D)
		(E) edge [loop below] node {a,b} (E);
		
		\end{tikzpicture}
		\caption*{Automate $O_5$}
	\end{figure}


Ayant reçu $aaba$, ce mot et tous ses préfixes sont ajoutés à la table. L'extension $R.\Sigma$ est recalculée et la table $O_3$ est construite.

Ensuite, la question de la \emph{fermeture} est posée. Un manquement est détecté : le mot $a$. En effet, en lui ajoutant le symbole $b$, on obtient $ab$ qui n'est ni dans $R$ ni en relation $R_O$ avec $a$. $ab$ est alors ajouté à $R$, et $R.\Sigma$ est étendu. La nouvelle table, $O_4$ est de nouveau testée.

$O_4$ ne respecte pas la fermeture : le mot $ab$, agrémenté du symbole $a$ donne le mot $aba$, qui n'est ni dans $R$ ni en relation avec $ab$. Le mot est ajouté à $R$, et la table est étendue. La nouvelle table, $O_5$ est à la fois fermée et cohérente.

L'automate $O_5$ est alors proposé à l'enseignant pour vérification. Celui-ci est accepté (isomorphe à $A_4$). L'algorithme s'arrête et un automate minimal pour le langage a été construit. 

\subsection{Algorithme}\label{ss:a_algo}

\subsection{Preuve}\label{ss:a_proof}

\subsection{Complexité}\label{ss:a_comp}


	\chapter{Apprentissage automatique de vérification de sécurité d'automates FIFO}\label{ch:learning}\input{vardhan}
	\section{Problème}\label{sec:prob}Cette section décrit le problème rencontré par \cite{Vardhan04} et la technique générale utilisée pour proposer une solution à ce problème.

Les automates FIFO définis à la section \ref{sec:fifo} sont plus puissants que les ADF définis dans le chapitre \ref{ch:bases}. Contrairement, à ceux-ci, les automates FIFO ont possiblement une infinité d'états. Dans ces conditions, il n'est pas possible d'en faire une exploration exhaustive pour trouver tous les états acceptants.

À la place, une propriété dite de sécurité est définie. Si un état respecte cette propriété, il est sécure. Si il y a moyen de prouver que la totalité des états de l'automate respectent cette propriété, l'automate est considéré comme sécure. Si au contraire, un exemple de violation de la propriété est trouvé, l'automate peut être déclaré comme insécure.

Les sections suivantes donnent les différents éléments utilisés par \cite{Vardhan04} pour répondre à cette question. Un langage proxy est donné pour représenter les différents états d'un automate FIFO. Celui-ci est construit pour être régulier. De la sorte, il est possible d'appliquer l'algorithme d'Angluin pour apprendre ce langage proxy.

Celui-ci n'étant qu'une approximation du langage de l'automate FIFO, certaines incertitudes peuvent apparaître. Cependant, l'article justifie ces différents cas en ramenant la question à la sécurité, qui peut être répondue en un temps polynomial.

Dès lors, les différentes section permettent de construire ce langage, de se prononcer sur l'appartenance et l'équivalence avec ce langage et d'arrêter l'apprentissage s'il est possible de se prononcer sur la propriété de sécurité pour l'automate FIFO considéré.
%problème et générale proposée par le papier
	\section{Automate FIFO}\label{sec:fifo}L'article de Vardhan \cite{Vardhan04} se concentre sur un automate plus général : l'automate FIFO. Celui-ci est Turing Complete. De la sorte, leur équipe propose une réponse pour un ensemble plus large de langage. Toutefois, tous les langages ne sont pas éligibles à l'algorithme. En effet, c'est un problème indécidable de façon générale. Cette section décrit les automates FIFO et lesquels sont éligibles à l'algorithme $L^*$ modifié.



% ██████  ███████ ███████
% ██   ██ ██      ██
% ██   ██ █████   █████
% ██   ██ ██      ██
% ██████  ███████ ██

\subsection{Définitions}\label{ss:fifodef}

\begin{definition}
  Un \emph{automate FIFO} \fifo est défini comme suit :
  \begin{itemize}
    \item $Q$ est un ensemble fini d'\emph{états de contrôle}
    \item $C$ est un ensemble fini de \emph{noms de canaux}
    \item $\Sigma$ est un alphabet
    \item $q_0 \in Q$ est l'\emph{état de contrôle initial}
    \item $\Theta$ est un ensemble fini de \emph{noms de transitions}
    \item $\delta$ est la \emph{fonction nommante}. $\delta : \Theta \rightarrow Q \times ((C \times \{?,!\} \times \Sigma) \bigcup \{\tau\}) \times Q$. Un nom de transition $\theta$ correspond à une transition de la forme $\delta(\theta)=(p,\text{"action"},q)$. Cette action a une des trois formes suivantes :
    \begin{itemize}
      \item $c!m$ : C'est une action d'envoi. Le symbole $m$ est ajouté en fin de canal $c$.
      \item $c?m$ : C'est une action de réception. Le symbole $m$ est consommé en début de canal $c$.
      \item $\tau$ : C'est une action interne. Aucun canal n'est manipulé.
    \end{itemize}
  \end{itemize}
\end{definition}

Un automate $F$ défini un \emph{système de transitions} \tsys. $\mathcal{T}$ est l'objet qui permet de passer d'un \emph{état} à un autre.

En effet, il existe les états de contrôles $q\in Q$, mais les états au sens d'un automate FIFO sont de forme $s \in S=Q\times(\Sigma^*)^C$. En particulier, un état $s=(q,w)$ avec $q\in Q$ un état de contrôle et $w\in (\Sigma^*)^C$ est un vecteur qui fait correspondre à chaque canal $c\in C$ un mot $w[c] \in \Sigma^*$ représentant le contenu de ce canal.

Dès lors, un état $s$ peut être compris comme étant composé d'un état et du contenu des différents canaux.

De plus, la \emph{fonction de transition} $\rightarrow:S\times\Theta\rightarrow S$ associe un état $s$ et un nom de transition $\theta$ à un état $s'$.

$\mathcal{T}$ respecte trois règles, correspondants chacune à un des types d'actions mentionnés précédemment. En plus de la notation $w[c]$, celles-ci utilisent la notation $w[c\mapsto c']$ signifiant $w$ à l'exception du canal $c$ dont le contenu a été remplacé par le mot $c'$.
\begin{itemize}
  \item Si $\delta(\theta)=(p,c?m,q)$ alors $(p,w)\xrightarrow{\theta}(q,w')$ si et seulement si $w=w'[c\mapsto mw'[c]]$
  \item Si $\delta(\theta)=(p,c!m,q)$ alors $(p,w)\xrightarrow{\theta}(q,w')$ si et seulement si $w'=w[c\mapsto mw[c]]$
  \item Si $\delta(\theta)=(p,\tau,q)$ alors $(p,w)\xrightarrow{\theta}(q,w')$ si et seulement si $w=w'$
\end{itemize}




\begin{example}
  Soit un automate FIFO $F$ tel que sont système de transitions corresponde à la figure \ref{fig:fifo1}. Chaque état de l'automate correspondant à un couple état de contrôle/mot, il est imprécis de référer au système de transitions comme étant l'automate. Cependant, par abus de langage, ceux-ci seront souvent confondus dans ce document.

  \begin{figure}[H]
    \centering
    \begin{tikzpicture}[->,>=stealth',shorten >=1pt,auto,node distance=2.5cm, semithick, bend angle=10]

      \tikzstyle{every state}=[circle]

      \node[initial,state] (A)                    {$q_0$};
      \node[state]         (B) [above right= 1cm and 3 cm of A] {$q_1$};
      \node[state]         (C) [below right= 1cm and 3 cm of A] {$q_2$};
      \node[state]         (D) [above right= 1cm and 3 cm of C] {$q_3$};

      \path
      (A) edge node {$\theta_1(a!0)$} (B)
      (A) edge node[below left] {$\theta_2(a!1)$} (C)
      (B) edge node {$\theta_4(a?0)$} (D)
      (B) edge[loop above] node {$\theta_3(b!1)$} (B)
      (C) edge node[below right] {$\theta_6(a?1)$} (D)
      (C) edge [loop below] node{$\theta_5(b!0)$} (C)
      (D) edge node[above] {$\theta_7(b?0)$} (A)
      ;
    \end{tikzpicture}
    \caption{Système de transitions de l'automate FIFO $F$}\label{fig:fifo1}
  \end{figure}

  On retrouve bien la définition d'un automate fifo \fifo avec :
  \begin{itemize}
    \item $Q=\{q_0,q_1,q_2,q_3\}$
    \item $C=\{a,b\}$
    \item $\Sigma=\{0,1\}$
    \item $q_0\in Q$
    \item $\Theta=\{\theta_1, \theta_2, \theta_3, \theta_4, \theta_5, \theta_6\}$
    \item $\delta$ associant à chaque $\theta_i$ un triplet état/action/état. Celui-ci est représenté entre parenthèses à côté du nom de transition associé
  \end{itemize}

  De plus, on peut déduire le système de transition $\mathcal{T}$ défini par $F$. Considérons le mot $w=[\epsilon,\epsilon]$ où le premier élément du vecteur est le contenu du canal $a$ et le second celui du canal $b$.
  Dans cet exemple, comme $\delta(\theta_1)=(q_0,a!0,q_1)$, alors $(q_0,w)\xrightarrow{\theta_1}(q_1,w')$. Dans ce cas, $w'=[0,\epsilon]$. A ce moment, on a bien $w'=w[a\mapsto 0w[a]]$.
  En utilisant ce nouveau mot $w'$, un nouvel état est atteignable : $q_3$. En effet, comme $\delta(\theta_4)=(q_1,a?0,q_3)$, alors $(q_1,w')\xrightarrow{\theta_4}(q_4,w'')$. Dans ce cas, $w''=[\epsilon,\epsilon]$. A ce moment, on a bien $w'=w''[a\mapsto 0w''[a]]$.

  Intuitivement, la première transition $\theta_1$ ajoute le symbole $0$ en tête du canal $a$ en passant de l'état $q_0$ à l'état $q_1$. La transition $\theta_4$, elle, permet de passer de l'état $q_1$ à $q_3$ en consommant $0$ en tête du canal $a$.

\end{example}


% ██       █████  ███    ██  ██████
% ██      ██   ██ ████   ██ ██
% ██      ███████ ██ ██  ██ ██   ███
% ██      ██   ██ ██  ██ ██ ██    ██
% ███████ ██   ██ ██   ████  ██████

\subsection{Langage tracé}

Une façon de définir un langage à partir d'un automate FIFO est de s'intéresser aux noms des transitions suivies lors de l'exécution. Cette section défini les éléments permettant d'arriver à la construction d'un tel langage.

Dans un système de transitions \tsys, la fonction de transition $\rightarrow:S\times\Theta\rightarrow S$ permet de définir le passage d'un état à un autre.

La \emph{fonction de transition étendue} $\xrightarrow{*}$ est la fermeture transitive et réflexive de $\rightarrow$.

Pour une suite de noms de transitions $\sigma=\theta_1\theta_2 ...\theta_n\in\Theta^*$, on note $(p,w)\xrightarrow{\sigma}(q,w')$ si il existe des états $(p_1,w_1)(p_2,w_2)...(p_{n-1},w_{n-1})$ tels que $(p,w)\xrightarrow{\theta_1}(p_1,w_1)\xrightarrow{\theta_2}...\xrightarrow{\theta_{n-1}}(p_{n-1},w_{n-1})\xrightarrow{\theta_n}(q,w')$. Dans ce cas, $\sigma$ est une \emph{trace de chemin}.

\begin{definition} Soit un automate FIFO $F$ et l'état initial $s_0=(q_0, \epsilon^C)$. Celui-ci est le couple état de contrôle initial $q_0$ ainsi que des mots $w[c]=\epsilon$ pour tout canal $c\in C$.

  Le \emph{langage de trace} d'un automate $F$ est

  $$
  L(F)=\{\sigma\in\Theta^*|\exists s=(p,w) \text{ tel quel } s_0\xrightarrow{\sigma}s\}
  $$
\end{definition}

\begin{example}
  Considérons l'automate FIFO $F$ de la figure \ref{fig:fifo1}.

  Pour celui-ci, $\sigma=\theta_1\theta_4\theta_7$ n'est pas un chemin. En effet,
  $$
  (q_0,[\epsilon,\epsilon])\xrightarrow{\theta_1}(q_1,[0,\epsilon])\xrightarrow{\theta_4}(q_3,[\epsilon,\epsilon])
  $$

  Mais, il n'existe pas d'état $s$ tel que $(q_3,[\epsilon,\epsilon])\xrightarrow{\theta_7}s$. En effet, pour appliquer cette transition, il aurait fallu que le canal $b$ contienne un symbole $0$. Ce n'est pas le cas.


  Par contre, $\sigma=\theta_2\theta_5\theta_5\theta_6\theta_7\theta_1\theta_4\theta_7$ est un chemin dans $F$ :
  \begin{equation*}
    \begin{gathered}
      (q_0,[\epsilon,\epsilon])\xrightarrow{\theta_2}
      (q_0,[1,\epsilon])\xrightarrow{\theta_5}
      (q_0,[1,0])\xrightarrow{\theta_5}
      (q_0,[1,00])\xrightarrow{\theta_6}
      (q_0,[\epsilon,00])\xrightarrow{\theta_7}\\
      (q_0,[\epsilon,0])\xrightarrow{\theta_1}
      (q_0,[0,0])\xrightarrow{\theta_4}
      (q_0,[\epsilon,0])\xrightarrow{\theta_7}
      (q_0,[\epsilon,\epsilon])
    \end{gathered}
  \end{equation*}

  On a bien un état $s$ (ici $s=(q_0,[\epsilon,\epsilon])=s_0$) tel que $s_0\xrightarrow{\sigma}s$.

\end{example}

\subsection{Produit cartésien}\label{ss:cartesien}

Par soucis de simplicité, un automate FIFO (et son système de transitions servant à le représenter) peut être représenté comme plusieurs systèmes de transitions utilisant les mêmes canaux. Le \emph{produit cartésien} entre deux automates FIFO $A$ et $B$ retourne un nouvel automate FIFO $F=A \times B$. Dès lors, il est possible de représenter un automate FIFO en se concentrant sur ses parties et en les isolant. Ce produit cartésien fonctionne comme suit.

Soient les automates FIFO \fifoA et \fifoB. Alors le système de transitions \tsys de l'automate FIFO $F=A\times B$ est composé de :
\begin{itemize}
  \item $S \subseteq (Q_A\times Q_B)\times (\Sigma^*)^C$ composé d'un couple d'états de contrôle de $Q_A$ et $Q_B$ et du contenu des différents canaux.
  \item $\Theta = \Theta_A \bigcup \Theta_B$
  \item $\rightarrow$ est construit comme suit. Soit un état $((q_A,q_B), w)\in S$. Soit un triplet $(p,a,q)$ avec $p,q \in (Q_A \bigcup Q_B)$ et $a \in ((C \times \{?,!\} \times \Sigma) \bigcup \{\tau\})$.
  $((q_A,q_B),w)\xrightarrow{\theta}((q_{A'},q_{B'}),w')$ si et seulement si l'une des trois conditions suivantes est remplie

  \begin{itemize}
    \item $\exists \theta_A\in\Theta_A, \delta_A(\theta_A)=(q_A,a,q_{A'})$ et  $(q_A,w)\xrightarrow{\theta_A}(q_{A'},w')$ dans l'automate $A$ et\\ $\exists \theta_B\in\Theta_B, \delta_B(\theta_B)=(q_B,a,q_{B'})$ et $(q_B,w)\xrightarrow{\theta_B}(q_{B'},w')$ dans l'automate $B$
    \item $\exists \theta_A\in\Theta_A, \delta_A(\theta_A)=(q_A,a,q_{A'})$ et  $(q_A,w)\xrightarrow{\theta_A}(q_{A'},w')$ dans l'automate $A$,\\
    $\forall \theta_B\in\Theta_B,\forall q \in Q_B,\delta_B(\theta_B)\neq(q_B,a,q)$ dans l'automate $B$ et $q_{B'}=q_B$
    \item $\forall \theta_A\in\Theta_A,\forall q \in Q_A,\delta_A(\theta_A)\neq(q_A,a,q)$ dans l'automate $A$ et $q_{A'}=q_A$,\\
    $\exists \theta_B\in\Theta_B, \delta_B(\theta_B)=(q_B,a,q_{B'})$ et  $(q_B,w)\xrightarrow{\theta_B}(q_{B'},w')$ dans l'automate $B$
  \end{itemize}
\end{itemize}


Dès lors, l'état initial est (($q_0,q_0$), [$\epsilon$,...,$\epsilon$]). \cite{Suresh20}


Alors, on peut construire l'automate FIFO \fifo grâce à l'algorithme de produit cartésien \ref{alg:crossfifo} :

\begin{algo}[Produit cartésien entre deux automates]\label{alg:crossfifo}
 \begin{algorithmic}[1]
   \REQUIRE Les automates FIFO $A=(Q_A,C,\Sigma,q_{0A}, \Theta_A,\delta_A)$ et $B=(Q_B,C,\Sigma,q_{0B},\Theta_B,\delta_B)$
   \ENSURE Un automate FIFO \fifo

   \STATE $Q=Q_A\times Q_B$
   \STATE $q_0 = (q_{0A},q_{0B})$ \COMMENT{l'état de contrôle à l'intersection de $q_{0A}$ et $q_{0B}$}
   \STATE $\Theta=\Theta_A \times \Theta_B$

   \FORALL {action $a \in (C\times\{?,!\}\times\Sigma)\bigcup\{\tau\}$}
    \FORALL {$\theta_A \in \Theta_A$ tels que $\delta_A(\theta_A)=(q_A,a,q_{A'})$ avec $q_A,q_{A'}\in Q_A$}
      \FORALL {$\theta_B \in \Theta_B$ tels que $\delta_B(\theta_B)=(q_B,a,q_{B'})$ avec $q_B,q_{B'}\in Q_B$}
        \STATE $\delta((\theta_A, \theta_B))=((q_A,q_B),a,(q_{A'},q_{B'}))$
      \ENDFOR
    \ENDFOR
    \FORALL {$\theta \in \Theta_A$ tels que $\delta_A(\theta)=(q_A,a,q_{A'})$ avec $q_A,q_{A'}\in Q_A$ qui n'ont pas été utilisés dans la boucle 6}
      \FORALL {$q_B\in Q_B$}
        \STATE  $\delta((\theta,))=((q_A,q_B),a,(q_{A'},_B))$
      \ENDFOR
    \ENDFOR

    \FORALL {$\theta \in \Theta_B$ tels que $\delta_B(\theta)=(q_B,a,q_{B'})$ avec $q_B,q_{B'}\in Q_B$ qui n'ont pas été utilisés dans la boucle 6}
      \FORALL {$q_A\in Q_A$}
        \STATE  $\delta((,\theta))=((q_A,q_B),a,(q_A,_{B'}))$
      \ENDFOR
    \ENDFOR

   \ENDFOR


   \RETURN \fifo
 \end{algorithmic}
\end{algo}

L'automate produit par cet algorithme \ref{alg:crossfifo} est différents des deux autres, il n'est alors pas pertinent de prouver une égalité. Il s'agit juste d'un autre mode de représentation.


\begin{example}
  Soient deux automates FIFO $A$ et $B$ tels que représentés par leur systèmes de transitions donnés par la figure \ref{fig:fifoAB}. L'automate FIFO $AB=A \times B$ est représenté par son sytème de transitions à la figure \ref{fig:fifocross}.

  \begin{figure}[H]
    \centering
    \begin{subfigure}{0.5\textwidth}
      \centering
      \begin{tikzpicture}[->,>=stealth',shorten >=1pt,auto,node distance=1.5cm, semithick, bend angle=10]
        \tikzstyle{every state}=[circle]

        \node[initial,state] (A)  {$q_{0}$};
        \node[state]         (B) [right=of A]  {$q_{1}$};
        \node[state]         (C) [right=of B]  {$q_{2}$};
        \path
        (A) edge node {$\theta_1(a!1)$} (B)
        (B) edge node {$\theta_2(a?1)$} (C)
        (C) edge[bend left=40] node {$\theta_3(a!0)$} (A)
        ;
      \end{tikzpicture}
      \caption{Automate FIFO A}
      \label{fig:fifoA}
      \end{subfigure}%
      \begin{subfigure}{0.5\textwidth}
        \centering
        \begin{tikzpicture}[->,>=stealth',shorten >=1pt,auto,node distance=1.5cm, semithick, bend angle=10]
          \tikzstyle{every state}=[circle]

          \node[initial,state] (A)  {$q_{A}$};
          \node[state]         (B) [right=of A]  {$q_{B}$};
          \path
          (A) edge[bend left=20] node {$\theta_5(a?0)$} (B)
          (B) edge[bend left=20] node {$\theta_6(a!0)$} (A)
          ;
        \end{tikzpicture}
        \caption{Automate FIFO B}
        \label{fig:fifoB}
      \end{subfigure}
      \caption{Automates FIFO A et B représentés par leur système de transitions}
      \label{fig:fifoAB}
    \end{figure}



    \begin{figure}[H]
      \centering
      \begin{tikzpicture}[->,>=stealth',shorten >=1pt,auto,node distance=1.5cm, semithick, bend angle=15]
        \tikzstyle{every state}=[circle]

        \node[initial,state] (A)  {$q_{0A}$};
        \node[state]         (B) [below=of A]  {$q_{1A}$};
        \node[state]         (C) [below=of B]  {$q_{2A}$};

        \node[state]         (D) [right=of A] {$q_{0B}$};
        \node[state]         (E) [below=of D]  {$q_{1B}$};
        \node[state]         (F) [below=of E]  {$q_{2B}$};

        \path
        (A) edge node {$\theta(a!1)$} (B)
        (A) edge[bend left] node {$\theta(a?0)$} (D)

        (B) edge node {$\theta(a?1)$} (C)
        (B) edge[bend left] node {$\theta{a?0}$} (E)

        (C) edge[bend left=35] node {$\theta(a!0)$} (A)
        (C) edge node {$\theta(a?0)$} (F)

        (D) edge[bend left] node {$\theta{a!0}$} (A)
        (D) edge node {$\theta{a!1}$} (E)

        (E) edge[bend left] node {$\theta{a!0}$} (B)
        (E) edge node {$\theta{a?1}$} (F)

        ;

        \draw [->] (F) ..  controls  ($(E)+(5cm,2cm)$) and
  ($(D)+(-0.5cm,3cm)$).. node[right] {$\theta(a!0)$} (A);
      \end{tikzpicture}
      \caption{Automate FIFO AB résultant du produit cartésien $A\times B$}
      \label{fig:fifocross}
    \end{figure}

  \end{example}



  %  █████  ██████  ██████
  % ██   ██ ██   ██ ██   ██
  % ███████ ██████  ██████
  % ██   ██ ██   ██ ██
  % ██   ██ ██████  ██

  \subsection{Alternating Bit Protocol}\label{ss:abp}

  \begin{figure}[H]
    \centering
    \begin{tikzpicture}[->,>=stealth',shorten >=1pt,auto,node distance=3.5cm, semithick, bend angle=10]

      \tikzstyle{every state}=[circle]

      \node[initial,state] (A)                    {$q_0$};
      \node[state]         (B) [right of=A] {$q_1$};
      \node[state]         (C) [below of=B] {$q_2$};
      \node[state]         (D) [left of=C] {$q_3$};

      \node[state,draw=none]         (i1) [right=0cm of B]      {};
      \node[state,draw=none]         (i2) [right=3.5cm of i1]      {};
      \node[state,draw=none]         (i3) [right=0cm of C]      {};
      \node[state,draw=none]         (i4) [right=3.5cm of i3]      {};

      \node[state] (E) [right=0cm of i2]               {$q_{0'}$};
      \node[state]         (F) [right of=E] {$q_{1'}$};
      \node[state]         (G) [below of=F] {$q_{2'}$};
      \node[state]         (H) [left of=G] {$q_{3'}$};



      \path
      (A) edge node {$\theta_1(A!0)$} (B)
      (B) edge node {$\theta_4(B?ACK0)$} (C)
      (B) edge[loop above] node {$\theta_2(A!0),\theta_3(B?ACK1)$} (B)
      (C) edge node {$\theta_5(A!1)$} (D)
      (D) edge node {$\theta_8(B?ACK1)$} (A)
      (D) edge[loop below] node {$\theta_6(A!1),\theta_7(B?ACK0)$} (D)


      (E) edge node {$\theta_{11}(A?0)$} (F)
      (E) edge[loop above] node {$\theta_9(B!ACK1),\theta_{10}(A?1)$} (E)
      (F) edge node {$\theta_{12}(B!ACK0)$} (G)
      (G) edge node {$\theta_{15}(A?1)$} (H)
      (G) edge[loop below] node {$\theta_{13}(B!ACK0),\theta_{14}(A?0)$} (G)
      (H) edge node {$\theta_{16}(B!ACK1)$} (E)
      ;

      \draw[<-] (E) -- node[above left] {start} ++(-1cm,-1cm);

      \draw[double,->] (i1) -- node[above] {canal A} (i2);
      \draw[double,->] (i4) -- node[below] {canal B} (i3);
    \end{tikzpicture}
    \caption{Automate Fifo du ABP (\cite{Finkel03}, Fig.1.)}\label{fig:fifoabp}
  \end{figure}

  \href{https://scanftree.com/automata/dfa-cross-product-property}{Produit cartésien de deux automates}. Grâce à ça, on peut représenter ce système comme un seul automate, mais en avoir deux sur un graphique, ce qui simplifie la compréhension.

	\section{Trace d'automate}\label{sec:trace}Cette section s'intéresse aux langages qui peuvent être associés à un automate. La section \ref{ss:trace} défini le langage de trace d'un automate. Celui-ci n'est pas nécessairement régulier. Les sections suivantes s'appuyent sur \cite{Vardhan04} pour proposer un langage régulier qui représente ce langage de trace.


% ██       █████  ███    ██  ██████
% ██      ██   ██ ████   ██ ██
% ██      ███████ ██ ██  ██ ██   ███
% ██      ██   ██ ██  ██ ██ ██    ██
% ███████ ██   ██ ██   ████  ██████

\subsection{Langage tracé}\label{ss:trace}

Une façon de définir un langage à partir d'un automate FIFO est de s'intéresser aux noms des transitions suivies lors de l'exécution. Cette section défini les éléments permettant d'arriver à la construction d'un tel langage.

Dans un système de transitions \tsys, la fonction de transition $\rightarrow:S\times\Theta\rightarrow S$ permet de définir le passage d'un état à un autre.

La \emph{fonction de transition étendue} $\xrightarrow{*}$ est la fermeture transitive et réflexive de $\rightarrow$.

Pour une suite de noms de transitions $\sigma=\theta_1\theta_2 ...\theta_n\in\Theta^*$, on note $(p,w)\xrightarrow{\sigma}(q,w')$ si il existe des états $(p_1,w_1)(p_2,w_2)...(p_{n-1},w_{n-1})$ tels que $(p,w)\xrightarrow{\theta_1}(p_1,w_1)\xrightarrow{\theta_2}...\xrightarrow{\theta_{n-1}}(p_{n-1},w_{n-1})\xrightarrow{\theta_n}(q,w')$. Dans ce cas, $\sigma$ est une \emph{trace de chemin}.

\begin{definition} Soit un automate FIFO $F$ et l'état initial $s_0=(q_0, \epsilon^C)$. Celui-ci est le couple état de contrôle initial $q_0$ ainsi que des mots $w[c]=\epsilon$ pour tout canal $c\in C$.

  Le \emph{langage de trace} d'un automate $F$ est

  $$
  L(F)=\{\sigma\in\Theta^*|\exists s=(p,w) \text{ tel quel } s_0\xrightarrow{\sigma}s\}
  $$
\end{definition}

\begin{example}
  Considérons l'automate FIFO $F$ de la figure \ref{fig:fifo1}.

  Pour celui-ci, $\sigma=\theta_1\theta_4\theta_7$ n'est pas un chemin. En effet,
  $$
  (q_0,[\epsilon,\epsilon])\xrightarrow{\theta_1}(q_1,[0,\epsilon])\xrightarrow{\theta_4}(q_3,[\epsilon,\epsilon])
  $$

  Mais, il n'existe pas d'état $s$ tel que $(q_3,[\epsilon,\epsilon])\xrightarrow{\theta_7}s$. En effet, pour appliquer cette transition, il aurait fallu que le canal $b$ contienne un symbole $0$. Ce n'est pas le cas.


  Par contre, $\sigma=\theta_2\theta_5\theta_5\theta_6\theta_7\theta_1\theta_4\theta_7$ est un chemin dans $F$ :
  \begin{equation*}
    \begin{gathered}
      (q_0,[\epsilon,\epsilon])\xrightarrow{\theta_2}
      (q_0,[1,\epsilon])\xrightarrow{\theta_5}
      (q_0,[1,0])\xrightarrow{\theta_5}
      (q_0,[1,00])\xrightarrow{\theta_6}
      (q_0,[\epsilon,00])\xrightarrow{\theta_7}\\
      (q_0,[\epsilon,0])\xrightarrow{\theta_1}
      (q_0,[0,0])\xrightarrow{\theta_4}
      (q_0,[\epsilon,0])\xrightarrow{\theta_7}
      (q_0,[\epsilon,\epsilon])
    \end{gathered}
  \end{equation*}

  On a bien un état $s$ (ici $s=(q_0,[\epsilon,\epsilon])=s_0$) tel que $s_0\xrightarrow{\sigma}s$.

\end{example}

  % ████████ ██   ██ ███████ ████████  █████
  %    ██    ██   ██ ██         ██    ██   ██
  %    ██    ███████ █████      ██    ███████
  %    ██    ██   ██ ██         ██    ██   ██
  %    ██    ██   ██ ███████    ██    ██   ██



\subsection{Alphabet d'annotation}

Le langage de trace n'est pas nécessairement régulier. Pour permettre l'apprentissage par l'algorithme d'Angluin, il faut en construire un qui est régulier et qui permette de reconstruire le langage de trace. Pour ce faire, ce nouveau langage devrait pouvoir représenter tout état atteignable ainsi qu'un ou plusieurs chemins ou mots témoins permettant d'atteindre ceux-ci.

Pour ce faire, pour chaque nom de transition correspondant à une action d'envoi, un \emph{co-nom} est défini :
$$
\bar{\Theta}=\{\bar{\theta}|\theta\in\Theta\wedge\exists p,q \in Q, c\in C, a\in\Sigma,\text{tels que } \delta(\theta)=(p,c!a,q)\}
$$

De plus, un \emph{symbole de contrôle} est créé pour chaque état de contrôle : $T_Q = \{t_q | q\in Q\}$.

En combinant les noms de transitions, les co-noms et les symboles de contrôlé, un nouvel alphabet peut-être défini, l'\emph{alphabet d'annotation} : $\Phi=(\Theta-\Theta_r)\bigcup\bar{\Theta}\bigcup T_Q$.

Avec $\Theta_r=\{\theta|\theta\in\Theta\wedge\exists p,q \in Q, c\in C, a\in\Sigma,\text{tels que } \delta(\theta)=(p,c?a,q)\}$, similaire à $\bar{\Theta}$ mais avec un nom pour chaque transition pour les actions de réception.


\subsection{Trace annotée}


Soit $\mathcal{A}:\Theta^*\rightarrow\Phi^*$ une fonction associant une \emph{trace annotée} à une trace d'automate. Cette fonction est décrite par l'algorithme \ref{alg:A}.


\begin{algorithm}[H]
  	\begin{algorithmic}[1]
    \REQUIRE un automate FIFO \fifo , une suite de noms de transitions $\sigma\in\Theta^*$
		\ENSURE une trace annotée $\gamma\in\Phi^*$ représentant $\sigma$

    \STATE $\gamma\leftarrow\epsilon$
    \FORALL {transition $\theta\in\sigma$}
      \IF {$\theta$ correspond à une action de réception}
        \STATE trouver $\theta_s\in\Theta$ correspondant à une action d'envoi antécédant dans $\sigma$ tel que les actions s'appliquent sur le même canal et le même symbole
        \STATE $\gamma\leftarrow$ $\gamma$ où $\theta_s$ est remplacé par $\bar{\theta_s}\in\bar{\Theta}$ \COMMENT {$\theta$ n'est pas ajouté à $\gamma$}
      \ELSIF {$\theta$ correspond à une action d'envoi}
        \STATE $\gamma\leftarrow\gamma\theta$
      \ENDIF
    \ENDFOR
    \STATE trouver $q$ l'état de contrôle tel que $\delta(\theta)=(p,a,q)$ avec $p\in Q,a\in((C \times \{?,!\} \times \Sigma) \bigcup \{\tau\})$
    \STATE $\gamma\leftarrow\gamma t_q$ avec $t_q\in T_Q$ le symbole de contrôle associé à $q$
		\RETURN $\gamma$
	\end{algorithmic}
	\caption{$\mathcal{A}:\Theta^*\rightarrow\Phi^*$}\label{alg:A}
\end{algorithm}

Soit $AL(F)=\{\mathcal{A}(\sigma)|\sigma \in L(F)\}$ le \emph{langage de traces annotées} de l'automate $F$. $AL(F)$ est un ensemble de traces annotées correspondant à des exécutions valides de l'automate $F$. Intuitivement, $AL(F)$ contient l'ensemble des états atteignables par $F$ ainsi que les traces annotées servant de témoins de cette atteignabilité des états.


Soit un mot $\gamma \in \Phi^*$. $\gamma$ est \emph{correctement formaté} si il fini par un symbole de $T_Q$ qui qu'aucun autre symbole de cet ensemble n'apparaît dans le mot. Soit un langage arbitraire $L$. $L$ est \emph{correctement formaté} si tous les mots de $L$ le sont.


\begin{example}
Soit l'automate $F$ représenté par la figure \ref{fig:fifoAB}. Soient les séquences $\sigma_1=\theta_2\theta_8$ et $\sigma_2=\theta_1\theta_3\theta_5\theta_2$. Alors, les traces annotées de ces traces sont : $\mathcal{A}(\sigma_1)=\theta_2\theta_8t_{(q_1,q_B)}=\gamma_1$ et $\mathcal{A}(\sigma_2)=\bar{\theta_1}\bar{\theta_5}t_{(q_0,q_B)}=\gamma_2$.
Bien qu'elles soient toutes deux correctement formatées, $\gamma_1$ ne correspond à aucune exécution valide de $F$. Dès lors, $\gamma_1$ n'appartient pas au langage de traces annotées de $AL(F)$ contrairement à $\gamma_2$.

Soit le mot $\gamma_3=t_{q_0,q_A}\theta_2 t_{q_0,q_B} \in \Phi^*$. $\gamma_3$ n'est pas correctement formaté : il est impossible que ce mot appartienne à $AL(F)$.
\end{example}
%trace d'automate, Theta, Theta-, AL, etc
	\section{États insécures}\label{sec:unsafe}Comme mentionné au début du chapitre, ce travail s'intéresse à la propriété de unsafe dans les automates FIFO. Contrairement aux ADF construits dans le chapitre \ref{ch:bases}, les automates FIFO ont un nombre potentiellement infini d'états. Dans ces conditions, il n'est pas possible d'énumérer l'ensemble des états acceptants.

Au lieu de proposer un ensemble d'état acceptants, on va fixer une propriété. Si un état respecte cette propriété, il est dit safe. Il est donc unsafe s'il ne respecte pas cette propriété de safety. Par la suite, la section \ref{ss:tracesafety} propose une technique permettant de calculer les états unsafe pour un langage de traces annotées. Ceci permet de répondre à la question de safety directement depuis ce langage au lieu de devoir construire le langage de l'automate FIFO.


\subsection{Définition}
Dans un automate FIFO \fifo, chaque état de contrôle $q\in Q$ est associé à un union finie de langage réguliers pour chacun des canaux $c\in C$.


$$\bigcup_{0 \leq i \leq n_q}\Pi_{0 \leq j \leq k}U_q(i,c_j)$$

Où $U_q(i,c_j)$ est un langage régulier pour le contenu du canal $c_j$ sur l'état $q$. $n_q$ est le nombre de langages réguliers utilisés pour définir cette propriété par union.

Un état $s=(q,[w_0,w_1,\dots,w_k])$ est \emph{unsafe} s'il existe $i,j \in \mathcal{N}$ tels que $w_j \in U_q(i,c_j)$.



\subsection{Traces annotées menant à des états unsafe}\label{ss:tracesafety}




$$
W(L)=\bigcup_{q\in Q}\big(\bigcup_{0\leq i\leq n_q}\big(\bigcap_{0\leq j \leq k}h_{c_j}^{-1}(U_q(i,c_j))\big)\big)
$$


\begin{theorem}
  Pour vérifier la propriété de safety des automates FIFO, l'algorithme LeVer respecte les propriétés suivantes :
  \begin{enumerate}
    \item Si l'algorithme retourne une réponse, celle-ci est correcte
    \item Si $AL(F)$ est régulier, la procédure s'arrête.
    \item Le nombre de test d'appartenance et d'éuivalence dépend principalement de l'algorithme d'Angluin. Le temps total est borné en temps polynomial du nombre d'états de l'automate minimal pour $AL(F)$ et linéaire en le temps pris pour une requête d'appartenance à $AL(F)$
  \end{enumerate}
\end{theorem}

La preuve est disponible en annexe du document \cite{Vardhan04} mais c'est un résultat tellement important, il est peut-être pertinent que celle-ci soit rediscutée ici.
%W(L) et tout ce qu'il faut pour le calculer
	\section{Algorithme}\label{sec:algo}%comment l'algorithme d'angluin s'adapte à la nouvelle méthode

	\chapter{Implémentation}\label{ch:impl}
	\section{Choix}\label{sec:choix}
	\section{Résultats}\label{sec:res}

	\chapter{Conclusion}\label{ch:ccl}


	\newpage
	\bibliographystyle{siam}
	\bibliography{refs.bib}

\end{document}
