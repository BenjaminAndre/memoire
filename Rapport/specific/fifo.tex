L'article \cite{Vardhan04} se concentre sur un automate plus général : l'automate FIFO. Celui-ci est Turing Complete. De la sorte, l'équipe propose une réponse pour un ensemble plus large de langage. Cette section décrit les automates FIFO.



% ██████  ███████ ███████
% ██   ██ ██      ██
% ██   ██ █████   █████
% ██   ██ ██      ██
% ██████  ███████ ██

\subsection{Définitions}\label{ss:fifodef}

\begin{definition}
  Un \emph{automate FIFO} \fifo est défini comme suit :
  \begin{itemize}
    \item $Q$ est un ensemble fini d'\emph{états de contrôle}
    \item $C$ est un ensemble fini de \emph{noms de canaux}
    \item $\Sigma$ est un alphabet
    \item $q_0 \in Q$ est l'\emph{état de contrôle initial}
    \item $\Theta$ est un ensemble fini de \emph{noms de transitions}
    \item $\delta$ est la \emph{fonction nommante}. $\delta : \Theta \rightarrow Q \times ((C \times \{?,!\} \times \Sigma) \bigcup \{\tau\}) \times Q$. Un nom de transition $\theta$ correspond à une transition de la forme $\delta(\theta)=(p,\text{"action"},q)$. Cette action a une des trois formes suivantes :
    \begin{itemize}
      \item $c!m$ : C'est une action d'envoi. Le symbole $m$ est ajouté en fin de canal $c$.
      \item $c?m$ : C'est une action de réception. Le symbole $m$ est consommé en début de canal $c$.
      \item $\tau$ : C'est une action interne. Aucun canal n'est manipulé.
    \end{itemize}
  \end{itemize}
\end{definition}

Un automate $F$ défini un \emph{système de transitions} \tsys. $\mathcal{T}$ est l'objet qui permet de passer d'un \emph{état} à un autre.

En effet, il existe les états de contrôles $q\in Q$, mais les états au sens d'un automate FIFO sont de forme $s \in S=Q\times(\Sigma^*)^C$. En particulier, un état $s=(q,w)$ avec $q\in Q$ un état de contrôle et $w\in (\Sigma^*)^C$ est un vecteur qui fait correspondre à chaque canal $c\in C$ un mot $w[c] \in \Sigma^*$ représentant le contenu de ce canal.

Dès lors, un état $s$ peut être compris comme étant composé d'un état et du contenu des différents canaux.

De plus, la \emph{fonction de transition} $\rightarrow:S\times\Theta\rightarrow S$ associe un état $s$ et un nom de transition $\theta$ à un état $s'$.

$\mathcal{T}$ respecte trois règles, correspondants chacune à un des types d'actions mentionnés précédemment. En plus de la notation $w[c]$, celles-ci utilisent la notation $w[c\mapsto c']$ signifiant $w$ à l'exception du canal $c$ dont le contenu a été remplacé par le mot $c'$.
\begin{itemize}
  \item Si $\delta(\theta)=(p,c?m,q)$ alors $(p,w)\xrightarrow{\theta}(q,w')$ si et seulement si $w=w'[c\mapsto mw'[c]]$
  \item Si $\delta(\theta)=(p,c!m,q)$ alors $(p,w)\xrightarrow{\theta}(q,w')$ si et seulement si $w'=w[c\mapsto mw[c]]$
  \item Si $\delta(\theta)=(p,\tau,q)$ alors $(p,w)\xrightarrow{\theta}(q,w')$ si et seulement si $w=w'$
\end{itemize}


% ███████ ██   ██
% ██       ██ ██
% █████     ███
% ██       ██ ██
% ███████ ██   ██


\begin{example}
  Soit un automate FIFO $F$ tel que sont système de transitions corresponde à la figure \ref{fig:fifo1}. Chaque état de l'automate correspondant à un couple état de contrôle/mot, il est imprécis de référer au système de transitions comme étant l'automate. Cependant, par abus de langage, ceux-ci seront souvent confondus dans ce document.

  \begin{figure}[H]
    \centering
    \begin{tikzpicture}[->,>=stealth',shorten >=1pt,auto,node distance=2.5cm, semithick, bend angle=10]

      \tikzstyle{every state}=[circle]

      \node[initial,state] (A)                    {$q_0$};
      \node[state]         (B) [above right= 1cm and 3 cm of A] {$q_1$};
      \node[state]         (C) [below right= 1cm and 3 cm of A] {$q_2$};
      \node[state]         (D) [above right= 1cm and 3 cm of C] {$q_3$};

      \path
      (A) edge node {$\theta_1(a!0)$} (B)
      (A) edge node[below left] {$\theta_2(a!1)$} (C)
      (B) edge node {$\theta_4(a?0)$} (D)
      (B) edge[loop above] node {$\theta_3(b!1)$} (B)
      (C) edge node[below right] {$\theta_6(a?1)$} (D)
      (C) edge [loop below] node{$\theta_5(b!0)$} (C)
      (D) edge node[above] {$\theta_7(b?0)$} (A)
      ;
    \end{tikzpicture}
    \caption{Système de transitions de l'automate FIFO $F$}\label{fig:fifo1}
  \end{figure}

  On retrouve bien la définition d'un automate fifo \fifo avec :
  \begin{itemize}
    \item $Q=\{q_0,q_1,q_2,q_3\}$
    \item $C=\{a,b\}$
    \item $\Sigma=\{0,1\}$
    \item $q_0\in Q$
    \item $\Theta=\{\theta_1, \theta_2, \theta_3, \theta_4, \theta_5, \theta_6\}$
    \item $\delta$ associant à chaque $\theta_i$ un triplet état/action/état. Celui-ci est représenté entre parenthèses à côté du nom de transition associé
  \end{itemize}

  De plus, on peut déduire le système de transition $\mathcal{T}$ défini par $F$. Considérons le mot $w=[\epsilon,\epsilon]$ où le premier élément du vecteur est le contenu du canal $a$ et le second celui du canal $b$.
  Dans cet exemple, comme $\delta(\theta_1)=(q_0,a!0,q_1)$, alors $(q_0,w)\xrightarrow{\theta_1}(q_1,w')$. Dans ce cas, $w'=[0,\epsilon]$. A ce moment, on a bien $w'=w[a\mapsto 0w[a]]$.
  En utilisant ce nouveau mot $w'$, un nouvel état est atteignable : $q_3$. En effet, comme $\delta(\theta_4)=(q_1,a?0,q_3)$, alors $(q_1,w')\xrightarrow{\theta_4}(q_4,w'')$. Dans ce cas, $w''=[\epsilon,\epsilon]$. A ce moment, on a bien $w'=w''[a\mapsto 0w''[a]]$.

  Intuitivement, la première transition $\theta_1$ ajoute le symbole $0$ en tête du canal $a$ en passant de l'état $q_0$ à l'état $q_1$. La transition $\theta_4$, elle, permet de passer de l'état $q_1$ à $q_3$ en consommant $0$ en tête du canal $a$.

\end{example}

%  ██████ ██████   ██████  ███████ ███████
% ██      ██   ██ ██    ██ ██      ██
% ██      ██████  ██    ██ ███████ ███████
% ██      ██   ██ ██    ██      ██      ██
%  ██████ ██   ██  ██████  ███████ ███████


\subsection{Produit cartésien}\label{ss:cartesien}

Par soucis de simplicité, un automate FIFO (et son système de transitions servant à le représenter) peut être représenté comme plusieurs systèmes de transitions utilisant les mêmes canaux. Le \emph{produit cartésien} entre deux automates FIFO $A$ et $B$ retourne un nouvel automate FIFO $F=A \times B$. Dès lors, il est possible de représenter un automate FIFO en se concentrant sur ses parties et en les isolant \cite{Suresh20}. Ce produit cartésien fonctionne comme suit.

Soient les automates FIFO \fifoA et \fifoB. Alors le système de transitions \tsys de l'automate FIFO $F=A\times B$ est composé de :
\begin{itemize}
  \item $S \subseteq (Q_A\times Q_B)\times (\Sigma^*)^C$ composé d'un couple d'états de contrôle de $Q_A$ et $Q_B$ et du contenu des différents canaux.
  \item $\Theta$ est un ensemble de noms de transitions.
  \item $\rightarrow$ est construit comme suit. Soit un état $((q_A,q_B), w)\in S$. Soit un triplet $(p,a,q)$ avec $p,q \in (Q_A \bigcup Q_B)$ et $a \in ((C \times \{?,!\} \times \Sigma) \bigcup \{\tau\})$.
  $((q_A,q_B),w)\xrightarrow{\theta}((q_{A'},q_{B'}),w')$ si et seulement si l'une des trois conditions suivantes est remplie.
  \begin{itemize}
    \item $\exists \theta_A\in\Theta_A, \delta_A(\theta_A)=(q_A,a,q_{A'})$ et  $(q_A,w)\xrightarrow{\theta_A}(q_{A'},w')$ dans l'automate $A$ et\\ $\exists \theta_B\in\Theta_B, \delta_B(\theta_B)=(q_B,a,q_{B'})$ et $(q_B,w)\xrightarrow{\theta_B}(q_{B'},w')$ dans l'automate $B$
    \item $\exists \theta_A\in\Theta_A, \delta_A(\theta_A)=(q_A,a,q_{A'})$ et  $(q_A,w)\xrightarrow{\theta_A}(q_{A'},w')$ dans l'automate $A$,\\
    $\forall \theta_B\in\Theta_B,\forall q \in Q_B,\delta_B(\theta_B)\neq(q_B,a,q)$ dans l'automate $B$ et $q_{B'}=q_B$
    \item $\forall \theta_A\in\Theta_A,\forall q \in Q_A,\delta_A(\theta_A)\neq(q_A,a,q)$ dans l'automate $A$ et $q_{A'}=q_A$,\\
    $\exists \theta_B\in\Theta_B, \delta_B(\theta_B)=(q_B,a,q_{B'})$ et  $(q_B,w)\xrightarrow{\theta_B}(q_{B'},w')$ dans l'automate $B$
  \end{itemize}
\end{itemize}

Le produit cartésien est un nouvel automate représenté par son système de transitions. Celui-ci étant suffisant pour déduire le langage tracé, il n'est pas necéssaire de décrire formellement \fifo.

De plus, ce nouvel automate est différent des deux autres, il n'est alors pas pertinent de prouver une égalité. Il s'agit juste d'un autre mode de représentation.

% ███████ ██   ██
% ██       ██ ██
% █████     ███
% ██       ██ ██
% ███████ ██   ██


\begin{example}
  Soient deux automates FIFO $A$ et $B$ tels que représentés par leur systèmes de transitions donnés par la figure \ref{fig:fifoAB}.

  \begin{figure}[H]
    \centering
    \begin{subfigure}{0.5\textwidth}
      \centering
      \begin{tikzpicture}[->,>=stealth',shorten >=1pt,auto,node distance=1.5cm, semithick, bend angle=10]
        \tikzstyle{every state}=[circle]

        \node[initial,state] (A)  {$q_{0}$};
        \node[state]         (B) [right=of A]  {$q_{1}$};
        \node[state]         (C) [right=of B]  {$q_{2}$};
        \path
        (A) edge node {$\theta_1(a!1)$} (B)
        (B) edge node {$\theta_2(a?1)$} (C)
        (C) edge[bend left=40] node {$\theta_3(a!0)$} (A)
        ;
      \end{tikzpicture}
      \caption{Automate FIFO A}
      \label{fig:fifoA}
      \end{subfigure}%
      \begin{subfigure}{0.5\textwidth}
        \centering
        \begin{tikzpicture}[->,>=stealth',shorten >=1pt,auto,node distance=1.5cm, semithick, bend angle=10]
          \tikzstyle{every state}=[circle]

          \node[initial,state] (A)  {$q_{A}$};
          \node[state]         (B) [right=of A]  {$q_{B}$};
          \path
          (A) edge[bend left=20] node {$\theta_5(a?0)$} (B)
          (B) edge[bend left=20] node {$\theta_6(a!0)$} (A)
          ;
        \end{tikzpicture}
        \caption{Automate FIFO B}
        \label{fig:fifoB}
      \end{subfigure}
      \caption{Automates FIFO A et B représentés par leur système de transitions}
      \label{fig:fifoAB}
    \end{figure}

   L'automate FIFO $AB=A \times B$ est représenté par son sytème de transitions à la figure \ref{fig:fifocross}.

    \begin{figure}[H]
      \centering
      \begin{tikzpicture}[->,>=stealth',shorten >=1pt,auto,node distance=1.5cm, semithick, bend angle=15]
        \tikzstyle{every state}=[circle]

        \node[initial,state] (A)  {$(q_0, q_A)$};
        \node[state]         (B) [below=of A]  {$(q_1, q_A)$};
        \node[state]         (C) [below=of B]  {$(q_2, q_A)$};

        \node[state]         (D) [right=of A] {$(q_0, q_B)$};
        \node[state]         (E) [below=of D]  {$(q_1, q_B)$};
        \node[state]         (F) [below=of E]  {$(q_2, q_B)$};

        \path
        (A) edge node {$\theta_1(a!1)$} (B)
        (A) edge[bend left] node {$\theta_2(a?0)$} (D)

        (B) edge node {$\theta_3(a?1)$} (C)
        (B) edge[bend left] node {$\theta_4(a?0)$} (E)

        (C) edge[bend left=35] node {$\theta_5(a!0)$} (A)
        (C) edge node {$\theta_6(a?0)$} (F)

        (D) edge[bend left] node {$\theta_7(a!0)$} (A)
        (D) edge node {$\theta_8(a!1)$} (E)

        (E) edge[bend left] node {$\theta_9(a!0)$} (B)
        (E) edge node {$\theta_{10}(a?1)$} (F)

        ;

        \draw [->] (F) ..  controls  ($(E)+(5cm,2cm)$) and
  ($(D)+(0cm,4cm)$).. node[right] {$\theta_{11}(a!0)$} (A);
      \end{tikzpicture}
      \caption{Automate FIFO AB résultant du produit cartésien $A\times B$}
      \label{fig:fifocross}
    \end{figure}
  \end{example}\label{ex:fifocross}


  L'exemple \ref{ex:fifocross} suffit à se convaincre qu'on peut parler indistinctement de plusieurs systèmes de transition comme représentant un seul automate. Dans ce cas, il est sous-entend que l'automate en question est celui obtenu par produit cartésien.
