La section \ref{sec:prob} défini plus clairement la problématique à l'aide des éléments du chapitre précédent. Elle explique comment les différentes sections de ce chapitre confluent vers une solution à la problématique en question.

Ensuite, la section \ref{sec:fifo} étend le concept des automates aux automates à files et défini une nouvelle opération. Une forme de langage sur ces nouveaux automates est proposée à la section \ref{sec:trace}.

Finalement, la notion de sécurité est définie dans la section \ref{sec:unsafe} avec une formule permettant de calculer les états concernés.

Tous ces éléments combinés permettent l'utilisation de LeVer, la technique proposée par \cite{Vardhan04} et d'en implémenter l'algorithme dans le chapitre \ref{ch:impl}.
