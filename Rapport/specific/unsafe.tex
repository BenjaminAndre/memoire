Comme mentionné au début du chapitre, ce travail s'intéresse à la propriété de sécurité dans les automates à files. Contrairement aux ADFs construits dans le chapitre \ref{ch:bases}, les automates à files ont un nombre potentiellement infini d'états. Dans ces conditions, il n'est pas possible d'énumérer l'ensemble des états acceptants.

Au lieu de proposer un ensemble d'état acceptants, on va fixer une propriété. Celle-ci permet de catégoriser les états qui sont souhaitables de ceux qui ne le sont pas. Par la suite, la section \ref{ss:tracesafety} propose une technique permettant de calculer l'ensemble des états indésirables à partir d'une trace annotée.


\subsection{Définitions}
Dans un automate à files \fifo, chaque état de contrôle $q\in Q$ est associé à un union finie de langage réguliers pour chacun des canaux $c\in C$.


$$\bigcup_{0 \leq i \leq n_q}\Pi_{0 \leq j \leq k}U_q(i,c_j)$$

Où $U_q(i,c_j)$ est un langage régulier pour le contenu du canal $c_j$ sur l'état $q$. $n_q$ est le nombre de langages réguliers utilisés pour définir cette propriété par union.

Un état $s=(q,[w_0,w_1,\dots,w_k])$ est \emph{sécurisé} s'il n'existe pas $i,j \in \mathbb{N}$ tels que $w_j \in U_q(i,c_j)$.

Si un état n'est pas sécurisé, il est \emph{à risque}.



\subsection{Traces annotées menant à des états à risques}\label{ss:tracesafety}

Soit la fonction $h_c:\Phi^*\rightarrow\Sigma^*$ qui, pour un trace annotée donnée, ne retourne que les messages envoyés mais non réceptionnés sur le canal $c$.

$h_c$ est l'unique homomorphisme qui étend la fonction suivante de $\Phi$ à $\Phi^*$:
$$ h_c(\theta) = \left\{ \begin{array}{ll}
      m & \text{si } \theta\in\Theta\text{ et }\delta(\theta)=(p,c!m,q)\\
      \epsilon & \text{sinon}\end{array} \right. $$



Si $L=L(F)$ alors $L_q$ est l'ensemble des mots correctements formattés pour $L$.

Si il existe un état à risque $s$, alors il existe une trace $\sigma \in \Theta^*$ telle que $s_0\xrightarrow{\sigma}s$ où $s_0$ est l'état initial. Si les transitions dénotant des actions d'envoi et de réception d'un même symbole sur un même canal sont enlevées par paires, il ne reste que les transitions participant au contenu final des différents canaux de $s$. Par définition de $h_c$, pour chaque contenu $w[c_j]$ de chaque canal $c_j$, $w[c_j]=h_{cj}(\mathcal{A}(\sigma))$. Dès lors, pour que $s$ soit atteignable, il faut qu'il existe une trace annotée $\gamma \in AL(F)$ telle que $s=(q_\gamma, [h_{c0}(\gamma),h_{c1}(\gamma),\dots,h_{ck}(\gamma)])$ où $q_\gamma$ est l'état de contrôle désigné par le symbole de contrôle à la fin de $\gamma$.

Soit la fonction $h^{-1}_{c}:\Sigma^*\rightarrow\Phi^*$ l'homomorphisme inverse de $h_c$. C'est-à-dire $h^{-1}_{cj}(U_q(i,c_j))$ retourne des traces annotées correspondant au contenu d'un canal. Dans ce cas particulier, un des langages réguliers servant à définir la propriété de sécurité. Comme un plusieurs traces annotées peuvent correspondre au même contenu de canaux par $h_c$, un contenu de canal peut correspondre à plusieurs traces annotées via $h^{-1}_c$.

En calculant cette fonction pour l'ensemble des états, canaux et langages réguliers définissant la sécurité de $F$ et en s'assurant que ces traces sont correctement formatées, on obtient un ensemble de traces menant à des états à risque.

Cet ensemble est décrit mathématiquement par $\mathcal{W}(L)$ :
$$
\mathcal{W}(L)=\bigcup_{q\in Q}\big(\bigcup_{0\leq i\leq n_q}\big(\bigcap_{0\leq j \leq k}h_{c_j}^{-1}(U_q(i,c_j))\big)\big)
$$
