\documentclass[french,letterpaper, 12pt]{article}

\usepackage[top = 1.6cm, left = 2cm, right = 2cm ]{geometry}
\usepackage[pdftex]{graphicx}
\usepackage{soulutf8}
\usepackage{amsmath}
\usepackage{tikz}
\usepackage[utf8]{inputenc}
\usepackage{longtable}
\usepackage[T1]{fontenc}
\usepackage{epigraph}
\usepackage{fancyhdr}
\usepackage{float}
\usepackage{xcolor}
\usepackage{eurosym}
\usepackage{calc}
\usepackage{hyperref}
\usepackage{multirow}
\usepackage{caption}
\usepackage[Algorithme]{algorithm}
\usepackage{algorithmic}
\usepackage{enumerate}
\usepackage[french]{babel}
\usepackage{tcolorbox}
\usepackage{multicol}
\usepackage{etoolbox,refcount}
\usepackage{listings}
\usepackage{amssymb}
\usepackage{subcaption}
\usepackage[standard,thref, framed, hyperref,standard,thmmarks]{ntheorem}
\usepackage{lmodern}
\usepackage{thmtools}
\usepackage{tikz-cd}

%
%%%%%%% Sub librairies

\usetikzlibrary{arrows,automata}
\usetikzlibrary{decorations.pathreplacing,shapes,arrows,positioning}
\usetikzlibrary{calc}
\usetikzlibrary{positioning}
\usetikzlibrary{snakes}
\usetikzlibrary{shapes,fit}


%
%%%%% Custom commands
%



\lstset{
	keywordstyle=\color{red},
	basicstyle=\scriptsize\ttfamily,
	commentstyle=\ttfamily\itshape\color{gray},
	stringstyle=\ttfamily,
	showstringspaces=false,
	breaklines=true,
	frameround=ffff,
	rulecolor=\color{black}
}

%
\def\changemargin#1#2{\list{}{\rightmargin#2\leftmargin#1}\item[]}
\let\endchangemargin=\endlist
%
\newcommand{\newlinealinea}{
	~\\ \hspace*{0.5cm}}
%
\newcommand{\alinea}{
	\hspace*{0.5cm}}
%
\newcommand{\alinealong}{
	\hspace*{1.1cm}}
%
\newcommand{\alignparagraph}{
	\hspace*{0.6cm}}
%
\newcommand{\red}[1]{
	\textcolor{red}{#1}}
%
\newcommand{\green}[1]{
	\textcolor{green}{#1}}
%
\newcommand{\point}{$\bullet\ $}
%
\makeatletter
\newcommand*{\whiten}[1]{\llap{\textcolor{white}{{\the\SOUL@token}}\hspace{#1pt}}}
\newcommand{\myul}[1]{
	\underline{\smash{#1}}
}
\makeatother
%
\setlength{\fboxsep}{2pt}
%
\DeclareMathOperator*{\argmax}{\arg\!\max}
%
%
%%%%% Custom text
%
%
\makeatletter
\@addtoreset{section}{part}
\makeatother
%
\renewcommand*\sfdefault{phv}
\renewcommand*\rmdefault{ppl}
%
\renewcommand\epigraphflush{flushright}
\renewcommand\epigraphsize{\normalsize}
\setlength\epigraphwidth{0.7\textwidth}
%
\definecolor{titlepagecolor}{cmyk}{0.24,0.92,0.78,0.25}
\definecolor{red}{cmyk}{0, 0.91, 0.91, 0.20}
%
\DeclareFixedFont{\titlefont}{T1}{phv}{\seriesdefault}{n}{0.375in}
%
%
%%%%% Header
%
%
\pagestyle{fancy}
\lhead{\student}
\rhead{\grade}
\cfoot{\thepage}
%
%
%%%%% Title page. The following code is borrowed from:
%%%%%       http://tex.stackexchange.com/a/86310/10898
%
%
\newcommand\titlepagedecoration{%
	\begin{tikzpicture}[remember picture,overlay,shorten >= -10pt]

	\coordinate (aux1) at ([yshift=-70pt]current page.north east);
	\coordinate (aux2) at ([yshift=-460pt]current page.north east);
	\coordinate (aux3) at ([xshift=-6cm]current page.north east);
	\coordinate (aux4) at ([yshift=-150pt]current page.north east);

	\begin{scope}[titlepagecolor!40,line width=12pt,rounded corners=12pt]
	\draw
	(aux1) -- coordinate (a)
	++(225:5) --
	++(-45:5.1) coordinate (b);
	\draw[shorten <= -10pt]
	(aux3) --
	(a) --
	(aux1);
	\draw[opacity=0.6,titlepagecolor,shorten <= -10pt]
	(b) --
	++(225:2.2) --
	++(-45:2.2);
	\end{scope}
	\draw[titlepagecolor,line width=8pt,rounded corners=8pt,shorten <= -10pt]
	(aux4) --
	++(225:0.8) --
	++(-45:0.8);
	\begin{scope}[titlepagecolor!70,line width=6pt,rounded corners=8pt]
	\draw[shorten <= -10pt]
	(aux2) --
	++(225:3) coordinate[pos=0.45] (c) --
	++(-45:3.1);
	\draw
	(aux2) --
	(c) --
	++(135:2.5) --
	++(45:2.5) --
	++(-45:2.5) coordinate[pos=0.3] (d);
	\draw
	(d) -- +(45:1);
	\end{scope}
	\end{tikzpicture}%
}


%  █████  ██       ██████   ██████
% ██   ██ ██      ██       ██    ██
% ███████ ██      ██   ███ ██    ██
% ██   ██ ██      ██    ██ ██    ██
% ██   ██ ███████  ██████   ██████


\renewcommand{\algorithmicrequire}{\textbf{Requis:}}
\renewcommand{\algorithmicensure}{\textbf{Promet:}}
\renewcommand{\algorithmicend}{\textbf{fin}}
\renewcommand{\algorithmicif}{\textbf{si}}
\renewcommand{\algorithmicthen}{\textbf{alors}}
\renewcommand{\algorithmicelse}{\textbf{sinon}}
\renewcommand{\algorithmicelsif}{\algorithmicelse\ \algorithmicif}
\renewcommand{\algorithmicendif}{\algorithmicend\ \algorithmicif}
\renewcommand{\algorithmicfor}{\textbf{pour}}
\renewcommand{\algorithmicforall}{\textbf{pour chaque}}
\renewcommand{\algorithmicdo}{\textbf{faire}}
\renewcommand{\algorithmicendfor}{\algorithmicend\ \algorithmicfor}
\renewcommand{\algorithmicwhile}{\textbf{tant que}}
\renewcommand{\algorithmicendwhile}{\algorithmicend\ \algorithmicwhile}
\renewcommand{\algorithmicloop}{\textbf{boucle}}
\renewcommand{\algorithmicendloop}{\algorithmicend\ \algorithmicloop}
\renewcommand{\algorithmicrepeat}{\textbf{répéter}}
\renewcommand{\algorithmicuntil}{\textbf{jusqu'à}}
\renewcommand{\algorithmicprint}{\textbf{afficher}}
\renewcommand{\algorithmicreturn}{\textbf{retourner}}
\renewcommand{\algorithmictrue}{\textbf{vrai}}
\renewcommand{\algorithmicfalse}{\textbf{faux}}

\renewcommand{\algorithmicand}{\textbf{et}}
\renewcommand{\algorithmicor}{\textbf{ou}}
%%%%%%%%%%%%%%%%%%%%%%%%%%%%%%%%%%%%%%%%%%%%%%%%%%%

\newcommand{\todo}[1]{\textcolor{red}{\emph{\textbf{TODO} : #1}}}
%%%%%%%%%%%%%%%%%%%%%%%%%%%%%%%%%%%%%%%%%%%%%%%%%%%%


\newcounter{countitems}
\newcounter{nextitemizecount}
\newcommand{\setupcountitems}{%
	\stepcounter{nextitemizecount}%
	\setcounter{countitems}{0}%
	\preto\item{\stepcounter{countitems}}%
}
\makeatletter
\newcommand{\computecountitems}{%
	\edef\@currentlabel{\number\c@countitems}%
	\label{countitems@\number\numexpr\value{nextitemizecount}-1\relax}%
}
\newcommand{\nextitemizecount}{%
	\getrefnumber{countitems@\number\c@nextitemizecount}%
}
\newcommand{\previtemizecount}{%
	\getrefnumber{countitems@\number\numexpr\value{nextitemizecount}-1\relax}%
}
\makeatother
\newenvironment{AutoMultiColItemize}{%
	\ifnumcomp{\nextitemizecount}{>}{3}{\begin{multicols}{3}}{}%
		\setupcountitems\begin{itemize}}%
		{\end{itemize}%
		\unskip\computecountitems\ifnumcomp{\previtemizecount}{>}{3}{\end{multicols}}{}}


%%%%%%%%%%%%%%%%%%%%%%%%%%%%%%%%%%%%%%%%%%%%%%%%%%%%

% ████████ ██   ██ ███    ███
%    ██    ██   ██ ████  ████
%    ██    ███████ ██ ████ ██
%    ██    ██   ██ ██  ██  ██
%    ██    ██   ██ ██      ██



\theoremstyle{plain}


\renewtheorem{theorem}{Théorème}[section]
\renewtheorem{lemma}[theorem]{Lemme}
\renewtheorem{proposition}[theorem]{Proposition}


\theoremstyle{break}

\newtheorem{algo}{Algorithme}[section]

\theoremheaderfont{\normalfont\smallskip\normalsize\itshape\bfseries}
\theorembodyfont{\normalfont\normalsize}



\renewtheorem{proof}{Preuve}[theorem]
\renewtheorem{example}{Exemple}
\renewtheorem{corollary}{Corollaire}[theorem]


\newtheorem{algoproof}{Preuve}[algo]
\newtheorem{complexity}[algoproof]{Complexité}


% ██   ██ ███████ ██      ██████
% ██   ██ ██      ██      ██   ██
% ███████ █████   ██      ██████
% ██   ██ ██      ██      ██
% ██   ██ ███████ ███████ ██



\newcommand{\hdelta}{\hat{\delta}}
\newcommand{\automaton}{$A=(Q,\Sigma, q_0, \delta, F)$\ }
\newcommand{\automatonbis}{$B=(Q_B,\Sigma_b, q_b, \delta_b, F_b)$\ }
\newcommand{\fifo}{$F=(Q,C, \Sigma, q_0, \Theta, \delta)$\ }
\newcommand{\tsys}{$\mathcal{T}=(S,\Theta, \rightarrow)$\ }

\newcommand{\re}{$R_E$\ }
\newcommand{\rb}{$R_B$\ }
\newcommand{\rl}{$R_L$\ }
\newcommand{\ro}{$R_O$\ }

%%%%%%%%%%%%%%%%%%%%%%%%%%%%%%%%%%%%%%%%%%%%%%%%%%%%
