\documentclass[french,letterpaper, 12pt]{article}
\usepackage[top = 1.6cm, left = 2cm, right = 2cm ]{geometry}
\usepackage[pdftex]{graphicx}
\usepackage{soulutf8}
\usepackage{amsmath}
\usepackage{tikz}
\usepackage[utf8]{inputenc}
\usepackage{longtable}
\usepackage[T1]{fontenc}
\usepackage{epigraph}
\usepackage{fancyhdr}
\usepackage{float}
\usepackage{subfig}
\usepackage{xcolor}
\usepackage{eurosym}
\usepackage{calc}
\usepackage{hyperref}
\usepackage{multirow}
\usepackage{caption}
\usepackage{amsmath}
\usepackage[Algorithme]{algorithm}
\usepackage{algorithmic}
\usepackage{enumerate}
\usepackage[french]{babel}
\usepackage{tcolorbox}
\usepackage{multicol}
\usepackage{etoolbox,refcount}
\usepackage{listings}
%
%%%%%%% Sub librairies

\usetikzlibrary{arrows,automata}
\usetikzlibrary{decorations.pathreplacing,shapes,arrows,positioning}
\usetikzlibrary{calc} 
\usetikzlibrary{positioning}

%
%%%%% Custom commands
%



\lstset{
	keywordstyle=\color{red},
	basicstyle=\scriptsize\ttfamily,
	commentstyle=\ttfamily\itshape\color{gray},
	stringstyle=\ttfamily,
	showstringspaces=false,
	breaklines=true,
	frameround=ffff,
	rulecolor=\color{black}
}

%
\def\changemargin#1#2{\list{}{\rightmargin#2\leftmargin#1}\item[]}
\let\endchangemargin=\endlist 
%
\newcommand{\newlinealinea}{
	~\\ \hspace*{0.5cm}}
%
\newcommand{\alinea}{
	\hspace*{0.5cm}}
%
\newcommand{\alinealong}{
	\hspace*{1.1cm}}
%
\newcommand{\alignparagraph}{
	\hspace*{0.6cm}}
%
\newcommand{\red}[1]{
	\textcolor{red}{#1}}
%
\newcommand{\green}[1]{
	\textcolor{green}{#1}}
%
\newcommand{\point}{$\bullet\ $}
%
\makeatletter
\newcommand*{\whiten}[1]{\llap{\textcolor{white}{{\the\SOUL@token}}\hspace{#1pt}}}
\newcommand{\myul}[1]{
	\underline{\smash{#1}}
}
\makeatother
%
\setlength{\fboxsep}{2pt}
%
\DeclareMathOperator*{\argmax}{\arg\!\max}
%
%
%%%%% Custom text
%
%
\makeatletter
\@addtoreset{section}{part}
\makeatother  
%
\renewcommand*\sfdefault{phv}
\renewcommand*\rmdefault{ppl}
%
\renewcommand\epigraphflush{flushright}
\renewcommand\epigraphsize{\normalsize}
\setlength\epigraphwidth{0.7\textwidth}
%
\definecolor{titlepagecolor}{cmyk}{0.24,0.92,0.78,0.25}
\definecolor{red}{cmyk}{0, 0.91, 0.91, 0.20}
%
\DeclareFixedFont{\titlefont}{T1}{phv}{\seriesdefault}{n}{0.375in}
%
%
%%%%% Header
%
%
\pagestyle{fancy}
\lhead{Benjamin André}
\rhead{MAB2 Sciences Informatiques}
\cfoot{\thepage}
%
%
%%%%% Title page. The following code is borrowed from: 
%%%%%       http://tex.stackexchange.com/a/86310/10898
%
%
\newcommand\titlepagedecoration{%
	\begin{tikzpicture}[remember picture,overlay,shorten >= -10pt]
	
	\coordinate (aux1) at ([yshift=-70pt]current page.north east);
	\coordinate (aux2) at ([yshift=-460pt]current page.north east);
	\coordinate (aux3) at ([xshift=-6cm]current page.north east);
	\coordinate (aux4) at ([yshift=-150pt]current page.north east);
	
	\begin{scope}[titlepagecolor!40,line width=12pt,rounded corners=12pt]
	\draw
	(aux1) -- coordinate (a)
	++(225:5) --
	++(-45:5.1) coordinate (b);
	\draw[shorten <= -10pt]
	(aux3) --
	(a) --
	(aux1);
	\draw[opacity=0.6,titlepagecolor,shorten <= -10pt]
	(b) --
	++(225:2.2) --
	++(-45:2.2);
	\end{scope}
	\draw[titlepagecolor,line width=8pt,rounded corners=8pt,shorten <= -10pt]
	(aux4) --
	++(225:0.8) --
	++(-45:0.8);
	\begin{scope}[titlepagecolor!70,line width=6pt,rounded corners=8pt]
	\draw[shorten <= -10pt]
	(aux2) --
	++(225:3) coordinate[pos=0.45] (c) --
	++(-45:3.1);
	\draw
	(aux2) --
	(c) --
	++(135:2.5) --
	++(45:2.5) --
	++(-45:2.5) coordinate[pos=0.3] (d);   
	\draw 
	(d) -- +(45:1);
	\end{scope}
	\end{tikzpicture}%
}
%%%%%%%%%%%%%%%%%%   ALGORITHM %%%%%%%%%%%%%%%%%%%%%

\renewcommand{\algorithmicrequire}{\textbf{Requis:}}
\renewcommand{\algorithmicensure}{\textbf{Promet:}}
\renewcommand{\algorithmicend}{\textbf{fin}}
\renewcommand{\algorithmicif}{\textbf{si}}
\renewcommand{\algorithmicthen}{\textbf{alors}}
\renewcommand{\algorithmicelse}{\textbf{sinon}}
\renewcommand{\algorithmicelsif}{\algorithmicelse\ \algorithmicif}
\renewcommand{\algorithmicendif}{\algorithmicend\ \algorithmicif}
\renewcommand{\algorithmicfor}{\textbf{pour}}
\renewcommand{\algorithmicforall}{\textbf{pour chaque}}
\renewcommand{\algorithmicdo}{\textbf{faire}}
\renewcommand{\algorithmicendfor}{\algorithmicend\ \algorithmicfor}
\renewcommand{\algorithmicwhile}{\textbf{tant que}}
\renewcommand{\algorithmicendwhile}{\algorithmicend\ \algorithmicwhile}
\renewcommand{\algorithmicloop}{\textbf{boucle}}
\renewcommand{\algorithmicendloop}{\algorithmicend\ \algorithmicloop}
\renewcommand{\algorithmicrepeat}{\textbf{répéter}}
\renewcommand{\algorithmicuntil}{\textbf{jusqu'à}}
\renewcommand{\algorithmicprint}{\textbf{afficher}}
\renewcommand{\algorithmicreturn}{\textbf{retourner}}
\renewcommand{\algorithmictrue}{\textbf{vrai}}
\renewcommand{\algorithmicfalse}{\textbf{faux}}

\renewcommand{\algorithmicand}{\textbf{et}}
\renewcommand{\algorithmicor}{\textbf{ou}}


%------------- MY ENVS --------%
\newcounter{note}[section]
\newenvironment{note}[1][]
{	
	
	\refstepcounter{note}
	
	
	
	\begin{center}
		\begin{tcolorbox}[note]
			\includegraphics[width=0.5cm]{/home/benjamin/.config/texstudio/templates/user/note}
			\textbf{\textcolor{lime!30!black}{Note \arabic{section}.\arabic{note}}} :
			\begin{em}
			}
			{
			\end{em}
		\end{tcolorbox}	
	\end{center}
}



\newcounter{question}[section]
\newenvironment{question}[1][]
{	
	
	\refstepcounter{question}
	
	
	
	\begin{center}
		\begin{tcolorbox}[question]
			\includegraphics[width=0.5cm]{/home/benjamin/.config/texstudio/templates/user/question}
			\textbf{\textcolor{teal}{Question \arabic{section}.\arabic{question}}} :
		}
		{
		\end{tcolorbox}	
	\end{center}
}


\newcounter{attention}[section]
\newenvironment{attention}[1][]
{	
	
	\refstepcounter{attention}
	
	
	
	\begin{center}
		\begin{tcolorbox}[attention]
			\includegraphics[width=0.5cm]{/home/benjamin/.config/texstudio/templates/user/attention}
			\textbf{\textcolor{red}{Attention \arabic{section}.\arabic{attention}}} :
			\begin{bf}
			}
			{
			\end{bf}
		\end{tcolorbox}	
	\end{center}
}

\newcommand{\todo}[1]{\textcolor{red}{\emph{\textbf{TODO} : #1}}}
%%%%%%%%%%%%%%%%%%%%%%%%%%%%%%%%%%%%%%%%%%%%%%%%%%%%


\newcounter{countitems}
\newcounter{nextitemizecount}
\newcommand{\setupcountitems}{%
	\stepcounter{nextitemizecount}%
	\setcounter{countitems}{0}%
	\preto\item{\stepcounter{countitems}}%
}
\makeatletter
\newcommand{\computecountitems}{%
	\edef\@currentlabel{\number\c@countitems}%
	\label{countitems@\number\numexpr\value{nextitemizecount}-1\relax}%
}
\newcommand{\nextitemizecount}{%
	\getrefnumber{countitems@\number\c@nextitemizecount}%
}
\newcommand{\previtemizecount}{%
	\getrefnumber{countitems@\number\numexpr\value{nextitemizecount}-1\relax}%
}
\makeatother    
\newenvironment{AutoMultiColItemize}{%
	\ifnumcomp{\nextitemizecount}{>}{3}{\begin{multicols}{3}}{}%
		\setupcountitems\begin{itemize}}%
		{\end{itemize}%
		\unskip\computecountitems\ifnumcomp{\previtemizecount}{>}{3}{\end{multicols}}{}}


%%%%%%%%%%%%%%%%%%%%%%%%%%%%%%%%%%%%%%%%%%%%%%%%%%%%

\newtheorem{theorem}{Théorème}[section]
\newtheorem{corollary}{Corrolaire}[theorem]
\newtheorem{lemma}[theorem]{Lemme}
\newtheorem{proof}{Preuve}[theorem]


%%%%%%%%%%%%%%%%%%%%%%%%%%%%%%%%%%%%%%%%%%%%%%%%%%%%
\begin{document}
	
	\begin{titlepage}
		%
		\noindent
		%
		\newgeometry{bottom = 2cm, top = 2.5cm}
		\begin{center}
			\includegraphics[scale=1.2]{res/UMONS}\\
			\vspace*{0.3cm}
			\includegraphics[scale=0.23]{res/FS_Logo}\\
			\vspace*{2.5cm}
			%
			\titlefont Automates \par
			%
		\end{center}
		\vspace*{3cm}
		\hfill
		%
		\begin{minipage}{0.18\linewidth}
			\begin{flushright}
				\rule{0.5pt}{50pt}
			\end{flushright}
		\end{minipage}
		%
		\begin{minipage}{0.8\linewidth}
			\begin{flushleft}
				\textsf{\textbf{Étudiant:}} Benjamin André\\
				\textsf{\textbf{Directrice:}} Véronique Bruyère\\
				\today
			\end{flushleft}
		\end{minipage}
		%
		\vspace*{\fill}                                                             
		%
		\begin{center}
			Faculté des Sciences $\bullet$ Université de Mons $\bullet$ 
			Place du Parc 20 $\bullet$ B-7000 Mons
		\end{center}
		%
		\titlepagedecoration
		%
	\end{titlepage}
	%
	%
	%%%% Tables des matières
	%
	%
	\newgeometry{top = 3cm, left = 2cm, right = 2cm, bottom=2.5cm}
	\cleardoublepage
	\tableofcontents
	\newpage
	
	
	\section{Automates utilisés}\label{sec:automatons}
	
	\begin{figure}[H]
		\centering
		\begin{tikzpicture}[->,>=stealth',shorten >=1pt,auto,node distance=3cm, semithick, bend angle=10]
		
		\tikzstyle{every state}=[circle]
		
		\node[initial,state] (A)                    {$q_a$};
		\node[state]         (B) [below right of=A] {$q_b$};
		\node[state]         (C) [below left of=A] {$q_c$};
		\node[accepting, state]         (D) [below right of=B] {$q_d$};
		\node[state]         (E) [below left of=C]       {$q_e$};
		\node[state]         (G) [below right of=E]       {$q_g$};
		\node[state]         (F) [above right of=G]       {$q_f$};
		
		\path 	(A) 	edge              node {0} (C)
		edge              node {1} (B)
		(B) 	edge              node {0} (D)
		edge [bend left]  node {1} (F)
		(C) 	edge              node {0} (E)
		edge              node {1} (F)
		(D) 	edge [loop above] node {0,1} (D)
		(E) 	edge [loop above] node {0,1} (E)
		(F) 	edge              node {0} (D)
		edge [bend left]  node {1} (B)
		(G) 	edge              node {0} (E)
		edge              node {1} (F);
		\end{tikzpicture}
		\caption{Automate $A_B$, exemple personnel}
	\end{figure}

	
	\begin{figure}[H]
		\centering
		\begin{tikzpicture}[->,>=stealth',shorten >=1pt,auto,node distance=2cm and 5cm, semithick, bend angle=10]
		
		\tikzstyle{every state}=[circle]
		
		\node[initial,state]	(A)					{$q_0$};
		\node[state]			(B)	[right= of A]	{$q_1$};
		\node[accepting,state]	(C) [below of=A]	{$q_2$};
		\node[accepting,state]	(D)	[below of=B]	{$q_3$};
		\node[accepting,state]	(E)	[below of=C]	{$q_4$};
		\node[state]			(F)	[below of=D]	{$q_5$};
		
		\path
		(A)	edge	[bend left]		node{0}		(B)
		(A)	edge					node{1}		(C)
		(B) edge	[bend left]		node{0}		(A)
		(B) edge					node{1}		(D)
		(C)	edge					node{0}		(E)
		(C)	edge					node[near start]{1}		(F)
		(D)	edge					node[near start, above]{0}		(E)
		(D)	edge					node{1}		(F)
		(E)	edge	[loop below]	node{0,1}	(E)
		(F)	edge	[loop below]	node{0,1}	(F)
		
		; 
		\end{tikzpicture}
		\caption{Automate $A_H$, exemple d'un livre de référence\cite{Hopcroft79}}
	\end{figure}
	
	
	\begin{figure}[H]
		\centering
		\begin{tikzpicture}[->,>=stealth',shorten >=1pt,auto,node distance=2cm and 2cm, semithick, bend angle=10]
		
		\tikzstyle{every state}=[circle]
		
		\node[initial,state]	(A)					{$q_0$};
		\node[accepting,state]	(B)	[right= of A]	{$q_1$};
		\node[state]			(C) [right= of B]	{$q_2$};
		
		\path
		(A)	edge	[bend left]		node{b}		(B)
		(A)	edge	[loop above]	node{a}		(A)
		(B) edge	[bend left]		node{a}		(A)
		(B) edge					node{b}		(C)
		(C)	edge	[loop above]	node{a,b}	(C)
		
		; 
		\end{tikzpicture}
		\caption{Automate $A_N$, exemple d'une thèse\cite{Neider14}}
	\end{figure}
	
	\section{Bases théoriques}\label{sec:theorie}
	
	\subsection{DFA}\label{sub:dfa}
	Soit un ensemble de symboles $\Sigma$. Soient $\Sigma^* = \{ a_1a_2a_3...a_n | a_1,a_2,a_3,...,a_n \in \Sigma \}$, l'ensemble des mots de taille arbitraire qu'il est possible de former à partir de $\Sigma$ et $|w|, w \in \Sigma$ la longueur de $w$, le nombre de symboles utilisés. Si $|w|=0$, on note $w=\epsilon$.
	
	
	Un automate est défini par $A = (Q, \Sigma, q_0, \delta, F)$ où
	\begin{itemize}
		\item $Q$ est un ensemble d'états, différenciés par leur indice $q_1, q_2, ..., q_n$ ou $n = |Q|$.
		\item $\Sigma$ est un ensemble de symboles
		\item $q_0 \in Q$ est l'état initial
		\item $\delta : Q x \Sigma \rightarrow Q$ est la fonction de transition. A partir d'un état de $Q$, en fonction d'un symbole, elle retourne un nouvel état faisant partie de $Q$.
		\item $F \subseteq Q$ est un ensemble d'état finaux.
	\end{itemize}
	 
	 A définir
	 \begin{itemize}
	 	\item Accepter un langage
	 	\item Congruence à droite
	 \end{itemize}
	 
	\subsection{Théorème de Myhill-Nerode}
	
	\begin{theorem}
		Les 3 énoncés suivants sont équivalents :
		\begin{enumerate}
			\item Un langage $L\subseteq\Sigma^*$ est accepté par un DFA
			\item $L$ est l'union de certaines classes d'équivalence d'index fini respectant une relation d'équivalence et de congruence à droite
			\item Soit la relation d'équivalence $R_L : xR_Ly \Leftrightarrow \forall z \in \Sigma^*, xz \in L \Leftrightarrow yz \in L$. $R_L$ est d'index fini.
		\end{enumerate}
	\end{theorem}
	
	\begin{proof}La preuve d'équivalence se fait en prouvant chaque implication de façon cyclique :\\
		
		$(1)\rightarrow(2)$ Soit un langage $L \subseteq \Sigma^*$ qui est accepté par un automate déterministe fini (ADF) $A$. Soit la relation $R_A : xR_Ay \Leftrightarrow \delta(q_0, x) = \delta(q_0, y)$ qui détermine si deux mots, une fois parcourus dans l'automates, finissent sur le même état.
		
		C'est une relation d'équivalence (réflexive, transitive et symétrique), et congruente à droite :
		
		\begin{itemize}
			\item \textbf{Réflexivité :} Soit le mot $x \in \Sigma^*$. Alors, par définition, $xR_Mx \Leftrightarrow \delta(q_0, x) = \delta(q_0, x)$.
			\item \textbf{Transitivité :} Soient les mots $x,y,z \in \Sigma^*$ tels que $xR_My$ et $yR_Mz$. Alors, $\delta(q_0, x) = \delta(q_0, y) = \delta(q_0, z)$ par la transitivité de l'égalité. Dès lors, $xR_Mz$
			\item \textbf{Symétrie : } Soient les mots $x,y \in \Sigma^*$ tels que $xR_My$. Comme $\delta(q_0, x) = \delta(q_0, y)$, $\delta(q_0, y) = \delta(q_0, x)$. Donc, $yR_Mx$.
			\item \textbf{Congruence à droite :} Soient les mots $x,y \in \Sigma^*$ tels que $xR_My$. Soit un mot $w \in \Sigma^*$. $\delta(q_0, xw) = \delta(\delta(q_0,x), w) = \delta(\delta(q_0,y), w) = \delta(q_0, yw)$.
		\end{itemize}
		
		Il peut au plus y avoir une classe d'équivalence par état (valeurs possibles retournées par $\delta$). Ce nombre d'état étant fini dans $M$, $R_M$ est donc d'index fini. Le langage $L$ correspond aux mots menant à un état appartenant à $F$, et $F$ peut être  écrit comme une union d'état, qui correspondent à une classe d'équivalence de $R_M$ qui est bien d'index fini et respectant la congruence à droite.
		
		
		$(2)\rightarrow(3)$ toute relation E de 2 est un refinement de RL du coup chaque c.eq est completement contenue dans une c.Eq de RL. on part de xRMy, cong droite
		
		$(3)\rightarrow(1)$ Mq RL cong droite xRLy, utiliser définitions 
	\end{proof}


	\begin{corollary}
		Possibilité de créer l'automate canonique...
	\end{corollary}

	 
	\subsection{Table Filling Algorithm}
	
	\begin{figure}[H]
		\centering
		\begin{tikzpicture}[->,>=stealth',shorten >=1pt,auto,node distance=3cm, semithick, bend angle=10]
		
		\tikzstyle{every state}=[circle]
		
		\node[initial,state] (A)                    {$q_a$};
		\node[state]         (B) [below right of=A] {$q_b$};
		\node[state]         (C) [below left of=A] {$q_c$};
		\node[accepting, state]         (D) [below right of=B] {$q_d$};
		\node[state]         (E) [below left of=C]       {$q_e$};
		\node[state]         (G) [below right of=E]       {$q_g$};
		\node[state]         (F) [above right of=G]       {$q_f$};
		
		\path 	(A) 	edge              node {0} (C)
						edge              node {1} (B)
				(B) 	edge              node {0} (D)
						edge [bend left]  node {1} (F)
				(C) 	edge              node {0} (E)
						edge              node {1} (F)
				(D) 	edge [loop above] node {0,1} (D)
				(E) 	edge [loop above] node {0,1} (E)
				(F) 	edge              node {0} (D)
						edge [bend left]  node {1} (B)
				(G) 	edge              node {0} (E)
						edge              node {1} (F);
		\end{tikzpicture}
		\caption{Automate $A_1$}
	\end{figure}
	
		
	L'état $q_g$ n'est pas atteignable : il peut être simplement supprimé.
		
	\begin{figure}[H]
		\centering
		\begin{tikzpicture}[->,>=stealth',shorten >=1pt,auto,node distance=3cm, semithick, bend angle=10]
		
		\tikzstyle{every state}=[circle]
		
		\node[initial,state] (A)                    {$q_a$};
		\node[state]         (B) [below right of=A] {$q_b$};
		\node[state]         (C) [below left of=A] {$q_c$};
		\node[accepting,state]         (D) [below right of=B] {$q_d$};
		\node[state]         (E) [below left of=C]       {$q_e$};
		\node[state]         (F) [below right of=C]       {$q_f$};
		
		\path 	(A) 	edge              node {0} (C)
		edge              node {1} (B)
		(B) 	edge              node {0} (D)
		edge [bend left]  node {1} (F)
		(C) 	edge              node {0} (E)
		edge              node {1} (F)
		(D) 	edge [loop above] node {0,1} (D)
		(E) 	edge [loop above] node {0,1} (E)
		(F) 	edge              node {0} (D)
		edge [bend left]  node {1} (B);
		\end{tikzpicture}
		\caption{Automate $A_2$}
	\end{figure}
	
	
	\begin{figure}
		\centering
		\begin{tabular}{ccccccc}
			\cline{2-2}
			\multicolumn{1}{c|}{B} & \multicolumn{1}{c|}{x} &&&&\\
			\cline{2-3}
			\multicolumn{1}{c|}{C} & \multicolumn{1}{c|}{x} &\multicolumn{1}{c|}{x}&&&\\
			\cline{2-4}
			\multicolumn{1}{c|}{D} & \multicolumn{1}{c|}{x} &\multicolumn{1}{c|}{x}&\multicolumn{1}{c|}{x}&&\\
			\cline{2-5}
			\multicolumn{1}{c|}{E} & \multicolumn{1}{c|}{x} &\multicolumn{1}{c|}{x}&\multicolumn{1}{c|}{x}&\multicolumn{1}{c|}{x}&\\
			\cline{2-6}
			\multicolumn{1}{c|}{F} & \multicolumn{1}{c|}{x} & \multicolumn{1}{c|}{}&\multicolumn{1}{c|}{x}&\multicolumn{1}{c|}{x}&\multicolumn{1}{c|}{x}\\
			\cline{2-6}
			\multicolumn{1}{c}{} & A&B&C&D&E\\
			
		\end{tabular}
		\caption{Table filling pour $A_2$, décelant des équivalences d'états}
	\end{figure}
	
	Par l'algorithme de minimisation, on obtient $A_3$. De cet automate, on peut déduire une écriture de $L$ sous forme d'expression régulière : $(1|01)1^*0(0|1)^*$
		
	\begin{figure}[H]
		\centering
		\begin{tikzpicture}[->,>=stealth',shorten >=1pt,auto,node distance=3cm, semithick, bend angle=10]
		
		\tikzstyle{every state}=[circle]
		
		\node[initial,state] (A)                    {$q_a$};
		\node[state]         (B) [below right of=A] {$q_b$};
		\node[state]         (C) [below left of=A] {$q_c$};
		\node[accepting, state]         (D) [below right of=B] {$q_d$};
		\node[state]         (E) [below left of=C]       {$q_e$};
		
		\path 	
		(A) 	edge              node {0} (C)
				edge              node {1} (B)
		(B) 	edge              node {0} (D)
				edge [loop above] node {1} (B)
		(C) 	edge              node {0} (E)
				edge              node {1} (B)
		(D) 	edge [loop above] node {0,1} (D)
		(E) 	edge [loop above] node {0,1} (E);
		\end{tikzpicture}
		\caption{Automate $A_3$}
	\end{figure}

	\section{Velleda}\label{sec:velleda}
	Complexité 4.4.2, bad pair pq, rs q et q' vont sur un meme p Attention a bien comprendre le cas de base, utilisation du mot témoin w qui différencie. La contradiction est sur la table pas sur w le plus petit (c'est un élemenet qu'on a introduit, ça nous avance à rien de le contredire)

	Fig 4.13 + Exemple + exemple preuve éq minimaux
	
	A rédiger, DFA/Notations, Rédiger, Avec des exemples persos
	
	Prouver Reflexif, transitif, symetrique
	
	Lire 4.23
	
	Important : la notion de congruence à droite
	
	Angluin : créer un contre-exemple
	
	4.24.3 tout état d'une classe S a une transition vers un état de la classe T par construction de la table
	
	
	\newpage
	\bibliographystyle{siam}
	\bibliography{refs.bib}
	
\end{document}