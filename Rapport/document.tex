\documentclass[french,letterpaper, 12pt]{article}

\usepackage[top = 1.6cm, left = 2cm, right = 2cm ]{geometry}
\usepackage[pdftex]{graphicx}
\usepackage{soulutf8}
\usepackage{amsmath}
\usepackage{tikz}
\usepackage[utf8]{inputenc}
\usepackage{longtable}
\usepackage[T1]{fontenc}
\usepackage{epigraph}
\usepackage{fancyhdr}
\usepackage{float}
\usepackage{xcolor}
\usepackage{eurosym}
\usepackage{calc}
\usepackage{hyperref}
\usepackage{multirow}
\usepackage{caption}
\usepackage[Algorithme]{algorithm}
\usepackage{algorithmic}
\usepackage{enumerate}
\usepackage[french]{babel}
\usepackage{tcolorbox}
\usepackage{multicol}
\usepackage{etoolbox,refcount}
\usepackage{listings}
\usepackage{amssymb}
\usepackage{subcaption}
\usepackage[standard,thref, framed, hyperref,standard,thmmarks]{ntheorem}
\usepackage{lmodern}
\usepackage{thmtools}
\usepackage{tikz-cd}

%
%%%%%%% Sub librairies

\usetikzlibrary{arrows,automata}
\usetikzlibrary{decorations.pathreplacing,shapes,arrows,positioning}
\usetikzlibrary{calc}
\usetikzlibrary{positioning}
\usetikzlibrary{snakes}
\usetikzlibrary{shapes,fit}
\usetikzlibrary{positioning}
\usetikzlibrary{automata,arrows,trees,positioning,shapes,calc}

%
%%%%% Custom commands
%



\lstset{
	keywordstyle=\color{red},
	basicstyle=\scriptsize\ttfamily,
	commentstyle=\ttfamily\itshape\color{gray},
	stringstyle=\ttfamily,
	showstringspaces=false,
	breaklines=true,
	frameround=ffff,
	rulecolor=\color{black}
}

%
\def\changemargin#1#2{\list{}{\rightmargin#2\leftmargin#1}\item[]}
\let\endchangemargin=\endlist
%
\newcommand{\newlinealinea}{
	~\\ \hspace*{0.5cm}}
%
\newcommand{\alinea}{
	\hspace*{0.5cm}}
%
\newcommand{\alinealong}{
	\hspace*{1.1cm}}
%
\newcommand{\alignparagraph}{
	\hspace*{0.6cm}}
%
\newcommand{\red}[1]{
	\textcolor{red}{#1}}
%
\newcommand{\green}[1]{
	\textcolor{green}{#1}}
%
\newcommand{\point}{$\bullet\ $}
%
\makeatletter
\newcommand*{\whiten}[1]{\llap{\textcolor{white}{{\the\SOUL@token}}\hspace{#1pt}}}
\newcommand{\myul}[1]{
	\underline{\smash{#1}}
}
\makeatother
%
\setlength{\fboxsep}{2pt}
%
\DeclareMathOperator*{\argmax}{\arg\!\max}
%
%
%%%%% Custom text
%
%
\makeatletter
\@addtoreset{section}{part}
\makeatother
%
\renewcommand*\sfdefault{phv}
\renewcommand*\rmdefault{ppl}
%
\renewcommand\epigraphflush{flushright}
\renewcommand\epigraphsize{\normalsize}
\setlength\epigraphwidth{0.7\textwidth}
%
\definecolor{titlepagecolor}{cmyk}{0.24,0.92,0.78,0.25}
\definecolor{red}{cmyk}{0, 0.91, 0.91, 0.20}
%
\DeclareFixedFont{\titlefont}{T1}{phv}{\seriesdefault}{n}{0.375in}
%
%
%%%%% Header
%
%
\pagestyle{fancy}
\lhead{\student}
\rhead{\grade}
\cfoot{\thepage}
%
%
%%%%% Title page. The following code is borrowed from:
%%%%%       http://tex.stackexchange.com/a/86310/10898
%
%
\newcommand\titlepagedecoration{%
	\begin{tikzpicture}[remember picture,overlay,shorten >= -10pt]

	\coordinate (aux1) at ([yshift=-70pt]current page.north east);
	\coordinate (aux2) at ([yshift=-460pt]current page.north east);
	\coordinate (aux3) at ([xshift=-6cm]current page.north east);
	\coordinate (aux4) at ([yshift=-150pt]current page.north east);

	\begin{scope}[titlepagecolor!40,line width=12pt,rounded corners=12pt]
	\draw
	(aux1) -- coordinate (a)
	++(225:5) --
	++(-45:5.1) coordinate (b);
	\draw[shorten <= -10pt]
	(aux3) --
	(a) --
	(aux1);
	\draw[opacity=0.6,titlepagecolor,shorten <= -10pt]
	(b) --
	++(225:2.2) --
	++(-45:2.2);
	\end{scope}
	\draw[titlepagecolor,line width=8pt,rounded corners=8pt,shorten <= -10pt]
	(aux4) --
	++(225:0.8) --
	++(-45:0.8);
	\begin{scope}[titlepagecolor!70,line width=6pt,rounded corners=8pt]
	\draw[shorten <= -10pt]
	(aux2) --
	++(225:3) coordinate[pos=0.45] (c) --
	++(-45:3.1);
	\draw
	(aux2) --
	(c) --
	++(135:2.5) --
	++(45:2.5) --
	++(-45:2.5) coordinate[pos=0.3] (d);
	\draw
	(d) -- +(45:1);
	\end{scope}
	\end{tikzpicture}%
}


%  █████  ██       ██████   ██████
% ██   ██ ██      ██       ██    ██
% ███████ ██      ██   ███ ██    ██
% ██   ██ ██      ██    ██ ██    ██
% ██   ██ ███████  ██████   ██████


\renewcommand{\algorithmicrequire}{\textbf{Requis:}}
\renewcommand{\algorithmicensure}{\textbf{Promet:}}
\renewcommand{\algorithmicend}{\textbf{fin}}
\renewcommand{\algorithmicif}{\textbf{si}}
\renewcommand{\algorithmicthen}{\textbf{alors}}
\renewcommand{\algorithmicelse}{\textbf{sinon}}
\renewcommand{\algorithmicelsif}{\algorithmicelse\ \algorithmicif}
\renewcommand{\algorithmicendif}{\algorithmicend\ \algorithmicif}
\renewcommand{\algorithmicfor}{\textbf{pour}}
\renewcommand{\algorithmicforall}{\textbf{pour chaque}}
\renewcommand{\algorithmicdo}{\textbf{faire}}
\renewcommand{\algorithmicendfor}{\algorithmicend\ \algorithmicfor}
\renewcommand{\algorithmicwhile}{\textbf{tant que}}
\renewcommand{\algorithmicendwhile}{\algorithmicend\ \algorithmicwhile}
\renewcommand{\algorithmicloop}{\textbf{boucle}}
\renewcommand{\algorithmicendloop}{\algorithmicend\ \algorithmicloop}
\renewcommand{\algorithmicrepeat}{\textbf{répéter}}
\renewcommand{\algorithmicuntil}{\textbf{jusqu'à}}
\renewcommand{\algorithmicprint}{\textbf{afficher}}
\renewcommand{\algorithmicreturn}{\textbf{retourner}}
\renewcommand{\algorithmictrue}{\textbf{vrai}}
\renewcommand{\algorithmicfalse}{\textbf{faux}}

\renewcommand{\algorithmicand}{\textbf{et}}
\renewcommand{\algorithmicor}{\textbf{ou}}
%%%%%%%%%%%%%%%%%%%%%%%%%%%%%%%%%%%%%%%%%%%%%%%%%%%

\newcommand{\todo}[1]{\textcolor{red}{\emph{\textbf{TODO} : #1}}}
%%%%%%%%%%%%%%%%%%%%%%%%%%%%%%%%%%%%%%%%%%%%%%%%%%%%


\newcounter{countitems}
\newcounter{nextitemizecount}
\newcommand{\setupcountitems}{%
	\stepcounter{nextitemizecount}%
	\setcounter{countitems}{0}%
	\preto\item{\stepcounter{countitems}}%
}
\makeatletter
\newcommand{\computecountitems}{%
	\edef\@currentlabel{\number\c@countitems}%
	\label{countitems@\number\numexpr\value{nextitemizecount}-1\relax}%
}
\newcommand{\nextitemizecount}{%
	\getrefnumber{countitems@\number\c@nextitemizecount}%
}
\newcommand{\previtemizecount}{%
	\getrefnumber{countitems@\number\numexpr\value{nextitemizecount}-1\relax}%
}
\makeatother
\newenvironment{AutoMultiColItemize}{%
	\ifnumcomp{\nextitemizecount}{>}{3}{\begin{multicols}{3}}{}%
		\setupcountitems\begin{itemize}}%
		{\end{itemize}%
		\unskip\computecountitems\ifnumcomp{\previtemizecount}{>}{3}{\end{multicols}}{}}


%%%%%%%%%%%%%%%%%%%%%%%%%%%%%%%%%%%%%%%%%%%%%%%%%%%%

% ████████ ██   ██ ███    ███
%    ██    ██   ██ ████  ████
%    ██    ███████ ██ ████ ██
%    ██    ██   ██ ██  ██  ██
%    ██    ██   ██ ██      ██



\theoremstyle{plain}


\renewtheorem{theorem}{Théorème}[section]
\renewtheorem{lemma}[theorem]{Lemme}
\renewtheorem{proposition}[theorem]{Proposition}


\theoremstyle{break}

\newtheorem{algo}{Algorithme}[section]

\theoremheaderfont{\normalfont\smallskip\normalsize\itshape\bfseries}
\theorembodyfont{\normalfont\normalsize}



\renewtheorem{proof}{Preuve}[theorem]
\renewtheorem{example}{Exemple}
\renewtheorem{corollary}{Corollaire}[theorem]


\newtheorem{algoproof}{Preuve}[algo]
\newtheorem{complexity}[algoproof]{Complexité}


% ██   ██ ███████ ██      ██████
% ██   ██ ██      ██      ██   ██
% ███████ █████   ██      ██████
% ██   ██ ██      ██      ██
% ██   ██ ███████ ███████ ██



\newcommand{\hdelta}{\hat{\delta}}
\newcommand{\automaton}{$A=(Q,\Sigma, q_0, \delta, F)$\ }
\newcommand{\automatonbis}{$B=(Q_B,\Sigma_b, q_b, \delta_b, F_b)$\ }
\newcommand{\fifo}{$F=(Q,C, \Sigma, q_0, \Theta, \delta)$\ }
\newcommand{\fifoA}{$A=(Q_A,C, \Sigma, q_{0A}, \Theta_A, \delta_A)$\ }
\newcommand{\fifoB}{$B=(Q_B,C, \Sigma, q_{0B}, \Theta_B, \delta_B)$\ }
\newcommand{\tsys}{$\mathcal{T}=(S,\Theta, \rightarrow)$\ }

\newcommand{\re}{$R_E$\ }
\newcommand{\rb}{$R_B$\ }
\newcommand{\rl}{$R_L$\ }
\newcommand{\ro}{$R_O$\ }

%%%%%%%%%%%%%%%%%%%%%%%%%%%%%%%%%%%%%%%%%%%%%%%%%%%%


\newcommand{\student}{Benjamin André}
\newcommand{\grade}{MAB2 Sciences Informatiques}
\newcommand{\director}{Véronique Bruyère}
\renewcommand{\title}{Automates}
\renewcommand{\date}{\today}

\begin{document}
	
	\begin{titlepage}
		%
		\noindent
		%
		\newgeometry{bottom = 2cm, top = 2.5cm}
		\begin{center}
			\includegraphics[scale=1.2]{res/UMONS}\\
			\vspace*{0.3cm}
			\includegraphics[scale=0.23]{res/FS_Logo}\\
			\vspace*{2.5cm}
			%
			\titlefont \title \par
			%
		\end{center}
		\vspace*{3cm}
		\hfill
		%
		\begin{minipage}{0.18\linewidth}
			\begin{flushright}
				\rule{0.5pt}{50pt}
			\end{flushright}
		\end{minipage}
		%
		\begin{minipage}{0.8\linewidth}
			\begin{flushleft}
				\textsf{\textbf{Étudiant:}} \student\\
				\textsf{\textbf{Directrice:}}\director\\
				\date
			\end{flushleft}
		\end{minipage}
		%
		\vspace*{\fill}                                                             
		%
		\begin{center}
			Faculté des Sciences $\bullet$ Université de Mons $\bullet$ 
			Place du Parc 20 $\bullet$ B-7000 Mons
		\end{center}
		%
		\titlepagedecoration
		%
	\end{titlepage}
	%
	%
	%%%% Tables des matières
	%
	%
	\newgeometry{top = 3cm, left = 2cm, right = 2cm, bottom=2.5cm}
	\cleardoublepage
	
	\tableofcontents
	\newpage
	
	\section{Introduction}
	Ce document a pour but d'amener à la compréhension de l'algorithme d'Angluin. Pour ce faire, des bases théoriques seront posées dans la section \ref{sec:theorie}. Ensuite, quelques algorithmes utilisés par la méthode d'Angluin sont présentés et analysés dans la section \ref{sec:algorithmes}. Dans la section \ref{sec:preuves}, certaines notions théoriques supplémentaires sont apportées, se reposant sur les différents algorithmes du point précédant. Finalement, dans la section \ref{sec:angluin}, l'algorithme d'Angluin est expliqué et illustré avec un exemple.
	
	
	\section{Bases théoriques}\label{sec:theorie}Cette section pose les bases théoriques et les conventions nécessaires à la compréhension des sections suivantes. De plus, certaines propriétés importantes sont énoncées et démontrées.

	\subsection{Alphabet}
	
	Un alphabet, nommé par convention $\Sigma$ est un ensemble fini et non vide de symboles.
	Voici certains exemples d'alphabets :
	\begin{itemize}
		\item $\Sigma = \{0,1,2,3,4,5,6,7,8,9\}$, l'alphabet des chiffres
		\item $\Sigma = \{a,b,c,...,z,A,B,C,...,Z\}$, l'alphabet latin
		\item $\Sigma = \{0,1\}$ l'alphabet binaire
	\end{itemize}

	\subsection{Mots}
	
	Comme $\Sigma$ est un ensemble, on peut définir $\Sigma^k$, qui donne des k-uples de symboles, appartenant tous à $\Sigma$.
	
	Un mot $w$ de taille $|w|=k$ est un ensemble de symboles provenant de $\Sigma^k$. Dans le cas particulier où $k=0$, on note le mot vide (sans symbole) $w=\epsilon$.
	
	De façon générale, $w$ est un mot sur $\Sigma$ si il existe $k$ tel que $w \in \Sigma^k$. Par convention, les mots sont nommés par une lettre minuscule, souvent $w,x,y,z$. 
	
	L'ensemble de tous ces mots possible sur $\Sigma$ est noté $\Sigma^*$. Cet ensemble est infini.
	

	
	\subsection{Langage}
	
	Un langage $L$ est défini sur un alphabet $\Sigma$. $L$ est un ensemble de mots sur cet alphabet : $L \subseteq \Sigma^*$. Comme $\Sigma^*$ contient une infinité de mots, $L$ est susceptible de ne pas être fini non plus.
	
	Par exemple, par rapport à tous les mots disponibles avec les lettres de l'alphabet latin ($\Sigma^*$), seulement certains font partie de la langue française ($L$).
	
	$L$ peut être défini  :
	\begin{itemize}
		\item en énumérant les mots en faisant partie : $L=\{12,35,42,7,0\}$
		\item via un notation ensembliste : $L=\{0^k1^j|k+j=7\}$ ou $L=\{w|w \text{ est un mot français}\}$
	\end{itemize}
	Dans ces deux cas, $\Sigma$ est souvent implicite.
	
	\subsubsection{Problème}
	Une notion liée aux langages est celle de problème. Ce que nous appelons couramment problème peut être exprimé en terme d'appartenance d'un mot $w$ à un langage $L$ sur un alphabet $\Sigma$.
	
	Par exemple, prenons l'alphabet $\Sigma=\{0,1,2,3,4,5,6,7,8,9\}$. Considérons ensuite le langage $L = \{w | \text{le nombre représenté par } w \text{ est pair}\}$.
	Demander si un mot $w$ appartient à $L$ revient à résoudre le problème sur le test de parité de $w$. (Le cas échéant, soit $w$ ne représente pas un nombre, soit il représente un nombre impair).
		
	\subsection{Expression régulière}
	
	Une expression régulière est une façon efficace de définir un langage. La construction de l'expression se fait de façon inductive.
	
	Le cas de base est l'expression $e = a, a \in \Sigma$. Dès lors $L_e = {a}$ où $L_e$ est le langage donné par l'expression $e$.
	
	Les autres règles sont inductives. Par ordre de priorité :
	\begin{itemize}
		\item $e = (e_0)$ La mise en évidence. Ici, $L_e = \{w|w \in L_{e_0}\}$
		\item $e = e_0a, a \in \Sigma$. La concaténation. Ici, $L_e = \{wa|w \in L_{e_0}\}$
		\item $e = e_0+e_1$. L'union. Ici, $L_e = L_{e_0} \cup L_{e_1}$
		\item $e = e_0^*$. La fermeture. Intuitivement, il s'agit de tous les mots qui peuvent être formé par une concaténation de mots définis par $e_0$, éventuellement aucun. Ici, $L_e = \{w_0w_1...w_k | k \in \mathbb{N}, w_0,w_1,...,w_k \in L_{e_0}\}$
		\item $e = e_0^+$. La fermeture non nulle. Il s'agit de la fermeture mais avec toujours au moins un mot venant du langage défini par $e_0$. Ici, $L_e = \{w_0w_1...w_k | k \in \mathbb{N}^0, w_0,w_1,...,w_k \in L_{e_0}\}$
	\end{itemize}
	
	Par exemple, on peut écrire l'expression $e_B = (1+01)1^*0(0+1)^*$.\\
	
	\begin{minipage}{0.5\linewidth}
		Les mots suivant en feraient partie :
		\begin{itemize}
			\item 10
			\item 010
			\item 0110
			\item 0111110
			\item 0101101
		\end{itemize}
	\end{minipage}
	\begin{minipage}{0.5\linewidth}
		Ceux-ci n'en feraient pas partie
		\begin{itemize}
			\item 00
			\item 1
			\item 01
			\item 0101
			\item 11
		\end{itemize}
	\end{minipage}

	Un langage pouvant être écrit sous la forme d'une expression régulière est appelé langage régulier. Un des conséquences de cette propriété est qu'il peut être représenté par un automate déterministe fini. \todo{a prouver}
	
	
	\subsection{Automate déterministe fini}\label{sub:dfa}
	Soit un ensemble de symboles $\Sigma$. Soient $\Sigma^* = \{ a_1a_2a_3...a_n | a_1,a_2,a_3,...,a_n \in \Sigma \}$, l'ensemble des mots de taille arbitraire qu'il est possible de former à partir de $\Sigma$ et $|w|, w \in \Sigma$ la longueur de $w$, le nombre de symboles utilisés. Si $|w|=0$, on note $w=\epsilon$.
	
	
	Un automate est défini par $A = (Q, \Sigma, q_0, \delta, F)$ où
	\begin{itemize}
		\item $Q$ est un ensemble d'états, différenciés par leur indice $q_1, q_2, ..., q_n$ ou $n = |Q|$.
		\item $\Sigma$ est un ensemble de symboles
		\item $q_0 \in Q$ est l'état initial
		\item $\delta : Q \times \Sigma \rightarrow Q$ est la fonction de transition. A partir d'un état de $Q$, en fonction d'un symbole, elle retourne un nouvel état faisant partie de $Q$.
		\item $F \subseteq Q$ est un ensemble d'état finaux.
	\end{itemize}
	 
	 
	 Exemple :
	 Soient 
	 \begin{itemize}
	 	\item $\Sigma=\{0,1\}$
	 	\item $Q=\{q_a,q_b,q_c,q_d,q_e,q_f,q_g\}$
	 	\item $q_0=q_a$
	 	\item $F=\{q_d\}$
	 \end{itemize}
 	
 	Pour obtenir un automate, il manque la description de $\delta$. Une façon efficace de noter cette fonction de transition est à l'aide d'une table dont les lignes reprennent des éléments de $Q$, les colonnes des symboles de $\Sigma$, et les cases des éléments de $Q$. Ainsi, on peut lire la relation $\delta : Q \times \Sigma \rightarrow Q$ en prenant une ligne et une colonne.
 	De plus, via cette notation, $Q$ et $\Sigma$ sont explicites. En dénotant l'état initial par $\rightarrow$ et les états acceptants par $*$, on obtient une définition complète d'un automate.
 	
 	\begin{figure}[H]
 	\centering
 	\begin{tabular}{|r||c|c|}
 		\hline
 		&0&1\\
 		\hline\hline
 		$\rightarrow q_a$&$q_c$&$q_b$\\\hline
 		$q_b$&$q_d$&$q_f$\\\hline
 		$q_c$&$q_e$&$q_f$\\\hline
 		$q_d$&$q_d$&$q_d$\\\hline
 		$q_e$&$q_e$&$q_e$\\\hline
 		$q_f$&$q_d$&$q_b$\\\hline
 		$q_g$&$q_e$&$q_f$\\\hline
 	\end{tabular}
	\caption{La table de transitions $\delta_B$}
 	\end{figure}
	 L'automate peut aussi être représenté graphiquement : 
	 
	 \begin{figure}[H]
	 	\centering
	 	\begin{tikzpicture}[->,>=stealth',shorten >=1pt,auto,node distance=3cm, semithick, bend angle=10]
	 	
	 	\tikzstyle{every state}=[circle]
	 	
	 	\node[initial,state] (A)                    {$q_a$};
	 	\node[state]         (B) [below right of=A] {$q_b$};
	 	\node[state]         (C) [below left of=A] {$q_c$};
	 	\node[accepting, state]         (D) [below right of=B] {$q_d$};
	 	\node[state]         (E) [below left of=C]       {$q_e$};
	 	\node[state]         (G) [below right of=E]       {$q_g$};
	 	\node[state]         (F) [above right of=G]       {$q_f$};
	 	
	 	\path 	(A) 	edge              node {0} (C)
	 	edge              node {1} (B)
	 	(B) 	edge              node {0} (D)
	 	edge [bend left]  node {1} (F)
	 	(C) 	edge              node {0} (E)
	 	edge              node {1} (F)
	 	(D) 	edge [loop above] node {0,1} (D)
	 	(E) 	edge [loop above] node {0,1} (E)
	 	(F) 	edge              node {0} (D)
	 	edge [bend left]  node {1} (B)
	 	(G) 	edge              node {0} (E)
	 	edge              node {1} (F);
	 	\end{tikzpicture}
	 	\caption{Automate $A_B$, exemple personnel}\label{fig:ab}
	 \end{figure}
	 
	 Cette représentation d'un automate peut sembler plus naturelle pour un humain alors que la table de transitions est plus proche d'un langage informatique. De plus, dans la représentation par graphe, les ensembles $Q$ et $\Sigma$ sont implicites et doivent être définis ou déduits à part. $q_0$ et $F$ sont respectivement représenté par la flèche entrante et le double cercle.
	 
	 \subsubsection{Chemin}
	 
	 Pour faire le lien avec la notion de langage, il doit exister une façon pour l'automate de représenter quels mots sont acceptés ou non.
	 
	 Pour ce faire, la fonction $\delta$ peut être étendue à un chemin $w$:
	 
	 \begin{itemize}
	 	\item Si $|w| \leq 1$, $\hat{\delta}(x, w) = \delta(x, w)$
	 	\item Sinon, c'est que $w$ peut s'écrire $au, u \in \Sigma^*$. Alors, $\hat{\delta}(x,w) = \hat{\delta}(x,au) = \hat{\delta}(\delta(x,a),u)$
	 \end{itemize}
	 
	 \subsubsection{Langage défini par un automate}
	 
	 Le langage représenté par un automate $A$ peut alors se définir comme les mots menant de l'état initial a un état acceptant :
	 $$
	 \{w \in \Sigma^* | \hat{\delta}(q_O,w) \in F_A\}
	 $$
	 Ainsi, pour tester l'appartenance d'un mot $w$ à un langage $L$ défini par l'automate $A$, il suffit de test $\hat{\delta}(q_O,w) \in F_A$.
	 
	 \subsubsection{La relation $R_M$}\label{ss:rm}
	 
	 Soit un automate $M$. Définissons la relation $R_M$ entre deux états : $$xR_My \iff (\forall w \in \Sigma^*, \hat{\delta}(x,w) \in F \iff \hat{\delta}(y,w) \in F)$$
	 
	 Intuitivement, ces deux états sont en relation si tout mot lu à partir de celui-ci mène à des états étant simultanément acceptants ou non. Il s'agit d'une relation d'équivalence. En effet, cette relation est :
	 
	 \begin{itemize}
	 	\item \textbf{Réflexive :} Soient un état $x \in Q_M$ et $w \in \Sigma^*$. Alors, $\hat{\delta}(x,w) \in F \iff \hat{\delta}(x,w) \in F$ et par définition, $xR_Mx$.
	 	\item \textbf{Transitive :} Soient les états $x,y,z \in Q_M$ tels que $xR_My$ et $yR_Mz$ ainsi que $w \in \Sigma^*$. Par hypothèse, $\hat{\delta}(x,w) \in F \iff \hat{\delta}(y,w)\in F$ et $\hat{\delta}(y,w) \in F\iff \hat{\delta}(z,w) \in F$. Par transitivité de l'implication, on obtient $\hat{\delta}(x,w) \in F \iff \hat{\delta}(z,w)\in F$. On a donc $xR_Mz$.
	 	\item \textbf{Symétrique : } Soient les états $x,y \in Q_M$ tels que $xR_My$ et un mot $w \in \Sigma^*$. Par hypothèse, $\hat{\delta}(x, w)\in F \iff \hat{\delta}(y, w)\in F$. En lisant la double implication depuis la droite, on a bien $\hat{\delta}(y, w) \in F\iff \hat{\delta}(x, w)\in F$ et donc $yR_Mx$.
	 	\item De plus, la relation est \textbf{congruente à droite :} si la relation est vraie pour deux état, elle reste valable pour les états atteints par la lecture d'un symbole quelconque. Soient les états $x,y \in Q_M$ tels que $xR_My$. Soit un symbole $a \in \Sigma$. Par hypothèse, 
	 	$$\forall w \in \Sigma^*, \hat{\delta}(x, w) \in F \iff \hat{\delta}(y, w) \in F$$
	 	C'est donc vrai en particulier pour $w = au, u \in \Sigma*$. Dès lors,
	 	$$\hat{\delta}(x, au) \in F\iff \hat{\delta}(y, au)\in F$$
	 	$$\hat{\delta}(\delta(x,a),u) \in F\iff\hat{\delta}(\delta(y,a),u)\in F$$
	 	$$\hat{\delta}(p,u) \in F\iff \hat{\delta}(q,u)\in F$$
	 \end{itemize}
	 
	 Deux informations importantes découlent des ces caractéristiques : 
	 \begin{enumerate}
	 	\item Les états forment des classes d'équivalence.
	 	\item Tout état dans une classe d'équivalence mène, pour un même symbole, à une même classe d'équivalence.
	 \end{enumerate}
	 
	 
	 
	\section{Preuves}\label{sec:preuves}\subsection{Théorème de Myhill-Nerode}
	
	\begin{theorem}
		Les 3 énoncés suivants sont équivalents :
		\begin{enumerate}
			\item Un langage $L\subseteq\Sigma^*$ est accepté par un DFA
			\item $L$ est l'union de certaines classes d'équivalence d'index fini respectant une relation d'équivalence et de congruence à droite
			\item Soit la relation d'équivalence $R_L : xR_Ly \Leftrightarrow \forall z \in \Sigma^*, xz \in L \Leftrightarrow yz \in L$. $R_L$ est d'index fini.
		\end{enumerate}
	\end{theorem}
	
	\begin{proof}La preuve d'équivalence se fait en prouvant chaque implication de façon cyclique :\\
		
		$(1)\rightarrow(2)$ 
		
		
		$(2)\rightarrow(3)$ toute relation E de 2 est un refinement de RL du coup chaque c.eq est completement contenue dans une c.Eq de RL. on part de xRMy, cong droite
		
		$(3)\rightarrow(1)$ Mq RL cong droite xRLy, utiliser définitions 
	\end{proof}


	\begin{corollary}
		Possibilité de créer l'automate canonique...
	\end{corollary}
	\section{Algorithmes}\label{sec:algorithmes}\subsection{Table Filling Algorithm}\label{ss:tfa}

Le \emph{Table Filling Algorithm} permet, pour un automate, de déterminer quels états sont équivalents. Il repose sur la définition de la relation $R_M$.

\subsubsection{La relation $R_M$}

Soit un automate $M$. Définissons la relation $R_M$ entre deux états : $$xR_My \iff (\forall w \in \Sigma^*, \hat{\delta}(x,w) \in L_M \iff \hat{\delta}(y,w) \in L_M)$$

Intuitivement, ces deux états sont en relation si tout mot lu à partir de celui-ci mène à la même conclusion sur l'appartenance au langage. Il s'agit en fait d'une relation d'équivalence. En effet, cette relation est :

\begin{itemize}
	\item \textbf{Réflexive :} Soient un état $x \in Q_M$ et $w \in \Sigma^*$. Alors, $\delta(x,w) \iff \hat{\delta}(x,w)$ et par définition, $xR_Mx$.
	\item \textbf{Transitive :} Soient les états $x,y,z \in Q_M$ tels que $xR_My$ et $yR_Mz$ ainsi que $w \in \Sigma^*$. Par hypothèse, $\hat{\delta}(x,w) \iff \hat{\delta}(y,w)$ et $\hat{\delta}(y,w) \iff \hat{\delta}(z,w)$. Par transitivité de l'implication, on obtient $\hat{\delta}(x,w) \iff \hat{\delta}(z,w)$. On a donc $xR_Mz$.
	\item \textbf{Symétrique : } Soient les états $x,y \in Q_M$ tels que $xR_My$ et un mot $w \in \Sigma^*$. Par hypothèse, $\hat{\delta}(x, w) \iff \hat{\delta}(y, w)$. En lisant la double implication depuis la droite, on a bien $\hat{\delta}(y, w) \iff \hat{\delta}(x, w)$ et donc $yR_Mx$.
	\item De plus, la relation est \textbf{congruente à droite :} si la relation est vraie pour deux état, elle reste valable pour les états atteints par la lecture d'un symbole quelconque. Soient les états $x,y \in Q_M$ tels que $xR_My$. Soit un symbole $a \in \Sigma$. Par hypothèse, $$\forall w \in \Sigma^*, \hat{\delta}(x, w) \iff \hat{\delta}(y, w)$$
	C'est donc vrai en particulier pour $w = au, u \in \Sigma*$. Dès lors,
	$$\hat{\delta}(x, au) \iff \hat{\delta}(y, au)$$
	$$\hat{\delta}(\delta(x,a),u) \iff\hat{\delta}(\delta(y,a),u)$$
	$$\hat{\delta}(p,u) \iff \hat{\delta}(q,u)$$
\end{itemize}

Deux informations importantes découlent des ces caractéristiques : 
\begin{enumerate}
	\item Les états forment des classes d'équivalence.
	\item Tout état dans une classe d'équivalence mène, pour un même symbole, à une même classe d'équivalence.
\end{enumerate}

\subsubsection{Construction de la table}

L'idée est de construire, par induction, une table nous disant pour chaque paire d'état si ceux-ci sont équivalents où non, suivant la relation $R_M$ définie précédemment:

\textbf{Cas de base :} Si $p$ est un état final et que $q$ ne l'est pas, alors la paire $\{p,q\}$ est différentiable.

\textbf{Induction : } Soient $p,q$ des états tels qu'il existe un symbole $a$ qui donne $\delta(p,a)=r$ et $\delta(q,a)=s$. Si $r$ et $s$ sont différentiables, alors $p$ et $q$ le sont aussi. En effet, il existe un mot témoin $w$ qui permet de différencier $r$ et $s$. Alors le mot $aw$ permet de différencier $p$ et $q$.

\begin{theorem}
	Si deux états ne sont pas distingués par l'algorithme de remplissage de table, les états sont équivalents.
\end{theorem}

\begin{proof}
	Considérons un automate déterministe fini quelconque $A = (Q, \Sigma, \delta, q_0, F)$, et faisons une preuve par l'absurde.
	
	Supposons qu'il existe une paire d'états $\{p,q\}$ tels que :
	\begin{enumerate}
		\item $p$ et $q$ ne sont pas distingués par l'algorithme de remplissage de table
		\item Les états ne sont pas équivalents, c'est à dire différentiables.
	\end{enumerate}
	
	L'hypothèse deux implique qu'il existe un mot $w \in \Sigma^*$ tel que de $\hat{\delta}(p,w)$ et $\hat{\delta}(q,w)$ un et un seul soit un état final.
	
	Une telle paire est une \emph{mauvaise paire}. Si il y a des mauvaises paires, chacune distinguée par un mot témoin, il doit exister un paire distinguée par le mot témoin le plus court. Posons $\{p,q\}$ comme étant cette paire et $w=a_1a_2\dots a_n$ le mot témoin le plus court qui les distingue. Encore une fois, un seul de $\hat{\delta}(p,w)$ et $\hat{\delta}(q,w)$ est acceptant.
	
	Ce mot $w$ ne peut pas être $\epsilon$. Auquel cas, la table aurait été remplie dès l'étape d'induction de l'algorithme.
	
	Ce mot $w$ doit forcément être de taille $\ge 1$ s'il n'est pas $\epsilon$. Considérons $r = \delta(p,a_1)$ et $s=\delta(q,a_1)$. Ces états sont différenciés par $a_2a_3\dots a_n$ puisque cette chaîne mène aux mêmes états que $\hat{\delta}(p,w)$ et $\hat{\delta}(q,w)$. Mais dans ce cas, cela signifie qu'il existe un mot plus petit que $w$ qui différencie deux états. Comme on a supposé que $w$ est le mot le plus petit qui différencie une mauvaise paire, $r$ et $s$ ne peuvent pas être une bad pair. Donc, l'algorithme a du découvrir qu'ils sont différentiables.
	
	Mais le pas d'induction stipule clairement que comme $\delta(p, a_1)$ et $\delta(q, a_1)$ mènent à deux états différentiables, $p$ et $q$ le sont aussi. On a une contradiction sur l'existence des mauvaises paires.
	
	Ainsi, s'il n'y en a pas, c'est que chaque paire différentiable est reconnue par l'algorithme.
\end{proof}


\subsubsection{Exemple}

\begin{figure}[H]
	\centering
	\begin{tikzpicture}[->,>=stealth',shorten >=1pt,auto,node distance=3cm, semithick, bend angle=10]
	
	\tikzstyle{every state}=[circle]
	
	\node[initial,state] (A)                    {$q_a$};
	\node[state]         (B) [below right of=A] {$q_b$};
	\node[state]         (C) [below left of=A] {$q_c$};
	\node[accepting,state]         (D) [below right of=B] {$q_d$};
	\node[state]         (E) [below left of=C]       {$q_e$};
	\node[state]         (F) [below right of=C]       {$q_f$};
	
	\path 	(A) 	edge              node {0} (C)
	edge              node {1} (B)
	(B) 	edge              node {0} (D)
	edge [bend left]  node {1} (F)
	(C) 	edge              node {0} (E)
	edge              node {1} (F)
	(D) 	edge [loop above] node {0,1} (D)
	(E) 	edge [loop above] node {0,1} (E)
	(F) 	edge              node {0} (D)
	edge [bend left]  node {1} (B);
	\end{tikzpicture}
	\caption{Automate $A_2$}\label{fig:a2}
\end{figure}

La première étape est de remplir la table avec l'algorithme précédant. Tout état est distinguable de $q_d$ : il est le seul état final. 5 cases peuvent déjà êtres cochées. Le reste de la table est remplie par induction.

\begin{figure}[H]
	\centering
	\begin{tabular}{ccccccc}
		\cline{2-2}
		\multicolumn{1}{c|}{B} & \multicolumn{1}{c|}{x} &&&&\\
		\cline{2-3}
		\multicolumn{1}{c|}{C} & \multicolumn{1}{c|}{x} &\multicolumn{1}{c|}{x}&&&\\
		\cline{2-4}
		\multicolumn{1}{c|}{D} & \multicolumn{1}{c|}{x} &\multicolumn{1}{c|}{x}&\multicolumn{1}{c|}{x}&&\\
		\cline{2-5}
		\multicolumn{1}{c|}{E} & \multicolumn{1}{c|}{x} &\multicolumn{1}{c|}{x}&\multicolumn{1}{c|}{x}&\multicolumn{1}{c|}{x}&\\
		\cline{2-6}
		\multicolumn{1}{c|}{F} & \multicolumn{1}{c|}{x} & \multicolumn{1}{c|}{}&\multicolumn{1}{c|}{x}&\multicolumn{1}{c|}{x}&\multicolumn{1}{c|}{x}\\
		\cline{2-6}
		\multicolumn{1}{c}{} & A&B&C&D&E\\
		
	\end{tabular}
	\caption{Table filling pour $A_2$, décelant des équivalences d'états}
	\label{fig:ta2}
\end{figure}




\subsection{Minimisation d'automate}

La minimisation d'automate se fait en deux étapes :
\begin{enumerate}
	\item Se débarrasser de tous les états injoignables : ils ne participent pas à la construction du langage représenté
	\item Grâce aux équivalences d'états trouvées grâce au TFA déifni au point \ref{ss:tfa}, construire un nouvel automate. 
\end{enumerate}

Ces étapes vont être accompagnées d'un exemple, à savoir l'automate $A_1$ représenté à la figure \ref{fig:a1}.

\begin{figure}[H]
	\centering
	\begin{tikzpicture}[->,>=stealth',shorten >=1pt,auto,node distance=3cm, semithick, bend angle=10]
	
	\tikzstyle{every state}=[circle]
	
	\node[initial,state] (A)                    {$q_a$};
	\node[state]         (B) [below right of=A] {$q_b$};
	\node[state]         (C) [below left of=A] {$q_c$};
	\node[accepting, state]         (D) [below right of=B] {$q_d$};
	\node[state]         (E) [below left of=C]       {$q_e$};
	\node[state]         (G) [below right of=E]       {$q_g$};
	\node[state]         (F) [above right of=G]       {$q_f$};
	
	\path 	(A) 	edge              node {0} (C)
	edge              node {1} (B)
	(B) 	edge              node {0} (D)
	edge [bend left]  node {1} (F)
	(C) 	edge              node {0} (E)
	edge              node {1} (F)
	(D) 	edge [loop above] node {0,1} (D)
	(E) 	edge [loop above] node {0,1} (E)
	(F) 	edge              node {0} (D)
	edge [bend left]  node {1} (B)
	(G) 	edge              node {0} (E)
	edge              node {1} (F);
	\end{tikzpicture}
	\caption{Automate $A_1$}\label{fig:a1}
\end{figure}

L'état $q_g$ n'est pas atteignable : il peut être simplement supprimé. On obtient ainsi l'automate $A_2$ qui a servi d'exemple pour le TFA, représenté à la figure \ref{fig:a2}.

\subsubsection{Minimisation par table de différenciation}

Pour minimiser l'automate $A_2 = (Q, \Sigma, \delta, q_0, F)$, il faut :
\begin{enumerate}
	\item Générer la table de différenciation (qui, pour cet exemple, est à la figure \ref{fig:ta2})
	\item Séparer $Q$ en classes d'équivalences
	\item Construire l'automate canonique $A_3$:
		\begin{itemize}
			\item Soit $S$ une des classes d'équivalence
			\item Soit $\gamma$ la fonction de transition sur $S$. Pour un symbole $a \in \Sigma$, alors il doit exister une classe d'équivalence $T$ tel que pour chaque état $q$ dans $S$, $\delta(q,a) \in T$. Sinon, c'est que deux états $p$ et $q$ dans $S$ menant à différentes classes d'équivalences. Ces deux états sont différenciables, et ne pourraient pas appartenir tous deux à $S$ par construction. On peut écrire $\gamma(S,a)=T$.
		\end{itemize}
	\item L'état initial de $A_3$ est la classe d'équivalence contenant l'état initial de $A_2$ (dans notre exemple, l'état s'y trouve seul)
	\item Les états finaux ($F$) de $A_3$ sont les classes d'équivalences qui contenaient des états acceptants de $A_2$.
\end{enumerate}

La table de la figure \ref{fig:ta2}. Peut servir de base à la construction du nouvel automate suivant cet algorithme.
\begin{figure}[H]
	\centering
	\begin{tikzpicture}[->,>=stealth',shorten >=1pt,auto,node distance=3cm, semithick, bend angle=10]
	
	\tikzstyle{every state}=[circle]
	
	\node[initial,state] (A)                    {$q_a$};
	\node[state]         (B) [below right of=A] {$q_b$};
	\node[state]         (C) [below left of=A] {$q_c$};
	\node[accepting, state]         (D) [below right of=B] {$q_d$};
	\node[state]         (E) [below left of=C]       {$q_e$};
	
	\path 	
	(A) 	edge              node {0} (C)
	edge              node {1} (B)
	(B) 	edge              node {0} (D)
	edge [loop above] node {1} (B)
	(C) 	edge              node {0} (E)
	edge              node {1} (B)
	(D) 	edge [loop above] node {0,1} (D)
	(E) 	edge [loop above] node {0,1} (E);
	\end{tikzpicture}
	\caption{Automate $A_3$}
\end{figure}

Une expression régulière ($(1+01)1^*0(0+1)^*$) peut être déduite pour $L$ grâce à cet automate.


\subsection{Équivalence d'automates}

\todo{Se base sur le TFA et la minimisation, on "colle" les deux}



\subsection{Construction d'automate depuis un langage}

Soit le langage $A_N = \{w | w \text{ fini par b et ne contient pas bb}\}$ défini sur $\Sigma_N = {a,b}$.

On peut diviser les mots en 3 ensembles : 

\begin{itemize}
	\item $W_0$ le sous-ensemble des mots ne finissant pas le symbole $b$
	\item $W_1$ celui des mots finissant par le symbole $b$ mais ne contenant pas $bb$
	\item $W_2$ celui des mots contenant au moins $bb$
\end{itemize}

Il y a d'autres façons de construire des sous-ensembles, mais celle-ci à l'avantage de rendre la question de l'appartenance à $L_N$ triviale : un mot appartient au second ensemble si et seulement si il fait partie du langage, par définition.

De plus, tous les éléments d'un sous-ensemble respectent la relation $R_L$ entre eux. ($R_L : xR_Ly \Leftrightarrow \forall z \in \Sigma^*, xz \in L \Leftrightarrow yz \in L$). Cela en fait des classes d'équivalence sur cette relation.

Cela peut être démontré pour chaque sous-ensemble :
\begin{itemize}
	\item Soient $x,y \in W_0$. Soit $z \in \Sigma^*$. Dès lors, si $xz \in L_N$, c'est que $z$ fini par $b$ mais ne contient pas $bb$, et donc $yz \in L_N$. Si $yz \in L_N$, le même argument peut être appliqué.
	\item Soient $x,y \in W_1$. Soit $z \in \Sigma^*$. Dès lors, si $xz \in L_N$, c'est que $z$ ne commençait pas le symbole $b$ et ne contenait pas $bb$, $yz$ ne contiendra donc pas $bb$, puisque cette chaîne n'est ni dans $z$ ni dans $y$, ni a cheval sur les deux, $z$ ne commençant pas par $b$. Ainsi, $yz \in L_N$. Si $yz \in L_N$, le même argument peut être appliqué.
	\item Soient $x,y \in W_2$. Soit $z \in \Sigma^*$. Comme $x$ contient déjà $bb$, $x \notin L_N$ et, a fortiori, $xz \notin L_N$. Comme la prémisse est fausse, l'implication $xz \in L \Rightarrow yz \in L$ est vraie. La même logique peut être appliquée à partir de $y$ pour justifier l'implication inverse.
\end{itemize}

De plus, ces sous-ensembles sont disjoints. Cela peut se prouver en invalidant la relation pour certains éléments entre eux, mais dans ce cas-ci, la propriété est assurée par définition.

Ceci revient à démontrer que $W_0,W_1,W_2$ sont des classes d'équivalence. De plus, $R_L$ respecte la congruence à droite, comme démontré dans la preuve du théorème de Myhill-Nérode. Ce même théorème donne une méthode pour construire un automate : prendre un représentant pour chaque classe et en faire un état.

\begin{itemize}
	\item $\Sigma=\{a,b\}$ est connu.
	\item $Q=\{[[\epsilon]]\, [[b]], [[bb]]\} = \{q_\epsilon, q_b, q_{bb}\}$
	\item $q_0 = q_\epsilon$ 
	\item $F = \{q_b\}$ l'union des classes acceptant
	\item $\delta$ défini en utilisant des exemples tirés des classes d'équivalence.
\end{itemize}

Ce qui donne l'automate de la figure \ref{fig:an}

\begin{figure}[H]
	\centering
	\begin{tikzpicture}[->,>=stealth',shorten >=1pt,auto,node distance=2cm and 2cm, semithick, bend angle=10]
	
	\tikzstyle{every state}=[circle]
	
	\node[initial,state]	(A)					{$q_\epsilon$};
	\node[accepting,state]	(B)	[right= of A]	{$q_b$};
	\node[state]			(C) [right= of B]	{$q_{bb}$};
	
	\path
	(A)	edge	[bend left]		node{b}		(B)
	(A)	edge	[loop above]	node{a}		(A)
	(B) edge	[bend left]		node{a}		(A)
	(B) edge					node{b}		(C)
	(C)	edge	[loop above]	node{a,b}	(C)
	
	; 
	\end{tikzpicture}
	\caption{Automate $A_N$, exemple d'une thèse\cite{Neider14}}\label{fig:an}
\end{figure}

Cet automate est bien une représentation du langage $L_N$. Seul un mot finissant par $b$ mais ne contenant pas $bb$ se termine à l'état $q_b$.


\subsection{Minimisation d'automate}

\begin{figure}[H]
	\centering
	\begin{tikzpicture}[->,>=stealth',shorten >=1pt,auto,node distance=2cm and 5cm, semithick, bend angle=10]
	
	\tikzstyle{every state}=[circle]
	
	\node[initial,state]	(A)					{$q_0$};
	\node[state]			(B)	[right= of A]	{$q_1$};
	\node[accepting,state]	(C) [below of=A]	{$q_2$};
	\node[accepting,state]	(D)	[below of=B]	{$q_3$};
	\node[accepting,state]	(E)	[below of=C]	{$q_4$};
	\node[state]			(F)	[below of=D]	{$q_5$};
	
	\path
	(A)	edge	[bend left]		node{0}		(B)
	(A)	edge					node{1}		(C)
	(B) edge	[bend left]		node{0}		(A)
	(B) edge					node{1}		(D)
	(C)	edge					node{0}		(E)
	(C)	edge					node[near start]{1}		(F)
	(D)	edge					node[near start, above]{0}		(E)
	(D)	edge					node{1}		(F)
	(E)	edge	[loop below]	node{0,1}	(E)
	(F)	edge	[loop below]	node{0,1}	(F)
	
	; 
	\end{tikzpicture}
	\caption{Automate $A_H$, exemple d'un livre de référence\cite{Hopcroft79}}
\end{figure}

	
	\section{Algorithme d'Angluin}\label{sec:angluin}
L'algorithme d'Angluin repose, en plus des éléments précédents sur quatre concepts :

\begin{itemize}
	\item Une table d'observation
	\item La relation $R_O$, se basant sur la table d'observation et semblable à la relation $R_L$
	\item La propriété de fermeture (closure en anglais)
	\item La propriété de cohérence (consistence en anglais)
\end{itemize}

Cette section commence par décrire cette table en \ref{ss:a_tblo}, la relation $R_O$ en \ref{ss:a_ro}, la fermeture en \ref{ss:a_fermeture}, la cohérence en \ref{ss:a_coherence}.

Une fois toutes ces bases posées, une exécution de l'algorithme sur un exemple est proposée en \ref{ss:a_exemple}, suivie du fonctionnement formel de l'algorithme et des preuves sur son exactitude et sa complexité en \ref{ss:a_algo}, \ref{ss:a_proof} et \ref{ss:a_comp}.


\subsection{Table d'observation}\label{ss:a_tblo}

\subsection{Relation $R_O$}\label{ss:a_ro}

\subsection{Fermeture}\label{ss:a_fermeture}

La propriété de fermeture (closure) s'exprime mathématiquement par 

$$ \forall u \in R, \forall a \in \Sigma, \exists v \in R, ua R_O v$$

En pratique, pour vérifier cette propriété, il suffit de de suivre cet algorithme, expliqué de façon visuelle sur la table O :

\begin{algorithm}[H]
	\begin{algorithmic}[1]
		\ENSURE si la fermeture est respectée ou non
		
		\FORALL {élément $w$ de la section $R$}
		\FORALL {symbole $a$ dans $\Sigma$}		
			\IF {$wa$ est dans $R$} 
				\STATE continuer
			\ELSE
				\STATE \COMMENT{$wa$ est dans $R.\Sigma$ par construction}
				\IF {La ligne de $wa$ dans $T$ est différente de celle de $w$}
					\RETURN Faux
				\ENDIF
			\ENDIF
		\ENDFOR
		\ENDFOR
		\RETURN Vrai
	\end{algorithmic}
	\caption{Vérification de la fermeture}\label{alg:closure}
\end{algorithm}

\subsection{Cohérence}\label{ss:a_coherence}

La propriété de cohérence (consistence) se définit mathématiquement comme 

$$ \forall u,v \in R, u R_O v \Rightarrow \forall a \in \Sigma, ua R_O va$$

Concrètement, il s'agit de prendre deux mots ($u,v$) dans $R$ ayant la même ligne dans $T$ et vérifier, pour chaque symbole ($a$), s'ils ($ua,va$) ont la même ligne dans $T$.



\subsection{Exemple}\label{ss:a_exemple}
Soit l'automate $A_3$ construit à la section \ref{ss:miniauto} sur la minimisation. L'automate $A_4$ recopié ici n'est qu'une isomorphie : les symboles $\{0,1\}$ ont été remplacés par $\{a,b\}$ pour plus de lisibilité dans les tables d'observation.
\begin{figure}[H]
	\centering
	\begin{tikzpicture}[->,>=stealth',shorten >=1pt,auto,node distance=3cm, semithick, bend angle=10]
	
	\tikzstyle{every state}=[circle]
	
	\node[initial,state] (A)                    {$q_a$};
	\node[state]         (B) [below right of=A] {$q_b$};
	\node[state]         (C) [below left of=A] {$q_c$};
	\node[accepting, state]         (D) [below right of=B] {$q_d$};
	\node[state]         (E) [below left of=C]       {$q_e$};
	
	\path 	
	(A) 	edge              node {a} (C)
	edge              node {b} (B)
	(B) 	edge              node {a} (D)
	edge [loop above] node {b} (B)
	(C) 	edge              node {a} (E)
	edge              node {b} (B)
	(D) 	edge [loop above] node {a,b} (D)
	(E) 	edge [loop above] node {a,b} (E);
	\end{tikzpicture}
	\caption{Automate $A_4$}
\end{figure}

\todo{Marquer la différence entre $R_L$ et $R_O$}

\subsubsection{Première itération}

L'algorithme d'Angluin précise, pour son cas de base, une initialisation de la table $T$ avec les ensembles $R$ et $S$ contenant uniquement $\epsilon$. Le champ $R.\{a,b\}$ ($R.\Sigma$) est rempli via des requête d'appartenance sur les mots $a$ et $b$.

\begin{minipage}{0.5\linewidth}
	\centering
	\begin{tabular}{|c|c|}
		\hline
		$O_0$ & $\epsilon$\\
		\hline
		$\epsilon$ & 0\\
		\hline
		$a$ & 0\\
		$b$ & 0\\
		\hline
	\end{tabular}
\end{minipage}
\begin{minipage}{0.5\linewidth}
	\centering
	\begin{figure}[H]
		\centering
		\begin{tikzpicture}[->,>=stealth',shorten >=1pt,auto,node distance=3cm, semithick, bend angle=10]
		\tikzstyle{every state}=[circle]
		\node[initial, state] (A) {$[[\epsilon]]$};
		\path (A) edge [loop above] node {a,b} (A);
		\end{tikzpicture}
		\caption*{Automate $O_0$}
	\end{figure}
\end{minipage}


\vspace{1cm}
L'étape suivante consiste à vérifier la \emph{closure} de la table d'observation $O_0$. Mathématiquement :

$$ \forall u \in R, \forall a \in \Sigma, \exists v \in R, ua R_L v$$

Intuitivement, pour chaque symbole (ici, $\{a,b\}$, et ce sera vrai jusqu'à la dernière itération), tout mot candidat (dans $R$, la partie supérieure de la table) doit se retrouver, complété de ce symbole, dans une classe d'équivalence d'un autre candidat de $R$. Ici, de toute évidence, les mots $a$ et $b$ sont dans la même classe d'équivalence que $\epsilon$. Dès lors, la propriété de \emph{closure} est respectée.

Si la \emph{closure} est respectée, alors la question de la \emph{consistence} (cohérence) se pose. Mathématiquement : 

$$ \forall u,v \in R, u R_L v \Rightarrow \forall a \in \Sigma, ua R_L va$$

Intuitivement, si deux candidats semblent être dans la même classe d'équivalence (leur lignes dans la table supérieure sont identiques), alors pour n'importe quel symbole, les deux nouveaux mots sont également dans une même classe d'équivalence (leur lignes, potentiellement dans la partie inférieure de la table, sont identiques). N'ayant qu'un seul candidat, cette propriété est forcément respectée ($R_L$ est réflexive).

Les deux propriétés étant respectées, les classes d'équivalences sont calculées (trivialement ici), et un automate $O_0$ est proposé à l'enseignant pour vérification.

Sur cette itération, un automate initial a été proposé, et aucun état final ne pouvant être atteint avec un seul symbole, la version est minime.

\subsubsection{Seconde itération}

L'enseignant répond que non, les automates ne sont pas équivalents. Il fourni le contre-exemple $ba$. Comme il est rejeté par $O_0$, cela signifie qu'il est accepté par $A_4$. Une nouvelle table est alors construite, en ajoutant $ba$ et ses préfixes (ici, juste $b$) à $R$. $R.\Sigma$ est calculé et les mots n'ayant pas encore reçu une valeur dans $T$ sont soumis à l'enseignant pour un test d'appartenance.
\vspace{1cm}

\begin{minipage}{0.25\linewidth}
	\centering
	\begin{tabular}{|c|c|}
		\hline
		$O_1$ & $\epsilon$\\
		\hline
		$\epsilon$ & 0\\
		\textcolor{red}{$b$} & \textcolor{red}{0}\\
		\textcolor{red}{$ba$} & \textcolor{red}{1}\\
		\hline
		$a$ & 0\\
		\textcolor{red}{$bb$} & \textcolor{red}{0}\\
		\textcolor{red}{$baa$} & \textcolor{red}{1}\\
		\textcolor{red}{$bab$} & \textcolor{red}{1}\\
		\hline
	\end{tabular}
\end{minipage}
\begin{minipage}{0.25\linewidth}
	\centering
	\begin{tabular}{|c|cc|}
		\hline
		$O_2$ & $\epsilon$ & \textcolor{red}{$a$}\\
		\hline
		$\epsilon$ & 0& \textcolor{red}{0}\\
		$b$ & 0&\textcolor{red}{1}\\
		$ba$ & 1&\textcolor{red}{1}\\
		\hline
		$a$ & 0&\textcolor{red}{0}\\
		$bb$ & 0&\textcolor{red}{1}\\
		$baa$ & 1&\textcolor{red}{1}\\
		$bab$ & 1&\textcolor{red}{1}\\
		\hline
	\end{tabular}
\end{minipage}
\begin{minipage}{0.5\linewidth}
	\centering
	\begin{figure}[H]
		\centering
		\begin{tikzpicture}[->,>=stealth',shorten >=1pt,auto,node distance=3cm, semithick, bend angle=10]
		\tikzstyle{every state}=[circle]
		
		\node[initial, state] (A) {$[[\epsilon]]$};
		\node[state] (B) [right of=A] {$[[b]]$};
		\node[accepting, state] [right of=B] (C) {$[[ba]]$};
		
		\path
		(A) edge [loop above] node {a} (A)
		(A) edge node {b} (B)
		(B) edge node {a} (C)
		(B) edge [loop above] node {b} (B)
		(C) edge [loop above] node {a,b} (C);
		
		
		\end{tikzpicture}
		\caption*{Automate $O_2$}
	\end{figure}
\end{minipage}

\vspace{1cm}
Comme pour la première itération, la \emph{fermeture} est testée, suivie de la \emph{cohérence}. Celle-ci n'est pas respectée : si on considère les mots $\epsilon$ et $b$, qui ont la même ligne dans la table $T$ ($\epsilon R_O b$), le symbole $a$, on obtient les mots $a$ et $ba$ qui n'ont pas la même ligne : ($\not a R_O ba$). Le symbole $a$ est alors ajouté à $S$ et une nouvelle table $O_2$ est calculée.

La fermeture étant déjà vérifiée, la question de la cohérence est reposée, et cette fois-ci elle est vérifiée ; l'automate est construit et proposé à l'enseignant.

Sur cette itération, l'algorithme a reçu le mot $ba$ comme étant accepté. Il a du ajouter $a$ à $S$ pour permettre de différencier certains états. L'automate se voit ajouter les états $[[b]]$ et $[[ba]]$.

\subsubsection{Troisième itération}

Suivant toujours l'algorithme de comparaison d'automates détaillé dans la section \ref{sec:algorithmes}, l'enseignant découvre qu'ils sont différents. 

Il sort le contre-exemple $aaba$. Si c'est un contre-exemple et qu'il est accepté par $O_2$, c'est qu'il ne l'est pas (0) par $A_4$. Une nouvelle table $O_3$ doit être construite.

\begin{minipage}{0.33\linewidth}
	\centering
	\begin{tabular}{|c|cc|}
		\hline
		$O_3$ & $\epsilon$ & $a$\\
		\hline
		$\epsilon$ & 0 &0\\
		\textcolor{red}{$a$}&\textcolor{red}{0}&\textcolor{red}{0}\\
		$b$&0&1\\
		\textcolor{red}{$aa$}&\textcolor{red}{0}&\textcolor{red}{0}\\
		$ba$&1&1\\
		\textcolor{red}{$aab$}&\textcolor{red}{0}&\textcolor{red}{0}\\
		\textcolor{red}{$aaba$}&\textcolor{red}{0}&\textcolor{red}{0}\\
		\hline
		\textcolor{red}{$ab$}&\textcolor{red}{0}&\textcolor{red}{1}\\
		$bb$&0&1\\
		\textcolor{red}{$aaa$}&\textcolor{red}{0}&\textcolor{red}{0}\\
		$baa$&1&1\\
		$bab$&1&1\\
		\textcolor{red}{$aabb$}&\textcolor{red}{0}&\textcolor{red}{0}\\
		\textcolor{red}{$aabaa$}&\textcolor{red}{0}&\textcolor{red}{0}\\
		\textcolor{red}{$aabab$}&\textcolor{red}{0}&\textcolor{red}{0}\\
		\hline
	\end{tabular}
\end{minipage}
\begin{minipage}{0.33\linewidth}
	\centering
	\begin{tabular}{|c|cc|}
		\hline
		$O_4$ & $\epsilon$ & $a$\\
		\hline
		$\epsilon$ & 0 &0\\
		$a$&0&0\\
		$b$&0&1\\
		$aa$&0&0\\
		\textcolor{red}{$ab$}&\textcolor{red}{0}&\textcolor{red}{1}\\
		$ba$&1&1\\
		$aab$&0&0\\
		$aaba$&0&0\\
		\hline
		$bb$&0&1\\
		$aaa$&0&0\\
		\textcolor{red}{$aba$}&\textcolor{red}{1}&\textcolor{red}{1}\\
		\textcolor{red}{$abb$}&\textcolor{red}{0}&\textcolor{red}{1}\\
		$baa$&1&1\\
		$bab$&1&1\\
		$aabb$&0&0\\
		$aabaa$&0&0\\
		$aabab$&0&0\\
		\hline
	\end{tabular}
\end{minipage}
\begin{minipage}{0.33\linewidth}
	\centering
	\begin{tabular}{|c|cc|}
		\hline
		$O_5$ & $\epsilon$ & $a$\\
		\hline
		$\epsilon$ & 0 &0\\
		$a$&0&0\\
		$b$&0&1\\
		$aa$&0&0\\
		$ab$&0&1\\
		$ba$&1&1\\
		$aab$&0&0\\
		\textcolor{red}{$aba$}&\textcolor{red}{1}&\textcolor{red}{1}\\
		$aaba$&0&0\\
		\hline
		$bb$&0&1\\
		$aaa$&0&0\\
		$abb$&0&1\\
		$baa$&1&1\\
		$bab$&1&1\\
		$aabb$&0&0\\
		\textcolor{red}{$abaa$}&\textcolor{red}{1}&\textcolor{red}{1}\\
		\textcolor{red}{$abab$}&\textcolor{red}{1}&\textcolor{red}{1}\\
		$aabaa$&0&0\\
		$aabab$&0&0\\
		\hline
	\end{tabular}
\end{minipage}



	\begin{figure}[H]
		\centering
		\begin{tikzpicture}[->,>=stealth',shorten >=1pt,auto,node distance=3cm, semithick, bend angle=10]
		\tikzstyle{every state}=[circle]
		
		\node[initial, state] (A) {$[[\epsilon]]$};
		\node[state] (B) [above right of=A] {$[[a]]$};
		\node [state] (E) [right of=B] {$[[aa]]$};
		\node[state] (C) [below right of =A] {$[[b]]$};
		\node[accepting, state] [right of=C] (D) {$[[ba]]$};
		
		\path
		(A) edge node {a} (B)
		(A) edge node {b} (C)
		(B) edge node {b} (C)
		(B) edge node {a} (E)
		(C) edge [loop below] node {b} (C)
		(C) edge node {a} (D)
		(D) edge [loop above] node {a,b} (D)
		(E) edge [loop below] node {a,b} (E);
		
		\end{tikzpicture}
		\caption*{Automate $O_5$}
	\end{figure}


Ayant reçu $aaba$, ce mot et tous ses préfixes sont ajoutés à la table. L'extension $R.\Sigma$ est recalculée et la table $O_3$ est construite.

Ensuite, la question de la \emph{fermeture} est posée. Un manquement est détecté : le mot $a$. En effet, en lui ajoutant le symbole $b$, on obtient $ab$ qui n'est ni dans $R$ ni en relation $R_O$ avec $a$. $ab$ est alors ajouté à $R$, et $R.\Sigma$ est étendu. La nouvelle table, $O_4$ est de nouveau testée.

$O_4$ ne respecte pas la fermeture : le mot $ab$, agrémenté du symbole $a$ donne le mot $aba$, qui n'est ni dans $R$ ni en relation avec $ab$. Le mot est ajouté à $R$, et la table est étendue. La nouvelle table, $O_5$ est à la fois fermée et cohérente.

L'automate $O_5$ est alors proposé à l'enseignant pour vérification. Celui-ci est accepté (isomorphe à $A_4$). L'algorithme s'arrête et un automate minimal pour le langage a été construit. 

\subsection{Algorithme}\label{ss:a_algo}

\subsection{Preuve}\label{ss:a_proof}

\subsection{Complexité}\label{ss:a_comp}

	
	

	
	\newpage
	\bibliographystyle{siam}
	\bibliography{refs.bib}
	
\end{document}