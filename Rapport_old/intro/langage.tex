Cette section pose les différents concepts et notations pour arriver à la notion de langage. Celles-ci reprennent les notations proposées par Hopcroft et al. \cite{Hopcroft00}. Un sous-ensemble de ces langages correspond à des automates, sujet central de ce document.

\subsection{Définitions}

Un \emph{alphabet} $\Sigma$ est un ensemble fini et non vide de \emph{symboles}. Un \emph{mot} sur cet alphabet $\Sigma$ est une suite finie de $k$ éléments de $\Sigma$ notée $ w = a_1a_2\dots a_k$ où $k$ est un nombre naturel. $k$ est la \emph{longueur} de ce mot aussi notée $|w|=k$. Le \emph{mot vide} est un mot de taille $k=0$ noté $w=\epsilon$.

La \emph{concaténation} de deux mots $w=a_1a_2\dots a_k$ et $x=b_1b_2\dots b_j$ est l'opération consistant à créer un nouveau mot $wx=a_1a_2\dots a_kb_1b_2\dots b_j$ de longueur $i=k+j$.

\begin{proposition}[$\epsilon$ et la concaténation]\emph{
$\epsilon$ est \emph{l'identité pour la concaténation}, à savoir pour tout mot $w$, $w\epsilon = \epsilon w = w$.}
\end{proposition}

Cette proposition est triviale par la définition de la concaténation.

\emph{L'exponentiation} d'un symbole $a$ à la puissance $k$, notée $a^k$, retourne un mot de longueur $k$ obtenu par la concaténation de $k$ copies du symbole $a$. Noter que $a^0=\epsilon$. $\Sigma^k$ est \emph{l'ensemble des mots sur $\Sigma$} de longueur $k$. L'ensemble de \emph{tous les mots possibles sur $\Sigma$} est noté $\Sigma^* = \bigcup_{k=0}^{\infty}\Sigma^k$.


Un ensemble quelconque de mots sur $\Sigma$ est un \emph{langage}, noté $L \subseteq \Sigma^*$. Étant donné que $\Sigma^*$ est infini, $L$ peut l'être également.

\begin{example}[Langages] Voici des exemples utilisant plusieurs modes de définition. $\Sigma$ y est implicite mais peut être donné explicitement.
	\begin{itemize}
		\item $L=\{12,35,42,7,0\}$, un langage défini explicitement
		\item $L=\{0^k1^j|k+j=7\}$, les mots de 7 symboles sur $\Sigma=\{0,1\}$ commençant par zéro, un ou plusieurs $0$ et finissant par zéro, un ou plusieurs $1$. Ici, $L$ est donné par notation ensembliste
		\item $L$ contient tous les noms de villes belges. Ici $L$ est défini en langage courant.
		\item $\emptyset$ est un langage sur tout alphabet.
		\item $L=\{\epsilon\}$ ne contient que le mot vide, et est un langage sur tout alphabet.
	\end{itemize}
\end{example}


\subsection{Opérations sur les langages}

Soient $L$ et $M$ deux langages. Le langage $L \cup M = \{w | w \in L\vee w \in M\}$ est l'\emph{union} de ces deux langages. Il est composé des mots venant d'un des deux langages.

Le langage composé de tous les mots produits par la concaténation d'un mot de $L$ avec un mot de $M$ est une \emph{concaténation} de ces deux langages et s'écrit $LM = \{vw | v \in L \wedge w \in M\}$.

La \emph{fermeture} de $L$ est un langage constitué de tous les mots qui peuvent être construits par un concaténation d'un nombre arbitraire de mots de $L$, noté $L^*=\{w_1w_2\dots w_n|n\in \mathbb{N},\forall i \in \{1,2,\dots,n\}, w_i \in L\}$.

% ██████  ███████  ██████  ███████ ██   ██
% ██   ██ ██      ██       ██       ██ ██
% ██████  █████   ██   ███ █████     ███
% ██   ██ ██      ██    ██ ██       ██ ██
% ██   ██ ███████  ██████  ███████ ██   ██

\subsection{Expressions régulières}\label{ss:regex}

Certains langages peuvent être exprimés par une \emph{expression régulière}.

Une expression régulière est un mot utilisant les symboles à représenter ainsi que les symboles (,),*,| qui sont réservés pour différentes opérations. Une expression régulière est construite à partir d'élements atomiques (les symboles du langage à représenter) et assemblés pour obtenir des langages plus complexes. Un language qui peut-être représenté par une expression régulière est dit \emph{langage régulier}.

Si plusieurs expressions régulières peuvent être composées en une expression plus complexe, une expression régulière peut aussi être décomposée en ses différents composants.

\textbf{Cas de base}
Certains langages peuvent être construits directement sans passer par l'induction:

\begin{itemize}
	\item $\epsilon$ est une expression régulière. Elle décrit le langage $L(\epsilon)=\{\epsilon\}$
	\item $\emptyset$ est une expression régulière décrivant $L(\emptyset)=\emptyset$
	\item Si $a$ est un symbole, alors $a$ est une expression régulière décrivant le langage $L(a) = \{a\}$.
\end{itemize}


\textbf{Induction}
Les autres langages réguliers sont construits suivant différentes règles d'induction présentées par ordre décroissant de priorité :

\begin{itemize}
	\item Si $E$ est une expression régulière, alors $(E)$ est une expression régulière et $L((E)) = L(E)$.
	\item Si $E$ est une expression régulière, alors $E^*$ est une expression régulière représentant la fermeture de $L(E)$, à savoir $L(E^*) = L(E)^*$.
	\item Si $E$ et $F$ sont des expressions régulières, alors $EF$ est une expression régulière décrivant la concaténation des deux langages représentés, à savoir $L(EF)=L(E)L(F)$.
	\item Si $E$ et $F$ sont des expressions régulières, alors $E+F$ est une expression régulière donnant l'union des deux langages représentés, à savoir $L(E+F)=L(E)\cup L(F)$. Ici encore, l'opération est associative et la priorité est à gauche.
\end{itemize}

\begin{example}[Expressions régulières]
	Soit l'expression $E = (b+ab)b^*a(a+b)^*$ qui décrit le langage $L=L(E)$.\\
	\begin{itemize}
		\item Le mot $ba$ fait partie de $L$. En effet, $ba=b\epsilon a \epsilon=(b)b^0a(a+b)^0$, ce qui respecte bien la définition de $E$.
		\item Le mot $ababbab$ fait partie de $L$. A nouveau, $ababbab=ab\epsilon a (a+b)(a+b)(a+b)(a+b)=(ab)b^0a(a+b)^4$.
		\item Le mot $aa$ ne fait \textbf{pas} partie de $L$. Supposons par l'absurde que $aa \in L$. Alors il existerait une façon de décomposer $E$ en $aa$. Or, les premiers symboles doivent être soit $b$, soit $ab$. Il y a contradiction. Donc, $aa \notin L$.
	\end{itemize}
	\label{ex:regex}
\end{example}

Un autre exemple d'expression régulière est $E=01^*0$. $E$ décrit le langage $L(E)$ constitué de tous les mots commençant et finissant par $0$ avec uniquement des $1$ entre les deux.
