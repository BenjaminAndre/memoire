
% ██      ██    █████
% ██       ██  ██   ██
% ██ █████  ██ ███████
% ██       ██  ██   ██
% ███████ ██   ██   ██

\subsection{Construction d'un automate depuis un langage régulier}

Comme vu dans la section \ref{ss:e-a}, il est intéressant de pouvoir convertir un problème sur un langage régulier en test d'appartenance à un automate. Il est cependant possible de le faire directement depuis le langage régulier grâce au théorème de Myhill-Nérode et à la relation \rl.

Soit le langage $A_N = \{w | w \text{finit par b et ne contient pas bb}\}$ défini sur $\Sigma_N = {a,b}$. Ce langage peut être écrit sous forme d'une expression régulière $E=a^*(ba)^*b$, ce qui confirme que le langage est régulier.

On peut diviser les mots de ce langage en 3 ensembles :

\begin{itemize}
 \item $W_0$ le sous-ensemble des mots ne finissant pas le symbole $b$
 \item $W_1$ celui des mots finissant par le symbole $b$ mais ne contenant pas $bb$
 \item $W_2$ celui des mots contenant au moins $bb$
\end{itemize}

Il y a d'autres façons de construire des sous-ensembles, mais celle-ci à l'avantage de rendre la question de l'appartenance à $L_N$ triviale : un mot appartient au second ensemble si et seulement si il fait partie du langage, par définition.

De plus, tous les éléments d'un sous-ensemble respectent la relation $R_L$ entre eux. ($R_L : xR_Ly \Leftrightarrow \forall z \in \Sigma^*, xz \in L \Leftrightarrow yz \in L$). Cela en fait des classes d'équivalence sur cette relation.

Cela peut être démontré pour chaque sous-ensemble :
\begin{itemize}
 \item Soient $x,y \in W_0$. Soit $z \in \Sigma^*$. Dès lors, si $xz \in L_N$, c'est que $z$ fini par $b$ mais ne contient pas $bb$, et donc $yz \in L_N$. Si $yz \in L_N$, le même argument peut être appliqué.
 \item Soient $x,y \in W_1$. Soit $z \in \Sigma^*$. Dès lors, si $xz \in L_N$, c'est que $z$ ne commençait pas le symbole $b$ et ne contenait pas $bb$, $yz$ ne contiendra donc pas $bb$, puisque cette chaîne n'est ni dans $z$ ni dans $y$, ni a cheval sur les deux, $z$ ne commençant pas par $b$. Ainsi, $yz \in L_N$. Si $yz \in L_N$, le même argument peut être appliqué.
 \item Soient $x,y \in W_2$. Soit $z \in \Sigma^*$. Comme $x$ contient déjà $bb$, $x \notin L_N$ et, a fortiori, $xz \notin L_N$. Comme la prémisse est fausse, l'implication $xz \in L \Rightarrow yz \in L$ est vraie. La même logique peut être appliquée à partir de $y$ pour justifier l'implication inverse.
\end{itemize}

De plus, ces sous-ensembles sont disjoints de par leur construction.

Ceci revient à démontrer que $W_0,W_1,W_2$ sont des classes d'équivalence. De plus, $R_L$ respecte la congruence à droite, comme démontré dans la preuve du théorème de Myhill-Nérode. Ce même théorème donne une méthode pour construire un automate : prendre un représentant pour chaque classe et en faire un état.

\begin{itemize}
 \item $\Sigma=\{a,b\}$ est connu.
 \item $Q=\{[[\epsilon]]\, [[b]], [[bb]]\} = \{q_\epsilon, q_b, q_{bb}\}$
 \item $q_0 = q_\epsilon$
 \item $F = \{q_b\}$ l'union des classes acceptant
 \item $\delta$ défini en utilisant des exemples tirés des classes d'équivalence.
\end{itemize}

Ce qui donne l'automate de la figure \ref{fig:an}

\begin{figure}[H]
 \centering
 \begin{tikzpicture}[->,>=stealth',shorten >=1pt,auto,node distance=2cm and 2cm, semithick, bend angle=10]

 \tikzstyle{every state}=[circle]

 \node[initial,state]	(A)					{$q_\epsilon$};
 \node[accepting,state]	(B)	[right= of A]	{$q_b$};
 \node[state]			(C) [right= of B]	{$q_{bb}$};

 \path
 (A)	edge	[bend left]		node{b}		(B)
 (A)	edge	[loop above]	node{a}		(A)
 (B) edge	[bend left]		node{a}		(A)
 (B) edge					node{b}		(C)
 (C)	edge	[loop above]	node{a,b}	(C)

 ;
 \end{tikzpicture}
 \caption{Automate $A_N$, exemple issu de la thèse\cite{Neider14}}\label{fig:an}
\end{figure}

Cet automate est bien une représentation du langage $L_N$. Seul un mot finissant par $b$ mais ne contenant pas $bb$ se termine à l'état $q_b$.



  %  █████  ██████  ██████
  % ██   ██ ██   ██ ██   ██
  % ███████ ██████  ██████
  % ██   ██ ██   ██ ██
  % ██   ██ ██████  ██

  \subsection{Alternating Bit Protocol}\label{ss:abp}

  \begin{figure}[H]
    \centering
    \begin{tikzpicture}[->,>=stealth',shorten >=1pt,auto,node distance=3.5cm, semithick, bend angle=10]

      \tikzstyle{every state}=[circle]

      \node[initial,state] (A)                    {$q_0$};
      \node[state]         (B) [right of=A] {$q_1$};
      \node[state]         (C) [below of=B] {$q_2$};
      \node[state]         (D) [left of=C] {$q_3$};

      \node[state,draw=none]         (i1) [right=0cm of B]      {};
      \node[state,draw=none]         (i2) [right=3.5cm of i1]      {};
      \node[state,draw=none]         (i3) [right=0cm of C]      {};
      \node[state,draw=none]         (i4) [right=3.5cm of i3]      {};

      \node[state] (E) [right=0cm of i2]               {$q_{0'}$};
      \node[state]         (F) [right of=E] {$q_{1'}$};
      \node[state]         (G) [below of=F] {$q_{2'}$};
      \node[state]         (H) [left of=G] {$q_{3'}$};



      \path
      (A) edge node {$\theta_1(A!0)$} (B)
      (B) edge node {$\theta_4(B?ACK0)$} (C)
      (B) edge[loop above] node {$\theta_2(A!0),\theta_3(B?ACK1)$} (B)
      (C) edge node {$\theta_5(A!1)$} (D)
      (D) edge node {$\theta_8(B?ACK1)$} (A)
      (D) edge[loop below] node {$\theta_6(A!1),\theta_7(B?ACK0)$} (D)


      (E) edge node {$\theta_{11}(A?0)$} (F)
      (E) edge[loop above] node {$\theta_9(B!ACK1),\theta_{10}(A?1)$} (E)
      (F) edge node {$\theta_{12}(B!ACK0)$} (G)
      (G) edge node {$\theta_{15}(A?1)$} (H)
      (G) edge[loop below] node {$\theta_{13}(B!ACK0),\theta_{14}(A?0)$} (G)
      (H) edge node {$\theta_{16}(B!ACK1)$} (E)
      ;

      \draw[<-] (E) -- node[above left] {start} ++(-1cm,-1cm);

      \draw[double,->] (i1) -- node[above] {canal A} (i2);
      \draw[double,->] (i4) -- node[below] {canal B} (i3);
    \end{tikzpicture}
    \caption{Automate Fifo du ABP (\cite{Finkel03}, Fig.1.)}\label{fig:fifoabp}
  \end{figure}


%  ██████ ██████   ██████  ███████ ███████
% ██      ██   ██ ██    ██ ██      ██
% ██      ██████  ██    ██ ███████ ███████
% ██      ██   ██ ██    ██      ██      ██
%  ██████ ██   ██  ██████  ███████ ███████


\subsection{Produit cartésien}\label{ss:cartesien}

Par soucis de simplicité, un automate à files (et son système de transitions servant à le représenter) peut être représenté comme plusieurs systèmes de transitions utilisant les mêmes canaux. Le \emph{produit cartésien} entre deux automates à files $A$ et $B$ retourne un nouvel automate $F=A \times B$. Dès lors, il est possible de représenter un automate à files en se concentrant sur ses parties et en les isolant \cite{Suresh20}. Cette propriété fait de l'automate à file un choix pertinent pour représenter des automates différents communiquant par des canaux. Dans une application pratique, cela pourrait être des programmes communiquant sur un réseau.

Ce produit cartésien fonctionne comme suit.

Soient les automates à files \fifoA et \fifoB. Alors le système de transitions \tsys de l'automate à files $F=A\times B$ est composé de :
\begin{itemize}
  \item $S \subseteq (Q_A\times Q_B)\times (\Sigma^*)^C$ composé d'un couple d'états de contrôle de $Q_A$ et $Q_B$ et du contenu des différents canaux.
  \item $\Theta$ est un ensemble de noms de transitions.
  \item $\rightarrow$ est construit comme suit. Soit un état $((q_A,q_B), w)\in S$. Soit un triplet $(p,a,q)$ avec $p,q \in (Q_A \bigcup Q_B)$ et $a \in ((C \times \{?,!\} \times \Sigma) \bigcup \{\tau\})$.
  $((q_A,q_B),w)\xrightarrow{\theta}((q_{A'},q_{B'}),w')$ si et seulement si l'une des trois conditions suivantes est remplie.
  \begin{itemize}
    \item $\exists \theta_A\in\Theta_A, \delta_A(\theta_A)=(q_A,a,q_{A'})$ et  $(q_A,w)\xrightarrow{\theta_A}(q_{A'},w')$ dans l'automate $A$ et\\ $\exists \theta_B\in\Theta_B, \delta_B(\theta_B)=(q_B,a,q_{B'})$ et $(q_B,w)\xrightarrow{\theta_B}(q_{B'},w')$ dans l'automate $B$
    \item $\exists \theta_A\in\Theta_A, \delta_A(\theta_A)=(q_A,a,q_{A'})$ et  $(q_A,w)\xrightarrow{\theta_A}(q_{A'},w')$ dans l'automate $A$,\\
    $\forall \theta_B\in\Theta_B,\forall q \in Q_B,\delta_B(\theta_B)\neq(q_B,a,q)$ dans l'automate $B$ et $q_{B'}=q_B$
    \item $\forall \theta_A\in\Theta_A,\forall q \in Q_A,\delta_A(\theta_A)\neq(q_A,a,q)$ dans l'automate $A$ et $q_{A'}=q_A$,\\
    $\exists \theta_B\in\Theta_B, \delta_B(\theta_B)=(q_B,a,q_{B'})$ et  $(q_B,w)\xrightarrow{\theta_B}(q_{B'},w')$ dans l'automate $B$
  \end{itemize}
\end{itemize}

Le produit cartésien est un nouvel automate représenté par son système de transitions. Celui-ci étant suffisant pour déduire le langage tracé, il n'est pas necéssaire de décrire formellement \fifo.

De plus, ce nouvel automate est différent des deux autres, il n'est alors pas pertinent de prouver une égalité. Il s'agit juste d'un autre mode de représentation.

% ███████ ██   ██
% ██       ██ ██
% █████     ███
% ██       ██ ██
% ███████ ██   ██


\begin{example}
  Soient deux automates à files $A$ et $B$ tels que représentés par leur systèmes de transitions donnés par la figure \ref{fig:fifoAB}.

  \begin{figure}[H]
    \centering
    \begin{subfigure}{0.5\textwidth}
      \centering
      \begin{tikzpicture}[->,>=stealth',shorten >=1pt,auto,node distance=1.5cm, semithick, bend angle=10]
        \tikzstyle{every state}=[circle]

        \node[initial,state] (A)  {$q_{0}$};
        \node[state]         (B) [right=of A]  {$q_{1}$};
        \node[state]         (C) [right=of B]  {$q_{2}$};
        \path
        (A) edge node {$\theta_1(a!1)$} (B)
        (B) edge node {$\theta_2(a?1)$} (C)
        (C) edge[bend left=40] node {$\theta_3(a!0)$} (A)
        ;
      \end{tikzpicture}
      \caption{Automate à files A}
      \label{fig:fifoA}
      \end{subfigure}%
      \begin{subfigure}{0.5\textwidth}
        \centering
        \begin{tikzpicture}[->,>=stealth',shorten >=1pt,auto,node distance=1.5cm, semithick, bend angle=10]
          \tikzstyle{every state}=[circle]

          \node[initial,state] (A)  {$q_{A}$};
          \node[state]         (B) [right=of A]  {$q_{B}$};
          \path
          (A) edge[bend left=20] node {$\theta_5(a?0)$} (B)
          (B) edge[bend left=20] node {$\theta_6(a!0)$} (A)
          ;
        \end{tikzpicture}
        \caption{Automate à files B}
        \label{fig:fifoB}
      \end{subfigure}
      \caption{Automates à files A et B représentés par leur système de transitions}
      \label{fig:fifoAB}
    \end{figure}

   L'automate $AB=A \times B$ est représenté par son sytème de transitions à la figure \ref{fig:fifocross}.

    \begin{figure}[H]
      \centering
      \begin{tikzpicture}[->,>=stealth',shorten >=1pt,auto,node distance=1.5cm, semithick, bend angle=15]
        \tikzstyle{every state}=[circle]

        \node[initial,state] (A)  {$(q_0, q_A)$};
        \node[state]         (B) [below=of A]  {$(q_1, q_A)$};
        \node[state]         (C) [below=of B]  {$(q_2, q_A)$};

        \node[state]         (D) [right=of A] {$(q_0, q_B)$};
        \node[state]         (E) [below=of D]  {$(q_1, q_B)$};
        \node[state]         (F) [below=of E]  {$(q_2, q_B)$};

        \path
        (A) edge node {$\theta_1(a!1)$} (B)
        (A) edge[bend left] node {$\theta_2(a?0)$} (D)

        (B) edge node {$\theta_3(a?1)$} (C)
        (B) edge[bend left] node {$\theta_4(a?0)$} (E)

        (C) edge[bend left=35] node {$\theta_5(a!0)$} (A)
        (C) edge node {$\theta_6(a?0)$} (F)

        (D) edge[bend left] node {$\theta_7(a!0)$} (A)
        (D) edge node {$\theta_8(a!1)$} (E)

        (E) edge[bend left] node {$\theta_9(a!0)$} (B)
        (E) edge node {$\theta_{10}(a?1)$} (F)

        ;

        \draw [->] (F) ..  controls  ($(E)+(5cm,2cm)$) and
  ($(D)+(0cm,4cm)$).. node[right] {$\theta_{11}(a!0)$} (A);
      \end{tikzpicture}
      \caption{Automate AB résultant du produit cartésien $A\times B$}
      \label{fig:fifocross}
    \end{figure}
  \end{example}\label{ex:fifocross}


  L'exemple \ref{ex:fifocross} suffit à se convaincre qu'on peut parler indistinctement de plusieurs systèmes de transitions comme représentant un seul automate. Dans ce cas, il est sous-entendu que l'automate en question est celui obtenu par produit cartésien.
